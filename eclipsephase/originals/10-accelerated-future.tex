%%% EDITION: 2nd Printing

\chapter{Accelerated Future}
\label{cha:accelerated-future}


%%% Local Variables: 
%%% mode: latex
%%% TeX-master: "ep"
%%% End: 

%%% 268
%%% 269
%%% 270
The future setting of \textit{Eclipse Phase} introduces a 
number of technological elements that have a strong 
impact on transhuman society. These include backups 
and uploading, resleeving, egocasting, forking, nano-fabrication
reputation systems, space habitats, and
space travel, among others.

\section{Backups And Uploading}

The transhuman mind is no longer a prisoner of the 
biological hardware on which it originates. Through 
various mechanisms, biological brains may be digitally 
emulated, allowing people to make a \textit{backup }of their 
minds, including their entire personality, memories, 
and skills—a process known as \textit{uploading.}

The primary use of backups is to ensure the person's 
ego can be retrieved in case of death, in which case 
they may be \textit{resleeved} (p. 271). For this reason, almost 
everyone in the solar system is equipped with a cortical
stack (p. 300). Backups may also be safely archived
in secure storage (p. 269) or used to create infomorphs 
(p. 264). A person may also egocast themselves across 
the solar system as a form of travel (p. 276).

\subsection{Cortical Stack Backups}

Cortical stack implants deploy a network of nanobots
throughout the brain that take a snapshot of
the mind's neural state, storing the data as a backup 
within the cortical stack. The average transhuman's 
cortical stack backs up their ego 86,400 times per day. 
Only the most recent backup is kept within the stack; 
older ones are overwritten. Pods and synthmorphs 
also can be equipped with cortical stacks (though 
AI-piloted bots often lack this feature), though these 
versions maintain an updated copy of the ego running 
in the morph's cyberbrain.

In the case of death, accidental or otherwise, a cortical
stack can be retrieved from a corpse and used to
recover the character, either as an infomorph or by 
resleeving them in a new morph. Cortical stacks are 
diamond-hardened and protected, so they may often 
be retrieved even if the corpse is badly mangled or 
damaged. If the corpse cannot be recovered or the 
cortical stack is destroyed, the backup is lost.

High rollers, well-equipped brinkers, and others 
in dangerous professions often opt for an emergency 
farcaster accessory (p. 306) that periodically (usually 
every 48 hours, but varying according to contract) 
transmits a backup from the cortical stack to a remote 
storage facility. This option is quite expensive, however
and so is generally only afforded by the wealthy.

\subsubsection{Retrieving A Cortical Stack}

Most cortical stacks are carefully excised from a 
corpse with surgery. In certain circumstances, however, 
a character may need to extract a cortical stack in the 
field, whether because transporting the corpse is impractical
or because the dead person is an enemy and
they either don't want them knowing who killed them 
or they want to interrogate them with psychosurgery 
in a simulspace.

The process of cutting out a cortical stack is called 
``popping,'' as a skilled extractor can usually get the 
smooth-shelled implant to pop right out by making 
an incision in the correct place and applying pressure. 
One does need to be careful that the tiny, blood-slick 
stack doesn't slip away once popped.

Popping can be done with a sharp knife and elbow 
grease, though it is grisly. Popping a stack is a Task 
Action that requires a Medicine: [any appropriate 
field] Test with a timeframe of 1 minute and a modifier
of +20. Morphs with stacks in non-standard locations
or with anatomical shielding (carapace plates,
etc.) around the stack may incur penalties to this test 
at the gamemaster's discretion. Of course, if you don't 
have the time for a precise extraction, you can always 
just cut the entire head off and take it with you.

Once a cortical stack is retrieved, it may be loaded 
into an ego bridge (p. 328) and used to bring the ego 
back, either as an infomorph or by resleeving.

\textbf{Living Subjects:} Cortical stacks may be excised 
from living people, but the process is usually fatal (or 
at least paralyzing) as it involves cutting through the 
spinal column. If the target is not unconscious or otherwise
incapacitated, they must first be immobilized
in melee combat (see \textit{Subdual,} p. 204). Cutting out 
the stack is handled like a Medicine Task Action as 
above, but this process inflicts 3d10 + 10 damage on 
the target. If the test fails, they still inflict 1d10 + 10 
damage to the target. If the person removing the stack 
wants to leave the target alive or harm them as little 
as possible, they suffer a –20 modifier on the test, but 
may reduce the damage by 1d10 per 10 full points of 
MoS. Living through the process of having your stack 
removed is traumatic; anyone who does so suffers 
1d10 mental stress.

\subsubsection{Destroying A Cortical Stack}

Cortical stacks have an Armor of 20 and a Durability 
of 20 for anyone attempting to destroy them.

\subsection{Uploading}

Uploading a backup into secure storage is usually 
handled with a brain scan at the storage facility's clinic 
using a bread box-sized unit called an \textit{ego bridge} (p. 
328). When activated, the ego bridge's sensor array 
twists open like a morning glory blossom, revealing an 
enclosure with a neck rest that automatically adjusts 
itself to morphs with oddly-sized or -shaped heads. 
The neck rest deploys millions of specialized nanobots 
%%% 271
into the brain and central nervous system. The petals 
are full of sensors that image the brain using a combination
of MRI, sonogram, and positional information
broadcast by the nanobot swarm in the morph's brain. 
The ego bridge then builds a digital copy of the person's
brain, which is stored away in the service's highly
secure, off-the-mesh, hardwired data vaults.

In the case of pods, the ego bridge scans the biological
brain bits and also accesses the cyberbrain to copy
the parts of the ego residing there. For synthmorphs, 
who have no biological brain, the process is much 
simpler, as it simply requires accessing and making a 
copy of their cyberbrain.

In a standard clinic with an undamaged morph, uploading
takes only 10 minutes, 5 with a pod. In other
situations, however, the process may take longer if the 
gamemaster so decides. Uploading from a synthmorph 
or extracted cortical stack is instantaneous. The ego 
bridge largely operates itself. While oversight by a 
medical specialist is a good idea, no test is necessary.

If an uploading character does not plan to return 
to their morph, it is usually put on ice until someone 
else resleeves into it. If a new resleeve is not ready 
and the uploading character doesn't want to leave a 
potential copy of themselves behind, they can have 
the morph's mind wiped by the nanobots as part of 
the uploading process.

\subsubsection{Uploading-Resleeving Continuity}

In ideal circumstances, a person who is intentionally 
resleeving (p. 271) can arrange for the uploading and 
resleeving process to occur without any noticeable 
loss of continuity. Though the experience of switching 
from one morph to another is still a bit jarring, the 
transition itself can be made into a seamless process, 
with no gaps in awareness or memory, which helps 
reduce associated mental stress.

In this case, during the process of uploading, the 
ego bridge is also connected to another ego bridge 
and the new sleeve. This connection can even be made 
wirelessly or by farcaster link (with a maximum distance
of 10,000 kilometers).

As the mind is uploaded, the ego bridge builds a 
virtual brain by copying the morph's brain bit by bit, 
using the data gained from the brain scan. At the same 
time, this data is slowly copied to the new sleeve as 
nanobots rewire the sleeve's brain structure (a much 
slower process). As the transfer occurs, the nanobots 
in the brain sever individual neural connections and 
re-route them to their duplicates in the virtual brain, 
and then eventually to the new brain. Effectively, the 
character's ego is running partially on the meat brain 
and partially on the virtual copy. By the time the nanobots
sever the last of the neural connections in the
old brain, the ego is running completely on the virtual 
brain and the new sleeve's brain. Once the resleeving 
is completed, the virtual brain is shut down.

In terms of perceptions, the character, who is awake 
during this process, experiences a very gradual shift 
from one morph to the other. As the process takes 
hours, however (or even longer if done via farcaster), 
the subject usually entertains themselves with some 
AR media, VR, or even XP to pass the time.

\subsubsection{Uploading After Death}

It is possible to upload the mind of a person who 
has recently died as long as the nanobots have time 
to scan the brain before cell deterioration kicks in 
too heavily, which takes approximately 2 hours. It 
is possible to sustain a corpse for longer by placing 
it in a healing vat (p. 326) for nanostasis. Post-death 
uploads may suffer integrity damage; see \textit{Backup }
\textit{Complications,} p. 270.
Cyberbrains may also be retrieved from a destroyed 
synthmorph and reactivated, assuming they are not 
damaged too heavily (gamemaster discretion).

\subsubsection{Destructive Uploading}

Though rare, some people engage in a process called 
destructive uploading, where the biological brain is literally
sliced apart and scanned piece by piece. Considered
abhorrent and wasteful by most transhumans, brain-peeling
is practiced by some bioconservative factions
who view it as the only ``pure'' method of uploading 
or the only real way to transfer the ``soul.'' Such people 
typically refuse to resleeve, living out the rest of their 
lives as infomorphs, quite often in dedicated simulspaces 
that are treated as a sort of virtual afterlife.

\subsection{Backup Insurance}

Almost everyone, with the exception of neo-primitivists 
and very young children, has a cortical stack. In the 
event of death, however, a cortical stack alone will not 
ensure resurrection unless you have acquired backup 
insurance (p. 330) to cover the costs of your resleeving
Going without backup insurance for any length of
time is taking a severe risk. Some jurisdictions (such 
as the Titanian Commonwealth) have a practice of 
bringing everyone back, even if only to an infomorph 
state, or at least filing the most recent backup away 
in dead storage just in case someone decides to pay 
to resurrect them later. Other authorities will simply 
destroy the stack or, worse, sell it on the black market 
to a soul-trading syndicate such as Nine Lives.
Backup insurance typically includes a subscription 
to an uploading facility, usually requiring a visit every 
6 months, to ensure that backup is held in safe storage
in case of cortical stack loss. People with risky
jobs (construction bot supervisor, hypercorp exoplanet
staff, girl who fights vicious giant eels for rich
jaded audiences, etc.) may back up once a week, or 
even daily. In the event of a verified death where the 
cortical stack could not be retrieved, the most recent 
backup is used to resleeve the person.
At the basic level, backup insurance will bring the 
character back as an infomorph, at which point they 
can access their credit and purchase a new morph. 
More expensive versions will automatically resleeve 
%%% 272
you in the pre-purchased morph of your choice. The 
exceedingly rich will often have customized clones 
(often of their original body) waiting on ice for them.

Backup insurance often involves a missing person 
clause, which states that a person will be brought 
back if they have not checked in for X amount of time 
(a calendar function automatically handled by your 
muse) and cannot be located.

It is worth noting that some criminal syndicates 
also offer backup insurance at a much reduced rate. 
The likelihood that copies of your backup are being 
used for illicit purposes, however, is quite high. For 
some people, however, what happens to a copy of 
themselves is of no concern.

\subsubsection{Backup Insurance Limitations}

Backup insurance is not always perfect. Though insurance
providers are required to make a reasonable
effort to retrieve your cortical stack, for many hypercorps
this is a simple cost-benefit analysis that often
will not work in the character's favor. If you died in a 
dangerous area such as the Zone on Mars, in a remote 
area such as the Kuiper Belt, or are simply difficult 
to track down (pushed out an airlock somewhere), 
odds are against your cortical stack being retrieved—
instead you will be re-instanced from a backup.

Jurisdiction can also play an important role. The 
insurance offered by many inner system providers is 
automatically nullified if you travel to an anarchist 
habitat, gatecrash, break the law, or engage in certain 
life-threatening activities like suicide sports or scavenging
in TITAN-infected ruins. At the least, they will
refuse to retrieve your stack in these circumstances. 
Likewise, if you struck a backup insurance deal with 
a medical collective from an autonomist habitat and 
then go and die on a hypercorp station, the hypercorp 
is very likely to refuse to recognize the authority of a 
bunch of anarchists and won't hand your stack over.

Even an archived backup and a missing person 
clause is no guarantee. A determined enemy could 
capture you, pry the backup insurance access codes 
from your muse, keep you on ice or quietly kill you, 
and then regularly ``check in'' on your behalf using 
the access codes so that the insurance provider never 
realizes you are dead or missing. Though this requires 
quite a bit of effort, it is often less difficult than dealing
with an immortal opponent who keeps coming
back no matter how often you kill them.

Other dangers also exist. An entire habitat may be 
destroyed, taking you, your backups, and your insurance
provider's records with it. A resourceful enemy
might penetrate a provider's security and delete your 
backups, or simply bribe the right people to make 
sure they get ``accidentally'' corrupted. Given these 
possibilities, the paranoid often make sure to get 
multiple redundant backup policies, assuming they 
can afford it.

\subsection{Backup Complications}

In most cases, backing up/uploading is risk free unless 
someone tampers with the equipment. If the character 
suffered brain or neurological damage, the backup is 
transferred via farcasting, or the upload is made from 
a dead character, then the backup may be damaged 
due to missing neural information. In any of these 
instances, make a LUC Test for the character. If the 
test fails, they suffer 1 point of mental stress per 10 
full points of MoF. Note that this stress (and possible
trauma applies to the backup, not the original
character. If the backup is used to re-instantiate the 
character, however, then the stress is applied.

\section{Resleeving}

\textit{Resleeving} (also called remorphing) is the process 
of giving a new body to an ego. Changing bodies 
is a normal part of life for hundreds of millions of 
transhumans, and it is an even more frequent occurrence
for people in certain professions. Characters
involved in specialized work may resleeve as often as 
once a month. Those who travel frequently may do so 
even more often. Also, given the number of infugees 
%%% 273
who died during the Fall but have now acquired a 
new morph, the vast majority of transhumanity has 
resleeved at least once. As such, most transhumans are 
accustomed to resleeving.

Adjusting to a new body takes time and a bit of 
effort (see \textit{Integration,} p. 272). Resleeving is also difficult
psychologically, as reflected by continuity (p. 272)
and alienation (p. 272).

Once an ego fully inhabits a new morph, the new 
morph's cortical stack needs ten minutes to amass a 
complete backup of the ego.

\subsection{Resleeving Biomorphs And Pods}

Resleeving takes about an hour in a properly equipped 
clinic. In essence, the process works like uploading in 
reverse. The new sleeve is hooked up to an ego bridge 
which infiltrates the brain with nanobots that physically
restructure the brain's neural structure and connections
according to the map provided by the backup.
Sleeving takes six times as long as uploading because 
the nanobot swarm working as a wet printer in the 
template brain needs to duplicate the entire physical 
structure of the ego's neural network. For resleeving, a 
``wet'' ego bridge is used, meaning that the sleeve and 
ego bridge are submerged in a vat filled with nanogel.

The resleeving process for pods takes only half an hour, 
as pods brains are half biological and half cyberbrain.

\begin{quotation}
\textbf{Resleeving And The Gamesmaster} %RESLEEVING AND THE GAMEMASTER}
\\
The gamemaster has a fine amount of control over what a character can obtain when resleeving. The
characters may be supplied with new morphs by Firewall or whatever employer/patron for whom they are
currently working. In this case, the gamemaster can simply assign whatever morphs they see fit — with complete
control over enhancements, traits, etc. While morphs should be tailored for the mission at hand, this
presents an opportunity for the gamemaster to throw the characters some new toys to play with and also
some new challenges to overcome. Gamemasters are encouraged to mix it up, have fun, and give players
something they can work with without necessarily giving them everything they want.\\
In other cases, the availability of desired morphs may be limited by the resleeving location. A small
outpost in the wilds of Mars is unlikely to have a wide selection of morphs — in fact, a few rusters and
synthmorphs may be all they have. Similarly, large habitats have a high demand for good morphs, so there
may be a waiting list for top-of-the-line sylphs or remade morphs, for example. In the same vein, available
morphs are going to be subject to local legalities, so getting that reaper morph may be out of the question.
Characters could always turn to black market morph providers, but these come with their own risks.\\
What this means is that gamemasters should never be afraid to say no if a character is pursuing a morph
that is unreasonable or potentially disruptive to the game. While it’s good form to give the players what they
want once in a while, it also makes for more interesting roleplaying to saddle them with morphs that are
a little different than what they were hoping for or that come with some interesting challenges, such as a
physical addiction. For extra fun, leave the character unaware of a morph’s negative traits or secret implants
until they reveal themselves. As always, the goal is to have fun, but variety often helps with that.
\end{quotation}

\subsection{Resleeving Synthmorphs}

Resleeving into the cyberbrain of a synthmorph is much 
easier and quicker, being a matter of copying the backup 
into the cyberbrain (an instantaneous affair) and then 
running the backup in its virtual brain state (1 Action 
Turn). The drawback to synthmorphs is that they are 
more difficult to acclimate to (see \textit{Integration,} p. 272), 
they are vulnerable to cyberbrain hacking (p. 261), and 
synthmorphs are viewed as low class in some cultures.

\subsubsection{Evacuating A Cyberbrain}

Characters inhabiting a synthmorph cyberbrain may 
voluntarily choose to evacuate by copying themselves 
as an infomorph onto another device. This takes 1 full 
Action Turn. See \textit{Infomorph Resleeving,} p. 273.

\subsection{Resleeving Costs}

The costs involved for the resleeving process itself are 
generally subsumed in the costs of the backup insurance
and/or the new sleeve itself. Costs for individual
morphs are noted in the descriptions starting on p. 
139. See \textit{Morph Brokerage} (p. 276) for rules on finding
and acquiring morphs.

\subsection{Integration}

Getting used to a new body typically takes some time. 
The character must become acclimated to the changes 
in height, weight, sex, and capabilities, which often 
requires unlearning ways of doing things that worked 
fine for their previous form. Resleeving in a synthetic 
morph or an uplift is also quite confusing at first, 
given the drastically different morphologies, change in 
limb structure (and sometimes amount of limbs), and 
so on. Luckily, transhuman minds are adaptive things, 
and this process is aided by the application of mental 
``patches'' during the resleeving process that give the 
character a bit of a boost for using their new body.

An ego in a new morph makes an Integration Test 
upon taking control of the body, rolling SOM x 3 (morph 
bonuses do not apply) and applying modifiers from the 
Integration and Alienation Modifiers table. The result of 
the test is explained on the Integration Test table, p. 272.
%%% 274
\\


\begin{table}
\caption{Integration Test}
\begin{tabular}{|l|l|}
%\hline
%\multicolumn{2}{|c|}{INTEGRATION TEST}\\
\hline
TEST RESULT & EFFECT\\
\hline
Critical Failure & Character is unable to acclimate to the new morph—\\
& something is just not right. Character suffers a –30\\
& modifier to all physical actions until resleeved. \\
\hline
Severe Failure (MoF 30+) & Character has serious trouble acclimating to the new \\
& morph. They suffer a –10 modifier to all actions for 2 days \\
& plus 1 day per 10 full points of MoF. \\
\hline
Failure & Character has some trouble acclimating to new morph. \\
& They suffer a –10 modifier to all physical actions for 2 \\
& days plus 1 day per 10 full points of MoF. \\
\hline
Success & Standard acclimation period. The character suffers a –10 \\
& modifier to all physical actions for 1 day. \\
\hline
Excellent Success (MoS 30+) & No ill effects. Character acclimates to new morph in no \\
& more than a few minutes. \\
\hline
Critical Success & Lookin’ good! This morph is an exceptionally good fit for \\
& the character. No ill effects; gain 1 Moxie point for use in \\
& that game session only. \\
\hline
\end{tabular} 
\end{table}

\begin{table}
\caption{Integration and alienation modifiers}
\begin{tabular}{|l|l|}
%\hline
%\multicolumn{2}{|c|}{INTEGRATION AND ALIENATION MODIFIERS} \\
\hline
TEST RESULT & EFFECT \\
\hline
Familiar; character has used this exact morph extensively in the past & +30 \\
\hline
Clone of prior morph & +20 \\
\hline
Character’s original morph type (what they were raised with) & +20 \\
\hline
Adaptability trait (Level 2) & +20 \\
\hline
Adaptability trait (Level 1) & +10 \\
\hline
Character has previously used this type of morph & +10 \\
\hline
First time resleeving & –10 \\
\hline
Character is an AGI sleeving into a physical body & –10 \\
\hline
Character is an uplift resleeving in a non-uplift (of their type) body & –10 \\
\hline
Synthetic morph & –10 \\
\hline
Sex change (from last morph) & –10 \\
\hline
Morph is heavily modified & –10 \\
\hline
Morphing Disorder trait (Level 1) & –10 \\
\hline
Morphing Disorder trait (Level 2) & –20 \\
\hline
Infomorph (does not apply to AGIs) (Alienation Test only) & –20 \\
\hline
Fork (Alienation Test only) & –20 \\
\hline
Morphing Disorder trait (Level 3) & –30 \\
\hline
Exotic morph (octomorph, neo-avian, novacrab, swarmanoid, etc.) & –30 \\
\hline
\end{tabular}
\end{table}

\subsection{Alienation}

After loss of continuity, the other major factor impacting resleeving characters is alienation. Once the ego 
has its new sleeve under control, it's time to look in 
the mirror. The alienation test reflects the experience of 
coming to terms with a new face, skin, and brain. For 
example, transferring to a radically different morph 
(such as a swarmanoid) can be difficult to grasp. 
Uplifts often have difficulty getting acquainted with 
the differing hormonal urges of a human biomorph 
and vice versa. While the character's ego is as it was 
in their last sleeve, the brains and neurochemistry of 
many morphs may alter aptitudes like WIL or COG. 
The effects of this can be frustrating or disorienting.
Every character makes an Alienation Test to reflect 
how mentally stressful it is to get a grip on their new 
body, rolling INT x 3 and apply modifiers from the 
Integration and Alienation Modifiers table. Consult 
the Alienation Test table to determine the effects.
\\

\begin{table}
\caption{Alienation test}
\begin{tabular}{|l|l|}
%\hline
%\multicolumn{2}{|c|}{ALIENATION TEST} \\
\hline
TEST RESULT & EFFECT \\
\hline
Critical Failure & Extreme Dysmorphia. The character doesn’t like their new sleeve at all \\
& and suffers 2 stress points per 10 full points of MoF. \\
\hline
Failure & Character is uneasy about the new morph and suffers 1 stress point \\
& per 10 full points of MoF. \\
\hline
Success & Character adapts to their new look well. No ill effects. \\
\hline
Critical Success & Best. Morph. Ever. The new morph jives perfectly with the character’s \\
& sense of self, and even enhances it somewhat. The character actually \\
& heals 1d10 $\div$ 2 (round up) stress points. \\
\hline
\end{tabular}
\end{table}

\subsection{Continuity Test}

Perhaps the biggest shock that strikes most resleeving 
characters is the loss of continuity of self. This is particularly true for characters who died. If their cortical 
stack was retrieved, they will remember their own 
death. If they were restored from an archived backup, 
they will not remember their death, but they will have 
lost an entire period of their life—all the way back to 
their last backup. In fact, if their body was not recovered, they may not even know that they are dead for 
certain—there may be a surviving copy of themselves 
out there. The driving point in this loss of continuity is 
a sort of existential crisis—they are no longer the original person they once were. This leads some to question 
whether they are who they think they are, or are they 
some poor imitation and not a real person at all?
To determine how this loss of continuity affects a 
character, make a Continuity Test by rolling WIL x 3. 
Every character suffers stress from loss of continuity, 
as noted on the Continuity Stress table. Reduce this 
stress damage by 1 point per 10 full points of MoS on 
the Continuity Test, or increase it by 1 point for every 
10 full points of MoF.

\subsection{Infomorph Resleeving}

Rather than resleeving into a physical body, a backup 
may instead by instantiated as an infomorph, a purely 
digital form. Infomorphs are distinct from backups in 
that backups are inert files. Infomorphs are backups 
imprinted onto a virtual brain template and run as 
a program. This virtual brain state must be run on a 
specific device and follows all of the rules noted for 
infomorphs on p. 264. Infomorphs may copy themselves to other devices, typically erasing themselves 
from the previous device as they go. Infomorphs that 
copy without erasing are treated as forks.
Characters instantiating as infomorphs must make 
Continuity and Alienation Tests, just like resleeving.
Infomorphs may be resleeved into physical morphs, 
following normal resleeving rules.
%%% 275

\begin{quotation}
I wake up with a taste like guava and umami
fresh on my tongue. Last night there was laughter.
We drank quinoa wine, and I was introduced
to people I had never met before, though I had
years of intimate knowledge of most of them.
Half of Illyria Module is curled naked around
me in my sleeping chamber. Last night we
made music with synthesizers, wood blocks,
and a lur. We drank mushroom tea brewed in
water from a rogue comet. Looking around me
as the morning sun starts to light the far orbital
horizon of Ceres, it appears we had an orgy.
Last night was my resleeving party. This version
of me—me 3.0—is ready for life.
—Zheng du Thierry, Carnival of the Goat
\end{quotation}


\section{Forking And Merging}

With all of these backups of transhuman minds on 
file and an abundance of mesh space on which to run 
them as virtual brains, one might wonder what's to 
stop post-Fall transhumanity from multiplying its 
numbers by running additional copies of them. The 
short answer is: nothing, aside from massive social 
stigma and thorny psychological issues. Taking a 
backup of a transhuman mind, copying it, and re-instancing
it as an infomorph is called \textit{forking.} It's one
of the most useful and still-controversial applications 
of transhumanity's brain science.

There are four classifications of forks: alpha, beta, 
delta, and gamma. Though typically copied as infomorphs
there is nothing preventing a fork from being
sleeved in a physical morph as well, other than legalities
and custom.

\begin{table}
\caption{Continuity stress}
\begin{tabular}{|l|l|}
\hline
SITUATION & STRESS VALUE \\
\hline
\textbf{Backup from cortical stack} & \\
\hline
Character remembers peaceful or not notable death & 1d10 $\div$ 2 (round down) \\
\hline
Character remembers sudden or violent death & 1d10 \\
\hline
\textbf{Backup from archive} & \\
\hline
Short memory gap (less than 1 day) & 1d10 $\div$ 2 (round down) \\
\hline
Memory gap greater than one day & 1d10 \\
\hline
Not knowing if/how you died & +2 \\
\hline
Uploading-to-resleeve with continuity (p. 269) & 0 \\
\hline
Uploading-to-resleeve without continuity & 1d10 $\div$ 2 (round down) \\
\hline
Character is a fork & 2 \\
\hline
\end{tabular}
\label{table:continuity-stress}
\end{table}

\subsection{Alpha Forks}

An \textit{alpha fork} is an exact copy of the original ego, 
re-instanced as a separate infomorph. An alpha fork 
may be created by copying and running an infomorph 
(from a backup, infomorph, synthmorph cyberbrain, 
or a removed cortical stack in an ego bridge). Alpha 
forks mat be generated from biomorph brains using 
an ego bridge and the same process as uploading (p. 
268). Alpha forks are an exact copy of the character's
ego, with all of the same skills, memories, stats,
traits, personality, etc. New alpha forks must make 
an Alienation Test (p. 272), and possibly a Continuity 
Test (p. 272) if copied from a backup.

Creating alpha forks is illegal in many jurisdictions
including most of the inner system and the
Jovian Republic. In others it tends to be viewed with 
distaste, though there are some habitats/cultures in 
which it is encouraged.

\subsection{Beta Forks}

Beta forks are partial copies of the ego. They are 
intentionally hobbled so as to not to be considered 
an equal to the character, for legal and other reasons. 
Beta forks have most of the same skills as the original 
ego, though sometimes reduced. Their memories are 
also drastically curtailed, usually tailored to whatever 
task they are intended to perform.
Beta forks are created by taking an alpha fork and 
running it through a process known as \textit{neural pruning} 
(p. 274). They are legal and even common in many 
places, except for bioconservative holdouts like the 
Jovian Republic, though delta forks are more favored. 
Beta forks rarely have anything resembling civil rights 
or citizenship and are usually treated as the property 
of the originating ego. They are commonly used as 
digital aids or to represent the original ego when communicating
with others over great distances. \\
A beta fork's stats are determined as follows:

\begin{itemize}
\item Reduce all aptitudes by 5 (to a minimum of 1). 
This affects all skills as well. Likewise, this reduces
LUC by 10 and INIT by 20.
\item Active skills have a maximum value of 60.
\item Moxie is reduced to 1.
\item The Psi trait is removed. At the gamemaster's discretion
other traits may no longer apply as well.
\end{itemize}

Additional changes may apply as determined by 
the neural pruning test. Beta forks take 1 minute 
to generate.

\subsection{Delta Forks}

Delta forks are extremely limited copies of an ego. 
They are more akin to AI templates upon which 
the ego's surface personality traits are imprinted. 
Also created via neural pruning, delta forks are 
highly functional (as competent as a beta fork or 
AI) but have extremely limited skills and heavily
edited memories, usually to the point of being
functional amnesiacs. \\
A delta fork's stats are determined as follows:

\begin{itemize}
\item Reduce all aptitudes by 10 (to a minimum of 1). 
This affects all skills as well. Likewise, this reduces
LUC by 20 and INIT by 40.
\item Active skills have a maximum value of 40. The 
fork may have no more than 5 Active skills.
%%% 276
\item Knowledge skills have a maximum of 80. The 
fork may have no more than 5 Knowledge skills.
\item Moxie is reduced to 0.
\item The Psi trait is removed. At the gamemaster's discretion, other traits may no longer apply as well.
\end{itemize}

Additional changes may apply as determined by the 
neural pruning test. Delta forks take 1 Action Turn 
to generate.

\subsection{Gamma Forks}

More commonly known as \textit{vapors,} gamma forks are 
massively incomplete, corrupted, or heavily damaged 
copies of an ego. Vapors are not intentionally created
and are instead the results of botched uploads,
scrambled backups, incomplete or jammed farcasts, 
or infomorphs/forks that were somehow damaged or 
went insane. It is extremely rare for anyone to purposely
create a vapor for anything other than research
use, although they can crop up in some interesting 
places. For example, poorly made skill software occasionally
includes enough of the personality traits and
memories of the person the skill was taken from that 
it can behave in a vapor-like fashion when used.

Because vapors are anomalies rather than purposeful
creations, the characteristics of individual gamma
forks are left to the gamemaster. They should have 
some or all of the following: reduced skills, reduced 
aptitudes, incomplete or incoherent memories, negative
mental traits, and persistent mental stress or traumas
including derangements and/or disorders.

\subsection{Neural Pruning}

Neural pruning is the art of taking a backup/infomorph
and trimming it down to size so that it functions
as either a beta or delta fork.

Beta forks are created by taking a virtual mind state 
that is intentionally inhibited and filtering a copy of 
the ego through it. Like a topiary shrub, the portions 
of the character's neural network that exceed the capacities
of the intended fork are trimmed away. In addition
to the changes noted under \textit{Beta Forks} (p. 273),
characters may voluntarily choose to delete/decrease 
skills and remove memories.

Delta forks are created by excising the ego's surface 
personality traits and applying them to an AI template. 
In this case the ego's memories are usually excluded 
entirely—it is easier to start with a blank delta fork 
and feed them the specific memories/knowledge they 
need. As with beta forks, characters making delta forks 
may voluntarily choose to delete/decrease skills and 
keep specific memories. If an alpha fork is not available
to prune, a delta fork can be whipped up from
a biomorph brain with an ego bridge and 1 minute. 
Many people sleeved in biomorphs keep delta forks on 
hand in storage, to whip up on the fly as needed.

Transhumanity's grasp of neuroscience extends to 
scanning and copying a mind, but the most intricate 
workings of memory are still imperfectly understood. 
Making precise edits to individual portions of a neural 
network (to alter recollections, skills, and the like) is 
still a black art. The difficulty with neural pruning is 
that taking a weed whacker to the tree of memory 
isn't an exact science. Specific memories may not be 
excised or chosen—at best, memories may be handled 
in broad clumps, typically grouped by time periods no 
finer than 6 months. For simplicity, most beta forks are 
created by removing all memories older than 1 year.

When creating a beta or delta fork, the character 
must make a Psychosurgery Test (other parties may 
make this test on the character's behalf, representing 
that the character is giving them access to prune the 
fork appropriately). If the character succeeds, the fork 
is created as desired. If the test fails, the gamemaster 
chooses one of the following penalties for every 10 
full points of MoF. Some of these penalties may be 
combined for a cumulative effect:

\begin{itemize}
\item 1 additional skill decreased by –20
\item Fork acquires a Negative mental trait worth 10 CP
\item Fork suffers 1d10 $\div$ 2 (round up) mental stress
\item Extra memory loss (gamemaster discretion; beta forks only)
\item 1 Positive trait lost
\end{itemize}

\subsubsection{Neural Pruning With Long-Term Psychosurgery}

Rather than generating forks on the fly, some characters
prefer to have carefully-pruned forks on hand,
stored as inert files that can be called up, copied, and 
run as needed. These forks are crafted with long-term 
psychosurgery, meaning that they suffer fewer drawbacks
and the memories may be more finely tuned.

Long-term neural pruning requires a Psychosurgery 
Test as above, but with a +30 modifier. Delta forks take 
1 week to prune this way, beta forks take 1 month. Additional
modifications may be made to the fork using
any of the normal rules for psychosurgery (p. 229).

It is worth noting that some people prefer to use 
forks of themselves or loved ones rather than a muse. 
Likewise, some wealthy hyperelites are known to keep 
copies of their younger backups on hand, sometimes 
for decades, and re-instance these when their prime 
ego has enough skill and experience to completely 
outclass its younger selves. Though technically these 
are alpha forks, their lag behind the original ego is 
comparable in degree to that of a beta fork. This is 
rumored to be the method used by the Pax Familiae 
in instancing her army of cloned selves.

\subsection{Handling Forks}

Gamemasters are encouraged to allow players to roleplay
their character's own forks. It is important to
note, however, that even with alpha forks, once the 
fork and originating ego diverge, they develop onward 
as separate people. The events that shape the primary 
ego's personality, character, and knowledge will not 
happen—or even if they do, probably not in the same 
way—to the fork, and vice versa. The exact dividing 
line between an ego and a fork is a central philosophical
and legal debate among many transhumans.
%%% 277
This means that gamemaster should not be afraid 
to pull a fork out of a player character's hands and 
make them into an NPC if they start too diverge too 
greatly. Similarly, if a fork begins to learn information 
that the main character does not (yet) have access to, 
it is probably also better to run the fork as an NPC in 
order to avoid metagaming.
It is entirely possible that a fork might decide that 
it will no longer obey the originating ego and carry 
about doing its own thing. This usually only occurs 
with alpha forks, who are essentially a full copy 
anyway, and as time passes the idea of merging back 
with the original ego becomes unappealing. Beta and 
delta forks are quite aware of their nature as incomplete
copies, and so usually return back home to the
ego for reintegration. In rare cases, however, even 
these might make a break for life on their own.

\subsection{Merging}

Merging is the process of re-integrating a previously-spawned
fork with the originating ego. Merging is
performed on conscious egos/forks, transferring both 
to a single, merged ego. The process is not difficult 
to undergo when two forks have only been apart a 
short time. As forks spend more time apart, though, 
merging bll, a Psychosurgery 
Test is called for (made either by the ego or another 
character overseeing the process). The Merging table 
lists modifiers for this test as well as the result of success
or failure.
For synthmorphs, merging takes one full Action 
Turn. For biomorphs, an ego bridge (p. 328) or mnemonic
augmentation (p. 307) is required to merge,
and the process takes 10 minutes.
The result of the process is a unified ego, whether 
or not the Merging Test succeeds. Psychotherapy (p. 
209) and psychosurgery (p. 229) can troubleshoot bad 
merges over time. \\

\begin{table}
\caption{Merging}
\begin{tabular}{|l|l|l|l|}
\hline
TIME APART & MODIFIER & SUCCESS & FAILURE \\
\hline
Under 1 hour & +30 & Seamless ego with memories & Memories intact, (1d10 $\div$ 2, round down) – 1 SV \\
& & intact from both &\\
\hline
1–4 hours & +20 & Solid bond, memories intact & Memories intact, (1d10 $\div$ 2, round down) SV \\
\hline
4–12 hours & +10 & Memories intact, 1 SV & Minor memory loss, (1d10 $\div$ 2, round up) SV \\
\hline
12 hours–1 day & +0 & Memories intact, 2 SV & Moderate memory loss, (1d10 $\div$ 2, round up) + 2 SV \\
\hline
1 day–3 days & –10 & Memories intact, 3 SV & Major memory loss, 1d10 + 2 SV \\
\hline
3 days–1 week & –20 & Memories intact, 4 SV & Major memory loss, 1d10 + 4 SV \\
\hline
1 week+ & –30 & Minor memory loss, 5 SV & Severe memory loss, 1d10 + 6 SV \\
\hline
\end{tabular}
\label{table:merging}
\end{table}

\begin{quotation}
\textbf{THE SELF}
\\
Forking and merging have changed the way
transhumanity thinks about the self and what
it means to have a well-integrated personality. \\
While forking is child’s play from a technological
standpoint, the psychological and
social effects of cloning a mind mean that
most people are cautious about employing
forks. Some jurisdictions ban forking outright
for all but medical uses, while others have
severe restrictions. In many hypercorp jurisdictions,
for instance, alpha forks are illegal
and letting a beta fork run for more than 4
hours without merging violates the modern
descendants of 20th-century anti-trust laws.
Similarly, the Jovian Junta and other bioconservatives
ban forking entirely. \\
Disposing of unwanted forks is another
thorny issue. In some places, it’s as simple as
deleting them, because a stored mind has no
legal status. In others, a fork that doesn’t wish
to merge back with its originating ego might
be accorded some rights, though these are
generally only granted to alpha forks. \\
Most significantly, though, running a shortterm
fork of oneself for periods of an hour
or less is an easy task for many transhumans.
Many people use forks of themselves to get
work done in everyday life, and almost everyone
has at least experimented with forking at
some point. \\
Transhumans view forking a bit like early
21st-century humans viewed drinking and drug
use. A bit might be okay, but someone overdoing
it will be stigmatized. This is because most
transhumans understand the psychological
consequences of overusing forks.
\end{quotation}

\section{Egocasting}

In spite of being a spacefaring civilization with outposts
throughout the solar system and beyond, transhumanity
makes scant use of spacecraft for interplanetary
travel. Shuttlecraft using a variety of propulsion
systems make regular trips between habitats, planetary 
surfaces, and moons. But for any trip longer than 1.5 
million kilometers—the distance a fusion drive craft 
can cover in a day—people egocast.
Egocasting is transhumanity's most advanced 
personal transportation technology, though only the 
character's ego actually travels. Egocasting combines 
the technologies of uploading and quantum farcasting 
to transfer a backup (or sometimes even a conscious 
ego, see p. 269) over interplanetary distances.
%%% 278

Though egocasting occurs at the speed of light, 
egocasting times vary drastically with distance. Egocasting
within a cluster or planetary system is usually
just a matter of minutes. Egocasting from the sun to 
the Kuiper Belt, however, takes between 40 and 70 
hours, and so egocasting all of the way across the 
solar system can take even longer.

Once an ego arrives at the destination receiver, it can 
be archived, run as an infomorph, or resleeved as normal.

\subsection{Egocaster Security}

Beaming yourself across interplanetary space is a 
mature technology and usually works seamlessly. Because
egocasting uses quantum farcasters, there is no
danger of radio interference cooking the signal and 
causing data loss. Normally the entire process is mediated
by the character's backup service, and security
breaches are uncommon.
\\
However, there are several risks involved in egocasting
The most obvious is that the character's consciousness
is transferred as a digital backup file at the
destination. If the egocaster on the other end is not 
trusted or the networks at the destination are privately 
controlled by the receiver, the character is potentially 
putting themself at the mercy of their host. Most hypercorps
consider meddling with a transmitted ego to
be a serious breach of etiquette, whereas autonomist 
types would find it unthinkably repressive. However, 
political extremist groups and criminal organizations 
in control of egocasters suffer from fewer restraints.
\\
A more subtle risk is the possibility for hackers to 
exploit security holes in the egocaster and its attached 
virtual space to steal a fork of the character. This is 
extremely difficult to do. It almost never happens 
during a normal upload, because the uploading services
are security conscious to the point of paranoia.
Even so, the forks stolen by such attempts more often 
than not end up being vapors, because the intruder 
is usually stopped before a full copy can be obtained.

\subsection{Darkcasting}

Characters who want to egocast without the attention
of public officials like Immigration and Customs
must seek out so-called darkcasting services—illegal 
farcaster transceivers typically operated by criminal 
syndicates and other clandestine groups. To locate 
such a service, a character must use their Networking 
skill and possibly their reputation (p. 285).

\section{Morph Brokerage}

Morphs are a major commodity in transhuman society. 
The technology and materials needed to grow new 
morphs are cheap and abundant, though they take 
time. Cloned biomorphs take at least a year and a half, 
even with accelerated growth. Pods, which are typically
pieced together from vat-grown parts, take about
6 months. Synthmorphs like cases and synths can be 
produced in a day, whereas more complicated models 
can take a week or more. Theoretically, supply will one 
day outstrip demand to the point where flesh is free.

Characters have several options for acquiring 
morphs when they travel by egocast, suffer heavy 
damage, or just feel like a new body. When egocasting, 
the most common method for travelers of middling 
means is to store their current morph in a body bank's 
secure facility and lease a morph at their destination
Less commonly, characters may rely on public
resleeving facilities, or, if they have the means, they 
may purchase a new morph outright. Characters who 
expect to stay at their destination indefinitely or who 
decide to resleeve but aren't traveling might instead 
opt for a trade-in on their old body, leaving it behind 
permanently in most cases.

\subsection{Morph Availability}

As noted under \textit{Resleeving and the Gamemaster} (p. 271), 
finding the model of morph you want is not always easy. 
While many basic morph types (cases, synths, splicers) 
are generally available, characters can also locate new 
While many basic morph types (cases, synths, splicers)
are generally available, characters can also locate new 
%%% 279
morphs using their Networking skills (see \textit{Reputation }
\textit{and Social Networks,} p. 285). Certain morph types 
are harder to find then others; the gamemaster should 
apply an appropriate modifier for any morphs that seem 
rare or unusual (for example, swarmanoids or reapers). 
Likewise, some morphs may simply be unavailable in a 
given locale. Rusters are rarely available off of Mars, for 
example, while on Europa, most morphs are exotic local 
aquatic varieties.

The gamemaster determines which factions are able 
to provide new morphs in a given locale. Factions 
will not provide morphs that are unavailable to that 
faction as starting characters. If the faction is not the 
dominant one in that locale, a penalty should be applied
ranging from –10 to –30. Despite having a presence
in a given locale, some factions may be unable to
provide morphs at all.

If the character is seeking a customized morph with 
specific implants or enhancements, the search will be 
more challenging. The gamemaster should apply a –10 
to –30 modifier here as well, depending on the extent 
and legality of the modifications sought.

\subsection{Morph Acquisition}

Once a morph is located, the character may call in 
favors (p. 285) or pay credits for it. Morph costs are 
noted on the Morph Costs table. In the inner system, 
morph prices are often inflated by demand in the 
market such that the most desirable morph types can 
cost a small fortune. Outsystem, prices in rep are more 
reasonable but still steep due to population pressures 
on life support-dependent outer system settlements. 
For travelers and frequent body hoppers, there are a 
number of ways to defray these costs.

\subsubsection{Brokerage And Matchmaking}

Finding morphs for travelers and the bodiless is a 
specialized skill demanding deep social networks and 
a flair for negotiation. In general, it's a seller's market, 
so brokers (or ``matchmakers,'' as they're called in 
the open economy) act as agents for the person seeking
a body. The Morph Costs table assumes a 10\%
fee paid to the broker. Characters wishing to cut out 
the middleman may reduce cost by 10\% but take a 
–30 penalty on their Networking Test to locate an 
available morph.

\subsubsection{Customized Morphs}

If a character seeks to have a customized morph 
(with extra bioware, cyberware, or nanoware implants
or robotic enhancements), the costs for these
enhancements are added to the morph's cost (if the 
gamemaster chooses, discount package deals may 
apply). Likewise, morphs may come saddled with 
positive or negative morph traits (p. 145). These 
traits raise or lower the morph's cost at a rate of 
+500 credits per CP for positive traits, or –200 credits
per CP of negative traits. Negative traits typically
reflect abuses the morph has suffered at the hands of 
previous occupants.

\subsubsection{Trade-In}

For those who wish to leave their old morph behind 
permanently, trade-ins on current morphs are an 
option. The high demand for bodies means that a buyer 
is almost always available unless the gamemaster finds 
extenuating circumstances. Morphs may be traded in 
for the value shown on the Morph Costs table adjusted
for any positive or negative traits), less a 10\%
physical exam and finder's fee. This is either paid to the 
morph broker in cred or rendered as a favor using rep.

\subsubsection{Patron Provisioning}

Characters on missions for rich or influential patrons 
may have morphs provided for them. Normally 
such provisions are made for the duration of a job, 
although less commonly the morph itself might be 
payment for services rendered. Gamemasters are 
encouraged to be creative with such arrangements, 
though players should be advised that such bargains 
can quickly turn Faustian.

\subsubsection{Black Market Morphs}

Black market body traders promise to provide the 
buyer with morphs and upgrades of choice regardless 
of a habitat's laws against weapons or implants, in 
addition to bypassing standard arrival registration 
via darkcasting. Illegal morphs usually come with a 
price markup (+25\% at least), whereas used morphs 
with unsavory backgrounds (and traits) can usually be 
acquired on the cheap (–25\%).

\begin{table}
\caption{Morph costs}
\begin{tabular}{|l|l|}
\hline
MORPH TYPE & COST \\
\hline
\multicolumn{2}{|c|}{Biomorphs} \\
\hline
Flats, Splicers & High \\
\hline
Octomorphs & Expensive (30,000+) \\
\hline
Furies, Ghosts, Remade & Expensive (40,000+) \\
\hline
Futuras & Expensive (50,000+) \\
\hline
All others & Expensive \\
\hline
\multicolumn{2}{|c|}{Pods} \\
\hline
Workers, Pleasure Pods & High \\
\hline
Novacrabs & Expensive (30,000+) \\
\hline
\multicolumn{2}{|c|}{Synthmorphs} \\
\hline
Cases & Moderate \\
\hline
Synths, Dragonflies & High \\
\hline
Slitheroids, Swarmanoids & Expensive \\
\hline
Flexbots & Expensive (30,000+) \\
\hline
Arachnoids & Expensive (40,000+) \\
\hline
Reapers & Expensive (50,000+) \\
\hline
Positive morph traits & +500 per CP \\
\hline
Negative morph traits & –200 per CP \\
\hline
\end{tabular}
\label{table:morph-costs}
\end{table}


\subsubsection{Indenture}

Characters who find themselves too destitute to 
afford a new morph can strike a deal for indentured 
service—a ``deal'' that is rarely advantageous to the 
%%% 280
new indenture. Typical contracts require years of indentured
labor—terraforming Mars, herding comets,
asteroid mining, constructing habitats, colonizing 
exoplanets, etc.—in exchange for a cheap synthetic 
morph or splicer at the end of the term. Gamemasters 
may use their discretion in offering such terms, though 
in many cases the terms offered will temporarily or 
permanently end the character's career as a free agent. 
Hypercorps using indentured labor are notorious for 
changing the terms at a whim, extending the service 
period, or slamming the indenture with a slew of 
hidden and outrageous charges that were not made 
clear up front. Characters may, of course, enter into 
such service fully intending to grab their morph and 
run at the first opportunity, but the hypercorps are 
very protective of their investments. Indentures are 
closely monitored and tracked, and the hypercorps are 
not above sending ego hunters to retrieve a runaway.

\subsubsection{Public Resleeving}

Some locales, notably Titan, have a well-developed 
public resleeving infrastructure intended to provide a 
body to anyone who needs one. Morphs provided are 
usually unremarkable cases, synths, or splicers with 
no Positive traits or optional implants. Anyone holding
citizenship in a locale with public resleeving may
apply for a body. Wait times are between a month and 
two years, with Reputation influencing wait times at 
the gamemaster's discretion.

\subsection{Renting Morphs}

For temporary visits where an infomorph won't do, 
morphs may be leased rather than bought. The cost 
to rent a morph is 1\% of its cost per day, plus a Low 
charge for resleeving. This cost includes rental insurance
(see below). If the rental insurance is waived
(not always possible unless you have a good Rep), the 
rental cost may be reduced by half.

Characters who are leasing a morph may also use 
their previous morph as collateral. In this case, deduct 
the cost of the character's current morph from the 
rental morph before calculating the 1\% cost per day, 
with a minimal rental cost of 10 credits per day.

\subsubsection{Penal Lease}

Characters visiting the inner system or Jovian 
Republic may be able to lease morphs belonging 
to prisoners. In most jurisdictions, criminals are 
sentenced to terms in rehabilitative simulspace with 
a stipulation that the prisoner's morph becomes 
state property during their term of incarceration. 
Morphs acquired this way often have complicated 
histories but also tend to have modifications useful 
to Firewall agents. Conversely, characters who find 
themselves imprisoned may be subject to having 
their body leased out during incarceration.

The effects of taking a penal lease are at the discretion
of the gamemaster. A character may have
to pull some strings with their Reputation in order 
to lease such a morph, especially if it has restricted 
or illegal modifications. Negative traits, cases of 
mistaken identity, and unfortunate encounters with 
friends and associates of the morph's former occupant
are among the possible drawbacks to this
type of arrangement. On the up side, penal leases 
may reduce costs for both leasing and insuring the 
morph, again subject to the gamemaster's discretion.

\subsubsection{Rental Insurance}

Leased morphs must be covered by an insurance 
policy, which often restricts the character from 
breaking the law or taking the morph anywhere 
too dangerous or lawless. Characters may purchase 
hazard insurance that will cover taking the morph 
into certain dangerous situations, but this will 
double the rental price at minimum.
%%% 281

If a character suffers extensive organic damage or 
death while insured, the insurance will cover 80\% 
of the morph's cost, meaning that the character is 
expected to pay the other 20\%. If they cannot pay, 
their possessions or their stored morph may be 
seized in payment.

If a character violates their insurance policy by 
intentionally putting themself in harm's way above 
the threat level at which the policy was purchased, 
without first communicating with and rendering payment
to the insurer, the policy may be declared void.
If the leased morph dies under a voided policy and 
the character cannot pay to replace it, their possessions
and stored morph may be subject to seizure.

Seizure takes different forms depending upon the 
local economy and legal system. In hypercorp space, 
it is a straightforward seizure of liquid assets, including
forced uploading if the character's morph is
seized. Elsewhere, the character is more likely to end 
up owing a lot of favors or taking severe hits to their 
reputation, but they are unlikely to undergo forced 
uploading or outright physical seizure of their morph.

\section{Identity}

Given the nature of resleeving technologies, identity 
is a fluid concept in \textit{Eclipse Phase.} Transhumans are 
used to the idea of identifying people by how they 
look or even by their biometric data, but this is no 
longer a certified method. What you look like may 
drastically change from one day to the next. You 
may see an olympian you recognize, but perhaps it's 
been awhile, so you're no longer certain that it's the 
same person still in that morph. If you're sleeved in a 
popular off-the-rack morph, there may be hundreds 
of other cloned morphs that look exactly like you out 
there—perhaps useful if you desire to blend in. Similarly
security services can no longer rely on biometric
technologies. Forensics may be able to identify an individual
morph's presence at a crime scene, but proving
who was in that morph at the time is another matter.

Identity is, of course, tied to ego, and various 
authorities have instituted verification and security 
measures based on this. Within the inner system, each 
ego is given an ID number, which is used to validate 
their identity, citizenship, legal status, credit accounts, 
licensing, etc. This ego ID is verifiable by the person's 
brainwave patterns, which remain the same even 
when resleeving. When an ego uploads, the uploading
service is required to incorporate this ego ID into
the person's backup/infomorph. Likewise, when that 
person resleeves, the service handling the procedure 
is required by law to verify the ego's ID before downloading
The ego ID is then hardcoded into the morph
itself in the form of a nanotattoo on the tip of the person's
index finger. This nanotat can be easily scanned
at security checkpoints to verify identity.

Though efficient, this system is far from perfect. For 
one, ID record-keeping is far from standardized and 
varies drastically from habitat to habitat. Most do not 
share records with each other unless they are part of the 
same political alliance in order to protect their citizens' 
privacy. For example, Lunar-Lagrange Alliance stations 
do not share citizenship ID data with the Planetary 
Consortium, though they do share with each other. 
On top of this, many identity records were lost 
during the Fall, a situation that was undoubtedly exploited
by those who preferred to erase their past or
adopt a new persona. These all make for a situation 
where identity records are patchwork at best. Officials 
must also rely on the security of other habitats for ID 
verification. If a person egocasts to Nectar on Mars 
from Qing Long in the Martian Trojans, and the 
Nectar officials have no record of this person, they 
can only trust that the Qing Long officials did their 
job when verifying the subject's ID and background.
To make matters worse, many autonomist habitats 
operate without identity checks altogether. Though 
some ID measures are still used, both to prevent 
reputation system gaming and to be able to identify 
bodies in the case of death, these uses are significantly 
more lax and few records are kept. Therefore, when 
autonomists and the like egocast to habitats that require
ID, they are assigned a temporary ID for the duration
of their stay (and sometimes any future visits).

\subsection{Identity Verification}

There are three ways to verify someone's identity: 
nanotat scan, brainwave scan, and checking the cryptographic
hash on a digital mind.

\subsubsection{Nanotat Scans}

Special encoded nanobots are used to create a small 
nanotat on a person's index finger. These  nanobots 
contain encoded information that includes their name 
and identity, brainwave pattern, citizenship/legal status, 
credit account number, insurance information, and 
licenses. Depending on the local habitat laws, it may 
include other information such as criminal history, 
travel history, restricted implants, employment records, 
and so on. This nanotat may be read by anyone with 
a special ID scanner that reads the nanobot encoding.
ID nanotats include information on the company 
that did the resleeving, so that the data may be accessed
and verified with their records online. The data
on the nanotat is also cryptographically signed with 
the company's public key, meaning that anyone who 
checks the data and the signature online can tell if the 
data has been altered.

\subsubsection{Brainwave Scans}

Brainwave scans are one of the few types of biometric 
prints that stay with an ego no matter what morph it 
is in. They are impractical for most security purposes 
as they require a scan with a combination electroencephalogram
and neuroimaging device, referred to as
a brainprint scanner, which takes approximately 5 
minutes. This device measures the subject's baseline 
brainwave pattern as well as the subject's brainwave 
signature responses when they think certain thoughts 
%%% 282
or sense certain patterns. These scans are all but 
impossible to fool, however, barring hacking of the 
brainprint scanner itself, and so are considered quite 
reliable. For this reason they are occasionally used in 
high-security facilities.

It is worth noting that infection by some variants 
of the Exsurgent virus, notably the Watts-Macleod 
strain (p. 367), sometimes alters a person's brainwave 
patterns, but not in every case.

\subsubsection{Digital Code}

Digital ID codes are often incorporated into backups and 
infomorphs. Not only does this help identify who the 
backup belongs to, but it serves as an electronic signature 
for verifying ID when the backup is to be resleeved. This 
digital code typically contains the same information as 
the nanotat ID, and is signed with a cryptographic hash 
that makes it difficult to forge and which can be verified 
online. AIs and AGIs also feature such built-in codes.

\subsection{Circumventing ID Checks}

Firewall sentinels and clandestine agents often have a 
need to hide or alter their identities. While ID system 
are challenging, they are not insurmountable.

\subsubsection{Fake IDs}

The easiest way to bypass security checks is to establish 
a fake ID. Given the patchwork nature of identity records
and the lack of any centralized authority, this is
not very difficult. Numerous crime syndicates and even 
some autonomist groups maintain a thriving ID fabrication
business, often with complete histories and medical
covers for implants that might be restricted or illegal. 

These IDs are usually registered with habitats that are 
either known criminal havens, have autonomist sympathies
or are isolated and remote. Though the ID is
actually verifiable and registered with these stations, the 
potential shady origins of such IDs is known to most 
inner system authorities and so the character may be 
exposed to extra scrutiny or monitoring. Fake IDs may 
be acquired that are registered with more respected authorities
but this often requires a much higher expense
or connections to hypercorp clandestine operations.

Black market darkcast and resleeving options offer 
fake IDs as a matter of course.

\subsubsection{Altering Nanotat IDs}

Special nanobot treatments may be manufactured 
to erase, rewrite, or replace nanotat IDs. Erasing a 
nanotat is easy, but not having one is a crime and immediate
grounds for suspicion in many habitats. Rewriting
a nanotat is also easy, though this means that
the nanotat will fail its authorization online unless the 
encryption has also been cracked (p. 253). Replacing 
a nanotat ID with a fake one is just as possible, and is 
part of the process of acquiring a fake ID.

\subsubsection{Digital ID Tampering}

Digital ID codes may also be tampered with, though like 
nanotat IDs this will mean that the ID fails online verification
unless the encryption is also defeated (p. 253).

\section{Life In Space}

Transhumanity is not just a spacefaring race, it is 
also largely space-dwelling. While a substantial portion
of transhumanity inhabits planetary bodies like
Mars, Luna, Venus, and the moons of the gas giants, 
the balance live in a variety of space habitats, ranging 
from the old-fashioned O'Neill cylinders of the inner 
system to the Cole bubbles of the outer system.

\subsection{Space Habitats}

Space habitats come in many sizes and configurations
from survivalist outposts designed to support
ten or fewer people to miniature worlds in resource-rich
areas housing as many as ten million people. In
heavily settled regions of space, such as Martian orbit, 
habitats may be integrated into local infrastructure, 
relying to some extent on supply shipments from other 
orbital installations.

More commonly, especially in the outer system, 
habitats are independent entities. This usually means 
that in addition to the main space station, the habitat 
is attended by a host of support structures, including 
zero-g factories, gas and volatiles refineries, foundries, 
defense satellites, and mining bases.

Habitats—especially large ones—sometimes have 
visitors, as well. Majors habitats are crossroads in 
space. In addition to scheduled bulk freighter stops, 
they may have hangers-on such as scum barges, prospectors
or out-of-work autonomous bot swarms.

Many habitats have some form of transportation 
network. This is most common in large cylindrical 
habitats with centrifugal gravity. Common solutions 
for public transit include monorail trains, trams, and 
dirigible skybuses. Common personal transit options 
included bicycles, scooters, motorcycles, and microlight
aircraft, with larger vehicles being uncommon
and usually reserved for official use.

Most habitats with large interior spaces also use augmented
reality overlays to create consensual hallucinations
of a sky and clouds, to which most residents keep
their AR channels tuned. One would think that in space, 
talking about the weather would have disappeared from 
transhumanity's repertoire of small talk, but the habit 
persists—only the weather discussed is usually virtual (if 
it's not real ``weather''—solar flare activity and the like).

\subsubsection{Cluster Colony}

Clusters are the most common form of microgravity 
habitat. Clusters consist of networks of spherical or 
rectangular modules made of light materials and connected
by floatways. Typically business and residential
modules are clustered around arterial floatways and infrastructure
modules such as farms, power, and waste
recycling. Limited artificial gravity areas may exists, 
frequently parks or other public places and specialized 
modules like resleeving facilities (morphs often keep 
better when stored in gravity). Arterial floatways in 
large clusters may have ``fast lanes'' where a constantly 
moving conveyor of grab-loops speeds people along.
%%% 283

Clusters are most commonly found in volatile-rich environments 
like the Trojans and the ring systems of the gas giants (particularly 
Saturn). Clusters are rare in the Jovian system because shielding a 
cluster of individual modules rather than one large station from 
Jupiter's intense magnetosphere is hideously inefficient.

Cluster colonies can have anywhere from 50 to 250,000 inhabitants.

\subsubsection{Cole Bubbles}

Cole bubbles (or ``bubbleworlds'') are found mostly in the main 
asteroid belt, where the large nickel-iron asteroids used to construct 
them are abundant. Bubbleworlds are less common in the Trojans 
and Greeks, where crusty ice asteroids predominate. A Cole bubble 
is similar in many respects to an O'Neill cylinder, but there are no 
longitudinal windows. Sunlight instead enters through axial mirror 
arrays. The bubbleworld is also constructed very differently, using a 
large solar array to heat a pocket of water inside of a metal asteroid 
so that the metal expands. Rotating the asteroid causes the malleable
material to form a cylinder, which is then capped off and the
water drained. The inside can then be pressurized, built out, and 
planted. Cole bubbles can also be spun for gravity, according to the 
whims of the inhabitants, though the gravity lowers as you near the 
poles of the bubble, with zero gravity at the axis of rotation.

Cole bubbles are among the largest structures transhumanity has 
created in space. The largest Cole habitat, Extropia, has a population
of 10 million.

\subsubsection{Hamilton Cylinders}

Hamilton cylinders are a new technology. There are only three fully 
operational Hamilton cylinders in the system, but the design shows 
great promise and is likely to be widely adopted over the coming 
period. Hamilton cylinders are grown using a complex genomic 
algorithm that orchestrates nanoscale building machines. These 
nanobots build the habitat slowly over time, a process more like 
growing than construction.

Similar to O'Neill cylinders and Cole bubbles, a Hamilton cylinder
is a cylindrical habitat rotating on its long axis to provide gravity
Two of the known Hamilton cylinders orbit Saturn in positions
skimming the rings near the Cassini division. From this position, 
they can graze on silicates and volatiles using harvester ships.

None of the currently-operating Hamilton cylinders have grown 
to full size yet, but estimates say they could each house up to 3 
million people.

\subsubsection{O'Neill Cylinders}

Found mostly in the orbits of Earth, Luna, Venus, and Mars, O'Neill 
cylinders were among transhumanity's first large space habitat designs
O'Neill cylinders are no longer built, having been replaced
by more efficient designs, but are still home to tens of millions of 
transhumans. O'Neill cylinders were constructed from metals mined 
on Luna or Mercury, Lunar volatiles (including Lunar polar ice), 
and asteroidal silicates.

A typical O'Neill habitat is thirty-five kilometers long, eight kilometers
in diameter, and rotates around its long axis at a speed
sufficient for centrifugal force to create one Earth gravity on the 
inner wall of the cylinder. Smaller cylinders exist, though these 
usually feature lower gravity (typically Mars standard). Cylinders 
are sometimes joined together, end-to-end, for extra long habitats. 
A spaceport is situated at one end on the rotational axis of the 
cylinder (where there is no gravity). Arrivals by space use a lift or 
microlight launch pad to get down to the habitat floor.
%%% 284

The inside of an O'Neill cylinder has six alternating 
strips of ground and window running from one cap 
of the cylinder to the other. One narrow end of an 
O'Neill cylinder points toward the sun. The opposite 
end is the mooring point for three immense reflectors 
angled to reflect sunlight into the windows. Smart 
materials coating the windows and reflectors prevent 
fluctuations in solar activity from delivering too much 
heat. The air inside the cylinder and its metal superstructure
provide radiation shielding.

The land in most O'Neill cylinders is one-third 
agricultural (a combination of food vats and high-yield
photosynthetic crops), one-third park land, and
one-third mixed use residential and business. O'Neill 
habitats have a day and night cycle regulated by the 
position of the external mirrors. The business and 
residential sections of the cylinder usually alternate 
with the park land over two of the strips of land; 
cropland usually takes up the third. Bridges cross 
the windows every kilometer or so, linking the land 
strips. The interior climate, the architectural style of 
the structures, and the types of vegetation and fauna 
present vary with the tastes of the habitats' designers.

Depending upon size, O'Neill cylinders can house 
from 25,000 to 2 million people.

\subsubsection{Tin Cans}

Antique research stations and survivalist prospector 
outposts often fit this description. Tin can habitats 
are only a few notches up from the early 21st-century 
International Space Station. Tin cans usually consist 
of one or more modules connected to solar panels 
and other utilities by an open truss. Deluxe models 
feature actual floatways or crawlways between 
modules, while barebones setups require a vacsuit or 
vac-resistant morph to go from room to room. Food 
growing capacity is severely limited and there may be 
no farcasters, but fabricators are available, as well as 
mooring for shuttles and perhaps prospecting craft.

Tin cans rarely house more than 50 people.

\subsubsection{Toruses}

Interchangeably called toruses, toroids, donuts, and 
wheels, these circular space habitats were a cheap 
alternative to the O'Neill cylinder used for smaller 
installations. Like O'Neill cylinders, toruses are seldom 
constructed anymore, but many are still encountered in 
the inner system, particularly in Earth and Lunar orbit.

A toroidal habitat looks like a donut 1 kilometer in 
diameter, rotating on great spokes. There is a zero-g 
spaceport at the wheel's hub. Visitors take a lift down 
one of the spokes to the level of the donut, where 
rotation creates one Earth gravity.

The plan of toroidal habitats varies greatly, as many 
were designed for specific scientific or military purposes 
and only later taken over as habitats by entrepreneurs 
or squatters. Many have a succession of decks in the 
donut. Most of those designed for long-term self-sufficient
habitation have smart material-covered glass
windows for growing plants along much of the inside 
surface of the torus. Toroidal habitats equipped for 
farming normally face the sun in a direction perpendicular
to their rotational axis, but then use a slow processional
wobble of that axis to create a day/night cycle.

Toruses were usually built to accommodate small 
crews of 500 or fewer people, though some larger 
ones exist, able to house 50,000. A few rare double-toruses
also exist, like two large wheels spinning in
opposite directions, joined at the axis.

\subsection{Immigration And Customs}

How characters gain entry to a habitat and what type 
of screening they're likely to undergo depends upon 
how they arrive. Some habitats are close to other 
settlements, while others are physically isolated by the 
vast, empty distances of interplanetary space.

Habitats in dense planetary systems receive most of 
their visitors via conventional space travel. Immigration
and customs infrastructure is geared toward receiving
visitors via their spaceport, and the processing
of arrivals is in most ways analogous to a twentieth 
century airport. Isolated habitats, on the other hand, 
tend to receive almost all of their visitors via egocast.

\subsubsection{Physical Arrivals}

Arrivals by spacecraft undergo, at minimum, an ego 
ID check, scans to detect pathogens, hostile nanobots, 
explosives, or radiation, and an inspection of their 
personal effects. Some habitats go farther, including 
rigorous secondary screenings using scout nanoswarms
scans of all electronic systems for malware,
and/or aggressive interrogation of a fork of the subject. 
Even autonomist enclaves enforce automated scans for 
anything that might pose a danger to the habitat or 
any signs of hypercorp saboteur efforts.

Restricted goods vary according to local legalities. 
Many habitats, particularly those controlled by autonomist
or criminal factions, allow personal weaponry
as long as its nothing you can use to blow a hole in 
the structure or indiscriminately kill dozens of people. 
Others, notably the Jovian Republic and hypercorp 
stations, disallow lethal weapons of all kinds, except 
for people who have acquired special permits and authorization
(sometimes available by bribing the right
people or pulling favors with rep). Nonlethal weapons 
are generally allowed. Other restricted items may 
include nanofabricators, nanoswarms, malware and 
hacker software, drugs and narcoalgorithms, certain 
types of XP recordings, covert operations tools, and 
so on. Certain types of morphs may also be restricted, 
such as reapers, furies, or uplifts.

Certain habitats may insist that visitors—or at 
least the ones they don't like the looks of—submit to 
specific forms of monitoring or surveillance for the 
duration of their stay. This might include taggant 
nanoswarms, hosting a police AI in your mesh inserts, 
or even physical tailing by an armed security drone. 
Other stations will require that their visitors leave a 
fork as a form of collateral at the door—in case they 
commit a crime, the fork can be interrogated.
%%% 285

Finally, though rare, some habitats go so far as to 
charge all visitors an ``air tax''—a fee for using the 
station's publicly available resources while they are 
present. This is generally only common in isolated 
habitats with strained resources, and is considered 
especially obnoxious by most autonomists.

Some syndicates run a good business in smuggling 
certain goods or even people into habitats. This is generally
accomplished through bribed security personnel,
but is also sometimes handled as falsified credentials 
that will allow the subject to breeze past security 
checks. Such services are typically quite expensive.

For those hoping to gain quiet and unobserved 
access, there is always the option of taking a spacewalk
and trying to break in through an unattended
airlock. Such attempts are quite often dangerous and 
futile, as most habitats have dedicated sensor and 
security systems to monitor their exterior surface and 
in particular any access points. Still, it is a possibility 
for a resourceful team with a skilled hacker, though 
armed sentry bots are a particular danger.

\subsubsection{Electronic Arrivals}

Arrivals by egocast are sometimes interviewed by 
habitat authorities in a simulspace before resleeving. 
Depending upon the habitat's attitude toward civil 
rights, this process can be relatively reasonable or 
quite invasive. A minimal entry inspection includes an 
ID check, a brief interview with a customs AI, and a 
review of the specs of the morph into which the arriving
ego plans to resleeve. Habitats with draconian
immigration measures may use harsh psychosurgery 
interrogation techniques on suspect infomorphs. Egocast
backups have little recourse to avoid this treatment—station
authorities can simply file them away
in cold storage if they choose—so it is wise to investigate
custom procedures before you send yourself over.

Because many people, particularly autonomists and 
brinkers, don't appreciate this kind of reception, various
uploading services have stepped in to provide pre-customs
resleeving for characters traveling to habitats
with suspect screening methods. For often-exorbitant 
fees, the traveler egocasts into an extraterritorial substation
close to their intended destination, resleeves
there, and then travels to their destination by rocket.

Various darkcast services, normally run by established
crime syndicates, sometimes offer an alternative
method of egocasting in and possibly even resleeving. 
Darkcast services are quite expensive, however, and 
the character is at the mercy of the syndicate operators
In rare cases, some political factions or even hypercorps
might operate their own darkcast systems,
which a character with good networking skills might 
be able to take advantage of.

\subsection{Space Travel}

In some circumstances, characters will prefer to travel 
physically through space rather than egocasting. In 
\textit{Eclipse Phase,} spacecraft are primarily dealt with as a 
setting environment rather than a vehicle/gear to use. 
Spacecraft largely pilot themselves via the onboard AI. 
Though characters can also take over with their Pilot: 
Spacecraft skill, the situation rarely calls for it.

\subsubsection{Local Travel}

In densely inhabited planetary systems such as Mars 
and Saturn, most travel between cities, surface stations, 
and orbital habitats within 200,000 kilometers is by 
small hydrogen-fueled (or sometimes methane-fueled) 
rockets. This form of travel is incredibly cheap, very 
fast, and avoids the occasional personality glitches 
that crop up during egocasting. LOTVs (lander and 
orbital transfer vehicles, p. 348) are commonly used. 
Spacecraft leaving a planetary body need to be able to 
generate enough thrust to escape the gravity well (see 
\textit{Escaping Gravity Wells,} p. 346).

\subsubsection{Distance Travel}

For distances of 200,000 to 1.5 million kilometers, 
somewhat larger (and more expensive) fusion- and 
plasma-drive craft make regular runs. Nuclear electric 
ion drives were once used on some of these routes, 
but the poor efficiency of these fission systems and the 
need for radioactive heavy metal reaction mass means 
that they are almost never used anymore. Faster antimatter-drive
couriers are also commonly used. These
ships lack the thrust to escape from the gravity wells 
of large planets or moons, so they station themselves 
in orbit and use smaller ships (typically LOTVs) with 
higher thrust to transport people to and from the 
planetary surface.
For distances beyond 1.5 million kilometers, almost 
everyone uses egocasting

\subsubsection{Space Travel Basics}

Spacecraft use various types of reaction drives (see 
\textit{Spacecraft Propulsion,} p. 347), meaning that they burn 
fuel (reaction mass) and direct the heated output in 
one direction, which pushes the spacecraft in the opposite
direction. Travel over any major distance typically
involves a period of high-acceleration burn for several 
hours at the beginning of the flight, where up to half of 
the reaction mass is spent to drive up the craft's velocity
The ship then coasts for the majority of the flight at
that speed, until it approaches its destination, where it 
flips over and burns an equal amount of reaction mass 
in the opposite direction to decrease velocity.
Though some craft burn half their reaction mass 
to get up to the best speed possible, this doesn't leave 
much room for additional maneuvering or emergencies
Many craft therefore only burn up to a quarter
or a third of their fuel in initial accelerations, so they 
have some to spare in case they need it. A few tricks 
can be used to save fuel and build speed, such as slingshotting
around the gravity wells of larger planets or
aerobraking in a planet's upper atmosphere.
Travel times between locations are constantly 
changing as various bodies move in their orbits 
around the solar system. Within a cluster or planetary 
system, travel takes a matter of hours. Within the 
286
inner system, travel can take days or weeks. Travel to, 
from, or within the outer system can take much longer, 
and is usually a matter of several months.

Most ships operate at zero-g, except for a few larger 
craft that are able to spin habitat modules for low 
gravity. Periods of high-acceleration also produce 
temporary gravity in a downward direction, towards 
the burn.

Space is a valuable commodity on board spacecraft, 
so room is often tight. Sleeping and personal quarters 
are rarely bigger than large closets, just enough room 
for a sleeping bag and personal effects. Depending 
on the size of the craft, there may be a communal 
recreation area. The crew tend to only be busy at the 
beginning and end of a trip, when they must deal with 
acceleration/deceleration and maneuvering around 
other space traffic. The rest of the trip they spend 
dealing with repairs or otherwise killing time, often 
by accessing XP or VR simulations or playing AR 
games. While spacecraft have their own local mesh 
network, they are usually too far to interact with the 
mesh networks of other habitats without significant 
communications lag, so they must make do with 
their own archive of entertainment options. Many 
long-haul ships are crewed by hibernoid morphs, who 
hunker down for a long nap.

\subsubsection{Spaceship Combat}

Combat in space tends to take place over long distances
using massive beam weapons, railguns, and
missiles. It also tends to be nasty, brutish, and short. 
Significant damage to a vessel can cause atmospheric 
decompression, killing any biomorph crew who aren't 
suited up and strapped down. 

For the most part, it is recommended that space 
combat be treated as a plot device, part of the background
story that helps create drama and tension,
rather than an event that characters actively participate
in. This is not to say the characters cannot play a
role in the combat, or that their actions will have no 
effect on the outcome. They may become involved in 
damage control, negotiate with hostile forces, repel 
boarders, target weapons with Gunnery skill, stage a 
mutiny, attempt to hack the networks of approaching
vessels, escape out the airlock, hide out while the
pirates sack the ship, or similar affairs. It is recommended
however, that gamemasters steer clear of
space combat situations that could easily lead to the 
whole team dying due to a few bad dice rolls.

\section{Nanofabrication}

In order to create an object in a nanofabricator 
(whether a cornucopia machine, fabber, or maker; see 
p. 327), three things are needed: raw materials, blueprints
and time.

\subsection{Raw Materials}

Raw materials are generally easy to acquire, as most 
nanofabricators are equipped with disassembler units 
that will break down just about anything into its constituent
molecules. Feedstock may also be purchased
(at a cost of Trivial). Many habitats route their recycling
and waste products directly into disassemblers.

\subsection{Blueprints}

Most nanofabricators are pre-loaded with blueprints 
for general purpose items: food, simple clothing, basic 
tools, etc. Blueprints for other goods may be acquired 
in several ways:

\begin{itemize}
\item They may be purchased online (legally or on the black market).
\item They may be found for free online (see below).
\item They may be acquired with Rep, following the usual rules for social networking (p. 285).
\item They may stolen (usually by hacking a mesh site or a nanofabricator containing such plans).
\item They may be self-programmed (see below).
\end{itemize}

\noindent Once the blueprints are acquired, they are simply 
loaded into the nanofabricator.

\subsubsection{Open Source Blueprints}

Blueprints for many goods may be found for free 
online, disseminated by an active open source 
software movement. The availability of such plans 
typically depends on the local mesh. In autonomist 
habitats, a simple Research Test is likely to turn up the 
open source blueprints you need (applying modifiers 
for unusual items). In more restricted habitats, open 
source blueprints may be harder to find, as they will 
be securely hidden from the prying eyes of the authorities
In this case, the character will need to use their
Rep to gain access, bribe a local hacker group, or do 
something similar.

Note that restricted nanofabricators may not accept 
open source blueprints (see \textit{Blueprint Restrictions}).

\subsubsection{Blueprint Restrictions}

Some nanofabricators are equipped with pre-programmed
restrictions not to accept blueprints for restricted
items (such as weapons) or non-licensed items
(such as black market or open source blueprints). 
These restrictions may be circumvented by hacking 
the nanofabricator and re-programming it, following 
normal hacking rules (p. 254).

\subsubsection{Programming Blueprints}

A dedicated character may simply decide to program 
their own blueprints, though this is a time-consuming 
endeavor. To do so, the character must make a Programming
(Nanofabrication) Test with a timeframe
of one week per cost level of the item. For example, a 
Trivial cost item takes 1 week, a Low cost item takes 2 
weeks, a Moderate item 3 weeks, and so on. Academics
Nanotechnology skill or a skill appropriate to the
object's design may be used as a complementary skill 
(p. 173) for this test. A fork or muse may also be assigned
to such a programming task.
%%% 287

\subsection{Time}

Once the raw materials and blueprints are in, most nanofabrication
is simply a matter of time. The exact timeframe to create an
object varies, but roughly approximates 1 hour per cost category of 
the item (1 hour for Trivial, 2 for Low, 3 for Moderate, etc.). The 
gamemaster may feel free to modify this period as appropriate for 
the object.

\subsection{The Programming Test}

Nanofabrication is typically handled as a Programming Nanofabrication
Test. In most cases, this can be treated as a Simple Success
Test (p. 118), with a failed roll simply indicating that the item has 
some minor imperfections, or perhaps took longer to make.

In some cases, the gamemaster may call for an actual Success Test, 
meaning that failure is more of a possibility. This should only be 
done for items that are exotic, extremely complicated, or for which 
the blueprints are incomplete or otherwise suspect. This test can also 
be made if the raw materials are limited.

The character operating the nanofabricator can make this test or 
it can be left up to the nanofabricator's built-in AI. Most such Such 
AIs have a Programming (Nanofabrication) skill of 30 (see \textit{AIs and }
\textit{Muses,} p. 331).

\section{Reputation And Social Networks}

\begin{quote}
``Once upon a time, there was a planet so incredibly primitive that its 
inhabitants still used money. That planet is called ‘Mars.'''

—Professor Magnus Ming, Titan Autonomous University
\end{quote}

The conflict between market capitalism and other forms of economics
is one of transhumanity's last great culture wars, and it's
still being fought. Transhumanity's expansion into the solar system 
created myriad opportunities to experiment with new economic systems
Many failed, but the reputation economies of the outer system
have proven both utilitarian and robust in a way that no previous 
challenger to market capitalism has managed.

The reputation economy, sometimes called the gift economy or 
open economy, is one in which the material plenty created by nanofabrication
and the longevity granted by uploading and backups
have removed considerations of supply versus scarcity from the 
economic equation—destroying classical economics in the process.

The regimented societies of the inner system and the Jovian 
Junta have used societal controls and careful regulation of the 
technologies of abundance on their populations, thus keeping to a 
transitional economy system that is largely an outgrowth of classical
economics. No one could get away with doing this in the outer
system. In the Trojans and Greeks, much of the belt, free Jupiter, and 
anywhere outward from Saturn, the reputation economy rules.

How did this happen? For one thing, money is a nuisance when 
you're an autonomous member of an autonomous collective whose 
nearest three neighbors (each 100,000 kilometers away) are also 
autonomous collectives. All of you are almost completely self-sufficient
in terms of material resources. You have a fleet of robots
that harvest water, volatiles, reactor mass, metals, and silicates. You 
have a nanofabricator to make all of your small items, a community 
factory for large ones, and a machine shop where you can build 
anything else—with help and advice from an AI with the combined 
knowledge and experience of a top flight engineering team, if you 
even need it. You grow your own food.
%%% 288

Money is for people who don't know how to take 
care of themselves. Transhumanity is only a few decades
away from being a mature Type I Kardashev
civilization, having largely mastered the material 
resources of its own solar system. A character from 
the outer system most likely finds the whole concept 
of money an embarrassment.

However, material abundance hasn't eliminated 
the value of certain goods and services. A transhuman's
lunch might be free, but innovative ideas, new
designs, health care, sex, and dirty work don't grow 
in fabricators. What if you need gene therapy on your 
morph to grow infrared sensing cells on your face? 
How about someone to assassinate your renegade 
beta fork after she set off a hallucinogen grenade at 
your gallery opening and kidnapped your boyfriend? 
What if you really need a spanking? You call on your 
social network. If your network is sufficiently deep 
and numerous, and your reputation is good enough, 
someone will help you out.

In the inner system, the reputation economy doesn't 
replace money for the exchange of goods and services, 
but it does hold sway over the network of favors and 
influence. Calling on contacts, getting information, 
and making sure you're in the best place to see and be 
seen all involve calling on your social network.

\subsection{Social Networks}

Social networks represent the people you know, and 
the people they know, and so on. It starts with your 
friends and family, spreads out to your co-workers, 
neighbors, and colleagues, and expands all the way 
out to your acquaintances, from the neo-hominid 
waitron at your favorite cafe to the sylph you flirt 
with at the club. In the always-online, fully-meshed 
universe of \textit{Eclipse Phase,} this goes even further, 
encompassing all of the people you've linked to via 
social mesh networks, everyone who watches your 
blog/lifelog/updates, and everyone you interact with 
on various mesh forums. Now add in the friend-of-a-friend
factor, and everyone has an impressive ability
to reach out to people they know, people they sort of 
know, and people you don't know but who are somehow
linked to you in one degree or another.

Of course, social networks are not homogeneous. 
Among the ever-diversifying ranks of transhumanity, 
there is a tendency to coalesce around various shared 
characteristics, whether those be cultural background, 
personal interests, professional ties, local connections, 
political affiliations, subcultural obsessions, or simply 
common interest from being part of the same subspecies
clade. The social network of an info-anarchist
hacker is likely to bear little resemblance to that of 
a hypercorp socialite or an isolate brinker. Nevertheless
social networks quite frequently overlap, often
in unexpected and interesting ways. Most people can 
be considered members of several different types of 
social networks. This overlap is what links disparate 
groupings of transhumans together.

\subsubsection{Networking}

Just being connected, of course, doesn't mean you have 
several thousand idle transhumans at your beck and 
call. If you hope to gather the latest gossip, get advice 
from an expert, find someone who can fix your problems
acquire a piece of gray market tech, or spread a
meme, you need to know both \textit{who} to talk to in that 
social network and \textit{how} to go about getting what you 
need, especially if you hope to keep things quiet and 
not raise any flags.

This is where your Networking: [Field] skills come 
in (p. 182). Networking represents your ability to maneuver
through this web of personal and impersonal
connections to find who and what you need. This 
could be handled by word-of-mouth, posting the right 
queries in the right places on the mesh, monitoring 
the right personal profiles and forums, harnessing the 
power of the mob with crowdsourcing, or any number 
of similar creative tactics.

Each field you have in Networking represents a 
particular network grouping, a common interest that 
ties people together. Most of these fields are based 
on factions (Autonomists, Hypercorp, etc.) and tie 
into a special reputation network (see the Reputation 
Networks table, p. 287). At the gamemaster's discretion
other groupings of people could be connected
through these skills and rep systems. For example, 
artists and journalists of all stripes can fall under 
the Networking: Media skill and f-rep, no matter if 
they are autonomist or hypercorp. Likewise, being a 
diverse group, brinkers do not universally fall into any 
of the categories, and are instead spread out between 
them. If the gamemaster and players agree, other 
Networking fields and rep networks may be added, 
representing other spheres of interest, such as AR 
Games, Sports, Slash Fiction, etc.

The exact uses for which you may exploit your social 
networks are noted below. While in some cases the defining
element is who you know and how good you are
at reaching out to them, in others the defining element 
is how known \textit{you} are. You might be connected to 
thousands of people, but if you don't have clout, your 
efforts to make use of these connections is limited. This 
is where Reputation comes into play.

\subsection{Reputation}

Reputation is a measurement of your social currency. 
In the gift economies of the outer system, social reputation
has effectively replaced money. Unlike credit,
however, reputation is far more stable.

Within \textit{Eclipse Phase,} reputation scores are facilitated
by online social networks. Almost everyone is a
member of one or more of these reputation networks. 
It is a trivial matter to ping the current Rep score and 
history of someone you are dealing with—your muse 
often does this automatically, marking an entoptic 
Rep score badge on anyone with whom you interact, 
updated in real time, so you will see if they suddenly 
take a hit or become popular. The 7 most common 
networks are noted on the Reputation Networks 
%%% 289
table. Gamemasters and characters may decide to add 
others appropriate to their game.
You purchase a Rep score in one or more of these 
networks during character creation. Rep scores are 
rated between 0 and 99, just like skills. These ratings 
determine your ability to acquire goods, services, and 
information and favors, as noted below. These scores 
may be raised or lowered during game play according 
to your character's actions.

\subsection{Using Networks And Rep}

In game terms, you take advantage of your connections
and personal cred every time you need a \textit{favor.}
A favor is broadly defined as anything you try to get 
via your social networks, whether that be information, 
aid, goods, and so on. Different types of favors are 
described under \textit{Favors,} p. 289.

\subsubsection{The Networking Test}

To pursue a favor, you start by looking around. This 
calls for a Networking Test to determine if you can 
find the person, people, or information you need. 
This represents talking to people you know, spreading
the word to people they know, posting queries to
the social network at large, digging through various 
profiles, chat rooms, etc. to find someone who might 
help you out, and so on.
Networking Tests are subject to modifiers for the 
level of the favor (see below), the amount the character
is trying to keep quiet about the request (see
below), and any other factors noted on the Networking
Modifiers table or determined by the gamemaster.
Networking Tests are Task Actions—it takes time 
to call in favors or track down information. The timeframe
depends on the level of favor, as noted on the
Favors table, p. 289.


\begin{table}
\caption{Reputation Networks}
\begin{tabularx}{\textwidth}{|X|l|l|X|}
\hline
%NETWORK NAME & REP NAME & NETWORKING FIELD & FACTIONS AND OTHERS \\
Network name & Rep name & Networking field & Factions and others \\
\hline
The Circle-A List & @-Rep & Autonomists & anarchists, Barsoomians, Extropians, Titanian, and scum \\
\hline
CivicNet & c-Rep & Hypercorps & hypercorps, Jovians, Lunars, Martians, Venusians \\
\hline
EcoWave & e-Rep & Ecologists & nano-ecologists, preservationists, and reclaimers \\
\hline
Fame & f-Rep & Media & socialites (also artists, glitterati, and media) \\
\hline
Guanxi & g-Rep & Criminals & criminals \\
\hline
The Eye & i-Rep & Firewall & Firewall \\
\hline
Research Network Associates & r-Rep & Scientists & argonauts (also technologists, researchers, and scientists) \\
\hline
\end{tabularx}
\end{table}

\subsubsection{Favor Levels And Modifiers}

Rep scores are broken down into five levels, reflecting 
your standing within that community. Every 20 points 
of Rep equals one level. See the Reputation Levels 
table for a breakdown.

Likewise, favors are also broken down into five 
levels, rated from Trivial to Scarce (see \textit{Favors,} p. 289, 
for specific examples). The standard level of favor 
you can expect to get from a social network is based 
on your level of Rep in that network. If you want to 
pursue a favor above your level, you can do so, but 
you will suffer a negative modifier on your Networking
Test. This reflects that someone with low standing
has a hard time getting people to go out of their way 
for them. Similarly, if you pursue a favor below your 
level, you receive a positive modifier to your Networking
Test, reflecting that your prestige makes it easier to
acquire minor things that you need. For each level the 
favor falls under or above your Rep score level, apply 
a + or –10 modifier, as appropriate.

\begin{table}
\caption{Networking modifiers}
\begin{tabular}{|l|l|}
\hline
SITUATION & MODIFIER \\
\hline
Favor level exceeds Rep level & –10 per level \\
\hline
Rep level exceeds favor level & +10 per level \\
\hline
Keeping quiet & –Variable (see p. 288) \\
\hline
Burning Rep & +Rep amount burned \\
\hline
Paying extra & +10 per level \\
\hline
\end{tabular}
\end{table}

\begin{table}
\caption{Reputation Levels}
\begin{tabular}{|l|l|}
\hline
REPUTATION SCORE & REPUTATION LEVEL \\
\hline
0–19 & Levek 1 \\
\hline
20–39 & Levek 2 \\
\hline
40–59 & Levek 3 \\
\hline
60–79 & Levek 4 \\
\hline
80–99 & Levek 5 \\
\hline
\end{tabular}
\end{table}

\begin{quotation}
Jaqui’s on a scum barge and she needs to get a
hold of a weapon fast. She has a specific weapon in
mind, but it’s pricey—its cost is High. She decides
her best approach is to try talking to the scum on
the ship to try and find someone who can lend or
sell her such a weapon, using her @-rep and her
Networking: Autonomist skill of 50. Acquiring a
High cost item counts as a Level 4 High favor (see
Acquire/Unloads Goods, p. 289). Jaqui’s @-rep
is 53, which is only Level 3. Since the favor is one
level higher than her rep level, she suffers a –10
modifier on her Networking Test. Jaqui must roll a
40 or less (50 – 10) to find a weapon supplier.
\end{quotation}

%%%% 290

\subsubsection{Paying/Exchanging For Favors}

Favors don't necessarily come for free. Depending on 
what you're after, you may also need to exchange for it.

In the capitalist and transitional economies of the 
inner system and Jovian Junta, you may need to buy 
the goods or services you are after with credit. Even information
might be paid for by bribing the right person.
Once spent, that credit is gone until you earn more.

In the anarchistic reputation economies of the outer 
system, you can get what you need for free. In this 
case, you are acquiring goods and services based on 
the strength of your reputation.

\begin{quotation}
Jaqui rolls a 39—she makes it! After posting some
public notices on the scum social network (she’s
not worried about legalities or hiding what she’s
doing—this is a scum ship after all), she gets directed
to a weapons dealer with a good rep. While
a scum arms merchant normally sells their wares
for credit, Jaqui is scum herself, so she’s able to use
her scum community standing and get the weapon
for free. This uses up a High favor, however.
\end{quotation}


\subsubsection{The Limits Of Reputation}

Even in the gift economies, reputation only gets you so 
far. There are limits to how often you can ask for help 
before you start coming across as pushy or a leech. In 
game terms, this is expressed as a \textit{refresh rate}—the 
amount of time you must wait to pass before you can 
seek out a favor of that level again without seeming 
demanding. Refresh rates are noted on the Favors 
table (p. 289).

If you need to seek another favor before the refresh 
rate has expired, you have two choices. You can 
expend a higher level favor instead, keeping in mind 
that higher level favors refresh more slowly. Alternatively
you can burn reputation (see below).

\begin{quotation}
Now that Jaqui’s got her weapon, she needs
another favor—she needs to find someone who
doesn’t want to be found. The person she’s after is
scum, so once again she turns to the scum for help.
The gamemaster decides that this is another Level
4 favor (see Acquire Information, p. 291). Once
again, with her Networking: Autonomist of 50 and
Level 3 rep, she must roll a 40 or less. She gets a
21, and finds someone who has the information
she needs.
Jaqui now has a choice. To get this information,
she either needs to pay the person in credits (a
High cost) or she she needs to expend another
Level 4 favor. She’s low on money, so she decides
to use her rep again. Level 4 favors only refresh
once a month, though, and Jaqui used her last one
just a few hours ago. Her only choice is to expend
a higher favor, so she expends a Level 5 to get the
intel she needs.
\end{quotation}


\subsubsection{Burning Reputation}

In some cases, getting what you need may be more 
important than not stepping on people's tentacles. In 
situations of dire need, you can \textit{burn }some of your 
Rep score to get the job done, meaning that you exchange
a loss of Rep for a shot at a favor. This reflects
that you are pushing the bounds of how far people are 
willing to go for you. While you still might get what 
you need, your online reputation rating takes a hit as 
people flag you for being needy.

There are two reasons to burn Rep score. The first 
is to get a bonus on your Networking Test. This indicates
that you are pulling strings and calling in markers
to get the favor you're after. This is particularly
useful when you are trying to obtain a favor that's of 
a level higher than your Rep, but abuse it too often 
and you will soon have no social standing at all. Every 
point of Rep you burn gives you an equivalent positive
modifier on the Networking Test, up to a maximum
of +30.

The second option is to burn Rep to seek a favor 
before it has refreshed. This reflects that you are asking 
for too much in a short period. The amount of Rep 
you must burn in this case depends on the level of favor 
you are seeking, as noted on the Favors table (p. 289).

\begin{quotation}
Jaqui’s got her weapon and her target’s whereabouts,
but she needs one more thing: a hacker.
She needs someone who can open some doors
and defeat some security systems so she can get
to the target she’s after in his hideout. Since she’s
on a scum barge, Jaqui feels that, once again, her
best option is to work her scum contacts. The
gamemaster determines that this will be another
Level 4 favor. Rolling against a target number of
40 again, she gets a 13—her luck is holding.
She finds a hacker, but now she needs to make
an exchange for their services. Once again she
decides not to spend credit and use her @-rep
instead. Jaqui’s already used up both her Level 4
and Level 5 @-rep favors, though, so she has no
choice but to burn reputation. A Level 4 favor costs
10 Rep to burn. Jaqui spends it, sending her @-rep
from 53 to 43—she’s been pulling in a lot of big
favors in a short amount of time, and her friends
and acquaintances are expressing their annoyance
by lowering her social standing.
\end{quotation}

\subsubsection{Keeping Quiet}

The problem with using social networks for favors is 
that you end up letting lots of other people know what 
you're up to. When you're involved in a clandestine 
operation, that could be exactly what you \textit{don't} want. 
The only way to diminish this is to take your requests 
to trusted friends and ask them to keep quiet, but this 
diminishes the pool of people at your disposal.
%%% 291
In game terms, you can try to keep word of what 
you're doing quiet, but this makes it harder to get 
what you need. For every negative modifier you apply 
to your Networking Test, the same negative modifier 
applies to anyone making a Networking Test to find 
out what you're up to.

\begin{quotation}
Revisiting one of our previous examples, we go
back to the point where Jaqui was trying to
ascertain someone’s hideout location. Because
the person she’s after is scum, they’re on a
scum ship, and Jaqui is using her Networking:
Autonomist skill to find them, there’s a good
chance that if she starts asking around to everyone,
word might trickle back to the person
she’s after. She doesn’t want them to know
she’s on their tail, though, so she decides to
make her inquiries more discreet. She applies
a –20 modifier to her Networking Test, which
lowers her target number from 40 to 20. As
noted before, she rolls a 21, which is a failure.
She spends a Moxie point to flip the roll,
though, making it a 12—a success.
Because Jaqui took that –20 hit, representing
the fact that she was keeping her research
quiet, her target will suffer a –20 modifier
when he makes his Networking Test to see if
he gets word that someone is asking around
about his hideout.
\end{quotation}

\subsection{Favors}

Creative players can undoubtedly come up with many 
uses for their social networks, but a few of the more 
common are detailed here.
Gamemasters should use their discretion as to how 
much roleplaying interaction and Networking Tests 
are included in using a social network. For normal 
goods, straightforward information queries, or small 
favors, neither dice rolling nor roleplaying may be required
For major requests, interactions with contacts,
and mission assistance, dice rolls and/or roleplaying 
interaction with contacts from the social network 
should usually occur. Gamemasters may wish to keep 
track of the NPC contacts in each character's social 
networks and make them recurring characters.

\begin{table}
\caption{Favors}
\begin{tabular}{|l|l|l|l|}
\hline
FAVOR LEVEL & TIMEFRAME & BURNING REP COST & REFRESH RATE \\
\hline
1 (Trivial) & 1 minute & 0 & 1 hour \\
\hline
2 (Low) & 30 minutes & 1 & 1 day \\
\hline
3 (Moderate) & 1 hour & 5 & 1 week \\
\hline
4 (High) & 1 day & 10 & 1 month \\
\hline
5 (Scarce) & 3 days & 20 & 3 months \\
\hline
\end{tabular}
\end{table}


\subsubsection{Acquire/Unload Goods}

Social networks are a good way to find items that 
you can't buy legally or make at home. Depending 
on who you're getting the goods from, this will cost 
you credit or require an appropriate Rep score. This 
favor can also be used to sell or give away such items, 
making some money or perhaps even some Rep in 
the process.

\begin{table}
\caption{Acquire/unload goods}
\begin{tabular}{|l|l|}
\hline
LEVEL & SERVICE \\
\hline
1 & Acquire/unload item with an expense of Trivial. \\
\hline
2 & Acquire/unload item with an expense of Low. \\
\hline
3 & Acquire/unload item with an expense of Moderate. \\
\hline
4 & Acquire/unload item with an expense of High. \\
\hline
5 & Acquire/unload item with an expense of Expensive \\
\hline
\end{tabular}
\end{table}


\subsubsection{Acquire Services}

When you lack the skills or education you need, or 
you just need another set of arms, you can call out 
to your social network to find someone to help you 
out. If you are looking for someone with a particular 
skill, the result of your successful Networking Test roll 
is the skill rating of the person you find. The higher 
your Networking skill, the better able you are to find 
highly-skilled professionals.

\begin{quotation}
Cole needs to find an astrobiologist who can help
him identify an alien critter. He rolls his Networking:
Scientist skill of 50 and gets a 43—a success.
He tracks down someone with Academics: Astrobiology
skill of 43 (his roll) who can help him out.
When the astrobiologist looks the critter over, the
gamemaster makes a roll for the NPC using that
skill of 43.
\end{quotation}


\subsubsection{Acquire Information}

When you can't find the information online or you 
don't have the time or capability to look, you can 
turn to people in your social network and tap their 
accumulated knowledge base.

\subsection{Reputation And Identity}

It is important to note that reputation is closely tied 
to identity. If you are undercover and using a fake ID, 
you can't really call on your Rep score without giving 
yourself away. As a result, many people using false 
identities end up building up a separate set of Rep 
scores for their alter ego.
Note that since many social network interactions 
take place online, it is possible for someone to secretly 
make use of their real identity while masquerading 
as someone else, as long as they're careful about it. If 
anyone happens to be spying on their activity via the 
mesh, they stand a chance of being found out.


\begin{table}
\caption{Acquire Services}
\begin{tabularx}{\textwidth}{|l|X|}
\hline
LEVEL & SERVICE \\
\hline
1 & \textbf{Trivial favor}: Get someone to perform services for 15 minutes. Move a chair. Browbeat someone. Catch a ride.  Research someone online. Borrow 50 credits. Other Trivial cost services. \\
\hline
2 & \textbf{Minor favor}: Get someone to perform services for an hour. Move to a new cubicle. Rough someone up. Loan a vehicle. Provide an alibi. Healing vat rental. Minor hacking assistance. Basic legal or police assistance. Borrow 250 credits. Other Low cost services. \\
\hline
3 & \textbf{Moderate favor}: Get someone to perform services for a day. Move to a habitat in the same cluster. Serious beating. Lookout. Short-distance egocast. Short shuttle trip (under 50,000 km). Minor psychosurgery. Uploading.  Reservations at the best restaurant ever. Major legal representation or police favors. Borrow 1,000 credits. Other Moderate cost services. \\
\hline
4 & \textbf{Major favor}: Get someone to perform services for a month. Move a body. Homicide. Getaway shuttle pilot.  Industrial sabotage. Large-volume shipping contract on bulk freighter. Medium-distance egocast. Mid-range shuttle trip (50,000–150,000 km). Moderate psychosurgery.  Resleeving. Get out of jail free. Borrow 5,000 credits. Other High cost services. \\
\hline
5 & \textbf{Partnership}: Get someone to perform services for a year. Move dismembered body. Mass murder. Major embezzlement. Acts of terrorism. Relocate a mid-size asteroid. Long-distance egocast. Long-range shuttle trip (150,000 km or more). Borrow 20,000 credits. Other Expensive cost services. \\
\hline
\end{tabularx}
\end{table}



\begin{table}
\caption{Acquire Services}
\begin{tabularx}{\textwidth}{|l|X|}
\hline
LEVEL & SERVICE \\
\hline
1 & \textbf{Common Information}: Where to eat. What biz a certain hypercorp is in. Who’s in charge. \\
\hline
2 & Public Information: Make gray market connections.  Where the “bad neighborhood” is. Obscure public database info. Who’s the local crime syndicate. Public hypercorp news. \\
\hline
3 & \textbf{Private Information}: Make black market connections.  Where an unlisted hypercorp facility is. Who’s a cop. Who’s a crime syndicate member. Where someone hangs out.  Internal hypercorp news. Who’s sleeping with whom. \\
\hline
4 & \textbf{Secret Information}: Make exotic black market connections. Where a secret corp facility is. Where someone’s hiding out. Secret hypercorp projects. Who’s cheating on whom. \\
\hline
5 & \textbf{Top Secret Intel}: Where a top secret black-budget lab is. Illegal hypercorp projects. Scandalous data.  Blackmail material. \\
\hline
\end{tabularx}
\end{table}


%%% 293

\section{Security}

Firewall sentinels make a regular habit of being in 
places where they are not supposed to be and bringing
things with them that others would prefer they
not have. Security has a different character post-Fall 
than in the 21st century. Due to hyper-abundance, 
physical security measures such as locks, doors, and 
walls are less important than in the past to common 
citizens. People don't worry about theft as much as 
in the past because most items can be replaced by a 
nanofabricator. The items that do tend to engender 
this type of security are irreplaceable or rare items 
such as artifacts of Earth.

Post-Fall physical security focuses heavily on 
surveillance—identifying intruders and tracking 
them so that they can be interdicted by transhuman 
or robotic defenders. Surveillance is more effective
than in pre-Fall societies because AIs with
near-human faculties of pattern recognition and 
indentured infomorphs can be employed to monitor 
surveillance data.

The emphasis on surveillance results from the ease 
with which most material barriers can be breached 
by high-powered hand weaponry and devices like 
the covert operations tool (p. 315). However, physical
barriers designed to actively resist intruders by
healing themselves or attacking tools used to damage 
them are used at key points in secure installations. 
Such barriers are typically very expensive and so are 
used sparingly.

Transhuman, animal, and infolife defenders are 
cornerstones of most security systems. The availability
of a huge pool of infomorph labor to guard
facilities means that someone is always on duty, 
whether as part of the surveillance system or in a 
robotic shell.

\subsection{Access Control}

The first step in any security system is simply to enact 
measure to keep unwanted people out. At a basic level 
this involves walls, locks, fencing, defensive landscaping
security lighting, and entoptic warnings.

Barriers of different sorts present an obstacle that 
must be cut through or blown apart in order to 
defeat. Barriers are treated just like other inanimate 
objects for purposes of attack sand damage; see \textit{Ob-}
\textit{jects and Structures,} p. 202.

\subsubsection{Bug Zappers}

Bug zappers create minute EMP pulses that are harmless
to most electronic equipment and implants but
wreak havoc on nanobot swarms, microbugs, and 
specks. Bug zappers are generally applied to surfaces, 
and as such they only destroy floating/flying swarms 
or specks if they land. In areas with heavily shielded 
electronics, they may be installed to destroy targets 
in an entire room. A zapper instantly destroys all 
free-crawling or flying nanobots and specks in a room 
when it goes off, but transhuman flesh is sufficient to 
prevent it destroying medichines or other implanted 
nanobots. Infiltrators trying to gather data in areas 
protected by zappers generally resort to going around 
them or trying to plant macroscale devices.

\subsubsection{Electronic Locks}

Electronic locks (e-locks) are commonly used as a 
means of maintaining privacy. They are easy to defeat, 
however, and so are rarely used in very secure areas. 
E-locks have several advantages over old-fashioned 
mechanical locks. Different users can have different 
authentication methods, they can log all events (entry, 
exit, failed authentications), and they can be connected
(usually hardwired but sometimes encrypted
wireless) to security systems for remote control and 
alarm triggering.

E-locks use one of several authentication systems, 
or sometimes a combination of systems:

\textbf{Biometric:} The lock scans one or more of the user's 
biometric prints. Common biometrics include DNA, 
facial thermographic, fingerprint, gait, hand veins, iris, 
keystroke, odor, palm, retinal, and voice prints.

\textbf{Keypad:} This is an alphanumeric keypad upon 
which users enter a specific code. Different users can 
have different codes.

\textbf{Token:} Authorized users must carry some sort of 
physical token that interacts with the lock to open the 
door, such as a keycard, electronic key, etc.

\textbf{Wireless Code:} Users must emit a cryptographic 
code via near-proximity wireless signal.

Though various technologies exist to defeat each 
of these systems, there are three methods that work 
against almost all e-locks. The first is use of a covert 
operations tool (p. 315), which infiltrates a lock with 
nanobots that swarm in and engage the electronic 
mechanism. The drawback to using a COT is that its 
use is immediately logged by the e-lock and an alarm 
is triggered. Some e-locks are equipped with guardian 
nanoswarms (p. 329) to defeat COTs, but the COT 
nanobots usually manage to open the lock before the 
guardians eat them.

The second method is to hack the e-lock. Most 
e-locks are slaved to a security system, so this often 
means intruding into the security system and then 
opening the lock from within. This can be difficult, 
however, especially if the security system is wirelessly
isolated or hardwired. The advantage is that,
if done right, all evidence of the lock being opened 
can be erased.
The third method is to physically open and manipulate
the lock. This requires first opening the
lock's case and then triggering the lock mechanism 
to open the door. Both of these are handled as 
separate Hardware: Electronics Task Actions with 
a timeframe of 1 minute each. In addition, most e-locks
have anti-tamper circuits that will set off an
alarm if the attacker does not achieve an Excellent 
Success when opening the case.

%%% 294

\subsubsection{Lockbots}

The 21st century saw a move from mechanical 
locks to e-locks and other largely electronic locking 
mechanisms. These devices worked well for about 
50 years, until electronic infiltration  capabilities 
rendered them largely useless. The more recent development
of lockbots has more in common with
their early mechanical forebears. They are unique, 
expensive, artisan items.

A typical lockbot is heavily integrated with the 
portal and barrier it protects. Lockbots usually 
include an AI or indentured infomorph, self-healing 
materials (treat as a self-healing barrier), and a 
swarm of guardian nanobots (p. 329). A lockbot 
monitors its surroundings and has visual recognition
software that knows what its users and its keys
look like (Perception skill 40). Picking a lockbot is 
thus incredibly difficult, because it will shut its orifice
and not accept a key that doesn't look right or
that comes from an unrecognized user. Unfamiliar 
nanobots trying to enter the orifice are targeted and 
destroyed by the guardian nanobots. Finally, external
tools used to harm the portal or the lock will be
attacked by fractal appendages extruded from the 
portal surface or the lock itself. These appendages 
have a range of 1 meter, attack with skill of 40, and 
inflict 1d10 +2 DV.

Lockbots are generally immune to being hacked 
because, for security, they aren't connected to 
the mesh. If attacked, however, lockbots are programmed
to send out an alarm signal via the mesh.

There are several ways to defeat a lockbot. One 
is to get a copy or image of the key and then forge 
a copy (using nanofabrication). Another is to attack 
the lockbot or the portal it guards with so much 
force that the lockbot is unable to repair it (usually 
using ranged weapons, as anything within a meter 
of a lockbot may be counterattacked). A third is to 
somehow image the cavity beyond the lockbot's orifice
without the imaging device being destroyed and
to then forge the key. All of these are difficult and 
time-consuming processes.

Some lockbots have the ability to destroy what 
they're protecting. For example, lockbots are a 
common protection for the physical interfaces to 
hardwired networks. If the lockbot is compromised, 
it may, as a last resort, destroy the interface it was 
protecting.

\subsubsection{Portal Denial System}

Installed in corridors or doorways, this is essentially 
a laser trap device. When an unauthorized person 
enters the portal denial system's area, it uses lasers 
to create a grid of plasma channels that are used to 
deliver a powerful electric current to the target. This 
system has both lethal and nonlethal settings.

\textbf{Nonlethal:} 1d10 DV + shock (p. 204)

\textbf{Lethal:} 2d10 +5 DV

\subsubsection{Self-Healing Barriers}

Walls and doors that are able to rapidly repair themselves
are sometimes found in high security installations
These barriers are made of materials that
automatically expand to ``heal'' small holes and that 
are equipped with nanosystems that slowly repair 
larger amounts of damage. The best of these barriers 
do no more than slow down the most determined 
assailants, but in combination with surveillance systems
they are a nuisance to invaders and can slow
down attempts to flee the scene.

Self-healing barriers heal any single source of 
damage that is less than 5 points of damage almost 
immediately, sealing the hole in 1 Action Turn. They 
will also seal the holes inflicted by a covert ops 
tool (p. 315) in the same time period. Additionally, 
these barriers repair larger themselves at the rate of 
1d10 damage per 2 hours; once all damage is fixed 
any wounds are repaired at the rate of 1 per day. 
Damage of 3 wounds or more may not be repaired 
by self-healing.

\subsubsection{Slippery Walls}

On planetary surfaces, high walls and fences are still 
common as a first line of defense against interlopers
Slippery walls are surface treated with the slip
chemical (p. 323), creating a virtually frictionless 
surface that is exceptionally difficult to climb.

\subsubsection{Wireless Inhibitors}

Wireless inhibitors are simple paint jobs or construction
materials that block radio signals. They are
used to create a contained area in which a wireless 
network may operate freely without worry that the 
signals will escape out of the area, where they can 
be intercepted. Wireless inhibitors allow the convenience
of using wireless links within a secure area
rather than the clumsier hardwired connections. If 
an intruder manages to gain access inside the area, 
however, they can intercept, sniff, and hack wireless 
devices as normal.

\subsection{Detection And Surveillance}

Should security measures fail to keep an intruder 
out, the second step is to detect an interloper and 
track their activity.

\subsubsection{Nanotagging}

A lot of post-Fall security centers not around keeping 
people out of private spaces, but tracking them after 
they come and go. What little privacy transhumans 
have, they cherish. Trespassing is a worse offense 
than theft in many places.

A room protected by a taggant nanoswarm (p. 329) 
usually has two or more hives, one each at floor and 
ceiling level (if in gravity; on the opposite side of the 
room if in microgravity) that generate and recycle 
nanobots. The taggants emerge from one hive, float 
through the room, and then return to the other for 
recharging and reuse. A feed line usually connects the 
%%% 295
hives so that they can share materials and power.

Anyone passing through the room is likely to be 
dosed with taggant nanobots. Once they lose proximity
to the rest of the hive, they hide and periodically
broadcast pulsed transmissions meant to give their position
to pursuers or investigators. Some may drop off
in clusters to form a breadcrumb trail to the interloper.

\subsubsection{Sensors}

Any of the various sensors described in the \textit{Gear}
chapter (p. 294) may be deployed within a facility 
to monitor and record the passage of personnel, 
both authorized and not. These sensors are typically 
slaved to the facility's security network and closely 
monitored by security AIs, meaning they are vulnerable
to hacking and possibly jamming. A few other
sensor types deserve mention here:

\textbf{Chemical Sniffers:} The chem sniffer described on 
p. 311 can also be set to detect the carbon dioxide 
exhaled in transhuman breaths. This is useful for 
detecting intruding biomorphs in areas that are 
abandoned/off-limits.

\textbf{Electrical Sensors:} Electrical sensors can be set in 
portals to detect a biomorph's electromagnetic field 
in addition to the electrical fields of synthmorphs.

\textbf{Heartbeat Sensors:} These sensitive sensors detect 
the vibration caused by transhuman heart beats. 
They can even be used to detect the heartbeats of 
passengers inside a large vehicle.

\textbf{Seismic Sensors:} Embedded in flooring, these sensors
pick up the pressure and vibration of weight
and movement.

\subsubsection{Weapon Scanners}

Weapon scanners come in several varieties, including
those that look for the rare elements used in
extremely destructive weapons such as nukes, those 
that attempt to locate personal weaponry, and those 
that look for detection taggants.

Rare element scanners are nearly flawless and are 
ubiquitous in habitat customs and spaceports. The 
only way to circumvent them is to find an alternate 
route into the protected area.

Personal weapon scanners can monitor a specific 
area, such as a small room or doorway. They use 
a number of sensing systems to detect and identify 
weapons and other dangerous objects, including 
chemical sniffers and radar/terahertz/infrared/x-ray/ultrasound imaging. They can detect the following 
items and substances:

\begin{itemize}
\item Metal used in kinetic weapons, seekers, and flechette weapons
\item Devices with onboard hives of metallic nano- bots (e.g., covert operations tools, spindles)
\item Magnetic elements in plasma guns and railguns
\item Propellant from firearms ammunition and seekers (–30 to conceal)
\item Chemical fuels used in torch spray weapons (–30 to conceal)
\item All explosives and grenades by their chemical particulate emissions (–30 to conceal)
\item Poisons and bioagents in flechette weapons
\item Any weapon or device larger than palm size (using sound waves and shape recognition)
\end{itemize}

Characters trying to sneak weapons and gear past 
personal weapon scanners must make a Palming
Test (if concealing) or an Infiltration
Test (if somehow maneuvering around 
without notice). This is opposed by a 
Perception Test from the character or 
AI manning the sensor system.

\subsubsection{Wireless Scanning}

Some high-security areas will intentionally
monitor for wireless
radio signals originating within 
their area as a way of detecting
intruders by their communications
emissions. These
signals can even be used to 
track the intruder's location via 
triangulation and other means 
(see  \textit{Physical Tracking,} p. 251). 
To bypass wireless detection systems
covert operatives can use
line-of-sight laser links (p. 313) 
for communication or touch-based 
skinlinks (p. 309).

\subsection{Active Countermeasures}

When all else fails, active countermeasures
may be deployed against
intruders. While live transhuman
guards are sometimes
used, robotic sentries are more 
common, typically AI-driven 
synthmorphs such as synths, 
slitheroids, arachnoids, or 
reapers, with guardian angels 
(p. 346) providing air support. 
Occasionally AI-operated gun 
emplacements—armored turrets
that pop out of walls and
ceilings—are also applied. In 
some circumstances, these shells 
are teleoperated or even jammed 
by transhuman security.

Additional countermeasures 
brought to bear will depend on the facility
in question. Some sites will engage in
active jamming, to deny the intruders any 
communication. Others will deploy hostile 
nanoswarms and even chemical weapons.