



\chapter{Action et Combat} \label{chap:action-combat} 

les jeux de rôles s'articulent autour du drame et de l'aventure, et cela signifie générallement de 'laction et du combat. Les scènes d'action et de combat sont les moments où le niveau d'adrénaline explose et où la vie et la mission des personnages sont en jeu. 

Les scénarios d'action et de combat peuvent être confus a  jouer, particulièrement si le maître de jeu doit également garder la trace des actions de nombreux PNJ. Pour ces raisons, il est important que le maître de jeu détaille ces actions d'une manière que tout le monde peut visualier, que cela signifie utiliser des cartes et des figurines, un logiciel, un tableau effaçable ou de rapide schémas sur une feuille de papier. Bien que la plupart des règles pour gérer les actiosn et le combat soient abstraites - permettant à l'interprétation et et au bidouillage des résultats de s'adapter à l'histoire - de nombreux facteurs tatciques sont également inclut, et même les plus petits détails peuvent faire la différence. Il est aussi utilse d'avoir les capacités des PNJ prédétermiénes et de les gérer en tant que groupe lorsque c'est possible, pour réduire le fardeau du maître de jeu au milieu d'une situation mouvementée. 



\section{Tour d'action} \label{sec:combat-action-turns} 

Les scènes d'action dans Eclipse Phase sont gérées en segments atomiques appelés Tour d'Action, chacun durant approximativement 3 secondes. Nou parlons "d'aproximativement" 3 secondes, car le système méthodoligue et par étape utilisé pour résoudere les actions ne se treanspose pas nécessairement de manière réaliste dans la vraie vie, où les gens prennent souvent des pauses, s'interrompent pour analyser la situation, reprennent leur souffle et ainsi de suite. Un combat qui commence et se termine en 5 Tour d'Action (15 secondes) dans Eclipse Phase pourrait de 30 seconde à plusierus minutes dans la vraie vie. D'un autre cœté, les personnages peuvent être dans une situation où leur environnement est en train de décompresser et sera vide d'air en 15 secondes, chaque seconde peut donc compter. Le maître de jeu devrait se tenir à la règle de 3 secondes par tour, mais ne devraient aps être effrayés de modifier le temps à chaque fois que la situation le demande. 

Les Tour d'Actions devraient être utilisé pour le combat et d'autres situations où le timing et l'ordre d'action est important. Si il n'est pas nécessaire de garder une trace de qui fait quoi de manière si précise, vous pouvez abandonner la structure en Tour d'Action et revenir à la forme de jeu libre "normale". 

Chaque Tour d'Action est divisé en plusieurs étapes distinctes, décrites ci-dessous. 



\subsection{Étape 1: Déterminer l'Initiative} \label{sec:roll-initiative} 

AU débit de chaque Tour d'Action, chaque JOUEUR impliqué dans la scène lance son Initiative pour déterminer l'ordre dans lequel chaque personnage agît. Pour plus de détails, voir \emph{ Initiative }. 



\subsection{Étape 2: Commencer la première phase d'action} \label{sec:begin-first-phase} 

Une fois que l'Initiative a été déterminée, la première Phase d'Action commence. Tout le monde peu agir dans la Première Phase d'Action (puisque tout le monde à une Vitesse minimale de 1), sauf si ils sont inconscients/mort/mis hors-jeu, en commençant par le personnage ayant le plus haut résultat d'Initiative réussi. 



\subsection{Éape 3: Déclarer et résoudre les actions} \label{sec:declare-resolve} 

le personnage qui agît en premier déclare puis résout maintenant les actions qu'il accomplira pendant sa première Phase d'Action. Puisque certaines actions que le personnage peut faire peuvent dépendre de l'issue d'autres actions, il n'est pas nécessaire de toutes les déclarer d'abord - elles doivent être annoncé et réglée les unes après les autres. 

Tel que décrit au chapitre Actions (p. 189), chaque personnage peut effectuer un nombre variable d'Action Rapide et/ou une seule Action Complexe pendant leur tour. Un personnage peut également commencer ou continuer une Actin de Tâche, ou retarder son action en attendant que les choses évoluent (voir Actions Retardée, p. 189). 

Un personnage qui a retardé son action peut interrompre un autre personnage à n'importe quel moment de cette étape. Ce personnage interrupteru doit accomplir entièrement son action, puis l'action revient au personnage interrompu qui finti alors le reste de son étape. 



\subsection{Éape 3: Continuer et recommencer} \label{sec:rotate-repeat} 

Une fois que le personnage a terminé ses actions pour cette phase, le personnage suivant dans l'ordre d'Initiative prend le relai, exécutant l'Étape 3 pour son compte. 

Si tous els personnages ont terminés leurs actions de cette phase, retournez à l'Étape 2 et démarrez la deuxième Phase d'Action. Chaque personnage avec une Vitesse de 2 ou plus peut repasser par l'Étape 3, avec el même ordre d'Initiative (modifié par les modificateurs de blessures). Une fois que la deuxième Phase d'Action est terminée, retournez à l'Étape 2 pour la troisième Phase d'Action, où tous les personnages qui ont une Vitesse de 3 ou plus peuvent agir. Finallement, une fois que toutes les personnegs pouvant agir pendant la troisième Phase d'Action ont agit, retournez à l"Étape 2 pour la quatrième et dernière Phase d'Action, où tout les personnages ayant une Vitesse de 4 peuvent agir une dernière fois. 

Arrivé à la fin de la quatrième Phase d'Action, retournez à l'Étape 1 et relancez l'Initiative pour le prochain Tour d'Action. 



\section{Initiative} \label{sec:initiative} 

Le minutage lors des Tours d'Action peut être critique - cela peut être la différence entre la vie et la mort pour un personnage qui a besoin de se mettre à couvert avant qu'un opposant ne dégaine et tire. La détermination de l'Inititative permet de savoir dans quel ordre agissent les protagonistes. 



\subsection{Ordre d'Initiative} \label{sec:initiative-order} 

La stat d'Initiative d'un personnage est égale à la somme de Intuition et Réflexes multipliée par 2. Ce score peut être ensuite modifié par le type de morph, les implants, la drogue, le psi ou les blessures. 

Lors de la première étape de chaque Tour d'Action, chaque personnage fait un Test d'Initiative, lançant 1d100 et y ajoutant sa stat d'Initiative. Celui qui obtiens le meilleru score agît en premier, suivi par les autres personnages en ordre décraoissanr, du plus élevé au plus bas. En cas dégalité, les personnages agissent en simultané. 

\begin{quotation} Adam, Bob, et Cami lancent leur Initiative. La stat d'Initiative d'Adam est de 80, celle de Bob est de 100 et celle de Cami est de 60. Adam obtient un 38, Bob un 24 et Cami un 76. Le score d'Initiative total d'Adam est de 118 (80 + 38), celui de Bob est de 134 (110 + 24) et celui de Cami est de 136 (60 + 76). Cami a le meilleur score, elle agît donc en premier, suivit de Bob et finalement d'Adam. Si Cami \& Bob avaient étés à égalité, ils auraient tous les deux agît en même temps. \end{quotation} 

\subsubsection{Initiative et dommages} 

Les personnages qui ont des blessures voient leur score d'Initiative temporairement réduit (voir Blessures, p. 207). Ce modificateur ets appliqué immédiatement lorsque la blessure est reçue, ce qui signifie qu'il peut modifier un score d'Initiative au milieu d'un tour d'Action. Si un personnage est blessé avant de pouvoir démarrer une Phase d'Action, son Initiative est réduite en fonction, ce qui peut signifier qu'il va maintenant agir après quelqu'un avant qui il aurait agit dans l'ordre d'Initiative. 

\begin{quotation} Avat la Phase d'Action de Bob, il prend deux blessures, réduisant son Initiative de 134 à 114. Cela signifie qu'Adam, avec une Initiative de 114, peut maintenant agir avant lui. \end{quotation} 

\subsubsection{Initiative, Moxie et Critiques} 

Un personnage peut dépenser un point de Moxie pour agir en premier lors d'une Phase d'Action, peu importe son jet d'Initiative (voir Moxie, p. 122). Si plsu d'un personnage choisit cette option, l'ordre est déterminé normalement parmi ceux qui ont dépensés du Moxie, suivit par ceux qui ne l'ont pas cfait. 

De manière similaire, tout personnage qui obtient un critique sur un jet d'Initiative agit automatiquement en premier, même avant ceux qui ont dépensés du Moxie. Si deux personnage ont obtenus des critiques, déterminez l'ordre entre eux de manière normale. 

\subsection{Simplifier l'initiative} \label{sec:simplifying-init} 

Pour des résolutions plsu rapide, demandez au personnage de ne faire un jet d'Initiative qu'une fois pour la scène entière. Ce résultat d'Initiative reste le leur pour tous les Tour d'Actions jusqu'à la fin du combat ou que le scénario soit terminé. De manière similaire, ignorez les modificateurs d'Initiative liés aux blessures. 

\subsection{Vitesse} \label{sec:speed} 

La Vitesse détermine le nombre de fois qu'un personnage peut agir pendant un Tour d'Action. Chaque personnage démarre avec une VItesse apr défaut de 1, ce qui signifie qu'ils peuvent agir que lors de la première Phase d'Action de chaque tout. Certaines morphs, implants, drogues, psi ou d'autres facteurs peuvent augmenter cumulativement la VItesse à 2, 3 ou même 4 (le maximum), leur permettant d'agir lors d'autres Phases d'Actiosn également. Par exemple, un personnage avec une Vitesse de 2 peut agi lors de la première et de la deuxième Phase d'Action, et un personnage avec une Vitesse de 3 peut agir de la première à la troisième Phase d'Action. Un peronnage avec une Vitesse de 4 peut agir dans toutes les Phases d'Actions. Cela représente les réflexes et la neurologie améliorés du personnage, lui permettant de penser et d'agir bien plsu vite que les personnages non améliorés. 

Si la Vitesse d'un personnage ne lui permet pas d'agir pendant une Phase d'Action, ile ne peuvent faire aucune action pendant la phase - ils doivent simplement attendre leur tour. le persqonnage peut cependant toujours se défendre et toute acton automatique reste "active." Notez que tout mouvement démarré par le personnage est toujours considéré comme en cours même durant les Phases d'Actions auquelles les personnages ne participents pas (coir \emph{Mouvement }, p. 190). 



\subsection{Actions retardées} \label{sec:delayed-actions} 

Lorsque vent votre tour d'agir pendant une Phase d'Action, vous pouvez décider que vous n'êtes pas encore prêt à agir. Vous pouvez être en train d'attendre le résultat d'une autre actiond 'un personnage, espérer interrompre l'action de quelqu'un d'autre ou simplement ne pas savoir quoi faire. Dans ces cas là, vous pouvez décider de retarder votre action. 

Quand vous retardez votre action, vous vous mettez en attente. À un moment ultérieur de la Phase d'Action, vous pouvez annoncer que vous effectuez votre action maintenant - même si vous interrompez l'actiond 'un autre personnage. Dans ce cas, toutes les autres activités sont mises en attentes jusqu'à ce que votre action soit résolue. Une fois que votre action s'est déroulée, l'ordre d'Initiative reprend là où vous l'avez interrompu. 

Vous pouvez retarder votre action jusque dans la prochaien phase d'Action, ou même jusqu'au prochain Tour d'Action, mais si vous ne l'avez prise au moment où arrive votre prochaine action dans l'ordre d'Initiative, alors vous la perdez. De plus, si vous délayez votre action dans une autre phase ou tour, vous perdez toute action que vous auriez eu dans cette Phase d'Action au moment où vous la prenez. 



\section{Actions} \label{sec:actions} 

Lorsque c'est à votre tour d'agir lors d'une Phase d'Action, vous avez de nombreuses options pour faire ce que vous voulez - bien trop pour toutes les lister ici. Il y a cependant une limite à ce que vous pouvez accomplir en 3 secondes et certaines limitations doivent être respéctées. La première chose est de détermienr quel type d'action vous voulez faire. À Eclipse Phase, les actiosn sont catégorisées comme AUtomatique, Rapide, Complexe ou de Tâche, en fonction de la quantité de temps et d'effort ils entraînent. 



\subsection{Actions automatiques} \label{sec:combat-automatic-actions} 

Les Atcions Automatiques ne nécessitent aucun effort. Ce sont les capacités ou les activités qui sont "toujours active" (du moment que vous êtes conscients) ou qui sont d'un du niveau des réflexes (elles arrivent automatiquement en réponse à certaines conditions, sans effort de votre part). Respirez, apr exemple, est une action automatique - votre corpos le fait sans effort conscient ou réflexion de votre part. 

Dans la plupart des cas, les Actions Automatiques ne sont pas quelque chsoe que vous initiez - elles sont toujours active, ou au moins en attente. Certaine circonstance, cependant, peuvent demander de s'occuper des Actiosn Automatiques. De telles Actions Automatiques sont gérées immédiatement quand elles entrent en action, sans nécessiter d'effort de votre personnage. 

\subsubsection{Résistance} 

Résister aux dégats - qu'ils viennent du combat, d'un poison ou d'une attaque psi - est l'un des exemples d'Actiosn Automatiques qui se décelncehnt en répponse à quelque chose d'autre. 

\subsubsection{Perception basique} 

Vos sens sont continuellement actifs, accmuluant des données sur le modne autour de vous. La perception basique est considérée comme une Action Automatique, et le maître de jeu peut vous demander de faire un Test de Perception lorsque vous recevez une entrée sensorielle que votre cerveau peut vouloir prendre en compte (voir perception, p. 182). De amnière similaire, vous pouvez demander au maître de jeu à n'importe quel moment - même pendant les actiosn d'autres personnages - de faire un Test de Perception basique juste pouv savoir ce que votre personnage remarque dans son environnement immédiat. 

En raison de la nature autoimatique, subconsciente, de la perception basique, vous subirez un modificateur de -20 à cause de la distraction - votre attention est concentrée ailleurs. Afin d'éviter le modificateur de distraction vous devez vous concentrer pleinement dans une perception détaillée ou utilisez un oracle implanté (p. 308). 



\subsection{Actions rapide} \label{sec:combat-quick-actions} 

Les actions rapide sont simples, elles peuvent donc être exécutées rapidement et simultanément à d'autres. Elles nécessitent un effort et une concentration minimales. Vous pouvez effectuer plusieurs Actions Rapide lors de chacun de vos Phases d'Actions, limitées uniquement par le jugement du maître de jeu. Si vous ne faites que des Actiosn Rapides durant une Phase d'Actions, vous devriez avoir le doit de faire au moins 3 Actiosn Rapides distinctes. Si vous effectuez également une Action Complexe ou une Action de Tâche pendant la même Phase d'Action, vous devriez avoir droit à un minimum d'une Action Rapide. C'est le maîþre de jeu qui a le dernier mot par rapport à ce que vous pouvez ou ne pouvez pas faire tenir dans une seule Phase d'Action. 

Quelques exemples d'Actions Rapides incluent: parler, basculer un cran de sureté, activer un implant, se relever, se jetter au sol, communiquer par geste, dégainer/préparer uen arme, manipuler un objet ou utiliser un appareil simple. 



\subsubsection{Ajuster} 

Ajuster est un cas particulier car c'est une Action Rapide mais qu'elle nécessite un certain niveau de concentration qui fait exceptions aux autres actions mineures. Si vous souhaiter ajuster avant d'effectuer une attaque dans la même Phase d'Action, ajuster est la seule ActioN Rapide que vous pouvez prendre pendant cette Phase d'Action (voir Tirs Ajustés, p. 193). 

\subsubsection{Perception détaillée} 

La perception détaillée implique de prendre un moment pour utiliser activement vos sens à la recherche d'information et analyser ce que vous percevez (voir Perception, p. 182). Cela recquiert un peu plus d'effort et d'effort cérébral (ou de puissance de calcul) que la perception basique, qui est une automatique. En tant qu'Action Rapide, vous ne pouvez vous livrer à une perception détaillée que lors d'une de vos Phases d'Actions, mais vous ne souffrez d'aucun modificateur pour la distraction (à moins que vous ne soyez dans un environnement extrêmement distractif, tels qu'une fusillade ou une foule agitée). 



\subsection{Actions complexe} \label{sec:combat-complex-actions} 

Les Actions Complexe nécessitent plus de concentration et d'effort que les Actions Rapides - elles monopolisent effectivement votre attention. Vous ne pouvez prendre qu'une Action Complexe à chacune de vos Phases d'Actions. Vous ne pouvez pas non plus effectuer une Action Complexe et une Action de Tâche pendant la même Phase d'Action. 

Des exemples d'Actions Complexes incluent: attaquer, tirer, faire des acrobaties, défense totale, désamorcer une bombe, utilsier un appareil complexe ou recharger une arme. 



\subsection{Actions de Tâche} \label{sec:combat-task-actions} 

Les actions de Tâches sont des actions qui nécessitent plus de temps qu'un Tour d'Action pour se réaliser. Chaque action de Tâche possède un intervalle pour déterminer le temps que mets l'action à s'accoplir. Cet intervalle peut varier de 2 Tours d'Actions à 2 ans. Lorsque vous effectuez une Action de Tâche, vous ne pouvez pas effectuer d'Action Complexe, même si dans certains cas vous pouvez mettre la tâche en pause et y revenir plus tard. Pour plus d'information, voir Action de Tâche, p. 120. 

Des exemples d'Actions de Tâche incluent: réparer un appareil, programmer, mener une analyse scientifique, fouiller une pièce, esacalader un mur ou cuisiner un repas. 



\section{Muovement} \label{sec:combat-movement} 

Le mouvement dans \emph{Eclipse Phase} est géré comme n'importe quelel autre action, et peut changer d'une Pahse d'Action à une autre. Marcher et courrir comptent tout les deux comme des Actiosn Rapide, car elle ne nécessitent pas tout votre concentration. La même règle s'applique pour le déplacement en ondulant, en rampant, en flottant, ou en glisant. Courrir peut cependant infliger un modificateur de -10 aux autres actions qui peuvent être affecté par votre mouvement de course. Le sprint est, plus que tout, une course tout aziumts et recquière donc uen Action Complexe (voir Sprinter, p. 191). 

À la discrétion du maître de jeu, d'autres mouvements peuvent peuvent nécessiter une Action Complexe. Franchir une clotûre, sauter à la perche, sauter d'un aplomb, nager ou traverser un habitat en zéro-g avec le parkour nécessitent un peu de finesse et d'attention, et devraient compter comme une Action Complese et devraient recevoir les mêmes modificateur que la course simple. Voler compte générallement comme une Action Rapide, les manœuvres plus délicates nécessitant une Action Complexe. 



\subsection{Allure} \label{sec:movement-rates} 

Parfois il est important de savoir non seulement comment un personnage se déplace, mais à quelle vitesse. Pour la plupart de la transhumanité, cette allure est la mrme: 4 mètres par Tour d'Action en marchant, 20 en courant. Pour déterminer quelle distance un personnage peut franchir en une Phase d'Action, divisés cette allure par le nombre total de Phase d'Actions dans ce tour. Dans un tour avec 4 Phases d'Actions, cela se divise en 1 mètre en marchant par Phase d'Action, 5 mètres en courant. 

Les déplacement en nageant ou en rampant partnet sur une base d'1 mètre par Tour d'Action, soit 0,25 mètres par Phase d'Action. Vous pouvez aussi Sprinter pour augmenter votre allure (voir Sprinter). Les véhicules, robots, créatures et morphs inhabituelles peuvent avoir des allures particulières, listées sous la forme vitesse de marche/de course en mètres par tour. 

Ces allures sont valables sur la gravité standard Terrestre. Si vous vous déplacez en environnement à faible gravité, en microgravité ou en haut-egravité, les choses changent. Voisr \emph{ Gravité}, p. 198. 



\subsubsection{Sauter} 

Les personnages effectuant un saut avec élan peuvent franchir SOM $\div$ 5 (arrondissez au suéprieur); mètres utilisez SOM $\div$ 20 (arronissez au supérieur) mètre pour les sauts sans élan. La hauteur de saut verticales est de 1 mètre. Les personnages faisant un test de Parkour peuvent augmenter la distance de saut de 1 mètre (saut avec élan) ou 0,25 mètres (saust sans élan/saut verticaux) par tranche de 10 points de MdR. 

\subsubsection{Sprinter} 

Vous pouvez utiliser Parkour pour augmenter votre distance de déplacement lors d'une Phase d'Action. Vous devez dépenser une Action Complexe pour sprinter et effectuer un test de Parkour. Chaque tranceh de 10 points de MdR augmente votre course de distance de cette Phase d'Action d'1 mètre, jusqu'à un maximum de +5 mètres. 



\section{Combat} \label{sec:combat} 

Parfois, la discussion échoue, et c'est le moment où les lames et les déchiqueteuse entrent en jeu. Tous les combats dans Eclipse Phase respectent la même mécanique de base, qu'ils soient gérés avec des griffes, des poings, des armes, des armes à feu ou du psi: c'estun Test en Oppoistion entre l'attaquant et le(s) défenseur(s). 

\subsection{Résoudre le combat} 

Utilisez la séquence d'étape suivante pour déterminer l'issue d'une attaque. 

\subsubsection{Étape 1: Déclarer une attaque} 

l'attaquant initie son action en prenant une Action Complexe pour attaquer lors de sa Phase d'Action. La compéternce utilisée dépend de la méthode utilisée pour attaquer. Si le personnage ne possède pas la compétence de Combat appropriée, il peut défausser sur l'aptitude liée. 

\subsubsection{Étape 2: Déclarer la défense} 

Une fois l'attaque déclarée, le défenseur choisit comment répondre. Se défendre est toujours considéré comme une Action AUtomatique sauf si le défenseur est surpis (voir Surprise, p. 204) ou rendu incapable de se défendre d'une manière ou d'une autre. 

\textbf{Contact:} Un personnage se défendant contre une attaque au contact utilsie sa compétence Esquive, représentant le fait d'esquiver les coups (si le personnage n'as pas cette compétence, il peut se défausser sur Réflexes). Le personnage peut aussi utiliser une compétence de combat au contact pour se défendre, représentant les blocages et les parades au lieu de l'esquive. 

\textbf{À distance:} Contre les attaques à distance, le personnage se défendant ne peut utiliser que la moitié de sa compétence Esquive (arrondie à l'inférieur). 

\textbf{Défense Totale:} Les personnages ayant utilisé une Action Complexe pour se mettre en défense totale (p. 198) reçoivent un modificateur de +30 à leur jet de défense. 

\textbf{Psi:} Un personange se défendant contre une attaque psi lance VOL $\times$ 2 (p. 222). UNe sorte de défense mentale totale peut également être utilisée contre les attaques psi. 

\subsubsection{Étape 3: Appliquer les modificateurs} 

Tout modificateur approprié est maintenant appliqué à la compétence de l'attaquant et du défenseur. Voir la table des Modificateurs de Combat (p. 193) pour des modificateurs situationnels communs. 

\subsubsection{Étape 4: Faire le test en opposition} 

L'attaquant et le défenseur lancent tous les deux 1d100 et comparent le résultat à leur seuil modifié. 

\subsubsection{Étape 5: Déterminez l'issue} 

Si l'attaquant réussit et que le défenseur échoue, l'attaque touche. Si lattaquant échoue, l'attaque rate complètement sa cible. 

Si l'attaquant et le défenseur réussissent tous les deux, comaprez leurs résultats. Si l'attaquant obtiens un résultat plsu élevé, l'attaque touche en dépit d'une défense inspirée; sinon l'attaque rate. 

\textbf{Réussite Execeptionnelle:} Si l'attaquant obtient un Succès Exceptionnel (MdR de 30+), une frappe solide est infligée. Augmentez la Valeur de Dommage (VD) infligée de +5. Si la MdR est de 6°+, augmentez la VD de +10. 

\textbf{Critique:} Si l'attaquant obtient un succès critique, l'attaque surclasse l'armure, signifiant que l'armure du défensuer est complètement ignorée - un défaut ou une faille a été exploitée, permettant à l'attaque de la traverser complètement. 

Si le défenseur obtient un succès critique, il esquive avec brio, atteint un couvert qui le protège des attaques à venir, manœuvre vers une position supérieure ou obtient un bénéfice quelconque. 

\subsubsection{Étape 6: Modifiez l'armure} 

Si la cible est touchée, son armure va l'aider à la protéger contre l'attaque (sauf si l'attaquant a obtenu un critique, voir au-dessus). Déterminez quelle type d'armure est approprié pour se défendre contre cette attaque (voir Armure, p. 194). La valeur de pénétration d'Armure (PA) de l'attaque réduit la valeur de l'armure représentant la capacité de l'arme à transpercer les mesures de protection. 

\subsubsection{Étape 7: Déterminez les dégâts} 

Chaque arme et tout type d'attaque possède une Valeur de Dégats (VD, voir p. 207). Ce total est réduit par la valeur de l'armure réduite par la PA de l'attaque. Si les dommages sont réduits à 0 ou moins, l'armure est efficace et l'attaque ne aprvient pas à blesser la cible. Sinon, les dommages supplémentaires sont appliquées au défenseur. Si les dégâts accumulés dépasse la Solidité du défenseur, ils sont rendus incapables de bouger et peuvent mourir (voir Solidité et Santé, p. 207). 

Notez que certaines attaques psi infligent du stress mentak au lieu de dommage physique (voir Santé Mentale, p. 209). Dans ce cas, la Valeur de Stress (VS) et gérée de la même manière que la VD. 

\subsubsection{Étape 8: Déterminez les blessures} 

Les dommages infligés par une seule attaque sont ensuite compérés au Seuild e Blessure de la victime. Si la VD modifée par l'armure égale ou dépasse le Seuil de Blessure, le personnage reçoit une blessure. Des blessurs multiples peuvent être appliquée avec uen seule attaque si la VD modifiée est un multiple du Seuil de Blessure. Les blessures représentent des blessures plus sérieuses et infligent des modificateurs et d'autres effets au personnage (voir Blessures, p. 207). 

\begin{quotation} Stoya essayes de se sortir rapidement de la station, mais l'assassin du  Night Cartel l'a repérée et l'a surprise dans la zone en microgravité de l'habitat. L'INIT de l'assassin est de 63, plus un jet de 23, pour une Initiative de 86. L'INIT de Stoya est de 55, plus un jet de 27, pour une Initiative de 82. 

L'assassin commence, dépensant une Action Rapide pour dégainer un déchiquetteur. Cette arme à fléchette est une arme à tir en rafale, avec une Action Complexe l'assassin peut donc tirer deux fois. Sa compétence d'Armes à Spray est de 65, il a une interface d'arme (+10), et ils sont à portée courte (+0), il a donc besoin de 75 ou moins. Stoya se défend avec sa compétence Esquive (60) divisée par 2, soit 30. 

L'assassin obtien un 08 avec son premier tir. Étonnament, Stoya obtient un 28. Ils réussissent tous les deux, mais Stoya a obtenu le meilleur jet, elle esquive donc le premier tir. 

l'assassin obtient un 20 pour son deuxième tir, un autre succès, et cette fois Stoya obtient un 83, un échec. L'assassin a également obtenu une Réussite Exceptionnelle avec une MdR de 55, augmentant sa VD de +5. 

Les dégats de base de l'assassin sont de 2d10 + 5, mais il tir en rafale contre uen seule cible pour +1d10 et c'est également une arme à zone d'effet conique à courte distance pour un +&d10 additionel, pour une VD totale de 4d10 + 5. L'assassin lance 4d10 et obtient 16, il y ajoute ensuite les +5 pour une VD totale de 21. 

Stoya porte une armure corporelle légère (VA 10/10) mais la pénétration d'Armure du déchiqueteur est de -10, son armure est donc entièrement annulée. Elle encaisse donc un dévastateur score de 21 de VD, dépassant son Seuil de Blessure de 10. Deux fois. Cela signifie que Stoya souffre de 2 blessures suite au tir, subissant un -20 à toutes ses actions. De plus, elle doit fair deux Tests de SOM $\times$ 3, l'un pour éviter d'être jettée au sol et l'autre pour éviter de sombrer dans l'incosncience. Sa SOM est de 30, signifiant qu'elle a besoin d'un 70 (30 $\times$ 3 = 90, 90 $-$ 20 de modificateurs de blessure = 70) sur ses deux jets. Elle obtient 40 et 27 les réussissant tous les deux. 

C'est maintenant à Stoya de joueur. Elle prend un Action Rapide pour attraper son arme: un assomeur. Sa compétence d'Armes à Rayon est de 47, modifiée par ses vlessure s(-20) et une interface d'arme (+10) pour 37. La compétence d'Esquive de l'assassin est de 48, divisée par 2 pour un 24 contre les attaques à distances. Stoya obtient un 22 - un succès critique - et l'assassin obtiens un 68. L'assomeur n'inflige qu'1d10 $\div$ 2 de VD, mais puisque l'attaque est un succès critique, elle surclasse l'armure. Stoya obtient un 8, pour 4 point de VD, en-dessous du Seuil de Blessure de l'assasin de 7. 

Les assomeurs sont cependat des armes à choc, l'assassin doit donc fair un Test de SOL + Armure Énergétique. Sa SOL est de 35 et il porte une veste blindée (VA 6/6), son seuil est donc de 41. Il obtient un 71 - Une MdE de 30, signifiant qu'il est immédiatement incapable d'agir pour 3 Tours d'Action. 

Ayant vaincu son opposant, Stoya en profite pour s'échapper rapidement. \end{quotation} 



\subsection{Résumé de combat} 

\begin{quotation} \textbf{Stats d'ego} \begin{itemize} \item L'attaquant lance sa compétence d'attaque +/- les modificateurs. \item Contact: Le défenseur lance Esquive ou sa compétence de combat au contact +/- les modificateurs. \item A distance: Le défenseur lance (Esquive $\div$ 2, arrondi à l'inférieur) +/- les modificateurs. \item Si l'attaquant réussit et obtient un meilleur score que le défenseur, l'atatque touche. \item Les résultats critique surclassent l'armure (l'armure n'est pas prise en compte). \item L'armure est réduite par la valeur de Pénétration d'Armure (PA) de l'attaque. \item Les dégats de l'arme sont réduit par la valeur d'Armure modifiée de la cible (sauf si l'attaque surclasse l'armure). \item Si les dégats dépassent le Seuil de Blessure de la cible, une blessure est également notée. (Si les dégats dépassent le Seuil de Blessure de plusieurs facteurs, de multipls blessures sont infligées.) \end{itemize} \end{quotation} 

\begin{table} \begin{tabularx}{\textwidth}{|X|l|} \hline

\multicolumn{2}{|c|}{\textbf{Modificateurs de Combat}} \\ \hline

\textbf{Situation} &\textbf{Modificateur}	\\ \hline

Le personnage utilsie sa mauvaise main	&-20	\\ \hline

Le personnage ets blessé/traumatisé	&-10 par blessure/trauma	\\ \hline

Le personnage a une position supérieure	&+20	\\ \hline

Attaque de toucher	&+20	\\ \hline

Tir visé	&-10	\\ \hline

Le personnage manie une armes à deux mains d'une seule main &-20	\\ \hline

Petite cible (taille d'un enfant)	&-10	\\ \hline

Trés petite cible 'souris ou insecte)	&-30	\\ \hline

Grosse cible (taille d'une voiture) &+10	\\ \hline

Trés grosse cible (taille d'une grange) &+30	\\ \hline

Visibilité génée (mineur: reflet, fumée légère, faible lumière) &-10	\\ \hline

Visibilité génée (majeur: fumée épaisse, obscurité) &-20	\\ \hline

Attaque aveugle &-30	\\ \hline

\textbf{Combat au contact} &\textbf{Modificateur}	\\ \hline

Le personnage a une meilleure allonge &+10	\\ \hline

Le personnage charge &-10	\\ \hline

Le personnage réceptionne une charge &+20	\\ \hline

\textbf{Combat à distance (attaquant)} &\textbf{Modificateur}	\\ \hline

L'attaquant utilise une interface d'arme ou un viseur laser	&+10	\\ \hline

L'attaquant est derrière un couvert &-10	\\ \hline

L'attaquant court &-20	\\ \hline

L'attaquant est en combat au contact &-30	\\ \hline

Le défenseur a un couvert mineur &-10	\\ \hline

Le défenseur a un couvert modéré &-20	\\ \hline

Le défenseur a un couvert majeur &-30	\\ \hline

Le défenseur est à plat ventre et loin (10+ mètres) &-10	\\ \hline

Le défenseur est caché &-60	\\ \hline

Tir ajusté (action rapide) &+10	\\ \hline

Tir ajusté (actoin complexe) &+30	\\ \hline

Tir de balayage avec une arme à rayon &+10 sur le deuxième tir	\\ \hline

Cibles multiples en une seule Phase d'Action &-20 par cible additionnelle \\ \hline

Tir indirect &-30	\\ \hline

Tire à bout portant (2 mètres ou moins) &+10	\\ \hline

Portée courte &--	\\ \hline

Portée moyenne &-10	\\ \hline

Portée longue &-20	\\ \hline

Portée extrême &-30	\\ \hline

\end{tabularx} \label{tab:combat-modifiers} \end{table} 



\section{Complication d'action et de combat} \label{sec:action-combat-comp} 

Le combat n'est pas aussi simple que de décider qi vous touchez ou ratez votre cible. Les armes, armures, munitions et de nombreux autres facteurs peuvent influer l'issue d'une attaque. De même, de nombreux facteurs peuvent avoir un impact sur une scène d'action, telles que le feu ou l'effet de la microgravité. 



\subsection{Tirs ajustés} \label{sec:aimed-shots} 

Comme noté au paragraphe AJuster, p. 190, un personnage peut sacrifier leur Action Rapide pour se concentrer sur la visée d'une attaque à distance et recevoir un modificateur de +10 à l'attaque. Vous pouvez également sacrifier une Action Complexe complète pour verrouiller votre mire sur la cible. Dans ce cas, et tant que la cible reste en vue jusqu'à la prochaine Phase d'Action, vous recevez un modificateur de +30 pour toucher. 



\subsection{Munitions et recharger} \label{sec:ammunition-reloading} 

Chaque armes possède une capacité de munitions qui indique combien de tir l'arme peut effectuer. Lorsque ces munitions sont épuisés, une nouveau stock doit être chargé. Les jouerus devraient garder une trace des tirs qu'ils effectuent. 

Recharger nécessite presque toujours une Action Complexe, que vous eneclenchier un nouveau chargeur de cartouches ou une batterie neuve pour laser. A la discrétion du maître de jeu, une recharge qui est immédiatement accessible (tel qu'un nouveau chargeur scotché à l'envers au chargeur enclenché, afin que la recharge n'écessite juste de retourner le chargeur et d'enclencher le plein) peut ne nécessiter qu'une Action Rapide. Les armes archaîques telles que les fusil à magasins peuvent nécessiter plus de temps pour être complètement rechargés. 



\subsection{Armes à aire d'effet} \label{sec:area-effect-weapons} 

Certaines attaques à distances sont conçue pour affecter plus d'une cible à la fois. Ces armes se répartissent en trois catégories: souffle, souffle réparti et cône 

\subsubsection{Effet de souffle} 

Les armes à effet de souffle telles que les grenades, les mines et les autres explosifs qui s'étendent autur d'un point de déflagration central. La plupart des attaques à souffle se développent en sphère, bien que certaines charge peuvent diriger une explosion dans une seule direction. La force d'explosion ets plus forte près de l'épicentre et s'affaiblit près des bors de la sphère. Poru chaque mètre de distance entre une cible et le centre, réduisez les dommages d'une armes à souffle de -2. 

\subsubsection{Souffle uniforme} 

Les attaques à souffle uniforme distibuent leur puissance de manière équitable sur toute la surface de l'effet. Cela inclut par exemple les bombes à vide et les arems thermobariques qui dispersent une mixture explosive dans un nuage de vapeur puis le fait détonner en une fois. Toutes els cibles dans le rayon d'explosion souffrent des même dommages. Les dommages contre les cibles hors de la sphère principale d'explosion sont réduits de -2 par mètres. 

\subsubsection{Cône} 

Les armes à effet de cône ont une aire d'effet qui commence au bout de l'arme et s'étend vers l'extérieur en un cône. Â courte portée, cette attaque affecte 1 cible; a portée moyenne elle en affecte 2 distante de moins d'un mètre; et à portée longue ou extrême elle affecte 3 cible ayant un écart maximal d'un mètre entre deux d'entre elles. Les attaque à effet de cône font +1d10 de dégat à courte portée et -1d10 à portée longue et extrème. 



\label{sec:combat-armor} 

Les armures ont rpogressé au même rythme technologique que les armes, permettant d'atteindre des niveaux de protectiosn sans précédents. Comme décrit à l'Étape 7: Déterminez les Dégats (voir p. 192), la valeur d'armure réduit les points de dégats d'une attaque. Pour une liste complète des types d'armures et de leur valeur, voir p. 311. 

\subsubsection{Énergétique contre Cinétique} 

Chaque type d'armure à une Valeur d'Armure (VA) avec deux valeurs - Énergétique et Cinétique - représentant la protection qu'elle fournit contre chaque type d'attaque. Elles sont listées sous la forme "Armure Énergétique/Armure Cinétique." Par exemple, un objet ayant une valeur d'armure "5/10" fournit 5 points d'armure contre les attaques éngertique et 10 points contre les attaques cinétiques. 

Les dégats énergétiques incluent ceux qui sont causés par les armes à rayons (laser, micro-onde, faisceau de particule, plasma, etc) ainsi que le feu et les explosifs à haute énergie. Les armures protégeant contre ce type de dégats sont fait de matériaux qui reflètent ou diffusent une telle énergie, dissipent et transfèrent la chaleur et les matériaux ablatifs. 

Les dégats cinétiques sont le transfert d'une énergie dévastatrice lorsqu'un objet en mouvement (tel qu'un point, un couteau, une massue ou une balle par exemple) entre en collision avec un autre objet(la cible). La pluaprt des armes de mélée et des armes à feu infligent des dégats cinétqiues, comme le font les rochers, les pendules ou les fragments propulsés par une explosion. Les armures cinétiques incluent des plaques résistante à l'impact, des fluides réoépaississant et des gels qui se durcissent à l'impact, ainsi que les fibres ballistique et renforcées. 

\subsubsection{Pénétration d'armure} 

Certaines armes disposent d'une valeur de Pénétration d'Armure (PA). Cette valeur représente la capacité d'une attaque à percer les différentes couches protectrices. La valeur d'AP réduit la valeur de l'armure utilisée pour se défendre contre l'attaque (voir Étape 6: Modifier l'Armure, p. 192). 

\subsubsection{Armures superposée} 

Si deux tyeps d'armure ou plus sont portées, les valeurs d'armures sont additionnées les unes aux autres. Cependant, porter de multiples couches d'armure est peu pratique et encombrant. Appliquez un modificateur de -20 aux actions du personnage pour chaque couche d'armure additionnelle portée. 

Notez que les objest spécifiquement notés comme étant des accessoires d'armures - casques, boucliers, etc - n'infligent pas de la pénalité d'armure superposés, ils ajoutent simplement leur bonus d'armur au total. Notez également que l'armure inhérente à une synthmorph ou à la structure d'un bot ne constituent pas une couche d'armure (càd que vous pouvez porter une armure sur une coque synthétique sans pénalité). 



\subsection{Asphyxie} \label{sec:asphyxiation} 

Le transshumain moyen peut retenir sa repsiration pour deux minutes avant de s'évanouir. Une activité fatiguante réduit dce total de temps. Pour chaque intervalle de 30 seconde après la première minute où une biomorph est empéchée de respirer, elle doit faire un Test de SOL. Appliquez un modificateur cumulatif de -10 à chaque fois que ce test est effectué. Si le personnage échoue au test, il tombe immédiatement inconscient et commence à souffrir des dommages de l'asphycie, au rythme de 10 oints par minutes, jusqu'à ce qu'il décède ou qu'il puisse respirer de nouveau. Ces dégats ne causent pas de blessure. 

L'asphyxie est un processus horrible, menant souvent à la panique. Les personnages qui ont été asphyxié doivent faire un Test de VOL $\times$ 3. Si ils échouent, ils souffrent de 1d10 $\div$ 2 (arrondissez au supérieur) point de stress mental et ne peuvent effectuer aucune action pour s'aider ce Tour d'Action. Un personnage qui réussit peuvent essayer de s'aider, et ils doivent en fait réussir un Test de VOL $\times$ pour effectuer toute action non directement reliée à leur survie immédiate (attaquer un autre personnage, une créature ou un objet maintenant le personnage sous l'eau sont exemptées de cette règle). 



\subsection{Armes à Rayons} \label{sec:combat-beam-weapons} 

Les arems à rayons sont plus facile à "pointer" sur la cible, en raison du rayon d'énergie continu qu'elles émettent en lieu et place des porjectiles. Cela signifie que l'une des deux règles suivantes peut-être utilisée lorsque vous attaquez avec une arme à rayons. Ces options ne sont pas disponibles pour les armes à "impulsions" (comme les pluseurs lasers), car de telels armes émettent des impulsions énergétique au lieu d'un faisceau continu. 

\subsubsection{Tir de balayage} 

Un attaquant effectuant deux attaques semi-automatiques (p. 198) avec une arme à rayon dans la même Action Complexe et qui rate sa première attaque peu traiter cette attaque comme une Action Rapide AJuster (p. 190), recevant un modificateur de +10 sur sa deuxième attaque. En d'autres terme, même si la première attaque rate sa cible, le personnage profite de l'opportunité pour balayer le fasiccau pour le rapprocher de la cible lors de la deuxième atatque. Cela ne s'applique que lorsque les deux attaques sont effectéues contre la même cible. 

\subsubsection{Tir concentré} 

Un personnage utilisant une arme à rayon semi-automatqiue et qui touche avec sa première atatque peut choisir de maintenir le rayon et de concentrer le faisceau, cuisant la cible. Dans ce cas, si le personnage abandonne sa deuxième attaque semi-atutomatique de cette Action Complexe, mais augmente automatique la VD de sa première atatque by $\times$ 1,5 (arrondissez au supérieur). Cette décision doit être prise avant que les dés de dégâts ne soient lancés. 



\subsection{Attaques aveugle} \label{sec:blind-attacks} 

Attaquer une cible que vous ne pouvez aps voir est, au mieux, difficile et est une questiond e chance dans le pire des cas. Si vous ne pouvez pas voir, vous devez faire un test de Perception en utilisant un autre sens disponible pour détecter votre cible. Si ce test est un succès, vous subissez un modificateur de -30 à votre attaque. Si votre Test de Perception échoue, votre ataque est essentiellement basée sur la chance - le seuil de votre attaque est égal à votre stat Moxie (sans aucun autre modificateur). 

\subsubsection{Tir indirect} 

Avec l'aide d'un désignateur, vous pouvez cibler un ennemi que vous ne pouvez pas voir en utilisant le tir indirect. Dans ce cas vous devez être meshé avec un personnage, un bot ou un système de cpateur qui a la cible en vue et qui vous pousse les donéns de ciblage (le maître de jeu peu demander un Test de Perception pour le désignateur). Les attaques indirectes souffrent d'un modificateur de -30. 

Les missiles à tête chercheuse (p. 340) peuvent atterir sur une cible qui est "éclairée" par l'énergie réfélchie d'un viseur laser (p. 342) ou d'un système de désignation de cible similaire. Une "atatque" doit d'abord être effectuée pour élcairer la cible avec le viseur laser en utilisant une compétence appropriée. Si l'attaque réussit, cela annule le modificateur de -30 du tir indirect pour le test d'attaque du lanceur. La ible doit rester en vue du désignateur (nécessitant une Action Complexe à chaque Phase d'Action) jusqu'à ce que le chercheur touche. 



\subsection{Bots, synthmorphs et véhciules} \label{sec:bots-synthmorphs-vehicles} 

Les robots manœuvrés par des IA et les morphs syntéhtique sont des choses commune à Eclipse Phase. Les robots sont utilisés pour une grande variété de tâches, de la surveillance, la maintenance et les services à la sécurité et à ploce - et peuvent donc souvent jouer un rôle dans les scènes d'actions et de combats. Bien que moins courants (au moins dans certains habitats), les véhicuels pilotés apr des IA sont également fréquemment utilisés et rencontrés. 

Notez que la différence entre un bots, un véhicule et une synthmorph est essentiellement sémantique. Les robots sont simplements des corps synthétique contrôlés par une IA. Les véhicules sont également des robots - contrôlé par IA - mais le terme "vhciule" dénote leur capacité à transporter des passagers. Les bots et les véhciules peuvent être utilisés comme morph synthétique - c'est à dire, habitées par un ego transhumain - du moment qu'ils sont équipés d'un cybercerveau (p. 300). Dans le cadre de ces règle, le mot "coque" est tilisé pour faire référence aux bots, aux véhicule et aux synthmorphs. 

Comme les synthmorphs, le sbot ste les véchiules sont traités comme tous les autres personnages: ils lancent leur Initiative, effectuent des actoins et utilisent des compétences. Quelques aspects de ces coques nécessitent cependant des considérations particulière qui sont couvertes ci-dessous. 

\subsubsection{Stats des coques} 

Tout comme les personnages en synthmorph, certaines stats des bots et de véhicules (Solidité, Seuil de Blessure, etc) et modificateur de stats (Initiative, Vitesse, etc) sont en fait déterminées par la coque physique. D'autre stats sont déterminés par l'IA contrlant le bot/véhciule (en lieu et place d'un ego). Les bots et les véhciules peuvent également avoir des traits qui s'appliquent à leur IA ou à leur coque physique. Pour des exemples de bto sou de véhicules, voir p. 342 du chapitre Équipement. 

\textbf{Maniabilité:} Le sbots et les véhicules ont une stat particulière appellée Maniabilité, qui est un modifictauer appliqué à tous les tests fait pour piloter le bot/véhicule. Ell représente la manœuvrabilité du bot/véhicule. 

\subsubsection{Compétences des coques} 

Les compétences et aptitudes utilisés par un bot ou un véhicule sont ceux possédés par son IA. Voir \emph{IA et Muses }, p. 264. 

\subsubsection{Déplacement des coques} 

Comme les personnages, les bots et les véhicules ont un score de Mouvement pour la marche et pour la course. Ils sont utilisés lorsque le bot/véhicule est impliqué dans une scène d'action ou de combat avec d'autres personnages. 

Les coques qui sont capables d'atteindre des vitesses plus élevées ont aussi une Vitesse Maximale - c'est la vitesse la plus élevé à laquelle la coque peut se déplacer de manière sûre, indiquée en kilomètre par heure. À la discrétion du maître de jeu, certaines coque peuvent dépasser leur limite et accélérer au-delà, masi avec une prise de risqaue significative - le mæître de jeu devrait appliquer des pénalités appropriées aux test de Piloter et aux autres tests. 

\subsubsection{Poursuites} 

Les coques qui se déplacent plus vite que leur score de course de Mouvement (jusqu'à leur Vitesse Max. ) sont gnéralement considérés comem se déplaçant trop vite pour intergair de manière normale avec les autres personnages lors de l'action ou du combat. C'est le moment ou la scène passe dans le mode "scène de poursuite" - un mouvement narratif composé de choix de maœuvre et de test avec différentes issues. Qu'une poursuite soit réellement en cours ou non, le maître de jeu devrait garder en tête que la Vitesse Maximale n'est pas le seul facteur à prendre en compte pour les situatiosn à haute-vitesse. Les facteur senvironneemntaux comme le terrain, les conditions météorologiques, la navigation, les piétons et le traffic peut fournir des obstacles que les coques doivent franchir. Une coque devant taverser un habitat pour atteindre une bombe avant qu'elle n'explose devra prendre plusieurs décisions et faire quelques tests qui détermineront si elle arrive à temps ou non. De manière similaire, une coque cherchant à semer des poursuivants devra effectuer quelques manœuvres imaginative et espérer trouver un raccourci ou deux pour semer ses opposants. 

\subsubsection{S'écraser} 

Les coques qui subissent des blessures en combat ou en poursuite pourraient être forcées de faire un Test de Piloter pour éviter de s'écraser ou s'écraser automatiquement. Les circonstances exacte d'un crash sont laissées à l'appréciation du maître de jeu en fonction de ce qui colle le mieux à l'histoire - la coque pourrait simplement s'arréter en glissant, percuter un arbre, tomber du ciel ou faire un tonneau et atterir sur un groupe de passants. Les coques qui percutent d'autres objets lorsqu'elles s'écrasent prennent générallement des dégats supplémenataire suite à la collision (voir Collisions) 

\subsubsection{Collisions} 

Si une coque s'écrase dans ou percute volontairement une personne ou un objet, quelqu'un sera blessé. Pour déterminer la quantité de VD infligée, lancez 1d10 et ajoutez-y la SOL de la coque divisée par 10 (arrondit au supérieur). Cela représente les dommages infligés à la vitesse de marche. Si la coque se déplaçait à la vitesse de coruse, multipliez la VD par 2. Si la coque se déplaçait à une vitesse de poursuite, multipliez la VD par la vitesse de la coque $\div$ 10 en mètres par tour. La coque et tout ce qu'elle percute subissent ces dégats, du moment que la collision a lieu vaec quelques choses équivalent en densité et en dureté. Les objets mous et déformable comem les biomoprhs feront moins de dégats à la coque (sauf si elles se trouvent dans une tenue rigide ou en armure de bataille), auquel cas la coque ne subira que la moitié des dommages suite à la collision. Les armures Cinétiques sont utilisées pour résister à la VD d'un crash. 

Si deux coques se percutent en face à face, calculez les dégats infligés par chacune d'elle et infligez les au deux. Si deux coques qui se déplacent dans la même sens, se percutent ne prenez en compte que la différence de vitesse. 

Les passagers dans les véhicules peuvent aussi être endommagés apr les collisions si ils n'utilisent pas les mesures de sécurité correctes. Ils ne subissent que la moitié de la VD subie par leur véhciule (réduite de leur propre armure Cinétique). 

\begin{table} \begin{tabular}{|l|l|} \hline

\multicolumn{2}{|c|}{\textbf{Dégats de Collision}}	\\ \hline

VD de Collision de Base	&1d10 + (SOL $\div$ 10)	\\ \hline

Course	&VD $\times$ 2	\\ \hline

Vitesses de Poursuites	&VD $\times$ (vitesse $\div$ 10)	\\ \hline

\end{tabular} \label{tab:collision-damage} \end{table} 

\subsubsection{Attaque des passagers des véhicules} 

Pendant le combat, les passagers à l'intérieur d'un véhciule peuvent être ciblés séparément du véhicule. Les attaques effectués contre les passagers de cette manière n'infligent aucun dégât au véhicule (à l'exception des armes à aire d'effet). Les passagers ciblés bénéficient à la fois d'un couvert (généralemennt Majeur, -30) et de la structure du véhciule, ajoutant l'Armure du Véhciule à la leur. 

Les passagers à l'intérieur d'un véhciule ne sont généralement pas blessés par les attaques effectuées contre le véhciule. Les armes à aire d'effet sont une exception à cette règle, mais dans ce cas les apssagers bénéficient aussi de toute l'Armure du Véhicule. 

\subsubsection{Coque contrôllées à distance} 

Toute coque (ou biomorph) avect l'option marionnette (inclut dans tous els cybercerveaux) peut être contrôllée à distance, soit par un personnage soit par une IA distante. Cela nécessite un lien de communication entre le téléopérateur et la coque (le "drône"). Le téélopérateur contrôlle le drône grâce à une interface entoptique, et reçoit les retour sensorielles et les autres données grâce à l'insert de mesh du drône. 

Lorsque soumis à un contrôlle direct, l'IA (ou l'égo résident) de la coque est surchargé et mis en veille. Le drône n'agît que selon les instructiosn reçues. Chauqe instruction coûte une Action Rapide. Dance cette situation, le drône agît avec la même Initiative que le téléopérateur. Les compétences et stats du téléopérateur ont utilisés en lieu et place de ceux de l'IA de la coque. En raison de la nature des opérations distantes tous les tests sont cependant fait avec un modificateur de -10. Plusieurs drônes peuvent être contrôllés en même temps, mais leur donner des ordres nécessite une Action Rapide distincte à moins qu'ils ne reçoivent la même commande. Le contrôle direct par téélopération 'est pas faisable sur des distances extrêmes, en raison de l'impact du décalage temporel sur les communications. 

Le téléopérateur peut également mettre led drœne en mode autonome, permettant à l'IA de le coque de reprendre sona ctivité normale. Le drône continue de suivre les ordres du téléopérateurs dans la mesure de ses capacités. Dans ce mode, le drœne fonctionne normalement, utilisant sa propre initiative et les compétences et stats de l'IA. 

\subsubsection{Interception de  coque} 

L'"interception" est le terme familier pour une forme plus directe de contrôle distant, en utilisant les technologies de RV et d'XP. Pendant l'interception, le port marionnette du drône alimente directement l'insert de mesh du téléopérateur en données sensorielles. Le téléopérateur se surcharge au sensorium du drône, "devenant" le drône. Dans ce cas, le téléopérateur abandonne le contrôle de leur propre morph, qui devient inerte. Pendant qu'il intercepte, il subit un modificateur de -60 sur tous les Test de Perception ou sur toute tentative d'agir avec sa morph. 

Intercepter nécessite une Action Complexe pour se connecter, et une autre pour se déconnecter. Un téléopérateur intercepteur contrôle un drône comme si il était dans sa propre morph. Comme pour la téléopération en contrôle direct, les compétences et Initiative de l'intercepteur sont utilisées en lieu et place de l'IA du drône. L'intercepteur ne subit atcun modificateur de téléopération, mais un seul drône peut être intercepté à la fois. 

Si le drône est tué ou détruit, l'intercepteur est immédiatement éjecté de sa connexion, reprenant le contrôle de sa propre morph comme d'habitude. Être éjecté de cette manière est extrêmement perturbant, au moins parce que l'intercepteur expérimente le fait d'être tué/détruit. L'intercepteur subit donc 1d10 points de stress mental. 



\subsection{Tirs visés} \label{sec:called-shots} 

Parfois il n'est pas suffisant de simplement toucher sa cible - vous devez tirer à travers une fenêtre, désarmer un adversaire ou tirer dans ce trou dans l'armure de votre cible. Vous devez déclarer que vous faites un tir visé avant de commencer votre attaque, en choisissant l'une des possibilités notées ci-dessous. Les tirs visés subissent un modificateur de -10 et nécessitent une réussite Exceptionnelle (MdR 30+). Si vous dépassez cette marge, vous réussissez votre tir visé et les résultats notés ci-sessous s'appliquent. Si vous réussissez mais que vous ne dépassez aps la marge reuise, vous touchez votre cible de manière habituelle. 

\subsubsection{Contourner l'armure} 

Les tirs visés peuvent être utilisés pour cibler un trou ou une faiblesse dans l'armure de votre adversaire. Si vous dépasez la MdR, vous obtenez un tir surclassant l'armure, et son armure ne s'applique pas. Notez que dans certaines circonstances, un maître de jeu peut décider que l'armure d'una dversaire n'a tout silmplement pas de point faible ou de zone non protégées, interdisant de tels tir visés. 

\subsubsection{Désarmer} 

Vous pouvez effectuer un tir visé pour tenter de faire tomber une arme des mains de votre adversaire. Si vous battez la MdR, la victime subit la moitié des dégats de l'attaque (réduit par l'armure comme d'habitude) et doit ensuite réussir un Test de SOM $\times$ 3 avec un modificateur de -30 pour conserver sa prise sur l'arme. 

\subsubsection{Ciblage spécifique} 

Vous pouvezeffectuer un tir visé dans le but de toucher un endroit spécifique ou un composant de votre cible - par exemple pour désactiver les senseurs d'un bot, balayer la jambe de quelqu'un ou taper dans l'œil d'une personne. Si vous battez la MdR, vous touchez le point spécifiquement visé. Le maîþre de jeu détermine le résultat de manière appropriée à l'attaque et à la cible - le composant peut-être déteuit, l'adversaire peut tomber ou temporairement aveuglé, et ainsi de suite. 

\subsection{Charger} \label{sec:charging} 

Un adversaire qui court puis attaque un adversaire au corps à corps dans la même phase d'Action est considéré comme chargeant. Un attaquant charegant souffre toujours du modificateur de -10 pour la course, mais ils reçoivent un bonus de dégats grâce au moment cinétique engendré: augmentez les dommages infligés de +1d10. 

\subsubsection{Réceptionner une charge} 

Vous pouvez retarder votre action (voir p. 189) afin de réceptionner une charge, vous préparant à l'impact, interrompant l'action et frappant juste avant que l'attaquant ne le fasse. Dans cette situation, vous recevez un modificateur de +20 pour frapper l'adevrsaire qui vous charge. 



\subsection{Démolitions} \label{sec:demolitions} 

L'utilisation la plus commune de la compétence Démolitions est le placemen t, le désarmement et la fabricatoin d'appareils explosifs, telels que les charges de superthermites (p. 330) ou les grenades (p. 340). 

\subsubsection{Placer des explosifs} 

Un démolisseur compétent peut placer des charges d'une manière qui va aigmnter leurs effets. Il peut identifier les vulnérabilité structurelle et les poinst faibles et concentrer une explosion sur ces zones. Il peut déterminer comment faire exploser un coffre sans end étruire le contenu. Il peut concentrer la force de l'explosion dans une direction aprticulière, augmentant la force dirigée tout en réduisant les effets de souffle. 

Chacun de ces scénarios demande un Test de Démolitions réussi. Le résultat exact est déterminé par le maître de jeu en fonction des spécificités du scénario. Par exemple, en utiliant les exemples ci-dessus, cibler un point faible peut doubler les dommages infligés sur cette structure. Modeler la charge pour orinter sa force peut tripler les dommages dans une direction, tel que noté dans la description de la superthermite (p. 330). Une Réussite Exceptionnelle augmentera les dommages de l'explosif de +5, alors qu'un succès critique permettra à l'explosion d'ignorer l'armure. 

\subsubsection{Désarmer} 

Désarmer un apapreil explosif est géré par un Test en Opposition entre les compétences Démolitions du démineur et de l'artificier. 



\subsubsection{Farbiquer des explosifs} 

Un personnage entraîné dans la Démolitions peut fabriquer des explosifs à partir de matériau bruts. Ces matériau peuvent être assemblés de manière traditionnelle ou être fabriqués en utilisant un nanofabeur. Même les nanofabeurs ayant des configurations restreintes pour empêcher la création d'explosifs peuvent être utilisés, les explosifs pouvant êtres construit de bien des manières différentes à aprtir de produits chimiques et de matériau inoffensifs. L'intervalle pour fariquer des explosifs est de 1 heure pour 1d10 points de dégat que l'explosif infligera. Si le démolisseur obtiens un échec critique, il peut se faire sauter accidentellement à moins que la charge ne soit extrêmement faible ou beaucoup plus puissante que prévu (quelle que soit al situation la plus désastreuse). 



\subsection{Tomber} \label{sec:falling} 

Si un personnage tombe, utilisez la table des Dégats de Chute pour déterminer quelles blessures il va subir. L'armure cinétqiue absorbera les dégats en utilisant la moitié de sa valeur (arrondissez à l'inférieur). À la discrétion du maître de jeu, les dégats peuvent aussi être réduits si quelquechose ralenti la chute (branches, surface molles). 

\begin{table} \begin{tabular}{|l|l|} \hline

\multicolumn{2}{|c|}{\textbf{Dégâts de chute}}	\\ \hline

\textbf{Distance de chute}	&\textbf{Dégâts}	\\ \hline

1-2 mètres	&1d10	\\ \hline

3-5 mètres	&2d10	\\ \hline

6-8 mètres	&3d10	\\ \hline

Plus de 8 mètres	&+1 par mètre	\\ \hline

\end{tabular} \label{tab:falling-damage} \end{table} 



\subsection{Feu} \label{sec:fire} 

Les objets qui entrent en contact avec une chaleur extême ou avec des flammes peuvent s'enflammer à la discrétion du maître de jeu, gardez à l'esprit à la fois le caractère inflammable de l'objet et la force de la chaleur/des flammes. Les objets (ou les personnages) qui brûlent souffrent de 1d10 $\div$ 2 (arrondissez au supérieur) points de dégats à chaque Tour d'Action à moins que quelquechose d'autre ne soit précisé. Les armures énergétiques vous protègerosn contre ces dommages, même si elles peuvent prendre feu, réduisant les dégats subit de leur valeur. Enf onction des conditions environnementalers, les feux ont tendance à grandir à moins que quelquechose ne cherche à le réduire. Tous les 5 Tours d'Actions, augmentez la VD infligée (d'abord à 1d10, puis à 2d10, ensuite à 3d10 et enfin par incrément de +5). Des oncidtions défavorables (telels que la pluie) ou des efforts pour éteindre le foyer réduiront la VD de manière appropriée. 

Notez que le feu ne brûle pas dans le vide. En microgravité, le feu brûle dans une sphère et se développe plsu lentement, les gazs d'expansions éloignant l'oxygène (augmentez la VD tous les 10 Tours d'Actions). Si il y a un manque de circulation d'air, des feux en microgravité peuvent s'éteindre d'exu-même. 



\subsection{Modes de tir et cadence de tir} \label{sec:firing-modes-rate} 

Toute les armes à distance d'\emph{Eclipse Phase} possèdent un ou plusieurs mode de tir qui détermine leur cadence. Ces modes de tirs sont détaillés ci-dessous. 

\subsubsection{Coup par coup (CC)} 

Les amres en coup par coup ne peuvent faire feu qu'une fois par Action Complexe. Ce sont typiquement les armes els plus grosses ou les plus archaïques. 

\subsubsection{Semi-automatique (SA)} 

Les armes semi-automatiques sont capable de tri rapide et répétés. Elle peuvent effectuer deux tirs avec la même Action Complexe. Chaque tir est considéré comme une attaque séparés. 

\subsubsection{Tir en rafale (TR)} 

Les amres qui peuvent tirer en rafales peuvent libérer plusierus tirs rapide (une "rafale") avec une seule pression de la détente. Deux rafales peuvent être tirées avec la même Action Complexe. Chaque rafale est gérée comem une attaque distincte. Les rafales tirent jusqu'à 3 munitions. 

Une rafale peut être tirée sur une seule cible (tir concentré) ou contre deux cibles écartées de moins d'un mètre l'une de l'autre. En ca de tir concentré contre une seule cible, augmenéte la VD de +1d10. 

\subsubsection{Tir Automatique (TA)} 

Les armes automatiques peuvent libérer une grêle de tirs avec une seule pressions sur la gachette. Une seule attaque en tir automatique ne peut être fait avec chaque Action Complexe. Cetet attaque peut être effectuée sur une à trois cibles différentes, tant qu'elles ne sont pas éloignées de plus d'un mètre l'une de l'autre. Dans le cas d'un tir concentré sur une seule cible, augmentée la VD de +1d10 +10. Tirer en automatique utilise 10 munitions. 



\subsection{Défense Totale} \label{sec:full-defense} 

Si vous vous attendez à être sous le feu, vous pouvez dépenser une Action Complexe pour apsser en défense totale. Cela représente le fait que vous dépensez toute votre énergie à esquiver, plonger, à parer les attaques et de manière générale à vous sortir de là jusqu'à votre prochaine Phase d'Action. Pendant ce temps, vous recevez un modificateur de +30 pour vous défendre contre les attaques en cours. Les personnages qui sont en défense totale peuvent utilsier leur compétence Parkour au lieu de leur compétence Esquive pour esquiver les attaques, représentant les mouvements acrobatiques qu'ils font pour éviter d'être touchés. 

\subsection{Gravité} \label{sec:gravity} 

La plupaert des personnages dans Eclipse Phase ont une expérience importantes des manœuvres en faible ou micro gravité et peuvent y effectuer des actiosn normales sans pénalités. Même les personanges qui ont grandit sur un corps planétaires ou dans des habitats tournant sont relativement familier aves le gravités alternatives grâce aux entraînements dans les systèmes éducatifs en simulspace. La même chsoe est également vraie dans l'autre sens; les personnages qui ont grandit en chute libre ont vécus des simulations de la vie dans un puit gravitationnel. 

À la discrétion du maître de jeu, les personnages qui ont passé de lingue périodes à s'acclimater à un type de gravité peuvent avoir du mal à s'habituer à un changement de gravité, au moins jusqu'à ce qu'ils se soient habitués à la nouvelle gravité. Dans ce cas, le maître de jeu peut appliquer un modificateur de -10 aux compétences sociales et physiques. La pénalité physique est la résultante des difficultés à manœuvrer. La pénalité sociale s'appliquent parcequ'il est difficile d'avori l'air impressionant, intimidant ou séduisant lorsque vous n'avez pas trouvé comment arranger vos habits pour qu'ils ne flottent pas devant vos yeux. La pénalité physique peut être augmentée à -20 pour les situatiosn impliquant des compétences de combat et des compétences nécessitants la manipulation rpécise, la construction ou la réparation d'objets. Ces pénalités s'aplliquent jusqu'à ce que le personnage finisse par s'adpater, généralement en 3 jours. 

Toute biomorph ayant les biomods de bases (p. 300) est immunisée au mal de l'espace et des effets de l'exposition à long-terme à la micro-gravité. 

\subsubsection{Microgravité} 

La microgravité inclut les environnements zéro-G ainsi que ceux qui ont une gravité légèrement plus élevée mais négligeable. Ces conditions sont trouvés dans l'espace, sur les astéroïdes et les petites lune ainsi que sur (une aprtie) des vaisseaux et habitats qui ne tournent pas pour générer de la gravité. Les objets en microgravité n'ont pas de poids, mais leur taile et leur masse sont toujours des facteurs à considérer. Les choses se comportent différement en microgravité. Par exemple: 

\begin{itemize} \item Les objets qui ne sont pas attachés ont tendance à dériver dans la direction de leur dernier déplacement. Les objets flottants peuvent éventuellement se diriger vers la aprtie la plus dense de l'habitat ou du vaisseau. \item Les objets lancés ou poussés vont voyager en ligne droite jusqu'à ce qu'ils percutent quelque chose. \item La fumée ne s'élève pas en un flot. Elle forme plutôt des halos vaguement sphérique autour de leur source. \item Les liquides n'ont que peu de cohésion, se répandant en nuage de petites gouttes si ils sont libérés dans l'air. Les boissions sont préparés dans des ampoules ou des bouteilles scellés. La nourriture est mangée d'une manière qui empêches les sauces et les parties liquide de s'échapper. Le sang se propage partout. \end{itemize} 

Se déplacer et manœuvrer en microgravité est géré par la compétence Shute Libre (p. 179). La plupart des activités quotidienne en chute libre ne nécessite pas de test. Le maître de jeu peu cependant demander un test de Chute Libre pour toute les manœuvre complexes, voler à travers de grandes distances, changer brutalement de direction ou de vitesse ou pour le combat au contact. Un échec signifie que le personnage a mal caculé sa trajectoire et termine dans une poistion qui n'était pas celle voulue. Un Échec Catastrophique signifie que le personnage s'est complètement planté, se fracassant dans un mur ou s'éjectant dans l'orbite en tourbillonant. 

Pour faciliter les choses, la plupart des habitats en microgravité ont des meubles recouvert de sangles élastiques et sont maillés de poches pour empêcher les objets de flotter dans tous els sens, ainsi que des bandes périphériques mobiles équipées de poignées le long des principales voies de circulations. les chaussures magnétiques ou à velcro sont également utilisée pour se promener plutôt que d'escalader ou de voler. Les environnements en zéro-g sont souvent conçus pour utiliser l'espace au maximum et tirent parti de l'absence de plafonds et de planchers. Comme les objets n'ont pas de poids, les personnages peuevnt même déplacer facilement des objets massifs. 

\textbf{Allure:} Les personnages qui grimpent, tirent ou se propulsent eux-même se déplace à la moitié de leur allure (p. 191) en microgarvité. 

\textbf{Vitesse Terminale:} Il n'est pas difficile d'atteindre la vitesse d'échappement sur les petits astéroïdes et les corps similaires - c'est quelque chose qu'il faut garder en tête avec les objets lancés et les armes à projectile. Dans certains cas, les personnages qui se déplacent suffisament rapidement et qui sautent peuvent atteindre la vitesse d'échappement, bien que ces situations soient laissées à l'appréciation du maître de jeu. 

\subsubsection{Faible gravité} 

La faible gravité inclut tout ce qui va de 0,5g à la microgravité. On trouve ces conditions sur la Lune, sur Mars, sur Titan et sur les parties rotatives de la plupart des vaisseuax et habitats tournant. Le faible gravité n'est pas trés différente de la gravité standard, bien que les personanges peuvent sauter deux fois plus loin et que les projectiles et objets lancés ont une portée plus longue (p. 203). Augmenter l'allure de course des personnages en faible gravité de x1,5. 

\subsubsection{Gravité élevée} 

La gravité élevée est tout ce qui est significativement plus élevée que la gravité Terrestre standard (1,2g et plus). La gravité élevée à Eclipse Phase n'existe typiquement que sur les exoplanètes. La gravité élevée peut être particvulièrement dure pour les personnages, leur corps étant soumis à des contraintes plus élevés en raison du poids plus important, les muscles se fatiguent à cause de la poussée supplémentaire nécessaire pour se déplacer et le cœur doit battre plsu fort pour envoyer le sang aprtout dans l'organisme. Pour tous les 0,2g au delà de 1 pour lequel un personnage 'nest pas acclimaté, considéré que le prsonnage souffre des effets d'1 blessure. A la discrétion du maîþre de jeu, les allures de déplacement peuvent également être modifiées. 

\subsection{Grenades et têtes chercheuses} \label{sec:combat-grenades-seekers} 

Les grenades, amres à tête chercheurse et explosifs simialire moderne ne détonnent pas forcément au moment où ils sont lancés ou lorsqu'ils touchent leur cible. En fait, plusieurs options de déclenchement sont disponible, chacune configurée par l'utilisateur lorsqu'il déploit l'arme. Les attaques ratées ou celles qui n'explsoent ni pendant le trajet ni à l'impact, sont sujettes à la dispersion (p. 204). 

\textbf{Explosion aérienne:} l'explosion aérienne signifie que l'appareil explose en l'air dés qu'il a parcouru une distance programmée au lancement. Dans ce cas, les effets de l'exposion sont résolu immédiatement, lors de cette Phase d'Action de l'utilisateur. Notez que les munitions à explosion aérienne sont programmées avec une sûreté qui empéchera la détaonation si elles n'arrivent pas à parcourir une distance minimale de précuation depuis le lanceur, bien que ceci puisse être contourné. 

\textbf{Impact:} La grenade ou le missile explose aussitôt qu'il touche quelque chose, que ce soit la cible, le sol ou un objet interrompant la trajectoire. Résolvez les effets imméditament, lors de cette Phase d'Action de l'utilisateur. 

\textbf{Signal:} La munition est préparée pour détonner lorsqu'elle reçoit un signal de commande par un lien sans-fil. L'appareile reste simplement en attente jusqu'à ce qu'il reçoive le signal correct (qui doit inclure la clef de chiffrement qui lui a été assigné lorsque la grenade a été préparée), détonnant immédiatement lorsqu'il le reçoit. 

\textbf{Minuteur:} l'appareil possède un miniteur inerne permettant à l'utilisateur de régler précisément le moment de la détonnation. Cela peut-être n'improte quand entre 1 seconde et plusieurs jours, mois ou années plus tard, transformant dans les fait l'appareil en bombe, mais augmentant également la probabilité qu'il soit découvert et neutralisé. La période dee détoantion minimale - 1 seconde- fait que la munition explosera avec le Score d'Initiative (actuel) de l'utilsiateur lors de la prochaine Phase d'Action. Un délai de 2 seconde durera deux phases d'Actions, un délai de 3 seconde dure trois Phases d'Action et ainsi de suite. 

\subsubsection{RRenvoyer les grendades} 

Il est possible qu'un personnage soit capable d'atteindre une grande avant qu'elle n'explose et qu'il al relance à son expéditeur (ou qu'il l'envoie au lieu, dans une direction sûre). Le personnage doit être à portée de mouvement de la localisation de la grenade, et doit dépenser une Action Complexe pour tenter un test de REF + COO + COO pour attraper la grenade. Si il réussit, il peut relancer al grenade dans la direction de son choix avec la même action (considérez l'action comme une attaque de lancer standard). 

Si le eprsonnage rate le test il peut cependant se trouver à l'épicentre de l'explosion. 

\subsubsection{Sauter sur l'explosion} 

Étant donné les possibilités offertes par la réincarnation, un personnage peut décider de se sacrifier pour l'équipe et de se jetter sur une grenade, se sacrifiant dans le but de protéger les autres. Le personnage doit être à portée de mouvement de la localisation de la grenade, et doit dépenser une Action Complexe pour tenter un test de REF + COO + VOL pour sauter sur la grenade et la recouvrir avec sa morph. Cela implique que le personnage subisse 1d10 de dégats supplémentaire lorsque la grenade explose. Le côté positif est que les dégats de la grenade sont réduits par l'armure du eprsonnage +10 lorsque ses effets sont appliqués aux autres dans le rayon de l'explosion. 

Si le maîþre de jeu pense que c'est approprié, un Test de VOL $\times$ 3 peut être demandé pour qu'un personnage se sacrifient de cette manière. 

\subsection{Environnements hostiles} \label{sec:hostile-environments} Le système solaire a beau être un point de départ idéal pour le développement de la vie, si vous vous retrouvez coincé dans le puits gravitationnel de Jupiter pendant une tempête magnétique, que vous essayer de respirer sans filtre sur Mars ou que vous essayez de nager dans le vide sans combinaison spatiale, il ne semble plus aussi amical. Cette section décrit quelques uns des environnements hostiles auquel les personnages d'\emph{Eclipse Phase} pourraient bien avoir à faire face. 

\subsubsection{Contamination atmosphérique} 

Les habitats tombent parfois malades. les effets d'un habitat souffrant d'un déséquilibre écologique ou de pathogène hors de contrôle peuvent aller des atmosphères légèrement allergène aux infections environnementales dévastatrice. les personnages sans système de respiration ou de filtration dans les environnements contaminés devraient subir des pénalités aux compétences physiques et, probablement, sociale allant de -10 (allergie légère) à -30 (atmosphère sévèrement infecté). En fonction de la contamnination, d'autres effets peuvent également s'appliquer en fonction des souhaits du maître de jeu. 

\subsubsection{Températures extrêmes} 

Les environnements planétaires varient de l'etrêmement chaud (Vénus, face ensolleillée de Mercure) à l'extrêmement froid (Neptune, Titan, Uranus). Les deux sont capables de tuer une biomorph non protégée et non adaptée en quelques minute, voire quelques secondes. Les synthmorpsh et les véhicules s'en sortent mieux, particulièrement dans le froid, mais même eux ont des chances de rapidement succomber aux fournaises flamboyantes des planètes intérieures sans de gros boucliers thermique et des systèmes de refroidissements. 

\subsubsection{Pression extrême} 

De manière similaire, la presison atmosphérique de Jupiter, Saturne et vénus devient rapidement mortellement écrasante après les premiers niveaux. Seuls les synthmorphs et les véchicules adaptés pour ces pressions extrêmes peuvent espérer survivre à de telles profondeurs. 

\subsubsection{Zones de transition gravitationnelle} 

l'usage important de la gravité artificielle dans les habitats spatiaux signifie que les personnages rencontrerons souvent des endroits où la direction du bas change brutalement. Dans la plupart des habitats rotatifs, la conception standard incluent une zone axiale où les vaisseaux spatiaux peuvent s'arrimer en microgravité et une zone de transition balisée et minutieusement concçues (géénralement un ascensseur) dans lqeulle els personnes et les cargaisons allant et venant vers le spatioport axial peuvent s'orienter vers le "bas" local et se tenir au bon endroit lorsque la gravité prend effet. Les transitions gravitationnelle dans les habitats sont rpesque toujours graduelles mais peuvent être extrêmement danegreuses si un personnage les franchit en étant au mauvais endroit au mauvais moment. 

Un personnage qui se retrouve à dériver dans la zone de microgravité axiale d'un habitat spatial rotatif va dériver lentement vers l'extérieur jusqu'à ce qu'il commence à subir la gravité, moment à partir duqeul il tombera. Le temps que prends ce phénomnèe varie en fonction de la taille de l'habitat. Une bonne règle de base est que pour chaque kilomètre de diamètre de l'habitat, le personnage a 30 seconde avent de commencer à tomber. Si le personnage a reçu une bonne poussée hors de l'axe lorsqu'il a commencé à dériver, ce temps devrait être divisé par deux, quatre ou plus, à la discrétion du maîþre de jeu. 

\subsubsection{Champs magnétiques} 

Le magnétismen'est pas un problème direct pour la plupart des personnages; les transhumains n'ont pas besoin de s'inquiéter des radiations générées par une magnétosphère puissante. Pour les appareils électroniques non blindés et les transhumains non protégés arborant du titanium les effets des champs magnétiques puissants peuvent être dévastateurs. Notez que la plupart des problèmes réusltatnts de l'expositions des véhicules, du matériel et des bots à une forte activité magéntique coincide avec une radioactivité élevée. 

Les champs magnétiques affectent les synthmorphs, les robots, les véhicules, les implants cybernétiques et l'électronique après 1 minute d'exposition. Comme l'exposition aux radiations, ces effets peuvent varier de manière radicale. Au minimum, les communications et les capteurs souffrirontd 'interférence et de portée réduite: au maximum, les systèmes électoniques vont simplement subir des dégâts et tomber en panne. 

\subsubsection{Radiation} 

Les radiations ionisantes sont l'un des dangers les plus fréquemments rencontrés dans le système solaire et l'un de sproblèmes les plus difficiles à vaincre pour la transhumanité. Les radiations abîment les matériaux génétiques, rendent malades et tuent en ionisant les composés chimiques impliqués dans la division cellullaire à l'intérieur du corps humain. Les effets s'étendent la nausée et la fatigue à des défectiosn d'organes massives et la mort. Les maladies radiactives ne sont pas qu'un phénomène somatique. La véritable terreur des radiations pour les transhumains, particulièrement ceux aux niveaux de dosage ls plus élevés tels qu'à la surface de ganylède ou d'autres lunes Jovienne, sont les dommages aux réseaux de neurones. Cela peu amener à des uploads et des sauvegardes buguées. La nanomédecine peut rapidement inverser l'ionisation des composés cellullaire et les nouveaux matériaux permettant une protection plus fine et de meilleru qualité aident, mais la simple magnitude de la radiation émise par certains corps dans le système solaire surapsse même ces derniers. 

L'empoisonnemetn par radioations est uen affaire complexe, et des règles détaillées sont hors du spectre de ce livre. Les sources de radiations incluent entre autres la ceinture Van Allen de la Terre, la ceinture radiocative de Jupiter, la magnétosphère de Saturne, les rayons cosmiques, les éruptions solaires, les matériaux fissibles, les explosions de fusion ou d'antimatière non protégée et les explosions nucléaires. Les effets peuvent varier drastiquement en fonction de la force de la source, le temps d'exposition et le niveau de protection disponible. Les effets immédiats sur les biomorphs (mettant de quelques minutes à 6 heures pour apparaître) incluent les nausées, les vomissements, la fatigue (SOM réduite) ainsi que des dégats physique et un faible montant de stress mental. Cette phase est suivi d'une période qui ressemblant à une rémission, durant de 6 heures à 2 semaines. Après cette étape, la phase terminale démarre, ce qui peut inclure une perte de pillosité, la stérilité, une SOM et une SOL réduites, des dommages importants aux tissus intestinaux et gastriques, des infections, des hémorragies incontrôlées puis la mort. 

Les synthmorphs ne sont pas aussi vulnérable que les biomorphs, mais elles peuvent être endommagées et handicapées par de trés fortes doses de radiations. 

\subsubsection{Atmosphère toxique} 

Neptune, Titan, Uranus, et Vénus ont toutes des atmosphères toxiques. Des atmosphères similaires peuvent être trouvées sur certaines exoplanète, ou avoir été intentionnellement créée comme mesure de sécurité dans un habitat ou une structure. 

Un personnage qui ignore la toxicité de l'atmosphère et qui ne retiens pas immédiatement sa respiration (nécessitant un Test de REF $\times$ 3) subit 10 points de dégats par Tour d'Action. Un personnage qui aprvient à retenir sa repsiration peut tenir un peu plus longtemps; appliquez les règles d'asphixie (p. 194). 

\textbf{Atmosphères Corrosive:} En plus d'être toxique, l'atmosphère de Vénus est la seule natturelement corrosive dans le système. Les amtopshères corrosives sont immédiatement dangereuse: les personnages subissent 10 points de dégats par Tour d'Action, peu importe qu'ils retiennent leur respiration ou pas. Les atmosphères corrosives endommagent également les véhicule et l'équipement ne disposant pas de protection anticorrosive. De tels objets subissent 1 points de dégat par minute. Avec des concentrations plsu élevée, telles que dans les nuages denses d'acide sulfurique de la haute atmosphère Vénusienne, les objets subissent 5 points de dégats par minute. 

\subsubsection{Atmosphère irrespirable} 

De très rare corps planétaire dans le système solaire ont réellement des atmosphère toxiques. Dans la plupart des atmosphère irrespirable, le danger principal pour les transhumains ne possédant pas les modifications ou le système respiratoire nécessaire est le manque d'oxygène. Coinsidérez l'exposition à une atmosphère irrespirable de la même manière que l'asphyxie. 

\subsubsection{Sous l'eau} 

En général, toute compétence physique utilisée sous l'eau subit une pénalité de -20 en raison de la résistance du milieu. Les compétences reposant sur l'équipement non adapté à l'utilisation sous-marine peuevent être encore plus difficile voire impossible à utiliser. L'allure d'un personnage en angeant ou en marchant sous l'eau et le quart de leur allure normale sur la terre ferme. Si un personnage comemnce à se noyer sous-l'eau, utilisez les règles d'asphixie (p. 194). Notez que les personnages en train de s noyer ne récupère pas immédiatement si ils sont sortis de l'eau; ils continueront à s'asphyxier jusqu'à ce qu'un traitement médical soit utilisé pour vider l'eau de leur poumon. 

\subsubsection{Vide} 

Les biomorphs dépourvues de système d'étanchéité au vide (p. 305) peuvent passer uen minute dans le vide spatial sans effet secondaire, du moment qu'ils se recroqueveillent en boule, vident leur poumons et gardent leurs yeux fermés (c'est quelque chose que les enfants des ahabitats spatiaux aprennent trés tôt). Contrairement aux croyances populaires liés aux médias pré-Chute, un personnage exposé au vide total n'explose pas en décompression, de même que ses fluides internes n'entrent pas en ébulitions (autres que les liquides relativement exposs, comme la salive sur al langue). En fait, le danger principal pour les personnages dans le vide sans exocombi est l'asphyxie due au manque d'oxygène et les complications associées telles qu'un œdème des poumons. 

Au delà d'une minute dans l'espace, le personnage commence à subir de l'asphyxie (p. 194). Les dégats sont doublés si le personnage essaye de garder de l'air dans ses poumons ou qu'il n'est aps recroquevillé dans une position de survie dans le vide telle que décrite plus-haut. Additionnelement, les personnages piégés dans l'espace sans protection thermique adéquate subissent 10 points de dégats par minute en raison du froid extrême. 

\subsection{Armes improvisées} \label{sec:improvised-weapons} 

Quelque fois les personnages seront pris par surprise et ils devront utiliser ce qu'ils auront sous la main comme arme - à moins qu'ils ne pensent avoir l'air cool à tabasser quelqu'un avec un mètre de chaines. La table des Armes Improvisées propose des statistiques pour quelques objets susceptibles d'être utilisés comme arme. Les maitres de jeu peuvent utiliser ces guides pour gérer les objets non listés. 

\begin{table} \begin{tabularx}{\textwidth}{|l|l|l|l|X|} \hline

\multicolumn{5}{|c|}{\textbf{Armes improvisées}} \\ \hline

\textbf{Armes}	&\textbf{PA}	&\textbf{Valeur de dégats (VD)}	&\textbf{VD moyenne}	&\textbf{Compétence}	\\ \hline

Balle de baseball	&$-$	&(1d10 $\div$ 10) + (SOM $\div$ 10)	&2 + (SOM $\div$ 10)	&Armes de jet	\\ \hline

Bouteille	&$-$	&1 + (SOM $\div$ 10), Se casse après une utilisation	&1 + (SOM $\div$ 10)	&Massues ou armes de jets	\\ \hline

Bouteille (cassée)	&$-$	&1d10 $-$ 1 (min: 1)	&4	&Lames	\\ \hline

Chaînes	&$-$	&1d10 + (SOM $\div$ 10)	&5 + (SOM $\div$ 10)	&Armes de mélée exotique\\ \hline

Casque	&$-$	&1d10 + (SOM $\div$ 10)	&5 + (SOM $\div$ 10)	&Massues ou armes de jets	\\ \hline

Torche à plasma	&$-$6	&2d10	&11	&Arme à Distance Exotique \\ \hline

Pied de biche	&$-$ &1d10 + (SOM $\div$ 10)	&5 + (SOM $\div$ 10)	&Massues	\\ \hline

\end{tabularx} \label{tab:improvised-weapons} \end{table} 



\subsection{Projetter} \label{sec:knockdown-knockback} 

Si le but d'un attaquant est simplement de mettre à terre ou d'éloigner un adversaire au contact au lieu de le blesser, lancer l'attaque et la défense comme d'habitude. Si l'attaquant réussit, le défenseur est projetté en arrière d'1 mètre par tranche complète de 10 points de MdR. Pour jetter à terre un adversaire, l'attaquant doit obtenir une Réussite Exceptionnelle (MdR de 30+) Une attaque de projection doit être déclarée avant que les dés ne soient jettés. 

À moins que l'attaquant n'obtiennent une réussite critique, aucun dégâts n'est infligé avec cette attaque, le défenseur est juste projetté/mis à terre. Cependant, si l'attaquant obtient une réussite critique, appliqez les dégats comme d'habitude en plus de l'effet de projection. 

Notez qu'un personnage blessé par une attaque peut également être jetté à terre (voir \emph{Effets des Blessures }, p. 207). 



\subsection{Bonus de dégat en mélée et des armes de jets} \label{sec:melee-thrown-damage-bonus} 

Chaque attaque réussit au corps à corps et avec les armes de jet, que ce soit à mains nue ou avec des armes, reçoit un bonus égal à la SOM de l'attaquant $/div$ 10, arrondissez à l'inférieur. Voir \emph{Bonus de dommage }, p. 123. 



\subsection{Cibles multiples} \label{sec:multiple-targets} 

Lorsque vous infligez des dégâts, il n'y a aucune raison de ne pas partager avec les autres. 

\subsubsection{Combat au contact} 

Un personnage qui prend une Action Complexe pour démarrer une attaque de mélée peut choisir d'attaquer deux adversaire ou plus avec la même action. Chaque adversaire doit être à moins d'un mètre d'une autre cible de l'attaquant. Ces attaques doivent être déclarée avant que les dés ne soient lancés pour la première attaque. Chaque attaque souffre d'un modificateur cumulatif de -20 pour chaque cible supplémentaire. Un personnage qui déclare qu'il va attaquer trois personnes avec la même action subira un modificateur cumulatif de -60 sur chaque attaque. 

\subsubsection{Combat à distance} 

Un eprsonnage faisant feu deux fois avec uen arme semi-automatique lors d'une Actrion Complexe peut cibler différent adversaire à chaque tir. Dans ce cas, l'attaquant subit un modificateur de -20 contre la deuxième cible. 

Un personnage tirant en rafale peut cibler jusqu'à deux cibles à chaque rafale, tant que ces cibles sont éloignées d'un mètre maximum l'une de l'autre. Cela est considéré comme une seule atatque; voir \emph{Tir en Rafale}, p. 198. 

Un personnage tirant en rafale deux fois avec une Action Complexe peut cibler une personne différente ou une paire d'autres personne à chaque rafale. Dans ce cas, la deuxième rafale subit un modificateur de -20. Cce modificateur ne s'applique pas si la même personne/pair de personne ciblée avec la première rafale est de nouveau ciblée. 

Les attaques en mode automatique peuvent aussi être dirigée sur plus d'une cible, tant que chaque cible est à moins d'un mètre la précédente. Cela est considéré comme une seule atatque; voir \emph{Tir Automatique}, p. 198. 



\subsection{Objets et structures} \label{sec:objects-structures} 

Comme tout apuvre mur dans le vosiinnage d'une violente fusillade pourra vous le dire, les objets et les structures ne sont pas immunisés à la violence et à la dégradation. Pour refléter ceci, les objets inanimés et les structures ont des scores de Solidité, de Seuil de Blessure et d'Armure, tout comme les personnages. La Solidité mesure la quantité de dommage que peut subir la structure avant d'être détruite. L'Armure réduit les dommages infligés par les attaques, comme pour les personnages. Pour simplifier, un seul score d'Armure est donné et compte à la fois comme une Armure Cinétique et une Armure Énergétique; à la discrétion du maître de jeu, ces scores peuvent être modifiées en focntion de al situation. 

Les Blessures subient par les objets et les structures n'ont aps les mêmes effets que les blessures infligés aux auttres personnages. Chaque blessure est simplement considérées comme un trou, une démolition partielle ou une fonctionnalité réduite selon ce que le maître de jeu pense convenir le mieux. De manière alternative, un appareil endommagé peut fonctionner de manière moisn efficace, et donc infliger un modificateur négatif aux tests de compétences fait pendant l'utilisation de cet objet (un modificateur cumulatif de -10 par blessure). 

Dans le cas des grosses structures, il est recommander de traiter les parties individuelles de la structure comme des entités séparées dans le cadre des dégats subit. 

\subsubsection{Attaques à distance} 

Les attaques à distance n'infligent qu'un tiers de leurd égats (arrondisse à l'inférieur) aux grosse structures telles que les portes, les murs, etc Cela reflète le fait que la plupart des attaques à distance ne font que pénétrer la structure, ne faisant que des dégats mineurs. 

Les agoniseurs et les étourdisseurs n'ont aucun effets sur les objets et les structures. 

\subsubsection{Tirer à travers} 

Si un personnage essaye de tirer à travers un objet ou une structure sur une cible de l'autre cpté, l'attaque subira un modificateur de tir en aveugle d'au moins -30 sauf si l'attaquant peut voir sa cible d'une manière ou d'une autre. En plus de ce modificateur, la cible recçoit un bonus d'armure égal au score d'Armure $\times$ 2 de l'objet/la structure. 

\begin{table} \begin{tabularx}{\textwidth}{|X|r|r|r|} \hline

\multicolumn{4}{|c|}{\textbf{Exemple d'objets et de structures}} \\ \hline

\textbf{Objet/structure} &\textbf{Armure} &\textbf{Solidité} &\textbf{Seuil de blessure}	\\ \hline

Composites avancées (coque de vaisseau/d'habitat)	&50	&1000	&200	\\ \hline

Aérogel (murs, fenêtre, etc)	&-	&50	&10	\\ \hline

Porte de sas	&15	&100	&25	\\ \hline

Alliage, bétons, polymère renforcés (portes/murs renforcés)	&30	&100	&20	\\ \hline

Verre blindé	&10	&50	&20	\\ \hline

Comptoir	&7	&60	&12	\\ \hline

Bureau	&5	&50	&10	\\ \hline

Lien ecto	&-	&6	&1	\\ \hline

Mousse metallique (murs, portes, etc.)	&20	&70	&15	\\ \hline

Vitres métalliques	&30	&150	&30	\\ \hline

Polymère ou bois (murs, portes, meubles, etc)	&10	&40	&8	\\ \hline

Lien de farcast quantique	&3	&20	&4	\\ \hline

Aluminium transparent (murs, meubles)	&5	&60	&12	\\ \hline

Arbre	&2	&40	&10	\\ \hline

Fenêtre	&-	&5	&1	\\ \hline

\end{tabularx} \label{tab:sample-objects-structures} \end{table} 

\subsection{Range} \label{sec:range} 

Every type of ranged weapon has a limited range, beyond which it is ineffective. The effective range of the weapon is further broken down into four categories: Short, Medium, Long, and Extreme. A modifier is applied for each category, as noted on the Combat Modifiers table, p. 193. For examples of specific weapon ranges, see the Weapon Ranges table. 

\begin{table} \begin{tabularx}{\textwidth}{|X|r|r|r|r|} \hline

\multicolumn{5}{|c|}{\textbf{Weapon ranges}} \\ \hline

\textbf{Weapon (type)} &\textbf{Short} &\textbf{Medium (-10)} &\textbf{Long (-20)} &\textbf{Extreme (-30)}\\ \hline

\multicolumn{5}{|l|}{\emph{Firearms}} \\ \hline

Light Pistol	&0-10	&11-25	&26-40	&41-60	\\ \hline

Medium Pistol	&0-10	&11-30	&31-50	&51-70	\\ \hline

Heavy Pistol	&0-10	&11-35	&36-60	&61-80	\\ \hline

SMG	&0-30	&31-80	&81-125	&126-230	\\ \hline

Assault Rifle	&0-150	&151-250	&251-500	&501-900	\\ \hline

Sniper Rifle	&0-180	&181-400	&401-1100	&1100-2300	\\ \hline

Machine Gun	&0-100	&101-400	&401-1000	&1001-2000	\\ \hline

\multicolumn{5}{|l|}{\emph{Railguns}}\\ \hline

\multicolumn{5}{|l|}{as Firearms but increase the effective range in each category by +50\%} \\ \hline

\multicolumn{5}{|l|}{\emph{Beam Weapons}} \\ \hline

Cybernetic Hand Laser	&0-30	&31-80	&81-125	&126-230 \\ \hline

Laser Pulser	&0-30	&31-100	&101-150	&151-250 \\ \hline

Microwave Agonizer	&0-5	&6-15	&16-30	&31-50 \\ \hline

Particle Beam Bolter	&0-30	&31-100	&101-150	&151-300 \\ \hline

Plasma Rifle	&0-20	&21-50	&51-100	&101-300 \\ \hline

Stunner	&0-10	&11-25	&26-40	&41-60 \\ \hline

\multicolumn{5}{|l|}{\emph{Seekers}} \\ \hline

Seeker Micromissile	&5-70	&71-180	&181-600	&601-2000 \\ \hline

Seeker Minimissile	&5-150	&151-300	&301-1000	&1001-3000 \\ \hline

Seeker Standard Missile	&5-300	&301-1000	&1001-3000	&3001-10000 \\ \hline

\multicolumn{5}{|l|}{\emph{Spray Weapons}} \\ \hline

Buzzer	&0-5	&6-15	&16-30	&31-50\\ \hline

Freezer	&0-5	&6-15	&16-30	&31-50\\ \hline

Shard Pistol	&0-10	&11-30	&31-50	&51-70\\ \hline

Shredder	&0-10	&11-40	&41-70	&71-100\\ \hline

Sprayer	&0-5	&6-15	&16-30	&31-50\\ \hline

Torch	&0-5	&6-15	&16-30	&31-50\\ \hline

Vortex Ring Gun	&0-5	&6-15	&16-30	&31-50\\ \hline

\multicolumn{5}{|l|}{\emph{Thrown Weapons}} \\ \hline

Lames &To SOM $\div$ 5 &To SOM $\div$ 2 &To SOM &To SOM $\times$ 2 \\ \hline

Minigrenades &To SOM $\div$ 2 &To SOM &To SOM $\times$ 2 &To SOM $\times$ 3 \\ \hline

Standard Grenades &To SOM $\div$ 5 &To SOM $\div$ 2 &To SOM &To SOM $\times$ 3 \\ \hline

\end{tabularx} \end{table} 



\subsubsection{Range, gravity and vacuum} 

The ranges listed on the Weapon Ranges table are for Earth-like gravity conditions (1 g). While the effective ranges of kinetic, seeker, spray, and thrown weapons can potentially increase in lower gravity environments due to lack of gravitational forces or aerodynamic drag, accuracy is still the defining factor for determining whether you hit or miss a target. In lower gravities, use the same effective ranges listed, but extend the maximum range by dividing it by the gravity (for example, a max range of 100 meters would be 200 meters in 0.5 g). In microgravity and zero g, the maximum range is effectively line of sight. Likewise, under high-gravity conditions (over 1 g), divide each range category maximum by the gravity (e.g., a short range of 10 meters would be 5 meters in 2 g). 

Beam weapons are not affected by gravity, but they do fare much better in non-atmospheric conditions. Maximum beam weapon range in vacuum is effectively line of sight. 



\subsection{Reach} \label{sec:reach} 

Some weapons extend a character’s reach, giving them a significant advantage over an opponent in melee combat. This applies to any weapon over half a meter long: axes, clubs, swords, shock batons, etc. Whenever one character has a reach advantage over another, they receive a +10 modifier for both attacking and defending. 



\subsection{Scatter} \label{sec:scatter} 

When you are using a blast weapon, you may still catch your target in the blast radius even if you fail to hit them directly. Weapons such as grenades must go somewhere when they miss, and exactly where they land may be important to the outcome of a battle. To determine where a missed blast attack falls, the scatter rules are called into play. 

To determine scatter, roll a d10 and note where the die ``points'' (using yourself as the reference point). This is the direction from the target that the missed blast lands. The die roll also determines how far away the blast lands, in meters. If the MoF on the attack is over 30, this distance is doubled. If the MoF exceeds 60, the distance is tripled. This point determines the epicenter of the blast; resolve the effects of damage against anyone caught within its sphere of effect as normal (see \emph{Blast Effect}, p. 193). 



\subsection{Shock attacks} \label{sec:shock-attacks} 

Shock weapons use high-voltage electrical jolts to physically stun and incapacitate targets. Shock weapons are particularly effective against biomorphs and pods, even when heavily armored. Synthmorphs, bots, and vehicles are immune to shock weapon effects. A biomorph struck with a shock weapon must make a DUR + Energy Armor Test (using their current DUR score, reduced by damage they have taken). If they fail, they immediately lose neuromuscular control, fall down, and are incapacitated for 1 Action Turn per 10 full points of MoF (minimum of 3 Action Turns). During this time they are stunned and incapable of taking any action, possibly convulsing, suffering vertigo, nausea, etc. After this period, they may act but they remain stunned and shaken, suffering a -30 modifier to all actions. This modifier reduces by 10 per minute (so -20 after 1 minute, -10 after 2 minutes, and no modifier after 3 minutes). Many shock weapons also inflict DV, which is handled as normal. 

A biomorph that succeeds the DUR Test is still shocked but not incapacitated. They suffer half the listed DV and suffer a -30 modifier until the end of the next Action Turn. This modifier reduces by 10 per Action Turn. Modifiers from additional shocks are not cumulative, but will boost the modifier back to its maximum value. 



\subsection{Subdual} \label{sec:subdual} 

To grapple an opponent in melee combat, you must declare your intent to subdue before making the die roll. Any appropriate melee skill may be used for the attack; if wielding a weapon, it may be used as part of the grappling technique. If you succeed in your attack with an Excellent Success (MoS of 30+), you have successfully subdued your opponent (for the moment, at least). Grappling attacks do not cause damage unless you roll a critical success (though even in this case you can choose not to). 

A subdued opponent is temporarily restrained or immobilized. They may communicate, use mental skills, and take mesh actions, but they may not take any physical actions other than trying to break free. (At the gamemaster’s discretion, they may make small, restrained physical actions, such as reaching for a knife in their pocket or grabbing an item dropped a few centimeters away on the floor, but these actions should suffer at least a -30 modifier and may be noticed by their grappler). 

To break free, a grappled character must take a Complex Action and succeed in either an Opposed Unarmed Combat Test or an Opposed SOM $\times$ 3 Test, though the subdued character suffers a -30 modifier on this test. 



\subsection{Suppressive fire} \label{sec:suppressive-fire} 

A character firing a weapon in full-auto mode (p. 198) may choose to lay down suppressive fire over an area rather than targeting anyone specifically, with the intent of making everyone in the suppressed area keep their heads down. This takes a Complex Action, uses up 20 shots, and lasts until the character’s next Action Phase. The suppressed area extends out in a cone, with the widest diameter of the cone being up to 20 meters across. Any character who is not behind cover or who does not immediately move behind cover on their action is at risk of getting hit by the suppressive fire. If they move out of cover inside the suppressed area, the character laying down suppressive fire gets one free attack against them, which they may defend against as normal. Apply no modifiers to these tests except for range, wounds, and full defense. If hit, the struck character must resist damage as if from a single shot. 



\subsection{Surprise} \label{sec:surprise} 

Characters who wish to ambush another must seek to gain the advantage of surprise. This typically means sneaking up on, lying in wait, or sniping from a hardto- perceive position in the distance. Any time an ambusher (or group of ambushers) attempts to surprise a target (or group of targets), make a secret Perception Test for the ambushee(s). Unless they are alert for surprises, this test should suffer the typical -20 modifier for being distracted. This is an Opposed Test against the ambusher(s) Infiltration skill. Depending on the attacker’s position, other modifiers may also apply (distance, visibility, cover, etc.). 

If the Perception Test fails, the character is surprised by the attack and cannot react to or defend against it. In this case, simply give the attacker(s) a free Action Phase to attack the surprised character(s). Once the attackers have taken their actions, roll Initiative as normal. 

If the Perception Test succeeds, the character is alerted to something a split-second before they are ambushed, giving them a chance to react. In this case, roll Initiative as normal, but the ambushed character(s) suffers a -30 modifier to the Initiative Test. The ambushed character may still defend as normal. 

In a group situation, things can get more complicated when some characters are surprised and others aren’t. In this case, roll Initiative as normal, with all non-ambushers suffering the -30 modifier. Any characters who are surprised are simply unable to take action on the first Action Phase, as they are caught off-guard and must take a moment to assess what’s going on and get caught up with the action. As above, surprised characters my not defend on this first Action Phase. 



\subsection{Tactical networks} \label{sec:tactical-networks} 

Tactical networks are specialized software programs used by teams that benefit from the sharing of tactical data. They are commonly used by sports teams, security outfits, military units, AR gamers, gatecrashers, surveyors, miners, traffic control, scavengers, and anyone else who needs a tactical overview of a situation. Firewall teams regularly take advantage of them. 

In game terms, tacnets provide specialized software skills and tools to a muse or AI, as best fits their tactical needs. These tools link together and share and analyze data between all of the participants in the network, creating a customizable entoptics display for each user that summarizes relevant data, highlights interactions and priorities, and alerts the user to matters that require their attention. 

\subsubsection{Combat tacnets} The following list is a sample of a typical combat tacnet’s features. Gamemasters are encouraged to modify and expand these options as appropriate to their game: 

\begin{itemize} \item \textbf{Maps:} Tacnets assemble all available maps and can present them to the user with a bird’s eye view or as a three-dimensional interactive, with distances between relevant features readily accessible. The AI or muse can also plot maps based on sensory input, breadcrumb positioning systems (p. 332), and other data. Plotted paths and other data from these maps can be displayed as entoptic images or other AR sensory input (e.g., a user who should be turning left might see a transparent red arrow or feel a tingling sensation on their left side). \item \textbf{Positioning:} The exact positioning of the user and all other participants are updated and mapped according to mesh positioning and GPS. Likewise, the positioning of known people, bots, vehicles, and other features can also be plotted according to sensory input. \item \textbf{Sensory Input:} Any sensory input available to a participating character or device in the network can be fed into the system and shared. This includes data from cybernetic senses, portable sensors, smartlink guncams, XP output, etc. This allows one user to immediately call up and access the sensor feed of another user. \item \textbf{Communications Management:} The tacnet maintains an encrypted link between all users and stays wary both of participants who drop out or of attempts to hack or interfere with the communications link. \item \textbf{Smartlink/Weapon Data:} The tacnet monitors the status of weapons, accessories, and other gear via the smartlink interface or wireless link, bringing damage, shortages, and other issues to the user’s attention. \item \textbf{Indirect Fire:} Members of a tacnet can provide targeting data to each other for purposes of indirect fire (p. 195). \item \textbf{Analysis:} The muses and AIs participating in the tacnet are bolstered with skill software and databases that enable them to interpret incoming data and sensory feeds. Perhaps the most useful aspect of tacnets, this means that the muse/AI may notice facts or details individual users are likely to have overlooked. For example, the tacnet can count shots fired by opponents, note when they are likely running low, and even analyze sensory input to determine the type of weaponry and ammunition being used. Opponents and their gear can also be scanned and analyzed to note potential weaknesses, injuries, and capabilities. If sensor contact with an opponent is lost, the last known location is memorized and potential movement vectors and distances are displayed. Opponent positioning can also identify lines of sight and fields of fire, alerting the user to areas of potential cover or danger. The tacnet can also suggest tactical maneuvers that will aid the user, such as flanking an opponent or acquiring better elevation. \end{itemize} 

Many of these features are immediately accessible to the user via their AR display; other data can be accessed with a Quick Action. Likewise, the gamemaster decides when the muse/AI provides important alerts to the user. At the gamemaster’s discretion, some of these features may apply modifiers to the character’s tests. 



\subsection{Touch-only attack} \label{sec:touch-only-attack} Some types of attacks simply require you to touch your target, rather than injure them, and are correspondingly easier. This might apply when trying to slap them with a dermal drug patch, spreading a contact poison on their skin, or making skin-to-skin contact for the use of a psi sleight. In situations like this, apply a +20 modifier to your melee attacks. 



\subsection{Two-handed weapons} \label{sec:two-handed-weapons} 

Any weapon noted as two-handed requires two hands (or other prehensile limbs) to wield effectively. This applies to some archaic melee weapons (large swords, spears, etc.) in addition to certain larger firearms and heavy weapons. Any character that attempts to use such a weapon single-handed suffers a -20 modifier. This modifier does not apply to mounted weapons. 



\subsection{Weilding two or more weapons} \label{sec:weilding-two-or-more} 

It is possible for a character to wield two weapons in combat, or even more if they are an octomorph or multi-limbed synthmorph. In this case, each weapon that is held in an off-hand suffers a -20 off-hand weapon modifier. This modifier may be offset with the Ambidextrous trait (p. 145). 

\subsubsection{Extra melee weapons} 

The use of two or more melee weapons is treated as a single attack, rather than multiple. Each additional weapon applies +1d10 damage to the attack (up to a maximum +3d10). Off-hand weapon modifiers are ignored. If the character attacks multiple targets with the same Complex Action (see \emph{Multiple Targets}, p. 202), these bonuses does not apply. The attacker must, of course, be capable of actually wielding the additional weapons. A splicer with only two hands cannot wield a knife and a two-handed sword, for example. Likewise, the gamemaster may ignore this damage bonus for extra weapons that are too dissimilar to use together effectively (like a whip and a pool cue). Note that extra limbs do not count as extra weapons in unarmed combat, nor do weapons that come as a pair (such as shock gloves). 

A character using more than one melee weapon receives a bonus for defending against melee attacks equal to +10 per extra weapon (maximum +30). 

\subsubsection{Extra ranged weapons} 

Similarly, an attacker can wield a pistol in each hand for ranged combat, or larger weapons if they have more limbs (an eight-limbed octomorph, for example, could conceivably hold four assault rifles). These weapons may all be fired at once towards the same target. In this case, each weapon is handled as a separate attack, with each off-hand weapon suffering a cumulative off-hand weapon modifier (no modifier for the first attack, -20 for the second, -40 for the third, and -60 for the fourth), offset by the Ambidextrous trait (p. 145) as usual. 

\section{Physical health} \label{sec:physical-health} 

In a setting as dangerous as \emph{Eclipse Phase}, characters are inevitably going to get hurt. Whether your morph is biological or synthetic, you can be injured by weapons, brawling, falling, accidents, extreme environments, psi attacks, and so on. This section discusses how to track such injuries and determine what effect they have on your character. Two methods are used to gauge a character’s physical health: damage points and wounds. 

\subsection{Damage points} \label{sec:damage-points} 

Any physical harm that befalls your character is measured in damage points. These points are cumulative, and are recorded on your character sheet. Damage points are characterized as fatigue, stun, bruises, bumps, sprains, minor cuts, and similar hurts that, while painful, do not significantly impair or threaten your character’s life unless they accumulate to a significant amount. Any source of harm that inflicts a large amount of damage points at once, however, is likely to have a more severe effect (see \emph{Wounds}, p. 207). 

Damage points may be reduced by rest, medical care, and/or repair (see \emph{Healing and Repair}, p. 208). 



\subsection{Damage types} \label{sec:damage-types} 

Physical damage comes in three forms: Energy, Kinetic, and Psi. 

\subsubsection{Energy damage} 

Energy damage includes lasers, plasma guns, fire, electrocution, explosions, and others sources of damaging energy. 

\subsubsection{Kinetic damage} 

Kinetic damage is caused by projectiles and other objects moving at great speeds that disperse their energy into the target upon impact. Kinetic attacks include slug-throwers, flechette weapons, knives, and punches. 

\subsubsection{Psi damage} 

Psi damage is caused by offensive psi sleights like Psychic Stab (p. 228). 



\subsection{Durability and health} \label{sec:durability-health} 

Your character’s physical health is measured by their Durability stat. For characters sleeved in biomorphs, this figure represents the point at which accumulated damage points overwhelm your character and they fall unconscious. Once you have accumulated damage points equal to or exceeding your Durability stat, you immediately collapse from exhaustion and physical abuse. You remain unconscious and may not be revived until your damage points are reduced below your Durability, either from medical care or natural healing. 

If you are morphed in a synthetic shell, Durability represents your structural integrity. You become physically disabled when accumulated damage points reach your Durability. Though your computer systems are likely still functioning and you can still mesh, your morph is broken and immobile until repaired. 



\subsubsection{Death} 

An extreme accumulation of damage points can threaten your character’s life. If the damage reaches your Durability $\times$ 1.5 (for biomorphs) or Durability $\times$ 2 (for synthetic morphs), your body dies. This known as your Death Rating. Synthetic morphs that reach this state are destroyed beyond repair. 

\subsection{Damage value} 

Weapons (and other sources of injury) in \emph{Eclipse Phase} have a listed Damage Value (DV) --- the base amount of damage points the weapon inflicts. This is often presented as a variable amount, in the form of a die roll; for example: 3d10. In this case, you roll three ten-sided dice and add up the results (counting 0 as 10). Sometimes the DV will be presented as a dice roll plus modifier; for example: 2d10 + 5. In this case you roll two ten-sided dice, add them together, and then add 5 to get the result. 

For simplicity, a static amount is also noted in parentheses after the variable amount. If you prefer to skip the dice rolling, you can just apply the static amount (usually close to the mean average) instead. For example, if the damage were noted 2d10 + 5 (15), you could simply apply 15 damage points instead of rolling dice. 

When damage is inflicted on a character, determine the DV (roll the dice) and subtract the modified armor value, as noted under \emph{Step 7: Determine Damage} (p. 192). 

\subsection{Wounds} \label{sec:wounds} 

Wounds represent more grievous injuries: bad cuts and hemorrhaging, fractures and breaks, mangled limbs, and other serious damage that impairs your ability to function and may lead to death or long-term damage. 

Any time your character sustains damage, compare the amount inflicted (after it has been reduced by armor) to your Wound Threshold. If the modified DV equals or exceeds your Wound Threshold, you have suffered a wound. If the inflicted damage is double your Wound Threshold, you suffer 2 wounds; if triple your Wound Threshold, you suffer 3 wounds; and so on. 

Wounds are cumulative, and must be marked on your character sheet. 

Note that these rules handle damage and wounds as an abstract concept. For drama and realism, gamemasters may wish to describe wounds in more detailed and grisly terms: a broken ankle, a severed tendon, internal bleeding, a lost ear, and so on. The nature of such descriptive injuries may help the gamemaster assign other effects. For example, a character with a crushed hand may not be able to pick up a gun, someone with excessive blood loss may leave a trail for their enemies to follow, or someone with a cut eye may suffer an additional visual perception modifier. Likewise, such details may impact how a character is treated or heals. 

\subsubsection{Wound effects} 

Each wound applies a cumulative -10 modifier to all of the character’s actions. A character with 3 wounds, for example, suffers -30 to all actions. Some traits, morphs, implants, drugs, and psi allow a character to ignore wound modifiers. These effects are cumulative, though the maximum amount of wound modifiers that may be negated is -30. 

\textbf{Knockdown:} Any time a character takes a wound, they must make an immediate SOM $\times$ 3 Test. Wound modifiers apply. If they fail, they are knocked down and must expend a Quick Action to get back up. Bots and vehicles must make a Pilot Test to avoid crashing. 

\textbf{Unconsciousness:} Any time a character receives 2 or more wounds at once (from the same attack), they must also make an immediate SOM $\times$ 3 Test; wound modifiers again apply. If they fail, they have been knocked unconscious (until they are awoken or heal). Bots and vehicles that take 2 or more wounds at once automatically crash (see \emph{Crashing}, p. 196). 

\textbf{Bleeding:} Any biomorph character who has suffered a wound and who takes damage that exceeds their Durability is in danger of bleeding to death. They incur 1 additional damage point per Action Turn (20 per minute) until they receive medical care or die. 

\subsection{Death} For many people in \emph{Eclipse Phase}, death is not the end of the line. If the character’s cortical stack can be retrieved, they can be resurrected and downloaded into a new morph (see \emph{Resleeving}, p. 271). This typically requires either backup insurance (p. 269) or the good graces of whomever ends up with their body/stack. 

If the cortical stack is not retrievable, the character can still be re-instantiated from an archived backup (p. 268). Again, this either requires backup insurance or someone who is willing to pay to have them revived. 

If the character’s cortical stack is not retrieved and they have no backup, then they are completely and utterly dead. Disparues. Kaput. (Unless they happen to have an alpha fork of themselves floating around somewhere; see \emph{Forking and Merging}, p. 273.) 



\section{Healing and repair} \label{sec:healing-repair} 

Use the follow rules for healing and repairing damaged and wounded characters. 

\subsection{Biomorph healing} 

Thanks to advanced medical technologies, there are many ways for characters in biological morphs (including pods) to heal injuries. Medichine nanoware (p. 308) helps characters to heal quickly, as do nanobandages (p. 333). Healing vats (p. 326) will heal even the most grievous wounds in a matter of days, and can even restore characters who recently died or have been reduced to just a head. 

Characters without access to these medical tools are not without hope, of course. The medical skills of a trained professional can abate the impact of wounds, and over time bodies will of course heal themselves. 

\subsubsection{Medical care} 

Characters with an appropriate Medicine skill (such as Medicine: Paramedic or Medicine: Trauma Surgery) can perform first aid on damaged or wounded characters. A successful Medicine Test, modified as the gamemaster deems fit according to situational conditions, will heal 1d10 points of damage and will remove 1 wound. This test must be made within 24 hours of the injury, and any particular injury may only be treated once. If the character is later injured again, however, this new damage may also be treated. Medical care of this sort is not effective against injuries that have been treated with medichines, nanobandages, or healing vats. 

\subsubsection{Natural healing} 

Characters trapped far from medical technology --- in a remote station, the wilds of Mars, or the like --- may be forced to heal naturally if injured. Natural healing is a slow process that’s heavily influenced by a number of factors. In order for a character to heal wounds, all normal damage must be healed first. Consult the Healing table. 

\subsubsection{Surgery} 

In \emph{Eclipse Phase}, most grievous injuries can be handled by time in a healing vat (p. 326) or simply rest and recovery. In circumstances where a healing vat is not available, the gamemaster may decide that a particular wound requires actual surgery from an intelligent being (whether a character or AI-driven medbot). Usually in this case the character will be incapable of further healing until the surgery occurs. The surgery is handled as a Medical Test using a field appropriate to the situation and with a timeframe of 1-8 hours. If successful, the character is healed of 1d10 damage and 1 wound and recovers from that point on as normal. 

\begin{table} \begin{tabularx}{\textwidth}{|X|r|r|} \hline

\multicolumn{3}{|c|}{\textbf{Healing}} \\ \hline

\textbf{Character situation}	&\textbf{Damage healing rate}	&\textbf{Wound healing rate} \\ \hline

Character without basic biomods	&1d10 (5) per day &1 per week	\\ \hline

Character with basic biomods	&1d10 (5) per 12 hours	&1 per 3 days	\\ \hline

Character using nanobandage	&1d10 (5) per 2 hours	&1 per day	\\ \hline

Character with medichines	&1d10 (5) per 1 hour	&1 per 12 hours	\\ \hline

Poor conditions (bad food, not enough rest/heavy activity, poor shelter and/or sanitation) &double timeframe &double timeframe \\ \hline

Harsh conditions (insufficient food, no rest/strenuous activity, little or no shelter and/or sanitation) &triple timeframe &no wound healing \\ \hline

\end{tabularx} \label{tab:healing} \end{table} 



\subsection{Synthmorph and Object repair} 

Unlike biomorphs, synthetic morphs and objects do not heal damage on their own and must be repaired. Some synthmorphs and devices have advanced nanotech selfrepair systems, similar to medichines for biomorphs (see Fixers, p. 329). Repair spray (p. 333) may also be used to conduct fixes and is an extremely useful option for non-technical people. Barring these options, technicians may also work repairs the old-fashioned way, using their skills and tools (see Physical Repairs, below). As a last resort, synthmorphs and objects may be repaired in a nanofabrication machine with the appropriate blueprints (using the same rules as healing vats, p. 326). 

\subsubsection{Physical repairs} 

Manually fixing a synthmorph or object requires a Hardware Test using a field appropriate to the item (Hardware: Robotics for synthmorphs and bots, Hardware: Aerospace for aircraft, etc.), with a -10 modifier per wound. Repair is a Task Action with a timeframe of 2 hours per 10 points of damage being restored, plus 8 hours per wound. Appropriate modifiers should be applied, based on conditions and available tools. For example, utilitools (p. 326) apply a +20 modifier to repair tests, while repair spray applies a +30 modifier. 

\subsubsection{Repairing armor} 

Armor may be repaired in the same manner as Durability, however, wounds do not impact the test with modifiers or extra time. 

\section{Mental health} \label{sec:mental-health} 

In a time when people can discard bodies and replace them with new ones, trauma inflicted on your mind and ego --- your sense of \emph{self} --- is often more frightening than grievous physical harm. There are many ways in which your sanity and mental wholeness can be threatened: experiencing physical death, extended isolation, loss of loved ones, alien situations, discontinuity of self from lost memories or switching morphs, psi attack, and so on. Two methods are used to gauge your mental health: \emph{stress points} and \emph{trauma}. 



\subsection{Stress points} \label{sec:stress-points} 

Stress points represent fractures in your ego’s integrity, cracks in the mental image of yourself. This mental damage is experienced as cerebral shocks, disorientation, cognitive disconnects, synaptic misfires, or an undermining of the intellectual faculties. On their own, these stress points do not significantly impair your character’s functioning, but if allowed to accumulate they can have severe repercussions. Additionally, any source that inflicts a large amount of stress points at once is likely to have a more severe impact (see \emph{Trauma}). 

Stress points may be reduced by long-term rest, psychiatric care, and/or psychosurgery (see p. 214). 



\subsection{Lucidity and stress} \label{sec:lucidity-stress} 

Your Lucidity stat benchmarks your character’s mental stability. If you build up an amount of stress points equal to or greater than your Lucidity score, your character’s ego immediately suffers a mental breakdown. You effectively go into shock and remain in a catatonic state until your stress points are reduced to a level below your Lucidity stat. Accumulated stress points will overwhelm egos housed inside synthetic shells or infomorphs just as they will biological brains --- the mental software effectively seizes up, incapable of functioning until it is debugged. 

\subsubsection{Insanity rating} 

Extreme amounts of built-up stress points can permanently damage your character’s sanity. If accumulated stress points reach your Lucidity $\times$ 2, your character’s ego undergoes a permanent meltdown. Your mind is lost, and no amount of psych help or rest will ever bring it back. 



\subsection{Stress value} \label{sec:stress-value} 

Any source capable of inflicting cognitive stress is given a Stress Value (SV). This indicates the amount of stress points the attack or experience inflicts upon a character. Like DV, SV is often presented as a variable amount, such as 2d10, or sometimes with a modifier, such as 2d10 + 10. Simply roll the dice and total the amounts to determine the stress points inflicted in that instance. To make things easier, a static SV is also given in parentheses after the variable amount; use that set amount when you wish to keep the game moving and don’t want to roll dice. 



\subsection{Trauma} \label{sec:trauma} 

Mental trauma is more severe than stress points. Traumas represent severe mental shocks, a crumbling of personality/self, delirium, paradigm shifts, and other serious cognitive malfunctions. Traumas impair your character’s functioning and may result in temporary derangements or permanent disorders. 

If your character receives a number of stress points at once that equals or exceeds their Trauma Threshold, they have suffered a trauma. If the inflicted stress points are double or triple the Trauma Threshold, they suffer 2 or 3 traumas, respectively, and so on. Traumas are cumulative and must be recorded on your character sheet. 

\subsubsection{Trauma effects} 

Each trauma applies a cumulative -10 modifier to all of the character’s actions. A character with 2 traumas, for example, suffers -20 to all actions. These modifiers are also cumulative with wound modifiers. 

\textbf{Disorientation:} Any time a character suffers a trauma, they must make an immediate WIL $\times$ 3 Test. Trauma modifiers apply. If they fail, they are temporarily stunned and disoriented, and must expend a Complex Action to regain their wits. 

\textbf{Derangements and Disorders:} Any time a character is hit with a trauma, they suffer a temporary derangement (see \emph{Derangements}). The first trauma inflicts a \emph{minor} derangement. If a second trauma is applied, the first derangement is either upgraded from minor to a \emph{moderate} derangement, or else a second minor derangement is applied (gamemaster’s discretion). Likewise, a third trauma may upgrade that derangement from moderate to \emph{major} or else inflict a new minor. It is generally recommended that derangements be upgraded in potency, especially when result from the same set of ongoing circumstances. In the case of traumas that result from distinctly separate situations and sources, separate derangements may be applied. 

\textbf{Disorder:} When four or more traumas have been inflicted on a character, a major derangement is upgraded to a disorder. Disorders represent long-lasting psychological afflictions that typically require weeks or even months of psychotherapy and/or psychosurgery to remedy (see \emph{Disorders}, p. 211). 



\subsection{Derangements} \label{sec:derangements} 

Derangements are temporary mental conditions that result from traumas. Derangements are measured as Minor, Moderate, or Major. The gamemaster and player should cooperate in choosing which derangement to apply, as appropriate to the scenario and character personality. 

Derangements last for 1d10 $\div$ 2 hours (round down), or until the character receives psychiatric help, whichever comes first. At the gamemaster’s discretion, a derangement may last longer if the character has not been distanced from the source of the stress, or if they remain embroiled in other stress-inducing situations. 

Derangement effects are meant to be role-played. The player should incorporate the derangement into their character’s words and actions. If the gamemaster doesn’t feel the player is stressing the effects enough, they can emphasize them. If the gamemaster feels it is appropriate, they may also call for additional modifiers or tests for certain actions. 

\subsubsection{Anxiety (minor)} 

You suffer a panic attack, exhibiting the physiological conditions of fear and worry: sweatiness, racing heart, trembling, shortness of breath, headaches, and so on. 

\subsubsection{Avoidance (minor)} 

You are psychologically incapable with dealing with the source of the stress, or some circumstance related to it, so you avoid it --- even covering your ears, curling up in a ball, or shutting off your sensors if you have to. 

\subsubsection{Dizziness (minor)} 

The stress makes you light-headed and disoriented. 

\subsubsection{Echoalia (minor)} 

You involuntarily repeat words and phrases spoken by others. 

\subsubsection{Fixation (minor)} 

You become fixated on something that you did wrong or some circumstance that led to your stress. You obsess over it, repeating the behavior, trying to fix it, running scenarios through your head and out loud, and so on. 

\subsubsection{Hunger (minor)} 

You are suddenly consumed by an irrational yet overwhelming desire to eat something --- perhaps even something unusual. 

\subsubsection{Indecisiveness (minor)} 

You are flustered by the cause of your stress, finding it difficult to make choices or select courses of action. 

\subsubsection{Logorrhoea (minor)} 

Your response to the trauma is to engage in excessive talking and babbling. You don’t shut up. 

\subsubsection{Nausea (minor)} 

The stress sickens you, forcing you to fight down queasiness. 

\subsubsection{Chills (moderate)} 

Your body temperature rises, making you feel cold, and shivering sets in. You just can’t get warm. 

\subsubsection{Confusion (moderate)} 

The trauma scrambles your concentration, making you forget what you’re doing, mix up simple tasks, and falter over easy decisions. 

\subsubsection{Echpraxia (moderate)} 

You involuntarily repeat and mimic the actions of others around you. 

\subsubsection{Mood swings (moderate)} 

You lose control of your emotions. You switch from ecstasy to tears and back to rage without warning. 

\subsubsection{Mute (moderate)} 

The trauma shocks you into speechlessness and a complete inability to effectively communicate. 

\subsubsection{Narcissism (moderate)} 

In the wake of the mental shock, all you can think about is yourself. You cease caring about those around you. 

\subsubsection{Panic (moderate)} 

You are overwhelmed by fear or anxiety and immediately seek to distance yourself from the cause of the stress. 

\subsubsection{Tremors (moderate)} 

You shake violently, making it difficult to hold things or stay still. 

\subsubsection{Blackout (major)} 

You operate on auto-pilot in a temporary fugue state. Later, you will be incapable of recalling what happened during this period. (Synthetic shells and infomorphs may call up memory records from storage.) 

\subsubsection{Frenzy (major)} 

You have a major freak out over the source of the stress and attack it. 

\subsubsection{Hallucinations (major)} 

You see, hear, or otherwise sense things that aren’t really there. 

\subsubsection{Hysteria (major)} 

You lose control, panicking over the source of the stress. This typically results in an emotional outburst of crying, laughing, or irrational fear. 

\subsubsection{Irrationality (major)} 

You are so jarred by the stress that your capacity for logical judgment breaks down. You are angered by imaginary offenses, hold unreasonable expectations, or otherwise accept things with unconvincing evidence. 

\subsubsection{Paralysis (major)} 

You are so shocked by the trauma that you are effectively frozen, incapable of making decisions or taking action. 

\subsubsection{Psychosomatic crippling (major)} 

The trauma overwhelms you, impairing some part of your physical functioning. You suffer from an inexplicable blindness, deafness, or phantom pain, or are suddenly incapable of using a limb or other extremity. 



\subsection{Disorders} \label{sec:disorders} 

Disorders reflect more permanent madness. In this case, ``permanent” does not necessarily mean forever, but the condition is ongoing until the character has received lengthy and effective psychiatric help. Disorders are inflicted whenever a character has accumulated 4 traumas. The gamemaster and player should choose a disorder that fits the situation and character. 

Disorders are not always ``active'' --- they may remain dormant until triggered by certain conditions. While it is certainly possible to act under a disorder, it represents a severe impairment to a person’s ability to maintain normal relationships and do a job successfully. Disorders should not be glamorized as cute role-playing quirks. They represent the best attempts of a damaged psyche to deal with a world that has failed it in some way. Additionally, people in many habitats, particularly those in the inner system, still regard disorders as a mark of social stigma and may react negatively towards impaired characters. 

Characters that acquire disorders over the course of their adventures may get rid of them in one of two ways, either through in-game attempts to treat them (p. 214) or by buying them off as they would a negative trait (p. 153). 

\subsubsection{Addiction} 

Addiction as a disorder can refer to any sort of addictive behavior focused toward a particular behavior or substance, to the point where the user is unable to function without the addiction but is also severely impaired due to the effects of the addiction. It is marked by a desire on the part of the subject to seek help or reduce the use of the addicting substance/act, but also by the subject spending large amounts of time in pursuit of their addiction to the exclusion of other activities. This is a step up from Addiction negative trait listed on p. 148 --- this is much more of a crippling behavior that compensates for spending time away from the addiction. Addictions are typically related to the trauma that caused the disorder (VR or drug addictions are encouraged). 

\textbf{Suggested Game Effects:} The addict functions in only two states: under the influence of their addiction or in withdrawal. Additionally, they spend large amounts of time away from their other responsibilities in pursuit of their addiction. 

\subsubsection{Atavism} 

Atavism is a disorder that mainly affects uplifts. It results in them regressing to an earlier un- or partially- uplifted state. They may exhibit behaviors more closely in line with their more animalistic forbears, or they may lose some of their uplift benefits such as the ability for abstract reasoning or speech. 

\textbf{Suggested Game Effects:} The player and gamemaster should discuss how much of the uplift’s nature is lost and adjust game penalties accordingly. It is important to note that other uplifts view atavistic uplifts with something akin to horror and will usually have nothing to do with them. 

\subsubsection{Attention deficit hyperactivity disorder (ADHD)} 

This disorder manifests as a marked inability to focus on any one task for an extended period of time, and also an inability to notice details in most situations. Sufferers may find themselves starting multiple tasks, beginning a new one after only a cursory attempt at the prior task. ADHD suffers may also have a manic edge that manifests as confidence in their ability to get a given job done, even though they will quickly lose all interest in it. 

\textbf{Suggested Game Effects:} Perception and related skill penalties. Increased difficulty modifiers to task actions, particularly as the action drags on. 

\subsubsection{Autophagy} 

This is a disorder that usually only occurs among uplifted octopi. It is a form of anxiety disorder characterized by self-cannibalism of the limbs. Subjects afflicted with autophagy will, under stress, begin to consume their limbs, if at all possible, causing themselves potentially serious harm. 

\textbf{Suggested Game Effects:} Anytime an uplifted octopi with this disorder is placed in a stressful situation they must make a successful WIL $\times$ 3 Test or begin to consume one of their limbs. 

\subsubsection{Bipolar disorder} 

Bipolar disorder is also called manic depression. It is similar to depression except for the fact that the periods of depression are interrupted by brief (a matter of days at most) periods of mania where the subject feels inexplicably ``up” about everything with heightened energy and a general disregard for consequences. The depressive stages are similar in all ways to depression. The manic stages are dangerous since the subject will take risks, spend wildly, and generally engage in behavior without much in the way of forethought or potential long term consequences. 

\textbf{Suggested Game Effects:} Similar to depression, but when manic the character must make a WIL $\times$ 3 Test to not do some action that may be potentially risky. They will also try to convince others to go along with the idea. 

\subsubsection{Body dysmorphia} 

Subjects afflicted with this disorder believe that they are so unspeakably hideous that they are unable to interact with others or function normally for fear of ridicule and humiliation at their appearance. They tend to be very secretive and reluctant to seek help because they are afraid others will think them vain --- or they may feel too embarrassed to do so. Ironically, BDD is often misunderstood as a vanity-driven obsession, whereas it is quite the opposite; people with BDD believe themselves to be irrevocably ugly or defective. A similar disorder, gender identity disorder, where the patient is upset with their entire sexual biology, often precipitates BDD-like feelings. Gender identity disorder is directed specifically at external sexually dimorphic features, which are in constant conflict with the patient’s internal psychiatric gender. 

\textbf{Suggested Game Effects:} Because of the nature of Eclipse Phase and the ability to swap out and modify a body, this is a fairly common disorder. It is suggested that characters with this suffer increased or prolonged resleeving penalties since they are unable to fully adjust to the reality of their new morph. 

\subsubsection{Borderline personality disorder} 

This disorder is marked by a general inability to fully experience one’s self any longer. Emotional states are variable and often marked by extremes and acting out. Simply put, the subject feels like they are losing their sense of self and seeks constant reassurance from others around them, yet is not fully able to act in an appropriate way. They may also engage in impulsive behaviors in an attempt to experience some sort of feeling. In extreme cases, there may be suicidal thoughts or attempts. 

\textbf{Suggested Game Effects:} The character needs to be around others and will not be left alone, however they also are not quite able to relate to others in a normal way and may also take risks or make impulsive decisions. 

\subsubsection{Depression} 

Clinical depression is characterized by intense feelings of hopelessness and worthlessness. Subjects usually report feeling as though nothing they do matters and no one would care anyway, so they are disinclined to attempt much in the way of anything. The character is depressed and finds it difficult to be motivated to do much of anything. Even simple acts such as eating and bathing can seem to be monumental tasks. 

\textbf{Suggested Game Effect:} Depressives often lack the will to take any sort of action, often to the point of requiring a WIL $\times$ 3 Test to engage in sustained activity. 

\subsubsection{Fugue} 

The character enters into a fugue state where they display little attention to external stimuli. They will still function physiologically but refrain from speaking and stare off into the distance, unable to focus on events around them. Unlike catatonia, a person in a fugue state will walk around if lead about by a helper, but is otherwise unresponsive. The fugue state is usually a persistent state, but it can be an occasional state that is triggered by some sort of external stimuli similar to the original trauma that triggered the disorder. 

\textbf{Suggested Game Effects:} Characters in a fugue state are totally non-responsive to most stimuli around them. They will not even defend themselves if attacked and will usually attempt to curl into a fetal position if physically assaulted. 

\subsubsection{General anxiety disorder (GAD)} 

GAD results in severe feelings of anxiety about nearly everything the character comes into contact with. Even simple tasks represent the potential for failure on a catastrophic scale and should be avoided or minimized. Additionally, negative outcomes for any action are always assumed to be the only possible outcomes. 

\textbf{Suggested Game Effects:} A character with GAD will be almost entirely useless unless convinced otherwise, and then only for a short period of time. Another character can attempt to use a relevant social skill to coax the GAD character into doing what is required of them. If the character with the disorder fails at the task, however, all future attempts to coax them will suffer a cumulative -10 penalty. 

\subsubsection{Hypochondria} 

Hypochondriacs suffer from a delusion that they are sick in ways that they are not. They will create disorders that they believe they suffer from, usually to get the attention of others. Often hypochondriacs will inflict harm on themselves or even ingest substances that will aid in producing symptoms similar to the disorder they believe they have. These attempts to simulate symptoms can and will cause actual harm to hypochondriacs. 

\textbf{Possible Game Effects:} A subject that is hypochondriac will often behave as though they are under the effects of some other disorder or physical malady. This can be consistent over time or can be different and ever changing. They will react with hostility to claims that they are faking or not actually ill. 

\subsubsection{Impulse control disorder} 

Subjects have a certain act or belief that they must engage in a certain activity that comes into their mind. This can be kleptomania, pyromania, sexual exhibitionism, etc. They feel a sense of building anxiety whenever they are prevented from engaging in this behavior for an extended period (usually several times a day to weekly, depending on the impulse) and will often attempt to engage in this behavior at inconvenient or inappropriate times. This is different from OCD in the sense that OCD is usually a single contained behavior that must be engaged in to reduce anxiety. Impulse control disorder is a variety of behaviors and can be virtually any sort of highly inappropriate action. 

\textbf{Suggested Game Effects:} Similar to OCD, if the character doesn’t engage in the behavior they will grow increasingly disturbed and suffer penalties to all actions until they are able to engage in the compulsion that alleviates their anxiety. 

\subsubsection{Insomnia} 

Insomniacs find themselves unable to sleep, or unable to sleep for an extended period of time. This is most often due to anxiety about their lives or as a result of depression and the accompanying negative thought patterns. This is not the sort of sleeplessness that is brought about as a result of normal stress but rather a near total inability to find rest in sleep when it is desired. Insomniacs may find themselves nodding off at inopportune times, but never for long, and never enough to gain any restful sleep. As a result, they are frequently lethargic and inattentive as their lack of sleep robs them of their edge and eventually any semblance of alertness. Additionally, insomniacs are frequently irritable due to being on edge and unable to rest. 

\textbf{Suggested Game Effects: }Due to the lack of meaningful sleep, insomniacs should suffer from blanket penalties to perception related tasks or anything requiring concentration or prolonged fine motor abilities. 

\subsubsection{Megalomania} 

A megalomaniac believes themselves to be the single most important person in the universe. Nothing is more important than the megalomaniac and everything around them must be done according to their whim. Failure to comply with the dictates of a megalomaniac can often result in rages or actual physical assaults by the subject. 

\textbf{Suggested Game Effects:} A character that has megalomania will demand attention and has diffi- culty in nearly any social situation. Additionally, they may be provoked to violence if they think they are being slighted. 

\subsubsection{Multiple personality disorder} 

This is the development of a separate, distinct personality from the original or control personality. The personalities may or may not be aware of each other and ``conscious” during the actions of the other personality. Usually there is some sort of trigger that results in the emergence of the non-control personality. Most subjects have only a single extra personality, but it is not unheard of to have several personalities. It is important to note that these are distinct individual personalities and not just crude caricatures of the Dr. Jekyll/Mr. Hyde sort. Each personality sees itself as a distinct person with their own wants, needs, and motivations. Additionally, they are usually unaware of the experiences of the others, though there is some basic information sharing (such as language and core skill sets). 

\textbf{Suggested Game Effects:} When the player is under the effects of another personality, they should be treated as an NPC. In some rare cases the player and the gamemaster can work out the second personality and allow the player to roleplay this. This does not however constitute an entire new character that can be ``turned on” at will. 

\subsubsection{Obsessive compulsive disorder (OCD)} 

Subjects with OCD are marked by intrusive or inappropriate thoughts or impulses that cause acute anxiety if a particular obsession or compulsion is not engaged in to alleviate them. These obsessions and compulsions can be nearly any sort of behavior that must be immediately engaged in to keep the rising anxiety at bay. Players and gamemasters are encouraged to come up with a behavior that is suitable. Examples of common behaviors include repetitive tics (touching every finger of each hand to another part of the body, tapping the right foot twenty times), pathological behaviors such as gambling or eating, or a mental ritual that must be completed (reciting a book passage). 

\textbf{Suggested Game Effects:} If the character doesn’t engage in the behavior they will grow increasingly disturbed and suffer penalties to all actions until they are able to engage in the compulsion that alleviates their anxiety. 

\subsubsection{Post traumatic stress disorder (PTSD)} 

PTSD occurs as a result of being exposed to either a single incident or a series of incidents where the sufferer had their own life, or saw the lives of others, threatened with death. These incidents are often marked by an inability on the part of the victim, either real or perceived, to do anything to alter the outcomes. As a result, they develop an acute anxiety and fixation on these incidents to the point where they lose sleep, become irritated or easily angered, or are depressed over feelings that they lack control in their own lives. 

\textbf{Suggested Game Effects:} Penalties to task actions, also treat situations similar to the initial episodes that caused the disorder as a phobia. 

\subsubsection{Schizophrenia} 

While schizophrenia is generally acknowledged as a genetic disorder that has an onset in early adulthood, it also seems to develop in a number of egos that undergo frequent morph changes. It has been theorized that this is due to some sort of repetitive error in the download process. Regardless, it remains a rare, yet persistent danger of dying and being brought back. Schizophrenia is a psychotic disorder where the subject loses their ability to discern reality from unreality. This can involve delusions, hallucinations (often in support of the delusions), and fragmented or disorganized speech. The subject will not be aware of these behaviors and will perceive themselves as functioning normally, often to the point of becoming paranoid that others are somehow involved in a grand deception. 

\textbf{Suggested Game Effects:} Schizophrenia represents a total break from reality. A character that is schizophrenic may see and hear things and act on those delusions and hallucinations while seeing attempts by their friends to stop or explain to them as part of a wider conspiracy. Adding to this is the difficulty of communicating coherently. Characters that have become schizophrenic are only marginally functional and only for short periods of time until they have the disorder treated. 



\subsection{Stressful situations} \label{sec:stressful-situations} 

The universe of \emph{Eclipse Phase} is ripe with experiences that might rattle a character’s sanity. Some of these are as mundane and human as extreme violence, extended isolation, or helplessness. Others are less common, but even more terrifying: encountering alien species, infection by the Exsurgent virus, or being sleeved inside a non-human morph. 



\subsection{Willpower stress tests} \label{sec:willpower-stress-tests} 

Whenever a character encounters a situation that might impact their ego’s psyche, the gamemaster may call for a (Willpower $\times$ 3) Test. This test determines if the character is able to cope with the unnerving situation or if the experience scars their mental landscape. If they succeed, the character is shaken but otherwise unaffected. If they fail, they suffer stress damage (and possibly trauma) as appropriate to the situation. A list of stress-inducing scenarios and suggested SVs are listed on the Stressful Experiences table, p. 215. The gamemaster should use these as a guideline, modifying them as appropriate to the situation at hand. 

Note that some incidents may be so horrific that a modifier is applied to the character’s (Willpower $\times$ 3) Test. 

\begin{table} \begin{tabularx}{\textwidth}{|X|l|} \hline

\multicolumn{2}{|c|}{\textbf{Stressful Experiences} } \\ \hline

\textbf{Situation}	&\textbf{SV} \\ \hline

Failing spectacularly in pursuit of a motivational goal	&1d10 $\div$ 2 (round down)	\\ \hline

Helplessness	&1d10 $\div$ 2 (round down)	\\ \hline

Betrayal by a trusted friend	&1d10 $\div$ 2 (round down)	\\ \hline

Extended isolation	&1d10 $\div$ 2 (round down)	\\ \hline

Extreme violence (viewing)	&1d10 $\div$ 2 (round down)	\\ \hline

Extreme violence (committing)	&1d10	\\ \hline

Awareness that your death is imminent	&1d10	\\ \hline

Experiencing someone’s death via XP	&1d10	\\ \hline

Losing a loved one	&1d10 $\div$ 2 (round down)	\\ \hline

Watching a loved one die	&1d10 + 2	\\ \hline

Being responsible for the death of a loved one	&1d10 + 5	\\ \hline

Encountering a gruesome murder scene	&1d10	\\ \hline

Torture (viewing)	&1d10 + 2	\\ \hline

Torture (moderate suffering)	&2d10 + 3	\\ \hline

Torture (severe suffering)	&3d10 + 5	\\ \hline

Encountering aliens (non-sentient)	&1d10 $\div$ 2 (round down)	\\ \hline

Encountering aliens (sentient)	&1d10	\\ \hline

Encountering hostile aliens	&1d10 + 3	\\ \hline

Encountering highly-advanced technology	&1d10 $\div$ 2 (round down)	\\ \hline

Encountering Exsurgent-modified technology	&1d10 $\div$ 2 (round down)	\\ \hline

Encountering Exsurgent-infected transhumans	&1d10	\\ \hline

Encountering Exsurgent life forms	&1d10 + 3	\\ \hline

Exsurgent virus infection	&Varies; see p. 366	\\ \hline

Witnessing psi-epsilon sleights	&1d10 + 2	\\ \hline

\end{tabularx} \label{tab:stressful-experiences} \end{table} 



\subsection{Hardening} \label{sec:hardening} 

The more you are exposed to horrible or terrifying things, the less scary they become. After repeated exposure, you become hardened to such things, able to shake them off without effect. 

Every time you succeed in a Willpower Test to avoid taking stress from a particular source, take note. If you successfully resist such a situation 5 times, you become effectively immune to taking stress from that source. 

The drawback to hardening yourself to such situations is that you grow detached and callous. In order to protect yourself, you have learned to cut off your emotions --- but it is such emotions that make you human. You have erected mental walls that will affect your empathy and ability to relate to others. 

Each time you harden yourself to one source of stress, your maximum Moxie stat is reduced by 1. Psychotherapy may be used to overcome such hardening, in the same way a disorder is treated. 



\subsection{Mental healing and psychotherapy} \label{sec:mental-healing-psychotherapy} 

Stress is trickier to heal than physical damage. There are no nano-treatments or quick fix options (other than killing yourself and reverting to a non-stressed backup). The options for recuperating are simply natural healing over time, psychotherapy, or psychosurgery. 

\subsubsection{Phychotherapy care} 

Characters with an appropriate skill --- Medicine: Psychiatry, Academics: Psychology, or Professional: Psychotherapy --- can assist a character suffering mental stress or trauma with psychotherapy. This treatment is a long-term process, involving methods such as psychoanalysis, counseling, roleplaying, relationship-building, hypnotherapy, behavioral modification, drugs, medical treatments, and even psychosurgery (p. 229). AIs skilled in psychotherapy are also available. 

Psychotherapy is a task action, with a timeframe of 1 hour per point of stress, 8 hours per trauma, and 40 hours per disorder. Note that this only counts the time actually spent in psychotherapy with a skilled professional. After each psychotherapy session, make a test to see if the session was successful. Successful psychosurgery adds a +30 modifier to this test; at the gamemaster’s discretion, other modifiers may apply. Likewise, each disorder the character holds inflicts a -10 modifier. Traumas may not be healed until all stress is eliminated. 

When a trauma is healed, the derangement associated with that trauma is eliminated or downgraded. Disorders are treated separately from the trauma that caused them, and may only be treated when all other traumas are removed. 

Gamemaster and players are encouraged to roleplay a character’s suffering and relief from traumas and disorders. Each is an experience that makes a profound impact on a character’s personality and psyche. The process of treatment may also change them, so in the end they may be a transformed from the person they once were. Even if treated, the scars are likely to remain for some time to come. According to some opinions, disorders are never truly eradicated, they are just eased into submission ... where they may linger beneath the surface, waiting for some trauma to come along. 

\subsubsection{Natural healing} 

Characters who eschew psychotherapy can hopefully work out the problems in their head on their own over time. For every month that passes without accruing new stress, the character may make a WIL $\times$ 3 Test. If successful, they heal 1d10 points of stress or 1 trauma (all stress must be healed first). Disorders are even more difficult to heal, requiring 3 months without stress or trauma, and even then only being eliminated with a successful WIL Test. As a result, disorders can linger for years until resolved with actual psychotherapy. 



