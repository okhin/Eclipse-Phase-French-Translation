<<<<<<< HEAD
<<<<<<< HEAD
\chapter{Enter the singularity} \label{chap:enter-the-singularity} 

<<<<<<< HEAD
We humans have a special way of pulling ourselves up and kicking ourselves down at the same time. We'd achieved more progress than ever before, at the cost of wrecking our planet and destabilizing our own governments. But things were starting to look up. 

With exponentially accelerating technologies, we reached out into the solar system, terraforming worlds and seeding new life. We re-forged our bodies and minds, casting off sickness and death. We achieved immortality through the digitization of our minds, resleeving from one biological or synthetic body to the next at will. We uplifted animals and AIs to be our equals. We acquired the means to build anything we desired from the molecular level up, so that no one need want again. 

Yet our race toward extinction was not slowed, and in fact received a machine-assist over the precipice. Billions died as our technologies rapidly bloomed into something beyond control ... further transforming humanity into something else, scattering us throughout the solar system, and reigniting vicious conflicts. Nuclear strikes, biowarfare plagues, nanoswarms, mass uploads ... a thousand horrors nearly wiped humanity from existence. 

We still survive, divided into a patchwork of restrictive inner system hypercorp-backed oligarchies and libertarian outer system collectivist habitats, tribal networks, and new experimental societal models. We have spread to the outer reaches of the solar system and even gained footholds in the galaxy beyond. But we are no longer solely ``human'' ... we have evolved into something simultaneously more and different— something transhuman. 
=======
Nous autres humains avons une manière très particulière de nous hisser vers le haut tout en se mettant des bâtons dans les roues. Nous avons accompli plus de progrès que jamais, au prix de la destruction de notre planète et de la déstabilisation de nos gouvernements. Mais ça n'était que le début. 

Avec une technologie progressant à une vitesse exponentielle, nous avons atteint les limites du système solaire. Nous avons re-forgé nos corps et nos esprits, abandonnant la maladie et la mort. Nous avons atteint l'immortalité grâce à la numérisation de nos esprists, passant d'un corps biologique ou synthétique au suivant. À volonté. Nous avons donné la conscience à des animaux et aux IA afin d'en faire nos égaux. Nous avons acquis les moyens d'assembler au niveau moléculaire tout ce que nous désirions, pour que plus personne ne soit jamais dans le besoin. 

Et cependant notre trajectoire vers l'extinction n'as pas été ralentie, et à même reçu une assistance des machines pour nous pousser dans le précipice. Des milliard sont mort alors que quelque chose hors de contrôle est sorti de l'œuf de notre technologie ... transformant encore l'humanité en quelque chose d'autre, nous dispersant à travers le système solaire, et rallumant les anciens conflits. Frappes nucléaires, armes biologiques, nuées de nanites, upload de masse ... un millier d'horreurs faillirent supprimer l'humanité de l'univers. 

Nous avons survécu, divisé entre le patchwork d'olligarchie restricitive des hypercorp du système intérieur d'un côté et des habitats libertariens et collectivistes, des réseaux tribaux et de nouveau modèles de sociétés expérimentaux dans le système extérieur de l'autre. Nous nous sommes propagés jusqu'à la limite extérieure du système solaire et nous avons même acquis quelques têtes de ponts dans la galaxie. Mais nous ne sommes plus seulement "humains" ... nous avons évolué en quelque chose qui est à la fois mieux et différent - quelque chose de transhumain. 
>>>>>>> e787973436878844dcbd819afabd2b0ac73eee89



\section{Starting out} \label{sec:starting-out} 

<<<<<<< HEAD
Eclipse Phase is a post-apocalyptic roleplaying game of transhuman conspiracy and horror. Humans are enhanced and improved, but humanity is battered and bitterly divided. Technology allows the re-shaping of bodies and minds and liberates us from material needs, but also creates opportunities for oppression and puts the capability for mass destruction in the hands of everyone. Many threats lurk in the devastated habitats of the Fall, dangers both familiar and alien. 
=======
Eclipse Phase est un jeux de rôle post-apocalyptique de conspiration et d'horreur transhumaniste. Les humains sont renforcés et améliorés, mais l'humanité est battue et profondément divisée. La technologie nous permet de retravailler nos corps et nos esprits et nous a libérée de nos besoins matériels, tout en créant des opportunités d'oppression et donnant à tout le monde des capacités de destruction massive. Des menaces rôdent dans les habitats dévastés par la Chute, des dangers à la fois familiers et étranger. 
>>>>>>> e787973436878844dcbd819afabd2b0ac73eee89



\subsection{What is a roleplaying game?} \label{sec:what-roleplaying} 

<<<<<<< HEAD
Have you ever read a book or seen a movie or a television show where a character does something really stupid, like heading into a basement at night when the character knows the serial killer is around? The whole time, you're thinking: ``I wouldn't walk down those creepy stairs to the dark basement, especially without a flashlight. I'd do X, Y, or Z instead!'' Since you're in the passenger's seat for the plot you're reading or watching, however, you simply have to sit back and let it unfold. 
=======
Avez-vous déjà lu un livre ou vu un film ou une série télévisée où l'un des personnages fait quelque chose de vraiment stupide, comme descendre à la cave de nuit quand le personnage sait qu'il y a un tueur en série dans le coin? A chaque fois, vous pensez: "Je ne descendrait jamais ces escaliers terrifiant dans cette cave obscure, surtout sans une lampe de poche. Je ferais plutôt ça, ou ça!" En étant un simple spectateur de l'intrigue que vous lisez ou regardez, vous ne pouvez que vous asseoir et laisser se dérouler l'histoire. 
>>>>>>> e787973436878844dcbd819afabd2b0ac73eee89

What if you could take hold of the driver's seat? What if you could take the plot in the direction you'd choose? That is the essence of a roleplaying game. 

<<<<<<< HEAD
A roleplaying game (or RPG, for short) is part improvisational theater, part storytelling, and part game. A single person (the gamemaster) runs the game for a group of players that pretend to be characters in a fictitious world. The world could be a mystery game set in the 1920s that takes you adventuring around the globe, a fantasy realm inhabited by dragons and trolls and sword-wielding barbarians, or a science fiction setting with aliens and spaceship and world-crushing weaponry. The players pick a setting that they find cool and want to play in. The players then craft their own characters, providing a detailed history and personality to bring each to life. These characters have a set of statistics (numerical values) that represent skills, attributes, and other abilities. The gamemaster then explains the situation in which the characters find themselves. The players, through their characters, interact with the storyline and each others' characters, acting out the plot. As the players roleplay through some scenarios, the gamemaster will probably ask a given player to roll some dice and the resulting numbers will determine the success or failure of a character's attempted action. The gamemaster uses the rules of the game to interpret the dice rolls and the outcome of the character's actions. 

As a group exercise, the players control the storyline (the adventure), which evolves much like any movie or book but within the flexible plot created by the gamemaster. This gamemaster plot provides a framework and ideas for potential courses of action and outcomes, but it is simply an outline of what might happen—it is not concrete until the players become involved. If you don't want to walk down those stairs, you don't. If you think you can talk yourself out of a situation in place of pulling a gun, then try and make it happen. The script of any roleplaying session is written by the players, and the story, based upon the character's actions and their responses to the events of the plot, will constantly change and evolve. 

The best part is that there is no ``right'' or ``wrong'' way to play an RPG. Some games may involve more combat and dice rolling-related situations, where other games may involve more storytelling and improvised dialogue to resolve a situation. Each group of players decides for themselves the type and style of game they enjoy playing! 
=======
Un jeu de rôle (ou JDR) est un mélange de théatre d'improvisation, d'art du conte et de jeu. Une personne (le maître de jeu) anime la partie pour un groupe de joueurs prétendant être des personnages dans un monde fictif. Le monde peu aussi bien être un décor mystérieux dans les années 1920 qui vous emmènerait dans des aventures autour du globe, un royaume mystérieux habité par des dragons, des trolls et des barbares maniant l'épée ou un cadre de science fiction avec des aliens, des vaisseaux spatiaux et une artillerie capable de détruire des planètes. Les joueurs choisissent un cadre qu'ils trouvent intéressant et dans lequel ils veulent jouer. Ils créent ensuite leurs propres personnages, leur fournissant un historique détaillé et une personnalité pour leur donner la vie. Ces personnages ont un ensemble de statistqiues (des valeurs numériques) qui représentent les compétences, attributs et autres capacités. Le maître de jeu explique ensuite la situation dans laquelle se trouve les personnages. Les joueurs, par l'intermédiaire de leur personnage, interagissent avec les évènements et les personnages des autres, agissant sur l'intrigue. Alors que les joueurs interpètent leur rôle à travers le scénario, le maître de jeu demandera probablement à l'un des joueurs de lancer quelques dés dont le résultat déterminera la réussite ou l'échec de l'action qu'à tenté le personnage. Le maître de jeu utilise les règles du jeu pour interpréter le jet de dé et les conséquences de l'action du personnage. 

En tant qu'exercice de groupe, les joueurs contrôlent la suite d'évènement (l'aventure), qui évolue à peu près comme n'importe quel film ou livre mais à l'inétrieur de l'intrigue flexible créée par le maître de jeu. L'intrigue du maître de jeu fournit un cadre et des idées pour déterminer les possibilités d'actions et leurs impact, mais c'est plus simplement un aperçu de ce qui pourrait arriver - cela devient concret uniquement au moment où les joueurs sont impliqués. Si vous ne voulez pas descendre ces escaliers, vous ne les descendez pas. Si vous pensez pouvoir vous sortir d'une situation en discutant calmement plutôt qu'en dégainant une arme, alors allez-y et tentez votre chance. Le script de chaque session de jeu est écrit par les joueurs, et l'histoire, basée sur les actions des personnages et leur réponse aux évènement de l'intrigue, changera et évoluera en permanence. 

Le meilleur c'est qu'il n'y a pas de "bonne" ou "mauvaise" façon de jouer à un JDR. Certains jeu impliqueront plus de combat et de situation liées à des jets de dés, là où d'autre impliquerontplus d'interprétation et de dialogue improvisés pour résoudre une situation. Chaque groupe de joueur décide le type et le style de jeu auquel ils ont envie de jouer! 
>>>>>>> e787973436878844dcbd819afabd2b0ac73eee89



\subsection{What is transhumanism?} \label{sec:what-transhumanism} 

<<<<<<< HEAD
Transhumanism is a term used synonymously to mean ``human enhancement.'' It is an international cultural and intellectual movement that endorses the use of science and technology to enhance the human condition, both mentally and physically. In support of this, transhumanism also embraces using emerging technologies to eliminate the undesirable elements of the human condition such as aging, disabilities, diseases, and involuntary death. Many transhumanists believe these technologies will be arriving in our near future at an exponentially accelerated pace and work to promote universal access to and democratic control of such technologies. In the long scheme of things, transhumanism can also be considered the transitional period between the current human condition and an entity so far advanced in capabilities (both physical and mental faculties) as to merit the label ``posthuman.'' 

As a theme, transhumanism embraces heady questions. What defines human? What does it mean to defeat death? If minds are software, where do you draw the line with programming them? If machines and animals can also be raised to sentience, what are our responsibilities to them? If you can copy yourself, where does ``you'' end and someone new begin? What are the potentials of these technologies in terms of both oppressive control and liberation? How will these technologies change our society, our cultures, and our lives? 
=======
Le transhumanisme est un mot utilisé comme synonyme de "amélioration de l'humain." C'est un mouvement culturel et intellectuel international qui approuve l'utilisation de la science et de la technologie pour améliorer la condition humaine, aussi bien mentalement que physiquement. Pour supporter cette théorie, le transhumanisme englobe également l'utilisation de la technologie pour éliminer les éléments indésirables de la condition humaine tels que le vieillissement, les maladies et la mort non volontaire. Beaucoup de transhumains croient que ces technologies arriveront dans un futur proche suivant un rythme exponentiel et travaillent à promouvoir l'accès universel et le contrôle démocratique de ces technologies. Sur le long terme, le transhumanisme peut aussi être considéré comme la période de transition entre l'humain actuel et une entité qui aurait tellement évoluée dans ses capacités (aussi bien mentales que physiques) qu'elle mérite d'être appellée "posthumain". 

En tant que thème, la transhumanité brasse des questions lourde de significations. Qu'est-ce qui défini l'humain? Qu'est ce que signifie vaincre la mort? Si l'esprit est un logiciel, où se situe la limite avec leur programmation? Si les machines et les animaux peuvent être amenée à ressentir, quelles sont nos responsabilité vis à vis d'eux? Si vous pouvez vous copier, où se situe la limite entre "vous" et quelqu'un de nouveau? Quels sont les potentiels de ces technologie en terme de contrôle oppressif et de libération? Comment ces technologies changeront nos société, nos cultures et nos vies? 
>>>>>>> e787973436878844dcbd819afabd2b0ac73eee89



\subsection{Post-apocalyptic, conspiracy and horror themes} \label{sec:post-apoc-consp} 

Several themes pervade Eclipse Phase, some of which the reader may not be intimately familiar with. The following helps define these themes so that as play ers read further into this rulebook, they gain a solid understanding of how Eclipse Phase builds on such themes to create its unique setting. 

<<<<<<< HEAD
Post-apocalyptic is a term used to describe fiction set after a cataclysmic event has ended human civili zation as we know it (usually accompanied by loss of human life on an almost unthinkable scale). The exact mechanism of the disaster is usually unimportant nuclear war, plague, asteroid strike, and so on. The importance of the theme is the human condition. If the world we know is torn away from us and humans suffer horrors beyond imagining in this transforma tion to a post-apocalyptic setting, how does human ity cope? Do we survive and thrive and overcome? Or do we lose our own humanity in the process, o ultimately fall to extinction? Those are the questions that drive this genre. 

To conspire means ``to join in a secret agreement to do an unlawful or wrongful act or to use such means to accomplish a lawful end.'' As such, a con spiracy theory attributes the ultimate cause of an event or a chain of events (whether political, societa or historical) to a secret group of individuals with immense power (including political, wealth and so on) who hide their activities from public view while manipulating events to achieve their goals, regard less of consequences. Many conspiracy theories contend that a host of the greatest events of history were initiated and ultimately controlled by such secret organizations. Of equal importance is the silent struggle between clandestine groups, waging a secret war behind the scenes to determine who influences the future. 

Horror takes many forms, but in Eclipse Phase it is more psychological than gore. It is the uncertainty of survival, the suspense of finding malevolent things among the stars, the fear of the unknown, the dread of facing Things That Should Not Be, the revulsion when encountering alien things, and the sickening realization of the wrong and ghastly things that transhumans are capable of doing to themselves and each other. Horror also arises both from the comprehension that there are scary things beyond our understanding nhabiting our universe and that transhumanity may be its own worst enemy. Despite all of the technological tools and advances available to future transhumans, they still face terrors like losing control of their own dentities, their perceptions, and their mental faculties—not to mention their future as a species. 

Eclipse Phase takes all of these themes and weaves them together in a transhuman setting. The postapocalyptic angle covers the understanding of all that transhumanity has lost, the fight against extinction, and how much of that is a struggle against our own nature. The conspiracy side delves into the nature of the secret organizations that play key roles n determining transhumanity's future and how the actions of determined individuals can change the ives of many. The horror perspective explores the results of humanity's self-inflicted transformations and how some of these changes effectively make us non-human. Tying it all together is an awareness of the massive indifference and the terrible alien-ness that pervades the universe and how transhumanity is insignificant against such a backdrop. 
=======
Post-apocalyptique est un terme utilisé pour décrire les fictions se déroulant après un évènement catacyclismique qui a mis fin à la civilisation humaine telle que nous la connaissont actuellement (habituellement accompagnée par la perte de vie humaine sur une échelle impensable). L'exacte mécanisme du désastre est habituellement sans importance. Il peut s'agir d'une guerre nucléaire, d'une épidémie, de la chute d'un astéroïde, etc. La part importante du thème est la condition humaine. Si le monde tel que nous le connaissons nous a été arraché et que les humains on subit des horreurs au-delà de l'imaginable lors de cette transformation à un cadre post-apocalyptique, comment l'humanité y fait face? Survivons-nous? Prospérons-nous? Surmontons-nous ces difficultés? Ou perdons-nous notre humanité dans le processus, succombant finalement à l'extinction? Ce sont les questions qui sont au cœur du thème. 
<<<<<<< HEAD
<<<<<<< HEAD
=======
>>>>>>> origin/french

Conspirer signifie "joindre une organisation secrète pour atteindre un but illégal ou mauvais ou utiliser ces moyens à un but légal." En tant que telle, une théorie conspirationiste attribue la cause ultime d'un évènement ou d'une succession d'évènements (qu'ils soient politique, sociaux ou historiques) à un groupe secret d'individus possédant un pouvoir immense (incluant politique et financier entre autres) et qui dissimulent leurs activités au public tout en manipulant les évènements afin d'atteindre leurs buts, sans se soucier des conséquences. Beaucoup de théories conspirationnistes prétendent qu'une grande partie des évènements historiques ont été initiés et finallement contrôllés par de telles organisations secrètes. Sur le même plan d'importance, on trouve la lutte silencieuse entre des groupes clandestins, s'affrontant dans une guerre occulte afin de déterminer qui influencera le futur. 

L'horreur prend beaucoup de formes, mais dans Eclipse Phase, il s'agît plus d'horreur psychologique que de gore. C'est la survie incertaine, le suspens de trouver des choses malveillantes parmi les étoiles, la peur de l'inconnu, l'angoisse d'affronter des Choses Qui Ne Devraient Pas Être, la révulsion lors de la rencontre de choses étrangères, et la prise de conscience maladive des choses horribles et mauvaises que les transhumains sont capable de s'infliger. L'horreur apparaît aussi du fait qu'il y a des choses effrayantes au delà de notre compréhension qui habitent notre univers et du fait que la transhumanité est probablement son pire ennemi. En dépit de tous les outils technologiques et des avancées disponibles aux futurs transhumains, ils doivent toujours affronter des horreurs comme la perte de contrôle de leur propre identité, de leur perceptions et de leur faculté mentale - sans parler du contrôle de leur futur en tant qu'espèce. 

Eclipse Phase mélange tout ces thèmes dans un contexte transhumain. L'angle postapocalyptique couvre la compréhension de tout ce que la transhumanité à perdu, la lutte contre l'extinction et la part de cette lutte qui est un combat contre notre propre nature. L'angle conspirationniste se situe dans la nature des organisations secrètes qui jouent un rôle clef dans la déterminatuion du futur de la transhumanité et comment les actions d'individus déterminés peut influer sur la vie de beaucoup. La perspective de l'horreur permet d'explorer les résultats des transformations que c'est infligé l'humanité, et comment certains de ces changements nous a effectivement transformés en non-humain. Mélanger ces thématiques permet une sensibilisation à l'indifférence massive et à la terrible étrangeté qui imprègne l'univers et souligne à quel point la transhumanité est insignifiante dans un tel contexte. 
>>>>>>> e787973436878844dcbd819afabd2b0ac73eee89
<<<<<<< HEAD

Au delà de ces thèmes, cependant, Eclipse Phase affirme également qu'il y a toujours de l'espoir, qu'il y a quelque chose quimérite d'être défendu, et que la transhumanité peut se frayer son propre chemin vers le futur. 



\subsection{Mais comment jouons-nous?} \label{sec:but-how-do} 

Pour jouer une partie d'Eclipse Phase, vous aurez besoin du matériel suivant: 

\end{itemize} \item Un groupe de joueur et un endroit où se rencontrer (dans le monde réel ou en-ligne!) \item Un joueur pour tenir le rôle de maître de jeu \item Le contenu de ce livre \item Quelque chose pour que les personnes puisse prendre des notes (bloc notes, ordinateur portable, peu importe!) \item Deux dés à dix-faces (ou un équivalent numérique) \item De l'imagination \end{itemize} 

\subsubsection{Un groupe de joueur et un endroit pour se rencontrer} \label{sec:group-players} 

Bien que jouer au jeu de rõle soit suffisament flexible pour permettre à n'importe quel nombre de personne de participer, la plupart des groupes de jeu sont composés de quatre à huits joueurs. Ce nombre de participant permet un mélange interessant des personnalités autour de la table et assure une bonne coopération pendant la partie. 

Une fois qu'un groupe de joueurs a déterminé qu'ils allaient jouer à Eclipse Phase, ils auront besoin de désigner quelqu'un pour être le maître de jeu (voir plus bas). Ils auront ensuite besoin de déterminer la date, l'heure et le lieu de la partie 

La plupart des groupes de jeu se retrouvent une fois par semaine à un intervalle de temps régulier: 19:00, Jeudi soir, chez Rob par exemple. Chaque groupe détermine cependant où, quand et comment ils veulent jouer. Un groupe peu décider qu'ils ne peuvent se rassembler qu'une fois par mois, alors qu'un autre est tellement excité à l'idée de se plonger dans les histoires potentielles d'Eclipse Phase, qu'ils voudront se rencontrer deux fois par semaine (ils décident d'une rotation entre leurs domiciles cependant, pour ne pas surcharger un joueur en particulier). Si un groupe à suffisament de chance pour que leur boutique de jeu préférée acceuille des parties, ils pourront décider de se retrouver là-bas. D'autres groupes se rencontrent dans des bibliothèques, dans des salles communes dans leurs écoles, dans des librairies qui possèdent des "salles de lecture" généreuses. Quelque soit ce qui convient à votre groupe de jeu, faites le! 

Lorsque les joueurs se retrouvent pour jouer, la plupart des JDR parlent de "session de jeu." La durée de chaque session de jeu dépend autant du consensus établi par le groupe de joueur que des limitations du local dans lequel ils jouent. L'histoire particulière qui se déroule dans une session donnée peut aussi impacter sur la durée de celle-ci. En jouant dans une boutique de jeu, le groupe de joueur peut n'avoir qu'un créneau de quatre heures ou bien le maître de jeu et le groupe peuvent avoir déterminés - après plusieurs sessions de jeu - que c'est un laps de temps parfait pour profiter de l'histoire à laquelle ils participent chaque semaine. Un autre groupe, cependant, pourrait vouloir un temps de jeu encore plus court. Un groupe encore différent pourrait décider que, même si d'habitude ils jouent par session de quatre heures, une fois par mois ils se regrouperont pour jouer tout le Samedi pour une super session de jeu étalée sur toute la journée. Les joueurs devraont s'immerger et commencer à jouer et faire preuve de souplesse pour décider ce qui leur procurera l'amusement ultime pour leur groupe de jeu. 

Alors que la camaraderie d'une expérience partagée en jouant face à face avec un groupe d'amis reste la force du jeu de rôle, les groupes ont besoin de ne pas se confiner à un seul mode de jeu. Des myriades d'options peuvent être utilisées. E-mail, messageires instantanées, forums, chat vidéo, appel téléphonique/en VoIP, SMS, wikis, (micro-)blogs: chacun d'entre eux peut être utilisé pour jouer sans se tenir chaud directement assis autour d'une table les uns en face des autres. 

Finalement, lorsqu'un groupe de jeu se rencontre pour la première fois, les joueurs devraient créer leurs personnage (par opposition à la création de personnage chacun de son côté). Alors qu'un groupe de joueur peut décider de générer leur personnages individuellement, il est souvent bien plus facile de le faire une fois que les joueurs sont rassemblés. Cela permet à ceux qui ont plus d'expérience dans le jeu de rôle d'aider ceux qui débutent. Encore plus important, cela permet au groupe entier de dimensionner les personnages pour qu'il n'y ait pas trop de doublons au niveau des compétences et des styles. Après tout, avec la richesse des types de personnages disponible, vous ne voulez pas arriver à la table de jeu avec un personnage presque identique à celui de votre voisin. 

\subsubsection{Le maître de jeu} \label{sec:gamemaster} 

Une fois qu'un groupe s'est organisé, quelqu'un doit s'avancer et prendre les rênes du maître de jeu. Certains groupes ont un seul maître de jeu qui anime toute leurs sessions de jeu mois après mois. D'autre groupes changent régulièrement de maître de jeu, avec l'un d'entre eux animant une portion donnée du déroulement de l'histoire pendant plusieurs session avant de passer le rôle à un autre joueur. Encore une fois, les participants doivent faire preuve de souplesse. Certains groupe peuvent avoir la personne idéale qui aime le travail induit par le rôle et qui est plus que volontaire pour animer une session après l'autre, alors que d'autres peuvent décider qu'ils veulent tous alterner entre être le maître de jeu et être un joueur. 
=======

Au delà de ces thèmes, cependant, Eclipse Phase affirme également qu'il y a toujours de l'espoir, qu'il y a quelque chose quimérite d'être défendu, et que la transhumanité peut se frayer son propre chemin vers le futur. 
>>>>>>> origin/french

Le maître de jeu contrôle l'histoire. Il garde une trace de tout ce qui est supposé arrivé et de quand cela arrive, il décrit les évènements au fur et à mesure qu'ils se produisent afin que les jouers (en tant que personnages) puissent réagir, il garde une trace des autres personnages de la partie (appelés personnages non joueur, ou PNJs), et résolve les actions en utilisant le système de jeu. Le sytème de jeu intervient lrosque les personnages cherchent à utiliser leur compétences ou à faire quelque chose qui nécessite un test pour détermienr si ils ont réussit ou pas. Des règles spécifiques sont présentées pour les situations qui impliquent de lancer des dés pour déterminer la réussite (voir Mécanique de Jeu, p.  112). 

Le maître de jeu décrit le monde tel que le voient les personnages, en utilisant leurs yeux, leurs oreilles, et leurs autres sens. Maîtriser n'est pas facile, mais les frissons de la création d'une aventure qui engage l'imagination des autres joueurs, et confrontent leur compétence de jeu et les compétences de leurs personnages dans l'univers de jeu, vaut largement l'investissement. Posthuman Studios et Catalyst Game Labs suivront la publication d'Eclipse Phase en publiant des suppléments et des aventures pour faciliter le processus, mais les maîtres de jeu expérimentés peuvent toujours adapter l'univers de jeu à leur propre style. En fait, puisque Eclipse Phase est publiée sous licence Creative Commons (voir p. 5), les joueurs sont encouragés à adapter l'univers à leur style de jeu et à partager leurs modifcications avec les autres joueurs. Vous ne saurez jamais quand un choix spécifique que vous faites en maîtrisant une campagne est exactement ce qu'un autre maître de jeu et son groupe recherchent. 

<<<<<<< HEAD
\subsubsection{Contenu de ce livre} \label{sec:contents-this-book} 
=======

Conspirer signifie "joindre une organisation secrète pour atteindre un but illégal ou mauvais ou utiliser ces moyens à un but légal." En tant que telle, une théorie conspirationiste attribue la cause ultime d'un évènement ou d'une succession d'évènements (qu'ils soient politique, sociaux ou historiques) à un groupe secret d'individus possédant un pouvoir immense (incluant politique et financier entre autres) et qui dissimulent leurs activités au public tout en manipulant les évènements afin d'atteindre leurs buts, sans se soucier des conséquences. Beaucoup de théories conspirationnistes prétendent qu'une grande partie des évènements historiques ont été initiés et finallement contrôllés par de telles organisations secrètes. Sur le même plan d'importance, on trouve la lutte silencieuse entre des groupes clandestins, s'affrontant dans une guerre occulte afin de déterminer qui influencera le futur. 

L'horreur prend beaucoup de formes, mais dans Eclipse Phase, il s'agît plus d'horreur psychologique que de gore. C'est la survie incertaine, le suspens de trouver des choses malveillantes parmi les étoiles, la peur de l'inconnu, l'angoisse d'affronter des Choses Qui Ne Devraient Pas Être, la révulsion lors de la rencontre de choses étrangères, et la prise de conscience maladive des choses horribles et mauvaises que les transhumains sont capable de s'infliger. L'horreur apparaît aussi du fait qu'il y a des choses effrayantes au delà de notre compréhension qui habitent notre univers et du fait que la transhumanité est probablement son pire ennemi. En dépit de tous les outils technologiques et des avancées disponibles aux futurs transhumains, ils doivent toujours affronter des horreurs comme la perte de contrôle de leur propre identité, de leur perceptions et de leur faculté mentale - sans parler du contrôle de leur futur en tant qu'espèce. 

Eclipse Phase mélange tout ces thèmes dans un contexte transhumain. L'angle postapocalyptique couvre la compréhension de tout ce que la transhumanité à perdu, la lutte contre l'extinction et la part de cette lutte qui est un combat contre notre propre nature. L'angle conspirationniste se situe dans la nature des organisations secrètes qui jouent un rôle clef dans la déterminatuion du futur de la transhumanité et comment les actions d'individus déterminés peut influer sur la vie de beaucoup. La perspective de l'horreur permet d'explorer les résultats des transformations que c'est infligé l'humanité, et comment certains de ces changements nous a effectivement transformés en non-humain. Mélanger ces thématiques permet une sensibilisation à l'indifférence massive et à la terrible étrangeté qui imprègne l'univers et souligne à quel point la transhumanité est insignifiante dans un tel contexte. 
>>>>>>> e787973436878844dcbd819afabd2b0ac73eee89

Au delà de ces thèmes, cependant, Eclipse Phase affirme également qu'il y a toujours de l'espoir, qu'il y a quelque chose quimérite d'être défendu, et que la transhumanité peut se frayer son propre chemin vers le futur. 



\subsection{Mais comment jouons-nous?} \label{sec:but-how-do} 

Pour jouer une partie d'Eclipse Phase, vous aurez besoin du matériel suivant: 

\end{itemize} \item Un groupe de joueur et un endroit où se rencontrer (dans le monde réel ou en-ligne!) \item Un joueur pour tenir le rôle de maître de jeu \item Le contenu de ce livre \item Quelque chose pour que les personnes puisse prendre des notes (bloc notes, ordinateur portable, peu importe!) \item Deux dés à dix-faces (ou un équivalent numérique) \item De l'imagination \end{itemize} 

\subsubsection{Un groupe de joueur et un endroit pour se rencontrer} \label{sec:group-players} 

Bien que jouer au jeu de rõle soit suffisament flexible pour permettre à n'importe quel nombre de personne de participer, la plupart des groupes de jeu sont composés de quatre à huits joueurs. Ce nombre de participant permet un mélange interessant des personnalités autour de la table et assure une bonne coopération pendant la partie. 

Une fois qu'un groupe de joueurs a déterminé qu'ils allaient jouer à Eclipse Phase, ils auront besoin de désigner quelqu'un pour être le maître de jeu (voir plus bas). Ils auront ensuite besoin de déterminer la date, l'heure et le lieu de la partie 

La plupart des groupes de jeu se retrouvent une fois par semaine à un intervalle de temps régulier: 19:00, Jeudi soir, chez Rob par exemple. Chaque groupe détermine cependant où, quand et comment ils veulent jouer. Un groupe peu décider qu'ils ne peuvent se rassembler qu'une fois par mois, alors qu'un autre est tellement excité à l'idée de se plonger dans les histoires potentielles d'Eclipse Phase, qu'ils voudront se rencontrer deux fois par semaine (ils décident d'une rotation entre leurs domiciles cependant, pour ne pas surcharger un joueur en particulier). Si un groupe à suffisament de chance pour que leur boutique de jeu préférée acceuille des parties, ils pourront décider de se retrouver là-bas. D'autres groupes se rencontrent dans des bibliothèques, dans des salles communes dans leurs écoles, dans des librairies qui possèdent des "salles de lecture" généreuses. Quelque soit ce qui convient à votre groupe de jeu, faites le! 

Lorsque les joueurs se retrouvent pour jouer, la plupart des JDR parlent de "session de jeu." La durée de chaque session de jeu dépend autant du consensus établi par le groupe de joueur que des limitations du local dans lequel ils jouent. L'histoire particulière qui se déroule dans une session donnée peut aussi impacter sur la durée de celle-ci. En jouant dans une boutique de jeu, le groupe de joueur peut n'avoir qu'un créneau de quatre heures ou bien le maître de jeu et le groupe peuvent avoir déterminés - après plusieurs sessions de jeu - que c'est un laps de temps parfait pour profiter de l'histoire à laquelle ils participent chaque semaine. Un autre groupe, cependant, pourrait vouloir un temps de jeu encore plus court. Un groupe encore différent pourrait décider que, même si d'habitude ils jouent par session de quatre heures, une fois par mois ils se regrouperont pour jouer tout le Samedi pour une super session de jeu étalée sur toute la journée. Les joueurs devraont s'immerger et commencer à jouer et faire preuve de souplesse pour décider ce qui leur procurera l'amusement ultime pour leur groupe de jeu. 

Alors que la camaraderie d'une expérience partagée en jouant face à face avec un groupe d'amis reste la force du jeu de rôle, les groupes ont besoin de ne pas se confiner à un seul mode de jeu. Des myriades d'options peuvent être utilisées. E-mail, messageires instantanées, forums, chat vidéo, appel téléphonique/en VoIP, SMS, wikis, (micro-)blogs: chacun d'entre eux peut être utilisé pour jouer sans se tenir chaud directement assis autour d'une table les uns en face des autres. 

Finalement, lorsqu'un groupe de jeu se rencontre pour la première fois, les joueurs devraient créer leurs personnage (par opposition à la création de personnage chacun de son côté). Alors qu'un groupe de joueur peut décider de générer leur personnages individuellement, il est souvent bien plus facile de le faire une fois que les joueurs sont rassemblés. Cela permet à ceux qui ont plus d'expérience dans le jeu de rôle d'aider ceux qui débutent. Encore plus important, cela permet au groupe entier de dimensionner les personnages pour qu'il n'y ait pas trop de doublons au niveau des compétences et des styles. Après tout, avec la richesse des types de personnages disponible, vous ne voulez pas arriver à la table de jeu avec un personnage presque identique à celui de votre voisin. 

\subsubsection{Le maître de jeu} \label{sec:gamemaster} 

Une fois qu'un groupe s'est organisé, quelqu'un doit s'avancer et prendre les rênes du maître de jeu. Certains groupes ont un seul maître de jeu qui anime toute leurs sessions de jeu mois après mois. D'autre groupes changent régulièrement de maître de jeu, avec l'un d'entre eux animant une portion donnée du déroulement de l'histoire pendant plusieurs session avant de passer le rôle à un autre joueur. Encore une fois, les participants doivent faire preuve de souplesse. Certains groupe peuvent avoir la personne idéale qui aime le travail induit par le rôle et qui est plus que volontaire pour animer une session après l'autre, alors que d'autres peuvent décider qu'ils veulent tous alterner entre être le maître de jeu et être un joueur. 

Le maître de jeu contrôle l'histoire. Il garde une trace de tout ce qui est supposé arrivé et de quand cela arrive, il décrit les évènements au fur et à mesure qu'ils se produisent afin que les jouers (en tant que personnages) puissent réagir, il garde une trace des autres personnages de la partie (appelés personnages non joueur, ou PNJs), et résolve les actions en utilisant le système de jeu. Le sytème de jeu intervient lrosque les personnages cherchent à utiliser leur compétences ou à faire quelque chose qui nécessite un test pour détermienr si ils ont réussit ou pas. Des règles spécifiques sont présentées pour les situations qui impliquent de lancer des dés pour déterminer la réussite (voir Mécanique de Jeu, p.  112). 

Le maître de jeu décrit le monde tel que le voient les personnages, en utilisant leurs yeux, leurs oreilles, et leurs autres sens. Maîtriser n'est pas facile, mais les frissons de la création d'une aventure qui engage l'imagination des autres joueurs, et confrontent leur compétence de jeu et les compétences de leurs personnages dans l'univers de jeu, vaut largement l'investissement. Posthuman Studios et Catalyst Game Labs suivront la publication d'Eclipse Phase en publiant des suppléments et des aventures pour faciliter le processus, mais les maîtres de jeu expérimentés peuvent toujours adapter l'univers de jeu à leur propre style. En fait, puisque Eclipse Phase est publiée sous licence Creative Commons (voir p. 5), les joueurs sont encouragés à adapter l'univers à leur style de jeu et à partager leurs modifcications avec les autres joueurs. Vous ne saurez jamais quand un choix spécifique que vous faites en maîtrisant une campagne est exactement ce qu'un autre maître de jeu et son groupe recherchent. 

\subsubsection{Contenu de ce livre} \label{sec:contents-this-book} 

Que vous vous soyez procuré la version imprimée ou électronique, ce livre est organisé spécifiquement pour présenter les informations que vous avez besoin de connaître pour commencer à raconter votre histoire dans l'univers d'Eclipse Phase. Vosu trouverez ci-dessous un résumé de chaque chapitre de ce livre 

\paragraph{Une Époque d'Éclipse:} Une histoire globale et un cadre entièrement détaillé décrivant l'univers d'Eclipse Phase et racontant comment l'humanité a effectué une transition d'ici à là-bas. Voir p. 30. 

\paragraph{Mécaniques de Jeu:} Les actions voulues par les joueurs deviennent réalité dans l'univers grâce à des mécaniques de jeu simple et facile à utiliser. Voir p. 112. 

\paragraph{Création de Personnage et Avancement:} Créer un personnage unique peut-être l'une des expéreinces les plus intéressantes du jeu de rôle. Encore plus gratifiant est de voir ce personnage évoluer et grandir au travers de nombreuses sessions de jeu, bien au-delà de tout ce que votre imagination avait envisagé. Voir p. 128. 
=======
\subsection{Mais comment jouons-nous?} \label{sec:but-how-do} 

Pour jouer une partie d'Eclipse Phase, vous aurez besoin du matériel suivant: 

\end{itemize} \item Un groupe de joueur et un endroit où se rencontrer (dans le monde réel ou en-ligne!) \item Un joueur pour tenir le rôle de maître de jeu \item Le contenu de ce livre \item Quelque chose pour que les personnes puisse prendre des notes (bloc notes, ordinateur portable, peu importe!) \item Deux dés à dix-faces (ou un équivalent numérique) \item De l'imagination \end{itemize} 

\subsubsection{Un groupe de joueur et un endroit pour se rencontrer} \label{sec:group-players} 

Bien que jouer au jeu de rõle soit suffisament flexible pour permettre à n'importe quel nombre de personne de participer, la plupart des groupes de jeu sont composés de quatre à huits joueurs. Ce nombre de participant permet un mélange interessant des personnalités autour de la table et assure une bonne coopération pendant la partie. 

Une fois qu'un groupe de joueurs a déterminé qu'ils allaient jouer à Eclipse Phase, ils auront besoin de désigner quelqu'un pour être le maître de jeu (voir plus bas). Ils auront ensuite besoin de déterminer la date, l'heure et le lieu de la partie 

La plupart des groupes de jeu se retrouvent une fois par semaine à un intervalle de temps régulier: 19:00, Jeudi soir, chez Rob par exemple. Chaque groupe détermine cependant où, quand et comment ils veulent jouer. Un groupe peu décider qu'ils ne peuvent se rassembler qu'une fois par mois, alors qu'un autre est tellement excité à l'idée de se plonger dans les histoires potentielles d'Eclipse Phase, qu'ils voudront se rencontrer deux fois par semaine (ils décident d'une rotation entre leurs domiciles cependant, pour ne pas surcharger un joueur en particulier). Si un groupe à suffisament de chance pour que leur boutique de jeu préférée acceuille des parties, ils pourront décider de se retrouver là-bas. D'autres groupes se rencontrent dans des bibliothèques, dans des salles communes dans leurs écoles, dans des librairies qui possèdent des "salles de lecture" généreuses. Quelque soit ce qui convient à votre groupe de jeu, faites le! 

Lorsque les joueurs se retrouvent pour jouer, la plupart des JDR parlent de "session de jeu." La durée de chaque session de jeu dépend autant du consensus établi par le groupe de joueur que des limitations du local dans lequel ils jouent. L'histoire particulière qui se déroule dans une session donnée peut aussi impacter sur la durée de celle-ci. En jouant dans une boutique de jeu, le groupe de joueur peut n'avoir qu'un créneau de quatre heures ou bien le maître de jeu et le groupe peuvent avoir déterminés - après plusieurs sessions de jeu - que c'est un laps de temps parfait pour profiter de l'histoire à laquelle ils participent chaque semaine. Un autre groupe, cependant, pourrait vouloir un temps de jeu encore plus court. Un groupe encore différent pourrait décider que, même si d'habitude ils jouent par session de quatre heures, une fois par mois ils se regrouperont pour jouer tout le Samedi pour une super session de jeu étalée sur toute la journée. Les joueurs devraont s'immerger et commencer à jouer et faire preuve de souplesse pour décider ce qui leur procurera l'amusement ultime pour leur groupe de jeu. 

Alors que la camaraderie d'une expérience partagée en jouant face à face avec un groupe d'amis reste la force du jeu de rôle, les groupes ont besoin de ne pas se confiner à un seul mode de jeu. Des myriades d'options peuvent être utilisées. E-mail, messageires instantanées, forums, chat vidéo, appel téléphonique/en VoIP, SMS, wikis, (micro-)blogs: chacun d'entre eux peut être utilisé pour jouer sans se tenir chaud directement assis autour d'une table les uns en face des autres. 

Finalement, lorsqu'un groupe de jeu se rencontre pour la première fois, les joueurs devraient créer leurs personnage (par opposition à la création de personnage chacun de son côté). Alors qu'un groupe de joueur peut décider de générer leur personnages individuellement, il est souvent bien plus facile de le faire une fois que les joueurs sont rassemblés. Cela permet à ceux qui ont plus d'expérience dans le jeu de rôle d'aider ceux qui débutent. Encore plus important, cela permet au groupe entier de dimensionner les personnages pour qu'il n'y ait pas trop de doublons au niveau des compétences et des styles. Après tout, avec la richesse des types de personnages disponible, vous ne voulez pas arriver à la table de jeu avec un personnage presque identique à celui de votre voisin. 

\subsubsection{Le maître de jeu} \label{sec:gamemaster} 

Une fois qu'un groupe s'est organisé, quelqu'un doit s'avancer et prendre les rênes du maître de jeu. Certains groupes ont un seul maître de jeu qui anime toute leurs sessions de jeu mois après mois. D'autre groupes changent régulièrement de maître de jeu, avec l'un d'entre eux animant une portion donnée du déroulement de l'histoire pendant plusieurs session avant de passer le rôle à un autre joueur. Encore une fois, les participants doivent faire preuve de souplesse. Certains groupe peuvent avoir la personne idéale qui aime le travail induit par le rôle et qui est plus que volontaire pour animer une session après l'autre, alors que d'autres peuvent décider qu'ils veulent tous alterner entre être le maître de jeu et être un joueur. 

Le maître de jeu contrôle l'histoire. Il garde une trace de tout ce qui est supposé arrivé et de quand cela arrive, il décrit les évènements au fur et à mesure qu'ils se produisent afin que les jouers (en tant que personnages) puissent réagir, il garde une trace des autres personnages de la partie (appelés personnages non joueur, ou PNJs), et résolve les actions en utilisant le système de jeu. Le sytème de jeu intervient lrosque les personnages cherchent à utiliser leur compétences ou à faire quelque chose qui nécessite un test pour détermienr si ils ont réussit ou pas. Des règles spécifiques sont présentées pour les situations qui impliquent de lancer des dés pour déterminer la réussite (voir Mécanique de Jeu, p.  112). 

Le maître de jeu décrit le monde tel que le voient les personnages, en utilisant leurs yeux, leurs oreilles, et leurs autres sens. Maîtriser n'est pas facile, mais les frissons de la création d'une aventure qui engage l'imagination des autres joueurs, et confrontent leur compétence de jeu et les compétences de leurs personnages dans l'univers de jeu, vaut largement l'investissement. Posthuman Studios et Catalyst Game Labs suivront la publication d'Eclipse Phase en publiant des suppléments et des aventures pour faciliter le processus, mais les maîtres de jeu expérimentés peuvent toujours adapter l'univers de jeu à leur propre style. En fait, puisque Eclipse Phase est publiée sous licence Creative Commons (voir p. 5), les joueurs sont encouragés à adapter l'univers à leur style de jeu et à partager leurs modifcications avec les autres joueurs. Vous ne saurez jamais quand un choix spécifique que vous faites en maîtrisant une campagne est exactement ce qu'un autre maître de jeu et son groupe recherchent. 

\subsubsection{Contenu de ce livre} \label{sec:contents-this-book} 

Que vous vous soyez procuré la version imprimée ou électronique, ce livre est organisé spécifiquement pour présenter les informations que vous avez besoin de connaître pour commencer à raconter votre histoire dans l'univers d'Eclipse Phase. Vosu trouverez ci-dessous un résumé de chaque chapitre de ce livre 

\paragraph{Une Époque d'Éclipse:} Une histoire globale et un cadre entièrement détaillé décrivant l'univers d'Eclipse Phase et racontant comment l'humanité a effectué une transition d'ici à là-bas. Voir p. 30. 

\paragraph{Mécaniques de Jeu:} Les actions voulues par les joueurs deviennent réalité dans l'univers grâce à des mécaniques de jeu simple et facile à utiliser. Voir p. 112. 

\paragraph{Création de Personnage et Avancement:} Créer un personnage unique peut-être l'une des expéreinces les plus intéressantes du jeu de rôle. Encore plus gratifiant est de voir ce personnage évoluer et grandir au travers de nombreuses sessions de jeu, bien au-delà de tout ce que votre imagination avait envisagé. Voir p. 128. 

\paragraph{Compétences:} Au delà des capacités innées d'un personnage, les compétences sont ce qui les distinguent. C'est ce que votre personnage sait et ce qu'ils savent faire. Voir p. 170. 

\paragraph{Action et Combat:} Qu'est-ce qu'une histoire dramatique sans action ni violence? Lorsque les mots échouent, les armes s'enflamment. Voir p. 186. 

\paragraph{Piratages Cognitifs:} Les possibilité inahbituelles offertes par les capacités psi et la reprogrammation mentale. Voir p. 216. 

\paragraph{Le Mesh:} La nature omniprésente du mesh assure que c'est un élément clef de toutes les histoires racontées. Voir p. 234. 

\paragraph{Futur Accéléré:} Les merveilles de la technologies avancées et comment elle fonctionne. Voir p. 266. 

\paragraph{Équipement:} Des améliorations personnelles, aux armes en passant par les robots. Voir p. 294. 

\paragraph{Informations de jeu:} La quintessence de l'ensemble des secrets destinés aux maîtres de jeu. Voir p. 350. 
>>>>>>> origin/french

\paragraph{Compétences:} Au delà des capacités innées d'un personnage, les compétences sont ce qui les distinguent. C'est ce que votre personnage sait et ce qu'ils savent faire. Voir p. 170. 

\paragraph{Action et Combat:} Qu'est-ce qu'une histoire dramatique sans action ni violence? Lorsque les mots échouent, les armes s'enflamment. Voir p. 186. 

<<<<<<< HEAD
\paragraph{Piratages Cognitifs:} Les possibilité inahbituelles offertes par les capacités psi et la reprogrammation mentale. Voir p. 216. 

\paragraph{Le Mesh:} La nature omniprésente du mesh assure que c'est un élément clef de toutes les histoires racontées. Voir p. 234. 

\paragraph{Futur Accéléré:} Les merveilles de la technologies avancées et comment elle fonctionne. Voir p. 266. 

\paragraph{Équipement:} Des améliorations personnelles, aux armes en passant par les robots. Voir p. 294. 
=======
\subsubsection{Prendre des notes} \label{sec:taking-notes} 

Que vous soyez maître de jeu ou joueur, vous avez besoin d'un moyen quelconque de suivre l'information. Lesjoueurs créeront des personnages et leur apporteront des changements d'une session à l'autre. Entre-temps, le maître de jeu devra suivre un paquet d'information: des notes sur la façon dont se déroule l'histoire suite aux interactions des personnages joueurs et dont vous devrez tenir compte lors de la session de la semaine suivante; des changements apportés aux PNJs; des changements apportés aux personnages joueurs que les joueurs ignorent encore (comme le fait qu'un personnage se soit fait pirater le cerveau mais qu'il ne le sait pas encore); et ainsi de suite. 

Additionellement, certains groupes apprécient un résumé de chaque session qui puisse être compilé et lu un peu plus tard pour appréicer et partager leurs exploits, de la même façon dont vour partageriez des extraits vidéos de votre jeu vidéo préféré pour montrer vos capacité à battre le méchant (traditionnellement cela s'appelle "tenir un journal"). Cela se révèle particulièrement utile si un joueur n'as pas pu partciiper à une session donnée, fournissant un résumé rapide qu'il peut lire avant de venir à la prochaine session de jeu et s'éviter ainsi de tomber dans un piège pendant qu'il essaye de se raccrocher aux évènements de la session de jeu actuelle. La prise de note est une responsabilité qui peut être partagée entre tous les joueurs ou assignée à un seul, tout dépend de ce qui fonctionne le mieux au sein du groupe. De la même manière, certains groupe de jeu réalisent un enregistrement audio de toute la session de jeu, à la fois pour pouvoir s'y reporter plus tard et pour effectuer des podcasts de "jeu live". 

Les vieux standard du crayon et du papier font encore des miracles. Un tas de techonolgies additionelles fournissent cependant plein de nouvelles options aux joueurs. D'un fichier texte sur un portable à un wiki partagé, la possibilité de suivre une somme d'information suffisament importante d'une manière simple et rapide - tout en fournissant des informations appropriées à chaque joueur d'une session de jeu à l'autre - réduit de manière significative le temps que chacun passe à suivre ces informations. Ce temps peut maintenant être réallouer au plaisir de participer à une historie grandiose. 
>>>>>>> origin/french

\paragraph{Informations de jeu:} La quintessence de l'ensemble des secrets destinés aux maîtres de jeu. Voir p. 350. 
>>>>>>> ab07e7255a391f48de47c3ed67a243ec64317163

Que vous vous soyez procuré la version imprimée ou électronique, ce livre est organisé spécifiquement pour présenter les informations que vous avez besoin de connaître pour commencer à raconter votre histoire dans l'univers d'Eclipse Phase. Vosu trouverez ci-dessous un résumé de chaque chapitre de ce livre 

<<<<<<< HEAD
\paragraph{Une Époque d'Éclipse:} Une histoire globale et un cadre entièrement détaillé décrivant l'univers d'Eclipse Phase et racontant comment l'humanité a effectué une transition d'ici à là-bas. Voir p. 30. 

<<<<<<< HEAD
\paragraph{Mécaniques de Jeu:} Les actions voulues par les joueurs deviennent réalité dans l'univers grâce à des mécaniques de jeu simple et facile à utiliser. Voir p. 112. 
=======
\subsubsection{Dés} \label{sec:dice} 

Comme décrit dans la section Mécanique de Jeu (p. 112), deux dés à dix faces sont nécessaires pour jouer à Eclipse Phase. Bien que la plupart des joueurs prennent plaisir à lancer les dés, il existe un nombre important de mécanisme pour parvenir à un résultat compris entre 00 et 99. Les joueurs qui font un usage intensif de techonolgies online pour jouer - tels que les discussions en ligne ou les blogs vidéos - peuvent trouver plus simple de créer et implémenter rapidement un petit programme de lancer de dés. 
>>>>>>> origin/french

\paragraph{Création de Personnage et Avancement:} Créer un personnage unique peut-être l'une des expéreinces les plus intéressantes du jeu de rôle. Encore plus gratifiant est de voir ce personnage évoluer et grandir au travers de nombreuses sessions de jeu, bien au-delà de tout ce que votre imagination avait envisagé. Voir p. 128. 

\paragraph{Compétences:} Au delà des capacités innées d'un personnage, les compétences sont ce qui les distinguent. C'est ce que votre personnage sait et ce qu'ils savent faire. Voir p. 170. 

\paragraph{Action et Combat:} Qu'est-ce qu'une histoire dramatique sans action ni violence? Lorsque les mots échouent, les armes s'enflamment. Voir p. 186. 
=======
\subsubsection{Prendre des notes} \label{sec:taking-notes} 

Que vous soyez maître de jeu ou joueur, vous avez besoin d'un moyen quelconque de suivre l'information. Lesjoueurs créeront des personnages et leur apporteront des changements d'une session à l'autre. Entre-temps, le maître de jeu devra suivre un paquet d'information: des notes sur la façon dont se déroule l'histoire suite aux interactions des personnages joueurs et dont vous devrez tenir compte lors de la session de la semaine suivante; des changements apportés aux PNJs; des changements apportés aux personnages joueurs que les joueurs ignorent encore (comme le fait qu'un personnage se soit fait pirater le cerveau mais qu'il ne le sait pas encore); et ainsi de suite. 

Additionellement, certains groupes apprécient un résumé de chaque session qui puisse être compilé et lu un peu plus tard pour appréicer et partager leurs exploits, de la même façon dont vour partageriez des extraits vidéos de votre jeu vidéo préféré pour montrer vos capacité à battre le méchant (traditionnellement cela s'appelle "tenir un journal"). Cela se révèle particulièrement utile si un joueur n'as pas pu partciiper à une session donnée, fournissant un résumé rapide qu'il peut lire avant de venir à la prochaine session de jeu et s'éviter ainsi de tomber dans un piège pendant qu'il essaye de se raccrocher aux évènements de la session de jeu actuelle. La prise de note est une responsabilité qui peut être partagée entre tous les joueurs ou assignée à un seul, tout dépend de ce qui fonctionne le mieux au sein du groupe. De la même manière, certains groupe de jeu réalisent un enregistrement audio de toute la session de jeu, à la fois pour pouvoir s'y reporter plus tard et pour effectuer des podcasts de "jeu live". 

Les vieux standard du crayon et du papier font encore des miracles. Un tas de techonolgies additionelles fournissent cependant plein de nouvelles options aux joueurs. D'un fichier texte sur un portable à un wiki partagé, la possibilité de suivre une somme d'information suffisament importante d'une manière simple et rapide - tout en fournissant des informations appropriées à chaque joueur d'une session de jeu à l'autre - réduit de manière significative le temps que chacun passe à suivre ces informations. Ce temps peut maintenant être réallouer au plaisir de participer à une historie grandiose. 
>>>>>>> ab07e7255a391f48de47c3ed67a243ec64317163

\paragraph{Piratages Cognitifs:} Les possibilité inahbituelles offertes par les capacités psi et la reprogrammation mentale. Voir p. 216. 

\paragraph{Le Mesh:} La nature omniprésente du mesh assure que c'est un élément clef de toutes les histoires racontées. Voir p. 234. 

<<<<<<< HEAD
\paragraph{Futur Accéléré:} Les merveilles de la technologies avancées et comment elle fonctionne. Voir p. 266. 

\paragraph{Équipement:} Des améliorations personnelles, aux armes en passant par les robots. Voir p. 294. 
=======
\subsubsection{Dés} \label{sec:dice} 

Comme décrit dans la section Mécanique de Jeu (p. 112), deux dés à dix faces sont nécessaires pour jouer à Eclipse Phase. Bien que la plupart des joueurs prennent plaisir à lancer les dés, il existe un nombre important de mécanisme pour parvenir à un résultat compris entre 00 et 99. Les joueurs qui font un usage intensif de techonolgies online pour jouer - tels que les discussions en ligne ou les blogs vidéos - peuvent trouver plus simple de créer et implémenter rapidement un petit programme de lancer de dés. 
>>>>>>> ab07e7255a391f48de47c3ed67a243ec64317163

\paragraph{Informations de jeu:} La quintessence de l'ensemble des secrets destinés aux maîtres de jeu. Voir p. 350. 


<<<<<<< HEAD

\subsubsection{Prendre des notes} \label{sec:taking-notes} 
=======
\subsubsection{Imagination} \label{sec:imagination} 

Bien trop souvent, il est facile pour quelqu'un qui regarde un JDR d'être intimidé. Il y a tellement de concepts à saisir, tellement d'idées qui semblent écrasantes. Cependant, comme décrit à la section Qu'est-ce qu'un Jeu de Rôle?, combien de fois avez-vous lus un livre ou vu un film et décidés que vous auriez fait mieux? C'est votre imagination qui est à l'œuvre. Laissez-vous simplement aller et vous serez étonnés de la vitesse à laquelle vous pouvez vous immerger dans l'univers d'Eclipse Phase. Bientôt vous serez au cœur d'histoires avec le meilleur de l'univers. 

N'oubliez pas non plus d'exploiter vos ressources. Votre groupe de jeu est la meilleure d'entre elles. Ce qu'il s'y passe, les idées pour gérer une situation ou pour vaincre le méchant: ce sont juste quelques unes des choses qui peuvent et doivent être disuctées par le groupe de joueur entre les sessions, et chacune de ces discussions est une opportunité de développer votre imagination. 

Une autre ressource est de simplement regader la télé ou de lire un bon livre. Faites attention à la manière dont l'histoire est assemblée, comment les personnages sont construits, et comment l'intrigue est dévoilée. Travaillez votre imagination et bientôt vous devinerez les sous-intrigues et qui est réellement le méchant bien avant qu'ils ne soient révélés. Savoir comment une histoire est construite vous permets d'assembler les votres lors de chaque session de jeu. 

Enfin, eclipsephase.com est le site officiel pour Eclipse Phase. Si vous avez des questions à propos du jeu, ou que vous voulez savoir comment un autre groupe de joueur gèrerai une situation, postez sur les forums. La communauté en ligne peut-être tout autant utile et plaisante qu'un groupe de jeu local. 
<<<<<<< HEAD
>>>>>>> ab07e7255a391f48de47c3ed67a243ec64317163
=======
>>>>>>> origin/french

Que vous soyez maître de jeu ou joueur, vous avez besoin d'un moyen quelconque de suivre l'information. Lesjoueurs créeront des personnages et leur apporteront des changements d'une session à l'autre. Entre-temps, le maître de jeu devra suivre un paquet d'information: des notes sur la façon dont se déroule l'histoire suite aux interactions des personnages joueurs et dont vous devrez tenir compte lors de la session de la semaine suivante; des changements apportés aux PNJs; des changements apportés aux personnages joueurs que les joueurs ignorent encore (comme le fait qu'un personnage se soit fait pirater le cerveau mais qu'il ne le sait pas encore); et ainsi de suite. 

Additionellement, certains groupes apprécient un résumé de chaque session qui puisse être compilé et lu un peu plus tard pour appréicer et partager leurs exploits, de la même façon dont vour partageriez des extraits vidéos de votre jeu vidéo préféré pour montrer vos capacité à battre le méchant (traditionnellement cela s'appelle "tenir un journal"). Cela se révèle particulièrement utile si un joueur n'as pas pu partciiper à une session donnée, fournissant un résumé rapide qu'il peut lire avant de venir à la prochaine session de jeu et s'éviter ainsi de tomber dans un piège pendant qu'il essaye de se raccrocher aux évènements de la session de jeu actuelle. La prise de note est une responsabilité qui peut être partagée entre tous les joueurs ou assignée à un seul, tout dépend de ce qui fonctionne le mieux au sein du groupe. De la même manière, certains groupe de jeu réalisent un enregistrement audio de toute la session de jeu, à la fois pour pouvoir s'y reporter plus tard et pour effectuer des podcasts de "jeu live". 

<<<<<<< HEAD
<<<<<<< HEAD
Les vieux standard du crayon et du papier font encore des miracles. Un tas de techonolgies additionelles fournissent cependant plein de nouvelles options aux joueurs. D'un fichier texte sur un portable à un wiki partagé, la possibilité de suivre une somme d'information suffisament importante d'une manière simple et rapide - tout en fournissant des informations appropriées à chaque joueur d'une session de jeu à l'autre - réduit de manière significative le temps que chacun passe à suivre ces informations. Ce temps peut maintenant être réallouer au plaisir de participer à une historie grandiose. 

=======
\subsection{Que font les joueurs?} \label{sec:what-do-players} 
=======
\subsection{Que font les joueurs?} \label{sec:what-do-players} 

Les joueurs peuvent remplir toute une variété de rôles dans Eclipse Phase. Suite à l'avancée dans les technologies de l'émulation digitale de l'esprit, l'upload et le download dans de nouvelles morphs (corps physiques, biologiques ou synthétique), il est possible d'être, littéralement, une nouvelle personne d'une session à l'autre. Avec les corps qui se retrouvent réduit au rôle d'équipement, les joueurs peuvent personnaliser leur apparence pour la tâche à venir. 
>>>>>>> origin/french

Les joueurs peuvent remplir toute une variété de rôles dans Eclipse Phase. Suite à l'avancée dans les technologies de l'émulation digitale de l'esprit, l'upload et le download dans de nouvelles morphs (corps physiques, biologiques ou synthétique), il est possible d'être, littéralement, une nouvelle personne d'une session à l'autre. Avec les corps qui se retrouvent réduit au rôle d'équipement, les joueurs peuvent personnaliser leur apparence pour la tâche à venir. 
>>>>>>> ab07e7255a391f48de47c3ed67a243ec64317163


<<<<<<< HEAD
\subsubsection{Dés} \label{sec:dice} 

<<<<<<< HEAD
Comme décrit dans la section Mécanique de Jeu (p. 112), deux dés à dix faces sont nécessaires pour jouer à Eclipse Phase. Bien que la plupart des joueurs prennent plaisir à lancer les dés, il existe un nombre important de mécanisme pour parvenir à un résultat compris entre 00 et 99. Les joueurs qui font un usage intensif de techonolgies online pour jouer - tels que les discussions en ligne ou les blogs vidéos - peuvent trouver plus simple de créer et implémenter rapidement un petit programme de lancer de dés. 
=======
\subsubsection{La campagne par défaut} \label{sec:default-campaign} 

Dans l'histoire par défaut (aussi appellée "cadre de campagne"), chaque personnage joueur est une "sentinelle", un agent disponible (ou une recrue potentielle) pour un réseau paralégal appelé "Firewall". Firewall est dédié à contrer les "risques existentiels" - des menaces à l'existence de la transhumanité. Ces risques peuvent inclure des fléaux de la guerre biologique, des invasions d'essaim de nanites, de la prolifération nucléaire, des terroristes avec des ADMs, des attaques informatiques destructrices, des IAs maligne, des rencontres aliens, et ainsi de suite. Firewall ne se contente évidement pas de simplement contrer ces menaces au moment où elles apparaissent, les personnages peuvent donc être également envoyés sur des missions de renseignement ou pour mettre en place des mesures de prévention ou de sécurité. Les personnages peuvent être chargés d'enquêter sur des personnes ou des lieux apparement innofensifs (et qui se rèveleront ne pas l'être), de négocier des arrangements avec d'obscurs réseaux criminels (qui se réveleront ne pas être digne de confiance), ou de voyager à travers le trou de ver d'une Porte de Pandore pour annalyser des reliques d'une ruine alien quelconque (et de vérifier si la menace qui les as tués est toujours là). Les sentinelles sont recrutés dans toutes les factions de la transhumanité; ceux qui ne sont pas idéologicallement loyaux à la cause sont recrutés en tant que mercenaires. Ces campagnes ont tendance à se méler à un peu de mystère et d'investigation avec des scènes d'actions et de combats achanés, se baignant dans une bonne dose de crainte et d'horreur. 
>>>>>>> origin/french

=======
\subsubsection{La campagne par défaut} \label{sec:default-campaign} 

Dans l'histoire par défaut (aussi appellée "cadre de campagne"), chaque personnage joueur est une "sentinelle", un agent disponible (ou une recrue potentielle) pour un réseau paralégal appelé "Firewall". Firewall est dédié à contrer les "risques existentiels" - des menaces à l'existence de la transhumanité. Ces risques peuvent inclure des fléaux de la guerre biologique, des invasions d'essaim de nanites, de la prolifération nucléaire, des terroristes avec des ADMs, des attaques informatiques destructrices, des IAs maligne, des rencontres aliens, et ainsi de suite. Firewall ne se contente évidement pas de simplement contrer ces menaces au moment où elles apparaissent, les personnages peuvent donc être également envoyés sur des missions de renseignement ou pour mettre en place des mesures de prévention ou de sécurité. Les personnages peuvent être chargés d'enquêter sur des personnes ou des lieux apparement innofensifs (et qui se rèveleront ne pas l'être), de négocier des arrangements avec d'obscurs réseaux criminels (qui se réveleront ne pas être digne de confiance), ou de voyager à travers le trou de ver d'une Porte de Pandore pour annalyser des reliques d'une ruine alien quelconque (et de vérifier si la menace qui les as tués est toujours là). Les sentinelles sont recrutés dans toutes les factions de la transhumanité; ceux qui ne sont pas idéologicallement loyaux à la cause sont recrutés en tant que mercenaires. Ces campagnes ont tendance à se méler à un peu de mystère et d'investigation avec des scènes d'actions et de combats achanés, se baignant dans une bonne dose de crainte et d'horreur. 
>>>>>>> ab07e7255a391f48de47c3ed67a243ec64317163

<<<<<<< HEAD

\subsubsection{Imagination} \label{sec:imagination} 

<<<<<<< HEAD
Bien trop souvent, il est facile pour quelqu'un qui regarde un JDR d'être intimidé. Il y a tellement de concepts à saisir, tellement d'idées qui semblent écrasantes. Cependant, comme décrit à la section Qu'est-ce qu'un Jeu de Rôle?, combien de fois avez-vous lus un livre ou vu un film et décidés que vous auriez fait mieux? C'est votre imagination qui est à l'œuvre. Laissez-vous simplement aller et vous serez étonnés de la vitesse à laquelle vous pouvez vous immerger dans l'univers d'Eclipse Phase. Bientôt vous serez au cœur d'histoires avec le meilleur de l'univers. 

N'oubliez pas non plus d'exploiter vos ressources. Votre groupe de jeu est la meilleure d'entre elles. Ce qu'il s'y passe, les idées pour gérer une situation ou pour vaincre le méchant: ce sont juste quelques unes des choses qui peuvent et doivent être disuctées par le groupe de joueur entre les sessions, et chacune de ces discussions est une opportunité de développer votre imagination. 

Une autre ressource est de simplement regader la télé ou de lire un bon livre. Faites attention à la manière dont l'histoire est assemblée, comment les personnages sont construits, et comment l'intrigue est dévoilée. Travaillez votre imagination et bientôt vous devinerez les sous-intrigues et qui est réellement le méchant bien avant qu'ils ne soient révélés. Savoir comment une histoire est construite vous permets d'assembler les votres lors de chaque session de jeu. 

Enfin, eclipsephase.com est le site officiel pour Eclipse Phase. Si vous avez des questions à propos du jeu, ou que vous voulez savoir comment un autre groupe de joueur gèrerai une situation, postez sur les forums. La communauté en ligne peut-être tout autant utile et plaisante qu'un groupe de jeu local. 



\subsection{Que font les joueurs?} \label{sec:what-do-players} 
=======
\subsubsection{Campagnes alternatives} \label{sec:alternate-campaigns} 
=======
\subsubsection{Campagnes alternatives} \label{sec:alternate-campaigns} 

Quand ils ne sont pas en train de sauver le système solaire, les sentinelles sont libres de poursuivre leurs propres buts. Le maître de jeu et les joueurs peuvent utiliser ce livre de règle pour générer tout type d'histoire qu'ils ont envie de raconter. Cependant, les exemples suivants fournissent un rapide aperçu des opportunités les plus évidentes pour des aventures dans Elcipse Phase. 

Après chaque variantes de campagnes ci-dessous, une liste "d'archétypes" pour Eclipse Phase est fournie entre parenthèses. Les archétypes sont les noms donnés au type de personnage le plus commun rencontrés dans ces scénarios. Par exemple, dans une histoire de détective classique, les archétypes seraient le Détective, la Demoiselle en Détresse, le Flic Dur à cuir, etc. Dans un film de cowboy, les archétypes seraient le Pistolero, le Barman, le Shériff, le Brave Indien, etc. Les joueurs noterons que plusieurs archétypes correspondant à de multiple cadre de scénario. Le système de création de personnage (p. 128) permet aux joueurs de créer tous les types d'archétype suggérés. De la même manière que les jeus de rôle sont conçus pour que les joueurs construisent leurs propres histoires, ces archétypes sont juste des suggestions et les joueurs peuvent les mélanger et choisir ceux qu'ils veulent. 

\paragraph{Mission de Récupération et de Sauvetage:} la Chute a laissé deux mondes et de nombreux habitats à l'état de ruines - mais ces cités et stations dévastées contiennent des trésors cachés pour ceux qui sont suffisament courageux et téméraires. Les trouvailles peuvent inclure: des systèmes d'armement; des ressources physiques; des banques de données perdues; des uploads abandonnés d'amis, de membres de la famille ou de personnes importantes; de nouvelles technologies développées et perdues lors du décollage brutal de la singularité; des héritages de valeur pour des oligarques immortels; et bien plus. À l'extérieur de ces royaumes autrefois habités, l'espace est en soi un endroit trés grand et beaucoup de personnes et de choses sont perdus là-bas dehors. Quelqu'uns doivent être sauvés et d'autres sont au delà du sauvetage. Cette option laisse les joueurs explorer l'inconnu ou chercher des cibles spécifiques pour un contrat. (Archéologiste/Charognard/Pirate/Libre-Échangiste/Contrebandier/Marchand Noir) 

\paragraph{Exploration:} Il y a pas mal d'opportunités à être un explorateur, un colon, ou un éclaireur avancé - peut-être même l'un des rare individus suffisament chanceux ou suicidaires et qui explorent via une Porte de Pandorre non testée. Même la Ceinture de Kuiper, à la limite de notre système solaire, est toujours partiellement explorée; il peut y avoir des trésors et des mystères à découvrir. Il y a aussi de nombreux dangers qui rôdent dans les recoins étranges du système, des factions posthumaines isolationnistes aux cartels criminels cachés, en passant par les pirates, les aliens et d'autres choses qui veulent demeurer hors de vue. (Explorateur/Archéologue/Charognard/Adepte de la Singularité/Techie/Médecin) 

\paragraph{Commerce:} Alors que la majorité du commerce dans le système intérieur est contrôlé par de brillantes hypercorporations, la plupart des stations les plus petites ou indépendantes dépendent des petits négociants. Dans les systèmes post-abondance à l'extérieur, le commerce prend une forme différente, avec l'information, les faveurs et la créativité qui servent de monnaie parmi ceux qui n'ont plus besoin de quoi que ce soit grâce à la disponibilité des machines d'abondances. (Libre-Échangistes/Contrebandier/Marchand Noir/Pirate) 

\paragraph{Crime:} L'assemblage hétéroclite d'habitats de la taille d'une ville et les lois variant grandmenet à travers le système ont créés d'amples opportunités pour ceux qui voudraient faire de cette situation leur gagne-pain. Les marchandises et activités du marché noir incluent l'échange d'esclave infomorph, les industries du plaisir et des sex pods, l'échange et le vol de données, l'exfiltration et le convoyage de technologies avancées et de scientifiques, el'spionnage politique et économique, l'assassinat, la vente de drogues et d'XP, l'échange d'âme, et bien d'autres choses. Que ce soit en tant qu'indépendant ou en tant que membre d'une organisation criminelle, il y a toujorus des opportunitées pour ceux qui ont soif d'aventure ou de profit et à la moralité douteuse. (Criminel/Contrebandier/Pirate/Arrangeur/Marchand Noir/GénoHacker/Hacker/Infiltrateur) 

\paragraph{Mercenaires:} Les manœuvres permanente des factions à caractère idéologique, les disputes autour de ressources contestées, et la ruée vers la colonisation de nouvelles exoplanètes au delà des Portes de Pandorre sont à l'origine de conflits sur des bases régulières. Certains d'entre eux couvent pendant des années en tant que conflits de faible intensité, dégénérant occasionnellement en raids et affrontements. D'autres choisissent de s'affronter dans une guerre sans pitié. Des femmes et des hommes voulant prendre les armes pour des crédits sont toujours à la recherche de bonnes paies. Les joueurs peuvent s'engager dans des campagne commando et militaire dans des habitats, au milieu des étoiles, ou dans l'environnement hostile d'une planète. (Mercenaire/Consultant en Sécurité/Arrangeur/Chasseur de prime/Ex-Flic/Médecin) 

\paragraph{intrigue Socio-Politique:} Les corporations et les factions politiques qui couvrent tout le système solaire ne respectent pas toujours les règles du jeu en jouant avec les autres, mais il n'est pas simple pour autant pour eux de se confronter ouvertement les uns aux autres sauf en de rares circonstances. Beaucoup de batailles ont été remportées par des manœuvres diplomatiques et politiques, en utilisant des mots et des idées plus puissantes que des armes. A l'intérieur même des factions, différents groupes sociaux peuvent se mener une concurrence sans pitié, ou les luttes de classes surchauffées  peuvent arriver à ébullition, déchirant une société de l'intérieur. Dans cette campagne, les joueurs peuvent commencer en tant que pions d'une entité quelconque et qui gravirons les échellons alors qu'ils seront de plus en plus impliqués dans les intrigues de leur soutien, ils peuvent joueur un groupe d'ambassadeur et d'espions stationnés dans la capitale de l'opposition, ou encore jouer un groupe d'activistes et de radicaux combattant pour des changements sociaux. (Politicien/Socialite/Infiltrateur/Hacker/Consultant en Sécurité/Journaliste/Memeticien) 
>>>>>>> origin/french

Quand ils ne sont pas en train de sauver le système solaire, les sentinelles sont libres de poursuivre leurs propres buts. Le maître de jeu et les joueurs peuvent utiliser ce livre de règle pour générer tout type d'histoire qu'ils ont envie de raconter. Cependant, les exemples suivants fournissent un rapide aperçu des opportunités les plus évidentes pour des aventures dans Elcipse Phase. 

Après chaque variantes de campagnes ci-dessous, une liste "d'archétypes" pour Eclipse Phase est fournie entre parenthèses. Les archétypes sont les noms donnés au type de personnage le plus commun rencontrés dans ces scénarios. Par exemple, dans une histoire de détective classique, les archétypes seraient le Détective, la Demoiselle en Détresse, le Flic Dur à cuir, etc. Dans un film de cowboy, les archétypes seraient le Pistolero, le Barman, le Shériff, le Brave Indien, etc. Les joueurs noterons que plusieurs archétypes correspondant à de multiple cadre de scénario. Le système de création de personnage (p. 128) permet aux joueurs de créer tous les types d'archétype suggérés. De la même manière que les jeus de rôle sont conçus pour que les joueurs construisent leurs propres histoires, ces archétypes sont juste des suggestions et les joueurs peuvent les mélanger et choisir ceux qu'ils veulent. 

<<<<<<< HEAD
\paragraph{Mission de Récupération et de Sauvetage:} la Chute a laissé deux mondes et de nombreux habitats à l'état de ruines - mais ces cités et stations dévastées contiennent des trésors cachés pour ceux qui sont suffisament courageux et téméraires. Les trouvailles peuvent inclure: des systèmes d'armement; des ressources physiques; des banques de données perdues; des uploads abandonnés d'amis, de membres de la famille ou de personnes importantes; de nouvelles technologies développées et perdues lors du décollage brutal de la singularité; des héritages de valeur pour des oligarques immortels; et bien plus. À l'extérieur de ces royaumes autrefois habités, l'espace est en soi un endroit trés grand et beaucoup de personnes et de choses sont perdus là-bas dehors. Quelqu'uns doivent être sauvés et d'autres sont au delà du sauvetage. Cette option laisse les joueurs explorer l'inconnu ou chercher des cibles spécifiques pour un contrat. (Archéologiste/Charognard/Pirate/Libre-Échangiste/Contrebandier/Marchand Noir) 

\paragraph{Exploration:} Il y a pas mal d'opportunités à être un explorateur, un colon, ou un éclaireur avancé - peut-être même l'un des rare individus suffisament chanceux ou suicidaires et qui explorent via une Porte de Pandorre non testée. Même la Ceinture de Kuiper, à la limite de notre système solaire, est toujours partiellement explorée; il peut y avoir des trésors et des mystères à découvrir. Il y a aussi de nombreux dangers qui rôdent dans les recoins étranges du système, des factions posthumaines isolationnistes aux cartels criminels cachés, en passant par les pirates, les aliens et d'autres choses qui veulent demeurer hors de vue. (Explorateur/Archéologue/Charognard/Adepte de la Singularité/Techie/Médecin) 
=======
\subsection{Où cela se passe-t-il?} \label{sec:where-does-it} 

Alors qu'Eclipse Phase se déroule dans un futur proche, les changements qui ont été effectués suite à l'avancée technologique ont transformés la Terre et ses habitants de manière méconnaissable. Alros que les joueurs plongent dans l'univers, ils rencontrerons généralement l'un des cadres suivants. 
>>>>>>> origin/french

\paragraph{Commerce:} Alors que la majorité du commerce dans le système intérieur est contrôlé par de brillantes hypercorporations, la plupart des stations les plus petites ou indépendantes dépendent des petits négociants. Dans les systèmes post-abondance à l'extérieur, le commerce prend une forme différente, avec l'information, les faveurs et la créativité qui servent de monnaie parmi ceux qui n'ont plus besoin de quoi que ce soit grâce à la disponibilité des machines d'abondances. (Libre-Échangistes/Contrebandier/Marchand Noir/Pirate) 

\paragraph{Crime:} L'assemblage hétéroclite d'habitats de la taille d'une ville et les lois variant grandmenet à travers le système ont créés d'amples opportunités pour ceux qui voudraient faire de cette situation leur gagne-pain. Les marchandises et activités du marché noir incluent l'échange d'esclave infomorph, les industries du plaisir et des sex pods, l'échange et le vol de données, l'exfiltration et le convoyage de technologies avancées et de scientifiques, el'spionnage politique et économique, l'assassinat, la vente de drogues et d'XP, l'échange d'âme, et bien d'autres choses. Que ce soit en tant qu'indépendant ou en tant que membre d'une organisation criminelle, il y a toujorus des opportunitées pour ceux qui ont soif d'aventure ou de profit et à la moralité douteuse. (Criminel/Contrebandier/Pirate/Arrangeur/Marchand Noir/GénoHacker/Hacker/Infiltrateur) 

<<<<<<< HEAD
\paragraph{Mercenaires:} Les manœuvres permanente des factions à caractère idéologique, les disputes autour de ressources contestées, et la ruée vers la colonisation de nouvelles exoplanètes au delà des Portes de Pandorre sont à l'origine de conflits sur des bases régulières. Certains d'entre eux couvent pendant des années en tant que conflits de faible intensité, dégénérant occasionnellement en raids et affrontements. D'autres choisissent de s'affronter dans une guerre sans pitié. Des femmes et des hommes voulant prendre les armes pour des crédits sont toujours à la recherche de bonnes paies. Les joueurs peuvent s'engager dans des campagne commando et militaire dans des habitats, au milieu des étoiles, ou dans l'environnement hostile d'une planète. (Mercenaire/Consultant en Sécurité/Arrangeur/Chasseur de prime/Ex-Flic/Médecin) 

\paragraph{intrigue Socio-Politique:} Les corporations et les factions politiques qui couvrent tout le système solaire ne respectent pas toujours les règles du jeu en jouant avec les autres, mais il n'est pas simple pour autant pour eux de se confronter ouvertement les uns aux autres sauf en de rares circonstances. Beaucoup de batailles ont été remportées par des manœuvres diplomatiques et politiques, en utilisant des mots et des idées plus puissantes que des armes. A l'intérieur même des factions, différents groupes sociaux peuvent se mener une concurrence sans pitié, ou les luttes de classes surchauffées  peuvent arriver à ébullition, déchirant une société de l'intérieur. Dans cette campagne, les joueurs peuvent commencer en tant que pions d'une entité quelconque et qui gravirons les échellons alors qu'ils seront de plus en plus impliqués dans les intrigues de leur soutien, ils peuvent joueur un groupe d'ambassadeur et d'espions stationnés dans la capitale de l'opposition, ou encore jouer un groupe d'activistes et de radicaux combattant pour des changements sociaux. (Politicien/Socialite/Infiltrateur/Hacker/Consultant en Sécurité/Journaliste/Memeticien) 
>>>>>>> ab07e7255a391f48de47c3ed67a243ec64317163

Les joueurs peuvent remplir toute une variété de rôles dans Eclipse Phase. Suite à l'avancée dans les technologies de l'émulation digitale de l'esprit, l'upload et le download dans de nouvelles morphs (corps physiques, biologiques ou synthétique), il est possible d'être, littéralement, une nouvelle personne d'une session à l'autre. Avec les corps qui se retrouvent réduit au rôle d'équipement, les joueurs peuvent personnaliser leur apparence pour la tâche à venir. 


<<<<<<< HEAD
=======
\subsubsection{Habitats de l'humanité} \label{sec:humanitys-habitats} 

La Terre est devenue une ruine écologiquement dévastée, mais l'humanité s'est envolée vers les étoiles. Lorsque la Terre a été abandonnée,  il en fut autant des derniers états-nation; la transhumanité manque maintenant d'un corps gouvernemental uni et est maintenant assujettie aux lois et aux régulations de quiconque contrôle un habitat donné. 

La majorité de la transhumanité est confinée dans des habitats orbitaux ou des stations sattelites éparpillées à traver tout le système Sol. Certains d'entre eux ont été construits à partir de rien en orbite ou aux points de Lagrange des corps planétraires, d'autres ont été creusés dans des sattelites et de gros astéroïdes. Ces stations possèdent des myriades de but, du commerce à la guerre, en passant par l'espionnage et la recherche. 

Mars continue d'être l'une des plus grandes colonnies de la transhumanité, même si elle a également été durement impactée par la Chute. De nombreuses cités et colonnies persistent, même si la terraformation de la planète n'est que partielle. Vénus, la Lune et Titan abritent aussi une pouplation significative. Il faut également compter un petit nombre de colonnies établies sur des exoplanètes (de l'autre côté des Porte de Pandorre) qui possèdent des environnement peu hostile envers l'humanité. 

Certains des transhumains préfèrent vivre sur de grands vaisseau coloniaux ou sur des essaims de plus petit vaisseaux liée entre eux, en bougeant nomadiquement. Certains de ces vagabond s'exilent intenstionnelement aux confins du système solaires, loin de tout le reste du monde, alors que d'autres commercent activement d'habitat en stations, de stations en habitats, servant de marché noirs mobiles. 
>>>>>>> origin/french

\subsubsection{La campagne par défaut} \label{sec:default-campaign} 
=======
\subsection{Où cela se passe-t-il?} \label{sec:where-does-it} 

Alors qu'Eclipse Phase se déroule dans un futur proche, les changements qui ont été effectués suite à l'avancée technologique ont transformés la Terre et ses habitants de manière méconnaissable. Alros que les joueurs plongent dans l'univers, ils rencontrerons généralement l'un des cadres suivants. 
>>>>>>> ab07e7255a391f48de47c3ed67a243ec64317163

<<<<<<< HEAD
Dans l'histoire par défaut (aussi appellée "cadre de campagne"), chaque personnage joueur est une "sentinelle", un agent disponible (ou une recrue potentielle) pour un réseau paralégal appelé "Firewall". Firewall est dédié à contrer les "risques existentiels" - des menaces à l'existence de la transhumanité. Ces risques peuvent inclure des fléaux de la guerre biologique, des invasions d'essaim de nanites, de la prolifération nucléaire, des terroristes avec des ADMs, des attaques informatiques destructrices, des IAs maligne, des rencontres aliens, et ainsi de suite. Firewall ne se contente évidement pas de simplement contrer ces menaces au moment où elles apparaissent, les personnages peuvent donc être également envoyés sur des missions de renseignement ou pour mettre en place des mesures de prévention ou de sécurité. Les personnages peuvent être chargés d'enquêter sur des personnes ou des lieux apparement innofensifs (et qui se rèveleront ne pas l'être), de négocier des arrangements avec d'obscurs réseaux criminels (qui se réveleront ne pas être digne de confiance), ou de voyager à travers le trou de ver d'une Porte de Pandore pour annalyser des reliques d'une ruine alien quelconque (et de vérifier si la menace qui les as tués est toujours là). Les sentinelles sont recrutés dans toutes les factions de la transhumanité; ceux qui ne sont pas idéologicallement loyaux à la cause sont recrutés en tant que mercenaires. Ces campagnes ont tendance à se méler à un peu de mystère et d'investigation avec des scènes d'actions et de combats achanés, se baignant dans une bonne dose de crainte et d'horreur. 

=======
\subsubsection{Le grand inconnu} \label{sec:great-unknown} 

Les zones de la galaxie dans lesquelles l'homme a posé le pied sont peu nombreuses et trés éloignées les unes des autres. Reposant entre ces avant-postes occasionnels de civilisations parfois douteuse se trouvent des mystères à la fois dangereux et merveilleux. Depuis la découverte des Portes de Pandore, il n'y a pas eu pénurie d'avnturiers suffisament courageux ou intrépides pour s'aventurer seuls dans les régions inconnues de l'espace dans l'espoir de trouver un artefact étranger, voire d'établir le contact avec l'une des autres espèces consciente de l'univers. 
>>>>>>> origin/french

<<<<<<< HEAD

\subsubsection{Campagnes alternatives} \label{sec:alternate-campaigns} 

<<<<<<< HEAD
Quand ils ne sont pas en train de sauver le système solaire, les sentinelles sont libres de poursuivre leurs propres buts. Le maître de jeu et les joueurs peuvent utiliser ce livre de règle pour générer tout type d'histoire qu'ils ont envie de raconter. Cependant, les exemples suivants fournissent un rapide aperçu des opportunités les plus évidentes pour des aventures dans Elcipse Phase. 

Après chaque variantes de campagnes ci-dessous, une liste "d'archétypes" pour Eclipse Phase est fournie entre parenthèses. Les archétypes sont les noms donnés au type de personnage le plus commun rencontrés dans ces scénarios. Par exemple, dans une histoire de détective classique, les archétypes seraient le Détective, la Demoiselle en Détresse, le Flic Dur à cuir, etc. Dans un film de cowboy, les archétypes seraient le Pistolero, le Barman, le Shériff, le Brave Indien, etc. Les joueurs noterons que plusieurs archétypes correspondant à de multiple cadre de scénario. Le système de création de personnage (p. 128) permet aux joueurs de créer tous les types d'archétype suggérés. De la même manière que les jeus de rôle sont conçus pour que les joueurs construisent leurs propres histoires, ces archétypes sont juste des suggestions et les joueurs peuvent les mélanger et choisir ceux qu'ils veulent. 
=======
\subsubsection{Le mesh} \label{sec:mesh} 

Bien que n'étant pas un "cadre" dans le sens traditionnel, contrairement aux sections décrites ci-dessus, le réseau informatique connu sous le nom de "mesh" est omniprésent. La nature omniprésente de l'environnement informatique a été rendu possible grâce aux techonogies informatique avancées et à la nanofabrication qui permettent un stockage de données illimité et des capacités de transmissions quasi-instantanée. Avec des émetteurs-récepeteurs sans fil microscopique, et peu cher à fabriquer et en surabondance, absolument tout possède une connexion sans-fil et est connecté. Via des implants ou de petits ordinateurs personnels, les personnages ont accès aux archives d'information qui éclipsent l'ensemble de l'internet du 21° siècle et à des systèmes de capteurs qui imprègent chaque lieu public. La vie entière de personnes est enregistrée et commentée, partagée avec d'autres sur l'un des nombreux réseaux sociaux qui relient les gens entre eux dans une toile de contact, de faveur et de systèmes réputationnels. 
>>>>>>> origin/french

\paragraph{Mission de Récupération et de Sauvetage:} la Chute a laissé deux mondes et de nombreux habitats à l'état de ruines - mais ces cités et stations dévastées contiennent des trésors cachés pour ceux qui sont suffisament courageux et téméraires. Les trouvailles peuvent inclure: des systèmes d'armement; des ressources physiques; des banques de données perdues; des uploads abandonnés d'amis, de membres de la famille ou de personnes importantes; de nouvelles technologies développées et perdues lors du décollage brutal de la singularité; des héritages de valeur pour des oligarques immortels; et bien plus. À l'extérieur de ces royaumes autrefois habités, l'espace est en soi un endroit trés grand et beaucoup de personnes et de choses sont perdus là-bas dehors. Quelqu'uns doivent être sauvés et d'autres sont au delà du sauvetage. Cette option laisse les joueurs explorer l'inconnu ou chercher des cibles spécifiques pour un contrat. (Archéologiste/Charognard/Pirate/Libre-Échangiste/Contrebandier/Marchand Noir) 
=======
\subsubsection{Habitats de l'humanité} \label{sec:humanitys-habitats} 

La Terre est devenue une ruine écologiquement dévastée, mais l'humanité s'est envolée vers les étoiles. Lorsque la Terre a été abandonnée,  il en fut autant des derniers états-nation; la transhumanité manque maintenant d'un corps gouvernemental uni et est maintenant assujettie aux lois et aux régulations de quiconque contrôle un habitat donné. 

<<<<<<< HEAD
La majorité de la transhumanité est confinée dans des habitats orbitaux ou des stations sattelites éparpillées à traver tout le système Sol. Certains d'entre eux ont été construits à partir de rien en orbite ou aux points de Lagrange des corps planétraires, d'autres ont été creusés dans des sattelites et de gros astéroïdes. Ces stations possèdent des myriades de but, du commerce à la guerre, en passant par l'espionnage et la recherche. 

Mars continue d'être l'une des plus grandes colonnies de la transhumanité, même si elle a également été durement impactée par la Chute. De nombreuses cités et colonnies persistent, même si la terraformation de la planète n'est que partielle. Vénus, la Lune et Titan abritent aussi une pouplation significative. Il faut également compter un petit nombre de colonnies établies sur des exoplanètes (de l'autre côté des Porte de Pandorre) qui possèdent des environnement peu hostile envers l'humanité. 

Certains des transhumains préfèrent vivre sur de grands vaisseau coloniaux ou sur des essaims de plus petit vaisseaux liée entre eux, en bougeant nomadiquement. Certains de ces vagabond s'exilent intenstionnelement aux confins du système solaires, loin de tout le reste du monde, alors que d'autres commercent activement d'habitat en stations, de stations en habitats, servant de marché noirs mobiles. 
>>>>>>> ab07e7255a391f48de47c3ed67a243ec64317163

\paragraph{Exploration:} Il y a pas mal d'opportunités à être un explorateur, un colon, ou un éclaireur avancé - peut-être même l'un des rare individus suffisament chanceux ou suicidaires et qui explorent via une Porte de Pandorre non testée. Même la Ceinture de Kuiper, à la limite de notre système solaire, est toujours partiellement explorée; il peut y avoir des trésors et des mystères à découvrir. Il y a aussi de nombreux dangers qui rôdent dans les recoins étranges du système, des factions posthumaines isolationnistes aux cartels criminels cachés, en passant par les pirates, les aliens et d'autres choses qui veulent demeurer hors de vue. (Explorateur/Archéologue/Charognard/Adepte de la Singularité/Techie/Médecin) 

\paragraph{Commerce:} Alors que la majorité du commerce dans le système intérieur est contrôlé par de brillantes hypercorporations, la plupart des stations les plus petites ou indépendantes dépendent des petits négociants. Dans les systèmes post-abondance à l'extérieur, le commerce prend une forme différente, avec l'information, les faveurs et la créativité qui servent de monnaie parmi ceux qui n'ont plus besoin de quoi que ce soit grâce à la disponibilité des machines d'abondances. (Libre-Échangistes/Contrebandier/Marchand Noir/Pirate) 
=======
\subsection{Ego contre Morph} \label{sec:ego-vs.-morph} 

La distinction entre ego (votre esprit et votre personnalité, incluant les souvenirs, les connaissances et les compétences) et morph (votre corps physique et ses capacités) et l'une des caractéristique d'Eclipse Phase. Une bonne compréhension de ce concept dés le départ offrira un aperçu de toutes les possibilités narratives aux joueurs. 

Votre corps est jetable. Si il devient vieux, malade ou trop gravement abimé, vous pouvez numériser votre conscience et la télécharger dans un nouveau. Le processuss n'est ni bon marché, ni simple, mais il vous garanti une immortalité effective - tant que vous vous rappelez de vous sauvegarder et que vous ne devenez pas fou. Le terme de morph est utilisé pour décrire tout type de forme que votre esprit habite, qu'il s'agisse d'une enveloppe clonée cultivé en cuve, d'une coquille robotique et synthétique, d'un "pod" en partie bio et en partie synthétique, ou même de l'état purement logiciel d'une informoph. 

La morph d'un personnage peu mourir, mais l'ego du personnage continuera à vivre, du moment que les mesures de sauvegardes nécessaires ont été prises. Les morphs sont immuables, mais l'ego de votre personnage représente la continuité des chemins pris par son esprit et sa personnalité tout au long de sa vie. Cette continuité peut-être interrompue par une mort inattendue (dépendant de la date de la dernière sauvegarde), mais elle représente la somme de l'état mental du personnage et de ses expériences. 

Des apects de votre personnage - en particulier les compétences, ainsi que quelques traits et statistiques - appartiennent à l'ego de votre personnage et ainsi l'acompagnent tout au long du développement du personnage. D'autres statistiques et traits sont cependant déterminés par une morph, comme noté précédemment, et changeront donc si votre personnage quitte son corps et en prend un autre. Les morphs peuvent aussi affecter d'autres compétences et statistiques, comme détaillée dans la description des morphs. 
>>>>>>> origin/french

<<<<<<< HEAD
\paragraph{Crime:} L'assemblage hétéroclite d'habitats de la taille d'une ville et les lois variant grandmenet à travers le système ont créés d'amples opportunités pour ceux qui voudraient faire de cette situation leur gagne-pain. Les marchandises et activités du marché noir incluent l'échange d'esclave infomorph, les industries du plaisir et des sex pods, l'échange et le vol de données, l'exfiltration et le convoyage de technologies avancées et de scientifiques, el'spionnage politique et économique, l'assassinat, la vente de drogues et d'XP, l'échange d'âme, et bien d'autres choses. Que ce soit en tant qu'indépendant ou en tant que membre d'une organisation criminelle, il y a toujorus des opportunitées pour ceux qui ont soif d'aventure ou de profit et à la moralité douteuse. (Criminel/Contrebandier/Pirate/Arrangeur/Marchand Noir/GénoHacker/Hacker/Infiltrateur) 

\paragraph{Mercenaires:} Les manœuvres permanente des factions à caractère idéologique, les disputes autour de ressources contestées, et la ruée vers la colonisation de nouvelles exoplanètes au delà des Portes de Pandorre sont à l'origine de conflits sur des bases régulières. Certains d'entre eux couvent pendant des années en tant que conflits de faible intensité, dégénérant occasionnellement en raids et affrontements. D'autres choisissent de s'affronter dans une guerre sans pitié. Des femmes et des hommes voulant prendre les armes pour des crédits sont toujours à la recherche de bonnes paies. Les joueurs peuvent s'engager dans des campagne commando et militaire dans des habitats, au milieu des étoiles, ou dans l'environnement hostile d'une planète. (Mercenaire/Consultant en Sécurité/Arrangeur/Chasseur de prime/Ex-Flic/Médecin) 
=======
\subsubsection{Le grand inconnu} \label{sec:great-unknown} 

<<<<<<< HEAD
Les zones de la galaxie dans lesquelles l'homme a posé le pied sont peu nombreuses et trés éloignées les unes des autres. Reposant entre ces avant-postes occasionnels de civilisations parfois douteuse se trouvent des mystères à la fois dangereux et merveilleux. Depuis la découverte des Portes de Pandore, il n'y a pas eu pénurie d'avnturiers suffisament courageux ou intrépides pour s'aventurer seuls dans les régions inconnues de l'espace dans l'espoir de trouver un artefact étranger, voire d'établir le contact avec l'une des autres espèces consciente de l'univers. 
>>>>>>> ab07e7255a391f48de47c3ed67a243ec64317163

\paragraph{intrigue Socio-Politique:} Les corporations et les factions politiques qui couvrent tout le système solaire ne respectent pas toujours les règles du jeu en jouant avec les autres, mais il n'est pas simple pour autant pour eux de se confronter ouvertement les uns aux autres sauf en de rares circonstances. Beaucoup de batailles ont été remportées par des manœuvres diplomatiques et politiques, en utilisant des mots et des idées plus puissantes que des armes. A l'intérieur même des factions, différents groupes sociaux peuvent se mener une concurrence sans pitié, ou les luttes de classes surchauffées  peuvent arriver à ébullition, déchirant une société de l'intérieur. Dans cette campagne, les joueurs peuvent commencer en tant que pions d'une entité quelconque et qui gravirons les échellons alors qu'ils seront de plus en plus impliqués dans les intrigues de leur soutien, ils peuvent joueur un groupe d'ambassadeur et d'espions stationnés dans la capitale de l'opposition, ou encore jouer un groupe d'activistes et de radicaux combattant pour des changements sociaux. (Politicien/Socialite/Infiltrateur/Hacker/Consultant en Sécurité/Journaliste/Memeticien) 
=======
\subsection{Où aller maintenant?} \label{sec:where-go-from} 

Maintenant que vous savez de quoi parle ce jeu, nous vous suggérons de lire le chapitre Une Époque d'Eclipse (p. 30), pour avoir une idée du cadre de jeu par défaut (que vous êtes, bien entendu, libre de changer pour  l'adapter à vos envies). Lisez ensuite le chapitre Mécaniques de Jeu (p. 112) pour avoir une idée des règles. Après ça, vous pouvez passer à la Création et Évolution de Personnage (p. 128) et créer votre premier personnage! 
>>>>>>> origin/french


<<<<<<< HEAD

<<<<<<< HEAD
\subsection{Où cela se passe-t-il?} \label{sec:where-does-it} 
=======
\subsubsection{Le mesh} \label{sec:mesh} 

Bien que n'étant pas un "cadre" dans le sens traditionnel, contrairement aux sections décrites ci-dessus, le réseau informatique connu sous le nom de "mesh" est omniprésent. La nature omniprésente de l'environnement informatique a été rendu possible grâce aux techonogies informatique avancées et à la nanofabrication qui permettent un stockage de données illimité et des capacités de transmissions quasi-instantanée. Avec des émetteurs-récepeteurs sans fil microscopique, et peu cher à fabriquer et en surabondance, absolument tout possède une connexion sans-fil et est connecté. Via des implants ou de petits ordinateurs personnels, les personnages ont accès aux archives d'information qui éclipsent l'ensemble de l'internet du 21° siècle et à des systèmes de capteurs qui imprègent chaque lieu public. La vie entière de personnes est enregistrée et commentée, partagée avec d'autres sur l'un des nombreux réseaux sociaux qui relient les gens entre eux dans une toile de contact, de faveur et de systèmes réputationnels. 
>>>>>>> ab07e7255a391f48de47c3ed67a243ec64317163

Alors qu'Eclipse Phase se déroule dans un futur proche, les changements qui ont été effectués suite à l'avancée technologique ont transformés la Terre et ses habitants de manière méconnaissable. Alros que les joueurs plongent dans l'univers, ils rencontrerons généralement l'un des cadres suivants. 

=======
\subsection{Terminologie} \label{sec:terminology} 

Eclipse Phase utilise tout un jargon pour transmettre simplement les nombreux concepts couverts par ce livre. Bien que non exhaustive, cette liste de terme permettra aux joueurs de s'acclimater rapidement à leur voyage dans Eclipse Phase. Si vous lisez quelque chose et que vous êtes perdus, ne vous inquiétez pas. Ces concepts sont entièrement détaillés dans d'autres sections du livre. 

Notez que plusieurs des mots sur la liste sont des termes scientifiques standard, souvent utilisé en astronomie. Comme Eclipse Phase essaye de rester aussi proche que possible du "hard science" - tout en permettant aux joueurs d'interagir avec les passionantes histoires qui attendent d'être révélées - de tels termes sont utilisés librement. 

\end{itemize} \item Aérostat: Un habitat conçu pour flotter comme un ballon dans la haute atmosphère d'une planète. \item AF: (After the Fall) Après la Chute (utilisé comme date de référence). \item IAG: Intelligence Artificielle Généraliste Une IA qui possède des facultés cognitives comparables ou supérieures à celle d'un humain. Aussi connues sous le nom de "IA forte" (par opposition aux "IA limitées" plus spécialisée). Voir aussi "IA germe." \item IA: Intelligence Artificielle. Généralement utilisé pour se référer à une IA faible; c'est à dire des IAs qui n'englobent pas (ou dans certains cas, qui sont complètement hors de) toute la portée des capacités cognitives humaines. Les IAs diffèrent des IAG dans le fait qu'elles sont généralement spécialisée et/ou intentionnellement bridée/limitée. \item Anarchiste: Quelq'un qui croit que le gouvernement n'est pas nécessaire, que le pouvoir corromp, et que les gens doivent contrôller leur propre vie à travers l'auto-organisation individuelle et l'action collective. \item Arachnoïde: Une synthmorph robotique ressemblant à une araignée. \item Argonautes: Une faction de scientifique tecno-progressistes qui font la promotion d'une utilisation responsable et éthique de la technologie. \item RA: Réalité Augmentée. Informations du mesh (réseau de donnée universel) qui sont superposées à vos sens du monde réel. Les données RA sont habituellement entoptique (visuelles), mais peuvent aussi être auditives, tactiles, olfactives, kinesthésique (conscience corporelle), émotionnelles ou tout autre type d'entrée. \item Async: Une personne avec des pouvoirs psi. \item UA: Unité Astronomique La distance entre la Terre et le Soleil, équivalent à 8,3 minutes lumières, ou à peu près 150 millions de kilomètres. \item Autonomistes: L'alliance des anarchistes, des Barsoomiens, des Extropiens, de la racaille et des Titaniens. \item Barsoomien: Un Martien rural, typiquement irrité par le contrôle des hypercorp. \item Piratage Basilique: Une image ou toute autre entrée sensorielle qui affecte le cortex visuel du cerveau aisni que ses capacités de recognition de motifs d'une manière à provoquer une erreur et de peut-être l'exploiter pour réécrire du code neuronal. \item Ruche: Un habitat à microgravité créé dans un astéroïde ou une lune évidée. \item BF: (Before the Fall) Avant la Chute (utilisé comme date de référence). \item Bioconservateurs: Un mouvement anti-technologie qui milite pour une régulation stricte de la nanofabrication, des IA, de l'upload, du fork, des améliorations cognitives et de toute autre techonologie perturbatrice. \item Biomorph: Un corps bilogique, qu'i ls'agisse d'un plat, d'un splicer, d'un transhumain génétiquement amélioré ou d'un pod. \item Banque de Corps: Un service pour louer, vendre, acheter  ou stocker une morph. Aussi appelé maison de poupée, morgue \item Bots: Robots. Desz coquilles synthétiques pilotée par des IA. \item Sonde Bracewell: Un type de sonde autonome de surveillance de l'espace profond conçue pour établir le contact avec des civilisations étrangères. \item Brinkeurs: Des éxilés qui vivent aux limites du système, ansi que dans tous les autres coins et recoins bien cachés du système. Aussi appellée isolés, limités, dériveurs. \item Caisse: Une coquille synthétique bon marché, commune et produite en masse. \item Chimère: Un transgénique, contenant des traits génétiques d'autre espèces. \item Circumjovien: Orbitant autour de Jupiter. \item Circumlunaire: Orbitant autour de la Lune. \item Circumsolaire: Orbitant autour du Soleil. \item Cislunaire: Entre la Terre et la Lune. \item Clade: Une espèce ou un groupe d'organisme partageant des caractéristique. Utilisé pour se référer aux sous-espèces transhumaines et au types de morph. \item Bulle Cole: Un habitat formé d'un astéroïde ou d'une lune évidée et en rotation pour obtenir une gravité. \item Machine d'Abondance: Un nanofabeur a but généraliste. \item Pile Corticalle: Une cellulle de mémoire implantée et utilisée pour sauvegarder un ego. Localisée là où l'épine dorsale rencontre le crâne; elle peut-être extraite. \item Cybercerveau: Un cerveau artificiel, hébergeant un ego. Utilisé à la fois dans les synthmorphs et dans les pods. \item Darkcast: Services de farcast et d'egocast ilégaux et trouvés au marché noir. \item Règles de Domaine: Les règles qui régissent la réalité dans un simulspace de réalité virtuelle. \item Drône: Un robot conrtrollé par téléopération (plutôt que par des IA embarquées). \item Ecto: Périphérique de mesh personnel, souple, étirable, auto-nettoyant, translucide et alimenté par énergie solaire. De ecto-lien (lien externe). \item Ego: La part de vous qui bascule d'un corps à l'autre. Aussi appelé ghost, âme, essence, esprit, persona. \item Egocaster: Terme pour envoyer un ego par farcasting. \item Entoptiques: Images de Réalité Augmentée que vous "voyez" dans votre tête. ("Entoptique" signifie "à l'intérieur des yeux") \item ETI: Intelligence extra-terrestre. Le terme utilisé par Firewall pour faire référence aux intelligence étrangères post-singularité de niveau divin théoriquement responsable du virus Exsurgent. \item Exaltés: Humains génétiquement améliorés (entre génétiquement réparés et transhumains). Aussi connus comme génomonstre, les ascendants, les élevés. \item Exoplanète: Une planète dans un autre système solaire. \item Exsurgent: Quelqu'un infecté par le virus Exsurgent \item Virus Exsurgent: Le virus multi-vecteur créé par un ETI inconnu et répandu dans la galaxie dans des sondes Bracewell. Le virus Exsurgent est mutant et peu infecter à la fois les systèmes informatique et les créatures biologiques. \item Extrasolaire: Hors du système solaire. \item Facteurs: La race étrangère ambassadrice qui fait affaire avec la transhumanité. Aussi appelés les Courtiers. \item La Chute: L'apocalypse; la singularité et les guerres qui ont presque amenées l'extinction de l'humanité. \item Farcasting: Communication intrasolaire utilisant des technologies de communication classiques (radio, laser, etc) et la téléportation quantique. Parfois appelée Hyeprdiff. \item Long Porteur: Transport spatiaux de longue distance. \item Firewall: La conspiration secrète, multifaction qui travaille à protéger la transhumanité des "risques existentiels" (risques qui menacent l'existence de la transhumanité). \item Bas de plancher: Quelqu'un qui est né ou habitué à vivre sur une planète ou une lune avec une gravité. \item Plats: Humains de base (sans modificatiosn génétique). Aussi appelés norms. \item Flexbot: Une synthmorph capable de changer de forme ou de rejoindre d'autre trasnformers afin de créer des forms plus grande et modulaires. Aussi appelé Transformers \item Forker: Copier un ego. Tous les forks ne sont pas des copies complètes. Aussi appelés sauvegardes. \item FTL: Faster-Than-Light. Plus rapide que la lumière. \item Fury: Une morph de combat transhumaine. \item Resquilleurs: Explorateurs qui tentent leur chances en utilisant une Porte de Pandorre pour aller vers un endroit pour l'instant inexploré. \item Génohacker: Quelqu'un qui manipule le code génétique pour créer des modifications génétiques voire même de nouvelles formes de vie. \item Ghost: Une morph de combat transhumaine optimisé pour la furtivisté et  l'infiltration. \item Ghost-riding: L'acte de transporter une infomorph dans un implant sépcial à l'intérieur de votre tête. \item Grecs: Astéroîdes ou lunes troyens qui partagent la même orbite qu'une planète ou lune plus grosse, mais qui ont 60 degrés d'avance sur l'orbite, au point de Lagrange L4. Le terme Grecs fait normalement référence aux astéroïdes orbitant autour du point L4 de Jupiter. Voir aussi à "Troyens." \item Habtech: Un technicien d'habitation. \item Héliopause: Le point auquel la pression des vents solaires s'équilibre avec les moyennes interstellaires (aux allentours de 100 AU). \item Hibernoïdes: Un transhumain modifié pour l'hibernation, pour des travaux prolongés dans l'espace. \item Glacetéroïde: Un astéroïde constitué essentiellemnt de glace au lieu de roche et de métal. \item Iktomi: Le nom donné à la mystérieuse race étrangère dont les reliques ont été retrouvées au delà des Portes de Pandorre. \item Contractés: Des esclaves sous contrats synallagmatique qui ont signé pour travailer avec une hypercorp ou une autre autorité, habituellement en échange d'une morph. \item Infovie: Intelligence artificielle généraliste et IAs germe. \item Infomorph: Un égo digitalisé; un corps virtuel. Aussi connus sous les noms de datamorph, uploads, sauvegardes. \item Infugié: ``Infomorph refugié,'' ou quelqu'un qui a tout abandonné sur Terre - y compris son corp - pendant la Chute. \item Isolés: Ceux qui vivent dans des communautés isolées loin au-delà des limites du système (dans la Ceinture de Kuiper et le Nuage d'Oort); aussi appelés outsters, limités. \item Saturer: l'acte de "devenir" un drône opéré à distance grâce à la technologie XP. Également utilisé pour l'accession à un flux XP en temps-réel de lifeblogeur et autre émetteurs en temps-réel. \item Ceinture de Kuiper: Une région de l'espace partant de l'orbite de Neptune et s'étalant sur envrion 55 UA, légèrement peuplée d'astéroïdes, de comètes et de planètes naines. \item Point de Lagrange: L'une des cinqs zones relative à un petit corps planétaire orbitant autour d'un plus gros dans lesquelles les forces gravitationnelles de ces deux corps sont neutralisées. Les point de Lagrange sont considérés comme stables et sont des positions idéales pour des habitats. \item Lifeblog: L'enregistrement de toute l'expérience de la vie de quelqu'un, rendue possible grâce aux capacités mémoires des ordinateurs quasi-illimitées. \item Generation Égarée: Dans une tentative de repeupler après la Chute, une génération d'enfant fut élevée en utilisant des techniques de croissance forcée. Les résultats furent désastreux: beaucoup sont morts ou devenus fous, et le reste a été stigmatisé. \item Ceinture Principale: La principale ceinture d'astéroïdes, un anneau torique orbitant entre Mars et Jupiter. \item Meme: Une idée virale. \item Mentalistes: Transhumains optimisés pour les compétences mentales et cognitive. \item Mercuriels: Les éléments conscient non-humains de la "famille" transhumaine; incluant les IAG et les animaux éveillés. \item Mesh: L'omniprésent maillage sans-fil de réseau de données. Egalement utilisé comme verbe (mesher) et comme adjectif (meshé ou nonmeshé). \item Mesh ID: La signature unique attachée à l'activité meshée de quelqu'un. \item Microgravité: Zéro-g ou environnement quasiment sans poids. \item Mist: Les nuages de données RA qui brouillent parfois votre perception et vos affichages. \item Morph: Un corps physique. Aussi appelé costume, veste, gaine, coquille, forme. \item Muse: IA d'assitant personnel. \item Nanobot: Une machine nanoscopique. \item Nano-écologie: Mouvement écologique pro-technologie. \item Essaim de nanite: Une masse de petits nanobots libérée dans un environnement. \item Neo-Aviens: Perroquets gris et corbeaux élevés. \item Néogenèse: La création d'une nouvelle forme de vie grâce aux manipulations génétiques et à la biotechnologies. \item Neo-Hominidés: Chimpanzés, gorilles et orang-outans élevés \item Néoteniques: Transhumains modifiés pour conserver une forme enfantine. \item Novacrabe: Un pod créé à partir de crabe araignés génétiquement conçus. \item Olympien: Une biomorph transhumaine modifiée pour l'athéltisme et l'endurance. \item Cylindre O'Neill: Un habitat en forme de canette, soumis à une rotation pour créer une gravité. \item Nuage d'Oort: Le "nuage" sphérique constitué de comètes qui entoure le système solaire et qui s'étend jusqu'à une année lumière du soleil. \item PAN: Personal area network/Réseau personnel. Le réseau créé lorsque vous asservissez tous vos périphériques électroniques mineur à votre ecto ou votre insert de mesh. \item Porte de Pandorre: Les portails de trou-de-ver abandonné par les TITANs. \item Pods: Des morphs à la fois biologique et synthétiques. Les clones utilisé pour créer les pods subissent une croissance forcée et possèdent un cerveau informatique. Aussi appelé bio-bots, pelure, répliquants. \item Posthumain: Un individu, humain ou un transhumain, ou une espèce qui a été génétiquement ou cognitivement modifié à un point qu'il n'est plus réellement humain (un cran au-dessus de transhumain). Aussi appelé parahumain. \item Prométhéens: Un groupe d'IAs germes pro-transhumaine créées par le Projet Canot de Sauvetage (précurseurs des argonautes) des années avant que les TITANS ne développent une conscience d'eux et qui ont (presque) évitées l'Infection Exsurgente. Les Prométhéens travaillent secrètement en soutien de Firewall et luttent contre les menaces existentielles. \item Proxys: Membres de la structure interne de Firewall. \item Psi: Pouvoirs parapsychologique développés suite à l'infection par la souche Watts-MacLeod du virus Exsurgent. \item Reapeur: Une synthmorph de combat. \item Réclamationnistes: Une faction transhumanistes qui cherche à lever l'interdiction et à récupérer la Terre. \item Redneck: Un Martien rural. Voir Barsoomien. Aussi appelés Reds. \item Reinstantiés: Réfugiés de la Terre qui se sont échappées sous la forme d'infomorph sans corps, mais qui ont depuis été réincarné. \item Se Réincarner: Changer de corps, ou être téléchargé dans un nouveau corps. Aussi appelé remorphing, regainage, basculer, renaissance. \item Rusteur: Biomorph optimisée pour la vie sur Mars. \item Scorcheur: Programme hostile qui peut endommager ou affecter un cybercerveau. \item Racaille: La faction nomade de punks/gitans de l'espace qui voyagent de stations en stations dans des barges lourdement modifiées ou dans des nuées de vaisseaux. Connus pour être des marchés noirs errants. \item IA germe: Une IAG capable d'auto-apprentissage récursif, lui permettant d'atteindre des niveaux d'intelligences similaire à ceux des dieux. \item Sentinelles: Agents de Firewall \item Coquille: Une morph physique synthétique. Aussi appellé synthmorph. \item Simulmorph: L'avatar que vous utilisez dans les simulspace RV. \item Simulspace: Environnement de réalité virtuelle permettant une immersion sensorielle complète. \item Singularité: Un point de progrés technologique rapide, exponentiel et récursif, au-delà duquel le futur devient impossible à prévoir. Souvent utilisé pour faire référence à l'ascenssion des IAs germe à des niveaux d'intelligence divins. \item Adepte de la Singularité: Des personnes qui cherchent des reliques et des preuves que les TITANs ou d'autres super-intelligences, soit pour en apprendre plus sur eux ou pour devenir une super-intelligence. \item Peau: Une morph physique biologique. Aussi appelé viande, chair. \item Habiller: Modifier son environnement perçu par la réalité augmenté grâce à des programmes. \item Exploit Psi: Un pouvoir psi. \item Slitheroïde: Une synthmorph robotique en forme de serpent. \item Animaux Intelligents: Espèces animales partiellement élevées (incluant chiens, chats, rats et cochons). D'autres gros animaux intelligents (baleines, éléphants) sont au bord de l'extinction. \item Spimes: Périphériques meshé, conscient et localisés. \item Spliceurs: Humains qui sont génétiquement modifiés pour éliminer les maladies génétiques et quelques autres aspects. Aussi connu comme génofixé, génolavés, bidouillés. \item Swarmanoïde: Une morph synthétique composé d'un essaim de robots de la taille d'un insecte. \item Sylphes: Biomorph transhumaine d'une exotique beauté  \item Synthmorph: Morphs syntéhtiques. Coquilles robotiques possédant des égos transhumains. \item Synths: Un type spécifique de synthmorph. Les synths sont des androïdes/gynoïdes classiques; des robots conçus pour être humanoïde, bien qu'ils soit facile de remarquer qu'ils ne sont pas humains. \item Téléopération: Contrôle à distance. \item Titanien: Quelqu'un qui vient de Titan, l'une des lunes de Saturne. \item TITANs: Les IAs germes créées par l'homme, capable d'apprentissage récursifs qui ont subit un décollage abrupte de la singularité et qui ont déclenchés la Chute. La désignation militaire originelle était TITAN: Total Information Tactical Awareness Network (Réseau Cognitif d'Information Tactique Complète). \item Tore: Un habitat en forme de donut, soumit à une rotation pour générer de la gravité. \item Transgénique: Qui contient des traits génétiques d'autres espèces. \item Transhumain: Un humain largement modifié. \item Troyens: Astéroïdes ou lunes qui partagent la même orbite qu'une autre planète ou lune, mais qui la suit avec 60° de décalage, à l'avant ou à l'arrière au poinst de Lagrange L4 ou L5. Le terme de Troyens fait noramellement référence aux astéroïdes orbitant aux point de Lagrange de Jupiter, mais Mars, Saturne, Neptune et d'autres corps ont aussi des Troyens. Voir aussi "Grecs." \item Élever: Élever à la conscience un animal en le transformant génétiquement. \item Travailleur du Vide: Ouvrier de l'espace. \item Vapor: Une émulation cognitive ratée ou un fork/une infomorph criblé de défaut (dérivé de vaporware). \item VPNs: Virtual private networks/Réseaux Privés Virtuels Des réseaux transitant à travers le mesh, habituellement chiffrés pour la protection de la vie privé et pour la sécurité. \item RV: Réalité Virtuelle. Imposer une réalité hyper-réaliste construite artificiellement par-dessus les sens physique de quelqu'un. \item X-Diffeur: Quelqu'un qui transmet et vends des enregistrement XP de leurs propres expérience (dérivé de X-Diffuseurs). \item Xénomorph: Forme de vie étrangères. \item Xé: Comme dans "X-é" - quelqu'un qui est accros ou obsédé par les XP. Fait parfois également référence aux personnes qui font de l'XP. \item XP: Experience Playback/Lecture d'expérience. Faire l'expérience des entrées sensorielle de quelqu'un d'autre (en temps réel ou après enregistrement). Aussi appelé experia, sim, simsense, playback. \item Risque X: Risque existentiel. Quelque chose qui menace l'existence même de la transhumanité. \item Zéros: Personnes sans accès sans-fil au mesh. Commun chez certains contractés. \end{itemize} 
>>>>>>> origin/french

<<<<<<< HEAD

<<<<<<< HEAD
\subsubsection{Habitats de l'humanité} \label{sec:humanitys-habitats} 

La Terre est devenue une ruine écologiquement dévastée, mais l'humanité s'est envolée vers les étoiles. Lorsque la Terre a été abandonnée,  il en fut autant des derniers états-nation; la transhumanité manque maintenant d'un corps gouvernemental uni et est maintenant assujettie aux lois et aux régulations de quiconque contrôle un habitat donné. 

La majorité de la transhumanité est confinée dans des habitats orbitaux ou des stations sattelites éparpillées à traver tout le système Sol. Certains d'entre eux ont été construits à partir de rien en orbite ou aux points de Lagrange des corps planétraires, d'autres ont été creusés dans des sattelites et de gros astéroïdes. Ces stations possèdent des myriades de but, du commerce à la guerre, en passant par l'espionnage et la recherche. 

Mars continue d'être l'une des plus grandes colonnies de la transhumanité, même si elle a également été durement impactée par la Chute. De nombreuses cités et colonnies persistent, même si la terraformation de la planète n'est que partielle. Vénus, la Lune et Titan abritent aussi une pouplation significative. Il faut également compter un petit nombre de colonnies établies sur des exoplanètes (de l'autre côté des Porte de Pandorre) qui possèdent des environnement peu hostile envers l'humanité. 
=======
\subsection{Ego contre Morph} \label{sec:ego-vs.-morph} 

La distinction entre ego (votre esprit et votre personnalité, incluant les souvenirs, les connaissances et les compétences) et morph (votre corps physique et ses capacités) et l'une des caractéristique d'Eclipse Phase. Une bonne compréhension de ce concept dés le départ offrira un aperçu de toutes les possibilités narratives aux joueurs. 

Votre corps est jetable. Si il devient vieux, malade ou trop gravement abimé, vous pouvez numériser votre conscience et la télécharger dans un nouveau. Le processuss n'est ni bon marché, ni simple, mais il vous garanti une immortalité effective - tant que vous vous rappelez de vous sauvegarder et que vous ne devenez pas fou. Le terme de morph est utilisé pour décrire tout type de forme que votre esprit habite, qu'il s'agisse d'une enveloppe clonée cultivé en cuve, d'une coquille robotique et synthétique, d'un "pod" en partie bio et en partie synthétique, ou même de l'état purement logiciel d'une informoph. 

La morph d'un personnage peu mourir, mais l'ego du personnage continuera à vivre, du moment que les mesures de sauvegardes nécessaires ont été prises. Les morphs sont immuables, mais l'ego de votre personnage représente la continuité des chemins pris par son esprit et sa personnalité tout au long de sa vie. Cette continuité peut-être interrompue par une mort inattendue (dépendant de la date de la dernière sauvegarde), mais elle représente la somme de l'état mental du personnage et de ses expériences. 

Des apects de votre personnage - en particulier les compétences, ainsi que quelques traits et statistiques - appartiennent à l'ego de votre personnage et ainsi l'acompagnent tout au long du développement du personnage. D'autres statistiques et traits sont cependant déterminés par une morph, comme noté précédemment, et changeront donc si votre personnage quitte son corps et en prend un autre. Les morphs peuvent aussi affecter d'autres compétences et statistiques, comme détaillée dans la description des morphs. 
>>>>>>> ab07e7255a391f48de47c3ed67a243ec64317163

Certains des transhumains préfèrent vivre sur de grands vaisseau coloniaux ou sur des essaims de plus petit vaisseaux liée entre eux, en bougeant nomadiquement. Certains de ces vagabond s'exilent intenstionnelement aux confins du système solaires, loin de tout le reste du monde, alors que d'autres commercent activement d'habitat en stations, de stations en habitats, servant de marché noirs mobiles. 


<<<<<<< HEAD

\subsubsection{Le grand inconnu} \label{sec:great-unknown} 
=======
\subsection{Où aller maintenant?} \label{sec:where-go-from} 

Maintenant que vous savez de quoi parle ce jeu, nous vous suggérons de lire le chapitre Une Époque d'Eclipse (p. 30), pour avoir une idée du cadre de jeu par défaut (que vous êtes, bien entendu, libre de changer pour  l'adapter à vos envies). Lisez ensuite le chapitre Mécaniques de Jeu (p. 112) pour avoir une idée des règles. Après ça, vous pouvez passer à la Création et Évolution de Personnage (p. 128) et créer votre premier personnage! 
>>>>>>> ab07e7255a391f48de47c3ed67a243ec64317163

Les zones de la galaxie dans lesquelles l'homme a posé le pied sont peu nombreuses et trés éloignées les unes des autres. Reposant entre ces avant-postes occasionnels de civilisations parfois douteuse se trouvent des mystères à la fois dangereux et merveilleux. Depuis la découverte des Portes de Pandore, il n'y a pas eu pénurie d'avnturiers suffisament courageux ou intrépides pour s'aventurer seuls dans les régions inconnues de l'espace dans l'espoir de trouver un artefact étranger, voire d'établir le contact avec l'une des autres espèces consciente de l'univers. 
=======
\textbf{Bienvenue à Firewall} 

[Message Entrant Reçus. Source: Inconnue] 

[Analyse Quantique: Pas d'interception Détectée] 

[Déchiffrement Complet] 

Salutations, 

Vos références et votre histoire ont été vérifiées trois fois et confirmées, et vous êtes maintenant validé en tant que processus sentinelle. Bienvenue à Firewall, l'ami. 

Pour ceux qui arrivent juste dans notre réseau privé, Firewall est une organisation dévouée à la protection de la transhumanité des menaces - à la fois internes et externes - et à la persistence de notre espèce. La Chute nous a peut-être rappelé que notre capacité à survivre et prospérer n'était pas garantie, mais les notres ont un spectre d'attention remarquablement réduit. En dépit de notre réalisation d'une quasi-immortalité fonctionnelle, nous continuons de à faire face à de nombreux dangers qui pourraient contribuer à notre extinction. Certains de ces risques viennent de notre propre factionnalisme et de nos divisions, combiné à de la technologie universellement disponible qui pourrait causer une destruction étenduée ou des décès indicible si elles tombaient dans les mauvaises mains. Certains viennent de notre manque de vision à long terme, incapables de voir les dangers dans lesquels nous nous sommes plongés entraînant notre nevironnement à cause d'actions imprudentes. D'autres proviennent de nos prorpres créations qui se sont retournées contre nous, comme les TITANs l'ont prouvé. D'autres risques peuvent venir d'intelligence étrangère aux motivations que nous ne pouvons pas encore deviner, et dont nous pourrions ne jamais avoir conscience. D'autres enfin pourraient nous menacer par pur hasard et la plus stupide, mais néanmoins meurtrière, causalité d'un univers dans lequel nous ne sommes rien d'autre que d'insignifiantes poussières. 

Firewall existe pour identifier, analyser et contrer ces risques. Nous sommes tous volontaires. Nous mettons tous nos vies en danger afin d'assurer la survie de la tranhsumanité. 

Firewall a existé, sous des noms et des formes différentes, bien avant la Chute. De nombreuses agences avec des plans similaires se sont regroupé à l'aube de ces évènements cataclysmique pour faire un point sur notre situation et nous préparer au pire. Maintenant, nous opérons sous une seule bannière. 

Nous sommes un réseau privé pour deux raisons. Premièrement, notre existence et nos capacités opératoires sont protégées par notre secret. Moins notre opposition sait de choses sur nous, plus nous pouvons les contrer de manière efficace. De manière similaire, certaines autorités pourraient être hostiles à une organisation telle que la notre opérant dans les territoires qu'elles proclament comme les leurs. Bien que certains d'entre eux doivent être au courant de notre existence, nous passons outre de nombreux obstacles juridiques et légaux qui pourraient sinon entraver nos actions et nos objectifs. Deuxièmement, il arrive que notre mission révèle des informations qui ne sont pas seulement dangereuses dans les mauvaises mains, mais qui pourraient en plus déclencher une panique généralisée si elles étaient rendues publique. Dans certains cas, l'existence même d'une telle connaissance peut-être problématique. En conservant ces secrets et en opérant sur lprincipe que vous savez ce que vous avez besoin de savoir, nous controns automatiquement certains risques. 

Firewall est un réseau décentralisé, de pair à pair. Nous avons une hiérarchie minimale et nous ne répondons à personne d'autre qu'à nous-même. Notre structure nodale nous permets de partager des ressources et des talents sans sacrifier la sécurité et la vie privée de nos agents de terrain. Vous avez été recrutés à cause de vos connaissances, possession ou compétences, et/ou parceque vous êtes entrés en contact avec certaines données d'accès restreint. Vous avez prouvé votre volonté à défendre nos objectifs. Nos vies et nos existences - et le futur de la transhumanité - peuvent reposer entre vos mains. 

Voici donc le futur - que nous puissions tous survivre pour le voir. 

[Fin du Message] 

[Ce message s'est auto-supprimé] 
>>>>>>> origin/french


<<<<<<< HEAD

\subsubsection{Le mesh} \label{sec:mesh} 

Bien que n'étant pas un "cadre" dans le sens traditionnel, contrairement aux sections décrites ci-dessus, le réseau informatique connu sous le nom de "mesh" est omniprésent. La nature omniprésente de l'environnement informatique a été rendu possible grâce aux techonogies informatique avancées et à la nanofabrication qui permettent un stockage de données illimité et des capacités de transmissions quasi-instantanée. Avec des émetteurs-récepeteurs sans fil microscopique, et peu cher à fabriquer et en surabondance, absolument tout possède une connexion sans-fil et est connecté. Via des implants ou de petits ordinateurs personnels, les personnages ont accès aux archives d'information qui éclipsent l'ensemble de l'internet du 21° siècle et à des systèmes de capteurs qui imprègent chaque lieu public. La vie entière de personnes est enregistrée et commentée, partagée avec d'autres sur l'un des nombreux réseaux sociaux qui relient les gens entre eux dans une toile de contact, de faveur et de systèmes réputationnels. 

=======
\subsection{Terminologie} \label{sec:terminology} 

Eclipse Phase utilise tout un jargon pour transmettre simplement les nombreux concepts couverts par ce livre. Bien que non exhaustive, cette liste de terme permettra aux joueurs de s'acclimater rapidement à leur voyage dans Eclipse Phase. Si vous lisez quelque chose et que vous êtes perdus, ne vous inquiétez pas. Ces concepts sont entièrement détaillés dans d'autres sections du livre. 

Notez que plusieurs des mots sur la liste sont des termes scientifiques standard, souvent utilisé en astronomie. Comme Eclipse Phase essaye de rester aussi proche que possible du "hard science" - tout en permettant aux joueurs d'interagir avec les passionantes histoires qui attendent d'être révélées - de tels termes sont utilisés librement. 

\end{itemize} \item Aérostat: Un habitat conçu pour flotter comme un ballon dans la haute atmosphère d'une planète. \item AF: (After the Fall) Après la Chute (utilisé comme date de référence). \item IAG: Intelligence Artificielle Généraliste Une IA qui possède des facultés cognitives comparables ou supérieures à celle d'un humain. Aussi connues sous le nom de "IA forte" (par opposition aux "IA limitées" plus spécialisée). Voir aussi "IA germe." \item IA: Intelligence Artificielle. Généralement utilisé pour se référer à une IA faible; c'est à dire des IAs qui n'englobent pas (ou dans certains cas, qui sont complètement hors de) toute la portée des capacités cognitives humaines. Les IAs diffèrent des IAG dans le fait qu'elles sont généralement spécialisée et/ou intentionnellement bridée/limitée. \item Anarchiste: Quelq'un qui croit que le gouvernement n'est pas nécessaire, que le pouvoir corromp, et que les gens doivent contrôller leur propre vie à travers l'auto-organisation individuelle et l'action collective. \item Arachnoïde: Une synthmorph robotique ressemblant à une araignée. \item Argonautes: Une faction de scientifique tecno-progressistes qui font la promotion d'une utilisation responsable et éthique de la technologie. \item RA: Réalité Augmentée. Informations du mesh (réseau de donnée universel) qui sont superposées à vos sens du monde réel. Les données RA sont habituellement entoptique (visuelles), mais peuvent aussi être auditives, tactiles, olfactives, kinesthésique (conscience corporelle), émotionnelles ou tout autre type d'entrée. \item Async: Une personne avec des pouvoirs psi. \item UA: Unité Astronomique La distance entre la Terre et le Soleil, équivalent à 8,3 minutes lumières, ou à peu près 150 millions de kilomètres. \item Autonomistes: L'alliance des anarchistes, des Barsoomiens, des Extropiens, de la racaille et des Titaniens. \item Barsoomien: Un Martien rural, typiquement irrité par le contrôle des hypercorp. \item Piratage Basilique: Une image ou toute autre entrée sensorielle qui affecte le cortex visuel du cerveau aisni que ses capacités de recognition de motifs d'une manière à provoquer une erreur et de peut-être l'exploiter pour réécrire du code neuronal. \item Ruche: Un habitat à microgravité créé dans un astéroïde ou une lune évidée. \item BF: (Before the Fall) Avant la Chute (utilisé comme date de référence). \item Bioconservateurs: Un mouvement anti-technologie qui milite pour une régulation stricte de la nanofabrication, des IA, de l'upload, du fork, des améliorations cognitives et de toute autre techonologie perturbatrice. \item Biomorph: Un corps bilogique, qu'i ls'agisse d'un plat, d'un splicer, d'un transhumain génétiquement amélioré ou d'un pod. \item Banque de Corps: Un service pour louer, vendre, acheter  ou stocker une morph. Aussi appelé maison de poupée, morgue \item Bots: Robots. Desz coquilles synthétiques pilotée par des IA. \item Sonde Bracewell: Un type de sonde autonome de surveillance de l'espace profond conçue pour établir le contact avec des civilisations étrangères. \item Brinkeurs: Des éxilés qui vivent aux limites du système, ansi que dans tous les autres coins et recoins bien cachés du système. Aussi appellée isolés, limités, dériveurs. \item Caisse: Une coquille synthétique bon marché, commune et produite en masse. \item Chimère: Un transgénique, contenant des traits génétiques d'autre espèces. \item Circumjovien: Orbitant autour de Jupiter. \item Circumlunaire: Orbitant autour de la Lune. \item Circumsolaire: Orbitant autour du Soleil. \item Cislunaire: Entre la Terre et la Lune. \item Clade: Une espèce ou un groupe d'organisme partageant des caractéristique. Utilisé pour se référer aux sous-espèces transhumaines et au types de morph. \item Bulle Cole: Un habitat formé d'un astéroïde ou d'une lune évidée et en rotation pour obtenir une gravité. \item Machine d'Abondance: Un nanofabeur a but généraliste. \item Pile Corticalle: Une cellulle de mémoire implantée et utilisée pour sauvegarder un ego. Localisée là où l'épine dorsale rencontre le crâne; elle peut-être extraite. \item Cybercerveau: Un cerveau artificiel, hébergeant un ego. Utilisé à la fois dans les synthmorphs et dans les pods. \item Darkcast: Services de farcast et d'egocast ilégaux et trouvés au marché noir. \item Règles de Domaine: Les règles qui régissent la réalité dans un simulspace de réalité virtuelle. \item Drône: Un robot conrtrollé par téléopération (plutôt que par des IA embarquées). \item Ecto: Périphérique de mesh personnel, souple, étirable, auto-nettoyant, translucide et alimenté par énergie solaire. De ecto-lien (lien externe). \item Ego: La part de vous qui bascule d'un corps à l'autre. Aussi appelé ghost, âme, essence, esprit, persona. \item Egocaster: Terme pour envoyer un ego par farcasting. \item Entoptiques: Images de Réalité Augmentée que vous "voyez" dans votre tête. ("Entoptique" signifie "à l'intérieur des yeux") \item ETI: Intelligence extra-terrestre. Le terme utilisé par Firewall pour faire référence aux intelligence étrangères post-singularité de niveau divin théoriquement responsable du virus Exsurgent. \item Exaltés: Humains génétiquement améliorés (entre génétiquement réparés et transhumains). Aussi connus comme génomonstre, les ascendants, les élevés. \item Exoplanète: Une planète dans un autre système solaire. \item Exsurgent: Quelqu'un infecté par le virus Exsurgent \item Virus Exsurgent: Le virus multi-vecteur créé par un ETI inconnu et répandu dans la galaxie dans des sondes Bracewell. Le virus Exsurgent est mutant et peu infecter à la fois les systèmes informatique et les créatures biologiques. \item Extrasolaire: Hors du système solaire. \item Facteurs: La race étrangère ambassadrice qui fait affaire avec la transhumanité. Aussi appelés les Courtiers. \item La Chute: L'apocalypse; la singularité et les guerres qui ont presque amenées l'extinction de l'humanité. \item Farcasting: Communication intrasolaire utilisant des technologies de communication classiques (radio, laser, etc) et la téléportation quantique. Parfois appelée Hyeprdiff. \item Long Porteur: Transport spatiaux de longue distance. \item Firewall: La conspiration secrète, multifaction qui travaille à protéger la transhumanité des "risques existentiels" (risques qui menacent l'existence de la transhumanité). \item Bas de plancher: Quelqu'un qui est né ou habitué à vivre sur une planète ou une lune avec une gravité. \item Plats: Humains de base (sans modificatiosn génétique). Aussi appelés norms. \item Flexbot: Une synthmorph capable de changer de forme ou de rejoindre d'autre trasnformers afin de créer des forms plus grande et modulaires. Aussi appelé Transformers \item Forker: Copier un ego. Tous les forks ne sont pas des copies complètes. Aussi appelés sauvegardes. \item FTL: Faster-Than-Light. Plus rapide que la lumière. \item Fury: Une morph de combat transhumaine. \item Resquilleurs: Explorateurs qui tentent leur chances en utilisant une Porte de Pandorre pour aller vers un endroit pour l'instant inexploré. \item Génohacker: Quelqu'un qui manipule le code génétique pour créer des modifications génétiques voire même de nouvelles formes de vie. \item Ghost: Une morph de combat transhumaine optimisé pour la furtivisté et  l'infiltration. \item Ghost-riding: L'acte de transporter une infomorph dans un implant sépcial à l'intérieur de votre tête. \item Grecs: Astéroîdes ou lunes troyens qui partagent la même orbite qu'une planète ou lune plus grosse, mais qui ont 60 degrés d'avance sur l'orbite, au point de Lagrange L4. Le terme Grecs fait normalement référence aux astéroïdes orbitant autour du point L4 de Jupiter. Voir aussi à "Troyens." \item Habtech: Un technicien d'habitation. \item Héliopause: Le point auquel la pression des vents solaires s'équilibre avec les moyennes interstellaires (aux allentours de 100 AU). \item Hibernoïdes: Un transhumain modifié pour l'hibernation, pour des travaux prolongés dans l'espace. \item Glacetéroïde: Un astéroïde constitué essentiellemnt de glace au lieu de roche et de métal. \item Iktomi: Le nom donné à la mystérieuse race étrangère dont les reliques ont été retrouvées au delà des Portes de Pandorre. \item Contractés: Des esclaves sous contrats synallagmatique qui ont signé pour travailer avec une hypercorp ou une autre autorité, habituellement en échange d'une morph. \item Infovie: Intelligence artificielle généraliste et IAs germe. \item Infomorph: Un égo digitalisé; un corps virtuel. Aussi connus sous les noms de datamorph, uploads, sauvegardes. \item Infugié: ``Infomorph refugié,'' ou quelqu'un qui a tout abandonné sur Terre - y compris son corp - pendant la Chute. \item Isolés: Ceux qui vivent dans des communautés isolées loin au-delà des limites du système (dans la Ceinture de Kuiper et le Nuage d'Oort); aussi appelés outsters, limités. \item Saturer: l'acte de "devenir" un drône opéré à distance grâce à la technologie XP. Également utilisé pour l'accession à un flux XP en temps-réel de lifeblogeur et autre émetteurs en temps-réel. \item Ceinture de Kuiper: Une région de l'espace partant de l'orbite de Neptune et s'étalant sur envrion 55 UA, légèrement peuplée d'astéroïdes, de comètes et de planètes naines. \item Point de Lagrange: L'une des cinqs zones relative à un petit corps planétaire orbitant autour d'un plus gros dans lesquelles les forces gravitationnelles de ces deux corps sont neutralisées. Les point de Lagrange sont considérés comme stables et sont des positions idéales pour des habitats. \item Lifeblog: L'enregistrement de toute l'expérience de la vie de quelqu'un, rendue possible grâce aux capacités mémoires des ordinateurs quasi-illimitées. \item Generation Égarée: Dans une tentative de repeupler après la Chute, une génération d'enfant fut élevée en utilisant des techniques de croissance forcée. Les résultats furent désastreux: beaucoup sont morts ou devenus fous, et le reste a été stigmatisé. \item Ceinture Principale: La principale ceinture d'astéroïdes, un anneau torique orbitant entre Mars et Jupiter. \item Meme: Une idée virale. \item Mentalistes: Transhumains optimisés pour les compétences mentales et cognitive. \item Mercuriels: Les éléments conscient non-humains de la "famille" transhumaine; incluant les IAG et les animaux éveillés. \item Mesh: L'omniprésent maillage sans-fil de réseau de données. Egalement utilisé comme verbe (mesher) et comme adjectif (meshé ou nonmeshé). \item Mesh ID: La signature unique attachée à l'activité meshée de quelqu'un. \item Microgravité: Zéro-g ou environnement quasiment sans poids. \item Mist: Les nuages de données RA qui brouillent parfois votre perception et vos affichages. \item Morph: Un corps physique. Aussi appelé costume, veste, gaine, coquille, forme. \item Muse: IA d'assitant personnel. \item Nanobot: Une machine nanoscopique. \item Nano-écologie: Mouvement écologique pro-technologie. \item Essaim de nanite: Une masse de petits nanobots libérée dans un environnement. \item Neo-Aviens: Perroquets gris et corbeaux élevés. \item Néogenèse: La création d'une nouvelle forme de vie grâce aux manipulations génétiques et à la biotechnologies. \item Neo-Hominidés: Chimpanzés, gorilles et orang-outans élevés \item Néoteniques: Transhumains modifiés pour conserver une forme enfantine. \item Novacrabe: Un pod créé à partir de crabe araignés génétiquement conçus. \item Olympien: Une biomorph transhumaine modifiée pour l'athéltisme et l'endurance. \item Cylindre O'Neill: Un habitat en forme de canette, soumis à une rotation pour créer une gravité. \item Nuage d'Oort: Le "nuage" sphérique constitué de comètes qui entoure le système solaire et qui s'étend jusqu'à une année lumière du soleil. \item PAN: Personal area network/Réseau personnel. Le réseau créé lorsque vous asservissez tous vos périphériques électroniques mineur à votre ecto ou votre insert de mesh. \item Porte de Pandorre: Les portails de trou-de-ver abandonné par les TITANs. \item Pods: Des morphs à la fois biologique et synthétiques. Les clones utilisé pour créer les pods subissent une croissance forcée et possèdent un cerveau informatique. Aussi appelé bio-bots, pelure, répliquants. \item Posthumain: Un individu, humain ou un transhumain, ou une espèce qui a été génétiquement ou cognitivement modifié à un point qu'il n'est plus réellement humain (un cran au-dessus de transhumain). Aussi appelé parahumain. \item Prométhéens: Un groupe d'IAs germes pro-transhumaine créées par le Projet Canot de Sauvetage (précurseurs des argonautes) des années avant que les TITANS ne développent une conscience d'eux et qui ont (presque) évitées l'Infection Exsurgente. Les Prométhéens travaillent secrètement en soutien de Firewall et luttent contre les menaces existentielles. \item Proxys: Membres de la structure interne de Firewall. \item Psi: Pouvoirs parapsychologique développés suite à l'infection par la souche Watts-MacLeod du virus Exsurgent. \item Reapeur: Une synthmorph de combat. \item Réclamationnistes: Une faction transhumanistes qui cherche à lever l'interdiction et à récupérer la Terre. \item Redneck: Un Martien rural. Voir Barsoomien. Aussi appelés Reds. \item Reinstantiés: Réfugiés de la Terre qui se sont échappées sous la forme d'infomorph sans corps, mais qui ont depuis été réincarné. \item Se Réincarner: Changer de corps, ou être téléchargé dans un nouveau corps. Aussi appelé remorphing, regainage, basculer, renaissance. \item Rusteur: Biomorph optimisée pour la vie sur Mars. \item Scorcheur: Programme hostile qui peut endommager ou affecter un cybercerveau. \item Racaille: La faction nomade de punks/gitans de l'espace qui voyagent de stations en stations dans des barges lourdement modifiées ou dans des nuées de vaisseaux. Connus pour être des marchés noirs errants. \item IA germe: Une IAG capable d'auto-apprentissage récursif, lui permettant d'atteindre des niveaux d'intelligences similaire à ceux des dieux. \item Sentinelles: Agents de Firewall \item Coquille: Une morph physique synthétique. Aussi appellé synthmorph. \item Simulmorph: L'avatar que vous utilisez dans les simulspace RV. \item Simulspace: Environnement de réalité virtuelle permettant une immersion sensorielle complète. \item Singularité: Un point de progrés technologique rapide, exponentiel et récursif, au-delà duquel le futur devient impossible à prévoir. Souvent utilisé pour faire référence à l'ascenssion des IAs germe à des niveaux d'intelligence divins. \item Adepte de la Singularité: Des personnes qui cherchent des reliques et des preuves que les TITANs ou d'autres super-intelligences, soit pour en apprendre plus sur eux ou pour devenir une super-intelligence. \item Peau: Une morph physique biologique. Aussi appelé viande, chair. \item Habiller: Modifier son environnement perçu par la réalité augmenté grâce à des programmes. \item Exploit Psi: Un pouvoir psi. \item Slitheroïde: Une synthmorph robotique en forme de serpent. \item Animaux Intelligents: Espèces animales partiellement élevées (incluant chiens, chats, rats et cochons). D'autres gros animaux intelligents (baleines, éléphants) sont au bord de l'extinction. \item Spimes: Périphériques meshé, conscient et localisés. \item Spliceurs: Humains qui sont génétiquement modifiés pour éliminer les maladies génétiques et quelques autres aspects. Aussi connu comme génofixé, génolavés, bidouillés. \item Swarmanoïde: Une morph synthétique composé d'un essaim de robots de la taille d'un insecte. \item Sylphes: Biomorph transhumaine d'une exotique beauté  \item Synthmorph: Morphs syntéhtiques. Coquilles robotiques possédant des égos transhumains. \item Synths: Un type spécifique de synthmorph. Les synths sont des androïdes/gynoïdes classiques; des robots conçus pour être humanoïde, bien qu'ils soit facile de remarquer qu'ils ne sont pas humains. \item Téléopération: Contrôle à distance. \item Titanien: Quelqu'un qui vient de Titan, l'une des lunes de Saturne. \item TITANs: Les IAs germes créées par l'homme, capable d'apprentissage récursifs qui ont subit un décollage abrupte de la singularité et qui ont déclenchés la Chute. La désignation militaire originelle était TITAN: Total Information Tactical Awareness Network (Réseau Cognitif d'Information Tactique Complète). \item Tore: Un habitat en forme de donut, soumit à une rotation pour générer de la gravité. \item Transgénique: Qui contient des traits génétiques d'autres espèces. \item Transhumain: Un humain largement modifié. \item Troyens: Astéroïdes ou lunes qui partagent la même orbite qu'une autre planète ou lune, mais qui la suit avec 60° de décalage, à l'avant ou à l'arrière au poinst de Lagrange L4 ou L5. Le terme de Troyens fait noramellement référence aux astéroïdes orbitant aux point de Lagrange de Jupiter, mais Mars, Saturne, Neptune et d'autres corps ont aussi des Troyens. Voir aussi "Grecs." \item Élever: Élever à la conscience un animal en le transformant génétiquement. \item Travailleur du Vide: Ouvrier de l'espace. \item Vapor: Une émulation cognitive ratée ou un fork/une infomorph criblé de défaut (dérivé de vaporware). \item VPNs: Virtual private networks/Réseaux Privés Virtuels Des réseaux transitant à travers le mesh, habituellement chiffrés pour la protection de la vie privé et pour la sécurité. \item RV: Réalité Virtuelle. Imposer une réalité hyper-réaliste construite artificiellement par-dessus les sens physique de quelqu'un. \item X-Diffeur: Quelqu'un qui transmet et vends des enregistrement XP de leurs propres expérience (dérivé de X-Diffuseurs). \item Xénomorph: Forme de vie étrangères. \item Xé: Comme dans "X-é" - quelqu'un qui est accros ou obsédé par les XP. Fait parfois également référence aux personnes qui font de l'XP. \item XP: Experience Playback/Lecture d'expérience. Faire l'expérience des entrées sensorielle de quelqu'un d'autre (en temps réel ou après enregistrement). Aussi appelé experia, sim, simsense, playback. \item Risque X: Risque existentiel. Quelque chose qui menace l'existence même de la transhumanité. \item Zéros: Personnes sans accès sans-fil au mesh. Commun chez certains contractés. \end{itemize} 
>>>>>>> ab07e7255a391f48de47c3ed67a243ec64317163


<<<<<<< HEAD
\subsection{Ego contre Morph} \label{sec:ego-vs.-morph} 

La distinction entre ego (votre esprit et votre personnalité, incluant les souvenirs, les connaissances et les compétences) et morph (votre corps physique et ses capacités) et l'une des caractéristique d'Eclipse Phase. Une bonne compréhension de ce concept dés le départ offrira un aperçu de toutes les possibilités narratives aux joueurs. 

Votre corps est jetable. Si il devient vieux, malade ou trop gravement abimé, vous pouvez numériser votre conscience et la télécharger dans un nouveau. Le processuss n'est ni bon marché, ni simple, mais il vous garanti une immortalité effective - tant que vous vous rappelez de vous sauvegarder et que vous ne devenez pas fou. Le terme de morph est utilisé pour décrire tout type de forme que votre esprit habite, qu'il s'agisse d'une enveloppe clonée cultivé en cuve, d'une coquille robotique et synthétique, d'un "pod" en partie bio et en partie synthétique, ou même de l'état purement logiciel d'une informoph. 

La morph d'un personnage peu mourir, mais l'ego du personnage continuera à vivre, du moment que les mesures de sauvegardes nécessaires ont été prises. Les morphs sont immuables, mais l'ego de votre personnage représente la continuité des chemins pris par son esprit et sa personnalité tout au long de sa vie. Cette continuité peut-être interrompue par une mort inattendue (dépendant de la date de la dernière sauvegarde), mais elle représente la somme de l'état mental du personnage et de ses expériences. 

Des apects de votre personnage - en particulier les compétences, ainsi que quelques traits et statistiques - appartiennent à l'ego de votre personnage et ainsi l'acompagnent tout au long du développement du personnage. D'autres statistiques et traits sont cependant déterminés par une morph, comme noté précédemment, et changeront donc si votre personnage quitte son corps et en prend un autre. Les morphs peuvent aussi affecter d'autres compétences et statistiques, comme détaillée dans la description des morphs. 



\subsection{Où aller maintenant?} \label{sec:where-go-from} 

Maintenant que vous savez de quoi parle ce jeu, nous vous suggérons de lire le chapitre Une Époque d'Eclipse (p. 30), pour avoir une idée du cadre de jeu par défaut (que vous êtes, bien entendu, libre de changer pour  l'adapter à vos envies). Lisez ensuite le chapitre Mécaniques de Jeu (p. 112) pour avoir une idée des règles. Après ça, vous pouvez passer à la Création et Évolution de Personnage (p. 128) et créer votre premier personnage! 



\subsection{Terminologie} \label{sec:terminology} 

Eclipse Phase utilise tout un jargon pour transmettre simplement les nombreux concepts couverts par ce livre. Bien que non exhaustive, cette liste de terme permettra aux joueurs de s'acclimater rapidement à leur voyage dans Eclipse Phase. Si vous lisez quelque chose et que vous êtes perdus, ne vous inquiétez pas. Ces concepts sont entièrement détaillés dans d'autres sections du livre. 

Notez que plusieurs des mots sur la liste sont des termes scientifiques standard, souvent utilisé en astronomie. Comme Eclipse Phase essaye de rester aussi proche que possible du "hard science" - tout en permettant aux joueurs d'interagir avec les passionantes histoires qui attendent d'être révélées - de tels termes sont utilisés librement. 
=======
\textbf{Bienvenue à Firewall} 

[Message Entrant Reçus. Source: Inconnue] 

[Analyse Quantique: Pas d'interception Détectée] 

[Déchiffrement Complet] 

Salutations, 

Vos références et votre histoire ont été vérifiées trois fois et confirmées, et vous êtes maintenant validé en tant que processus sentinelle. Bienvenue à Firewall, l'ami. 

Pour ceux qui arrivent juste dans notre réseau privé, Firewall est une organisation dévouée à la protection de la transhumanité des menaces - à la fois internes et externes - et à la persistence de notre espèce. La Chute nous a peut-être rappelé que notre capacité à survivre et prospérer n'était pas garantie, mais les notres ont un spectre d'attention remarquablement réduit. En dépit de notre réalisation d'une quasi-immortalité fonctionnelle, nous continuons de à faire face à de nombreux dangers qui pourraient contribuer à notre extinction. Certains de ces risques viennent de notre propre factionnalisme et de nos divisions, combiné à de la technologie universellement disponible qui pourrait causer une destruction étenduée ou des décès indicible si elles tombaient dans les mauvaises mains. Certains viennent de notre manque de vision à long terme, incapables de voir les dangers dans lesquels nous nous sommes plongés entraînant notre nevironnement à cause d'actions imprudentes. D'autres proviennent de nos prorpres créations qui se sont retournées contre nous, comme les TITANs l'ont prouvé. D'autres risques peuvent venir d'intelligence étrangère aux motivations que nous ne pouvons pas encore deviner, et dont nous pourrions ne jamais avoir conscience. D'autres enfin pourraient nous menacer par pur hasard et la plus stupide, mais néanmoins meurtrière, causalité d'un univers dans lequel nous ne sommes rien d'autre que d'insignifiantes poussières. 

Firewall existe pour identifier, analyser et contrer ces risques. Nous sommes tous volontaires. Nous mettons tous nos vies en danger afin d'assurer la survie de la tranhsumanité. 

Firewall a existé, sous des noms et des formes différentes, bien avant la Chute. De nombreuses agences avec des plans similaires se sont regroupé à l'aube de ces évènements cataclysmique pour faire un point sur notre situation et nous préparer au pire. Maintenant, nous opérons sous une seule bannière. 

Nous sommes un réseau privé pour deux raisons. Premièrement, notre existence et nos capacités opératoires sont protégées par notre secret. Moins notre opposition sait de choses sur nous, plus nous pouvons les contrer de manière efficace. De manière similaire, certaines autorités pourraient être hostiles à une organisation telle que la notre opérant dans les territoires qu'elles proclament comme les leurs. Bien que certains d'entre eux doivent être au courant de notre existence, nous passons outre de nombreux obstacles juridiques et légaux qui pourraient sinon entraver nos actions et nos objectifs. Deuxièmement, il arrive que notre mission révèle des informations qui ne sont pas seulement dangereuses dans les mauvaises mains, mais qui pourraient en plus déclencher une panique généralisée si elles étaient rendues publique. Dans certains cas, l'existence même d'une telle connaissance peut-être problématique. En conservant ces secrets et en opérant sur lprincipe que vous savez ce que vous avez besoin de savoir, nous controns automatiquement certains risques. 

Firewall est un réseau décentralisé, de pair à pair. Nous avons une hiérarchie minimale et nous ne répondons à personne d'autre qu'à nous-même. Notre structure nodale nous permets de partager des ressources et des talents sans sacrifier la sécurité et la vie privée de nos agents de terrain. Vous avez été recrutés à cause de vos connaissances, possession ou compétences, et/ou parceque vous êtes entrés en contact avec certaines données d'accès restreint. Vous avez prouvé votre volonté à défendre nos objectifs. Nos vies et nos existences - et le futur de la transhumanité - peuvent reposer entre vos mains. 

Voici donc le futur - que nous puissions tous survivre pour le voir. 

[Fin du Message] 

[Ce message s'est auto-supprimé] 
>>>>>>> ab07e7255a391f48de47c3ed67a243ec64317163

\end{itemize} \item Aérostat: Un habitat conçu pour flotter comme un ballon dans la haute atmosphère d'une planète. \item AF: (After the Fall) Après la Chute (utilisé comme date de référence). \item IAG: Intelligence Artificielle Généraliste Une IA qui possède des facultés cognitives comparables ou supérieures à celle d'un humain. Aussi connues sous le nom de "IA forte" (par opposition aux "IA limitées" plus spécialisée). Voir aussi "IA germe." \item IA: Intelligence Artificielle. Généralement utilisé pour se référer à une IA faible; c'est à dire des IAs qui n'englobent pas (ou dans certains cas, qui sont complètement hors de) toute la portée des capacités cognitives humaines. Les IAs diffèrent des IAG dans le fait qu'elles sont généralement spécialisée et/ou intentionnellement bridée/limitée. \item Anarchiste: Quelq'un qui croit que le gouvernement n'est pas nécessaire, que le pouvoir corromp, et que les gens doivent contrôller leur propre vie à travers l'auto-organisation individuelle et l'action collective. \item Arachnoïde: Une synthmorph robotique ressemblant à une araignée. \item Argonautes: Une faction de scientifique tecno-progressistes qui font la promotion d'une utilisation responsable et éthique de la technologie. \item RA: Réalité Augmentée. Informations du mesh (réseau de donnée universel) qui sont superposées à vos sens du monde réel. Les données RA sont habituellement entoptique (visuelles), mais peuvent aussi être auditives, tactiles, olfactives, kinesthésique (conscience corporelle), émotionnelles ou tout autre type d'entrée. \item Async: Une personne avec des pouvoirs psi. \item UA: Unité Astronomique La distance entre la Terre et le Soleil, équivalent à 8,3 minutes lumières, ou à peu près 150 millions de kilomètres. \item Autonomistes: L'alliance des anarchistes, des Barsoomiens, des Extropiens, de la racaille et des Titaniens. \item Barsoomien: Un Martien rural, typiquement irrité par le contrôle des hypercorp. \item Piratage Basilique: Une image ou toute autre entrée sensorielle qui affecte le cortex visuel du cerveau aisni que ses capacités de recognition de motifs d'une manière à provoquer une erreur et de peut-être l'exploiter pour réécrire du code neuronal. \item Ruche: Un habitat à microgravité créé dans un astéroïde ou une lune évidée. \item BF: (Before the Fall) Avant la Chute (utilisé comme date de référence). \item Bioconservateurs: Un mouvement anti-technologie qui milite pour une régulation stricte de la nanofabrication, des IA, de l'upload, du fork, des améliorations cognitives et de toute autre techonologie perturbatrice. \item Biomorph: Un corps bilogique, qu'i ls'agisse d'un plat, d'un splicer, d'un transhumain génétiquement amélioré ou d'un pod. \item Banque de Corps: Un service pour louer, vendre, acheter  ou stocker une morph. Aussi appelé maison de poupée, morgue \item Bots: Robots. Desz coquilles synthétiques pilotée par des IA. \item Sonde Bracewell: Un type de sonde autonome de surveillance de l'espace profond conçue pour établir le contact avec des civilisations étrangères. \item Brinkeurs: Des éxilés qui vivent aux limites du système, ansi que dans tous les autres coins et recoins bien cachés du système. Aussi appellée isolés, limités, dériveurs. \item Caisse: Une coquille synthétique bon marché, commune et produite en masse. \item Chimère: Un transgénique, contenant des traits génétiques d'autre espèces. \item Circumjovien: Orbitant autour de Jupiter. \item Circumlunaire: Orbitant autour de la Lune. \item Circumsolaire: Orbitant autour du Soleil. \item Cislunaire: Entre la Terre et la Lune. \item Clade: Une espèce ou un groupe d'organisme partageant des caractéristique. Utilisé pour se référer aux sous-espèces transhumaines et au types de morph. \item Bulle Cole: Un habitat formé d'un astéroïde ou d'une lune évidée et en rotation pour obtenir une gravité. \item Machine d'Abondance: Un nanofabeur a but généraliste. \item Pile Corticalle: Une cellulle de mémoire implantée et utilisée pour sauvegarder un ego. Localisée là où l'épine dorsale rencontre le crâne; elle peut-être extraite. \item Cybercerveau: Un cerveau artificiel, hébergeant un ego. Utilisé à la fois dans les synthmorphs et dans les pods. \item Darkcast: Services de farcast et d'egocast ilégaux et trouvés au marché noir. \item Règles de Domaine: Les règles qui régissent la réalité dans un simulspace de réalité virtuelle. \item Drône: Un robot conrtrollé par téléopération (plutôt que par des IA embarquées). \item Ecto: Périphérique de mesh personnel, souple, étirable, auto-nettoyant, translucide et alimenté par énergie solaire. De ecto-lien (lien externe). \item Ego: La part de vous qui bascule d'un corps à l'autre. Aussi appelé ghost, âme, essence, esprit, persona. \item Egocaster: Terme pour envoyer un ego par farcasting. \item Entoptiques: Images de Réalité Augmentée que vous "voyez" dans votre tête. ("Entoptique" signifie "à l'intérieur des yeux") \item ETI: Intelligence extra-terrestre. Le terme utilisé par Firewall pour faire référence aux intelligence étrangères post-singularité de niveau divin théoriquement responsable du virus Exsurgent. \item Exaltés: Humains génétiquement améliorés (entre génétiquement réparés et transhumains). Aussi connus comme génomonstre, les ascendants, les élevés. \item Exoplanète: Une planète dans un autre système solaire. \item Exsurgent: Quelqu'un infecté par le virus Exsurgent \item Virus Exsurgent: Le virus multi-vecteur créé par un ETI inconnu et répandu dans la galaxie dans des sondes Bracewell. Le virus Exsurgent est mutant et peu infecter à la fois les systèmes informatique et les créatures biologiques. \item Extrasolaire: Hors du système solaire. \item Facteurs: La race étrangère ambassadrice qui fait affaire avec la transhumanité. Aussi appelés les Courtiers. \item La Chute: L'apocalypse; la singularité et les guerres qui ont presque amenées l'extinction de l'humanité. \item Farcasting: Communication intrasolaire utilisant des technologies de communication classiques (radio, laser, etc) et la téléportation quantique. Parfois appelée Hyeprdiff. \item Long Porteur: Transport spatiaux de longue distance. \item Firewall: La conspiration secrète, multifaction qui travaille à protéger la transhumanité des "risques existentiels" (risques qui menacent l'existence de la transhumanité). \item Bas de plancher: Quelqu'un qui est né ou habitué à vivre sur une planète ou une lune avec une gravité. \item Plats: Humains de base (sans modificatiosn génétique). Aussi appelés norms. \item Flexbot: Une synthmorph capable de changer de forme ou de rejoindre d'autre trasnformers afin de créer des forms plus grande et modulaires. Aussi appelé Transformers \item Forker: Copier un ego. Tous les forks ne sont pas des copies complètes. Aussi appelés sauvegardes. \item FTL: Faster-Than-Light. Plus rapide que la lumière. \item Fury: Une morph de combat transhumaine. \item Resquilleurs: Explorateurs qui tentent leur chances en utilisant une Porte de Pandorre pour aller vers un endroit pour l'instant inexploré. \item Génohacker: Quelqu'un qui manipule le code génétique pour créer des modifications génétiques voire même de nouvelles formes de vie. \item Ghost: Une morph de combat transhumaine optimisé pour la furtivisté et  l'infiltration. \item Ghost-riding: L'acte de transporter une infomorph dans un implant sépcial à l'intérieur de votre tête. \item Grecs: Astéroîdes ou lunes troyens qui partagent la même orbite qu'une planète ou lune plus grosse, mais qui ont 60 degrés d'avance sur l'orbite, au point de Lagrange L4. Le terme Grecs fait normalement référence aux astéroïdes orbitant autour du point L4 de Jupiter. Voir aussi à "Troyens." \item Habtech: Un technicien d'habitation. \item Héliopause: Le point auquel la pression des vents solaires s'équilibre avec les moyennes interstellaires (aux allentours de 100 AU). \item Hibernoïdes: Un transhumain modifié pour l'hibernation, pour des travaux prolongés dans l'espace. \item Glacetéroïde: Un astéroïde constitué essentiellemnt de glace au lieu de roche et de métal. \item Iktomi: Le nom donné à la mystérieuse race étrangère dont les reliques ont été retrouvées au delà des Portes de Pandorre. \item Contractés: Des esclaves sous contrats synallagmatique qui ont signé pour travailer avec une hypercorp ou une autre autorité, habituellement en échange d'une morph. \item Infovie: Intelligence artificielle généraliste et IAs germe. \item Infomorph: Un égo digitalisé; un corps virtuel. Aussi connus sous les noms de datamorph, uploads, sauvegardes. \item Infugié: ``Infomorph refugié,'' ou quelqu'un qui a tout abandonné sur Terre - y compris son corp - pendant la Chute. \item Isolés: Ceux qui vivent dans des communautés isolées loin au-delà des limites du système (dans la Ceinture de Kuiper et le Nuage d'Oort); aussi appelés outsters, limités. \item Saturer: l'acte de "devenir" un drône opéré à distance grâce à la technologie XP. Également utilisé pour l'accession à un flux XP en temps-réel de lifeblogeur et autre émetteurs en temps-réel. \item Ceinture de Kuiper: Une région de l'espace partant de l'orbite de Neptune et s'étalant sur envrion 55 UA, légèrement peuplée d'astéroïdes, de comètes et de planètes naines. \item Point de Lagrange: L'une des cinqs zones relative à un petit corps planétaire orbitant autour d'un plus gros dans lesquelles les forces gravitationnelles de ces deux corps sont neutralisées. Les point de Lagrange sont considérés comme stables et sont des positions idéales pour des habitats. \item Lifeblog: L'enregistrement de toute l'expérience de la vie de quelqu'un, rendue possible grâce aux capacités mémoires des ordinateurs quasi-illimitées. \item Generation Égarée: Dans une tentative de repeupler après la Chute, une génération d'enfant fut élevée en utilisant des techniques de croissance forcée. Les résultats furent désastreux: beaucoup sont morts ou devenus fous, et le reste a été stigmatisé. \item Ceinture Principale: La principale ceinture d'astéroïdes, un anneau torique orbitant entre Mars et Jupiter. \item Meme: Une idée virale. \item Mentalistes: Transhumains optimisés pour les compétences mentales et cognitive. \item Mercuriels: Les éléments conscient non-humains de la "famille" transhumaine; incluant les IAG et les animaux éveillés. \item Mesh: L'omniprésent maillage sans-fil de réseau de données. Egalement utilisé comme verbe (mesher) et comme adjectif (meshé ou nonmeshé). \item Mesh ID: La signature unique attachée à l'activité meshée de quelqu'un. \item Microgravité: Zéro-g ou environnement quasiment sans poids. \item Mist: Les nuages de données RA qui brouillent parfois votre perception et vos affichages. \item Morph: Un corps physique. Aussi appelé costume, veste, gaine, coquille, forme. \item Muse: IA d'assitant personnel. \item Nanobot: Une machine nanoscopique. \item Nano-écologie: Mouvement écologique pro-technologie. \item Essaim de nanite: Une masse de petits nanobots libérée dans un environnement. \item Neo-Aviens: Perroquets gris et corbeaux élevés. \item Néogenèse: La création d'une nouvelle forme de vie grâce aux manipulations génétiques et à la biotechnologies. \item Neo-Hominidés: Chimpanzés, gorilles et orang-outans élevés \item Néoteniques: Transhumains modifiés pour conserver une forme enfantine. \item Novacrabe: Un pod créé à partir de crabe araignés génétiquement conçus. \item Olympien: Une biomorph transhumaine modifiée pour l'athéltisme et l'endurance. \item Cylindre O'Neill: Un habitat en forme de canette, soumis à une rotation pour créer une gravité. \item Nuage d'Oort: Le "nuage" sphérique constitué de comètes qui entoure le système solaire et qui s'étend jusqu'à une année lumière du soleil. \item PAN: Personal area network/Réseau personnel. Le réseau créé lorsque vous asservissez tous vos périphériques électroniques mineur à votre ecto ou votre insert de mesh. \item Porte de Pandorre: Les portails de trou-de-ver abandonné par les TITANs. \item Pods: Des morphs à la fois biologique et synthétiques. Les clones utilisé pour créer les pods subissent une croissance forcée et possèdent un cerveau informatique. Aussi appelé bio-bots, pelure, répliquants. \item Posthumain: Un individu, humain ou un transhumain, ou une espèce qui a été génétiquement ou cognitivement modifié à un point qu'il n'est plus réellement humain (un cran au-dessus de transhumain). Aussi appelé parahumain. \item Prométhéens: Un groupe d'IAs germes pro-transhumaine créées par le Projet Canot de Sauvetage (précurseurs des argonautes) des années avant que les TITANS ne développent une conscience d'eux et qui ont (presque) évitées l'Infection Exsurgente. Les Prométhéens travaillent secrètement en soutien de Firewall et luttent contre les menaces existentielles. \item Proxys: Membres de la structure interne de Firewall. \item Psi: Pouvoirs parapsychologique développés suite à l'infection par la souche Watts-MacLeod du virus Exsurgent. \item Reapeur: Une synthmorph de combat. \item Réclamationnistes: Une faction transhumanistes qui cherche à lever l'interdiction et à récupérer la Terre. \item Redneck: Un Martien rural. Voir Barsoomien. Aussi appelés Reds. \item Reinstantiés: Réfugiés de la Terre qui se sont échappées sous la forme d'infomorph sans corps, mais qui ont depuis été réincarné. \item Se Réincarner: Changer de corps, ou être téléchargé dans un nouveau corps. Aussi appelé remorphing, regainage, basculer, renaissance. \item Rusteur: Biomorph optimisée pour la vie sur Mars. \item Scorcheur: Programme hostile qui peut endommager ou affecter un cybercerveau. \item Racaille: La faction nomade de punks/gitans de l'espace qui voyagent de stations en stations dans des barges lourdement modifiées ou dans des nuées de vaisseaux. Connus pour être des marchés noirs errants. \item IA germe: Une IAG capable d'auto-apprentissage récursif, lui permettant d'atteindre des niveaux d'intelligences similaire à ceux des dieux. \item Sentinelles: Agents de Firewall \item Coquille: Une morph physique synthétique. Aussi appellé synthmorph. \item Simulmorph: L'avatar que vous utilisez dans les simulspace RV. \item Simulspace: Environnement de réalité virtuelle permettant une immersion sensorielle complète. \item Singularité: Un point de progrés technologique rapide, exponentiel et récursif, au-delà duquel le futur devient impossible à prévoir. Souvent utilisé pour faire référence à l'ascenssion des IAs germe à des niveaux d'intelligence divins. \item Adepte de la Singularité: Des personnes qui cherchent des reliques et des preuves que les TITANs ou d'autres super-intelligences, soit pour en apprendre plus sur eux ou pour devenir une super-intelligence. \item Peau: Une morph physique biologique. Aussi appelé viande, chair. \item Habiller: Modifier son environnement perçu par la réalité augmenté grâce à des programmes. \item Exploit Psi: Un pouvoir psi. \item Slitheroïde: Une synthmorph robotique en forme de serpent. \item Animaux Intelligents: Espèces animales partiellement élevées (incluant chiens, chats, rats et cochons). D'autres gros animaux intelligents (baleines, éléphants) sont au bord de l'extinction. \item Spimes: Périphériques meshé, conscient et localisés. \item Spliceurs: Humains qui sont génétiquement modifiés pour éliminer les maladies génétiques et quelques autres aspects. Aussi connu comme génofixé, génolavés, bidouillés. \item Swarmanoïde: Une morph synthétique composé d'un essaim de robots de la taille d'un insecte. \item Sylphes: Biomorph transhumaine d'une exotique beauté  \item Synthmorph: Morphs syntéhtiques. Coquilles robotiques possédant des égos transhumains. \item Synths: Un type spécifique de synthmorph. Les synths sont des androïdes/gynoïdes classiques; des robots conçus pour être humanoïde, bien qu'ils soit facile de remarquer qu'ils ne sont pas humains. \item Téléopération: Contrôle à distance. \item Titanien: Quelqu'un qui vient de Titan, l'une des lunes de Saturne. \item TITANs: Les IAs germes créées par l'homme, capable d'apprentissage récursifs qui ont subit un décollage abrupte de la singularité et qui ont déclenchés la Chute. La désignation militaire originelle était TITAN: Total Information Tactical Awareness Network (Réseau Cognitif d'Information Tactique Complète). \item Tore: Un habitat en forme de donut, soumit à une rotation pour générer de la gravité. \item Transgénique: Qui contient des traits génétiques d'autres espèces. \item Transhumain: Un humain largement modifié. \item Troyens: Astéroïdes ou lunes qui partagent la même orbite qu'une autre planète ou lune, mais qui la suit avec 60° de décalage, à l'avant ou à l'arrière au poinst de Lagrange L4 ou L5. Le terme de Troyens fait noramellement référence aux astéroïdes orbitant aux point de Lagrange de Jupiter, mais Mars, Saturne, Neptune et d'autres corps ont aussi des Troyens. Voir aussi "Grecs." \item Élever: Élever à la conscience un animal en le transformant génétiquement. \item Travailleur du Vide: Ouvrier de l'espace. \item Vapor: Une émulation cognitive ratée ou un fork/une infomorph criblé de défaut (dérivé de vaporware). \item VPNs: Virtual private networks/Réseaux Privés Virtuels Des réseaux transitant à travers le mesh, habituellement chiffrés pour la protection de la vie privé et pour la sécurité. \item RV: Réalité Virtuelle. Imposer une réalité hyper-réaliste construite artificiellement par-dessus les sens physique de quelqu'un. \item X-Diffeur: Quelqu'un qui transmet et vends des enregistrement XP de leurs propres expérience (dérivé de X-Diffuseurs). \item Xénomorph: Forme de vie étrangères. \item Xé: Comme dans "X-é" - quelqu'un qui est accros ou obsédé par les XP. Fait parfois également référence aux personnes qui font de l'XP. \item XP: Experience Playback/Lecture d'expérience. Faire l'expérience des entrées sensorielle de quelqu'un d'autre (en temps réel ou après enregistrement). Aussi appelé experia, sim, simsense, playback. \item Risque X: Risque existentiel. Quelque chose qui menace l'existence même de la transhumanité. \item Zéros: Personnes sans accès sans-fil au mesh. Commun chez certains contractés. \end{itemize} 

\begin{quotation} 

<<<<<<< HEAD
<<<<<<< HEAD
\textbf{Bienvenue à Firewall} 

[Message Entrant Reçus. Source: Inconnue] 

[Analyse Quantique: Pas d'interception Détectée] 

[Déchiffrement Complet] 

Salutations, 

Vos références et votre histoire ont été vérifiées trois fois et confirmées, et vous êtes maintenant validé en tant que processus sentinelle. Bienvenue à Firewall, l'ami. 

Pour ceux qui arrivent juste dans notre réseau privé, Firewall est une organisation dévouée à la protection de la transhumanité des menaces - à la fois internes et externes - et à la persistence de notre espèce. La Chute nous a peut-être rappelé que notre capacité à survivre et prospérer n'était pas garantie, mais les notres ont un spectre d'attention remarquablement réduit. En dépit de notre réalisation d'une quasi-immortalité fonctionnelle, nous continuons de à faire face à de nombreux dangers qui pourraient contribuer à notre extinction. Certains de ces risques viennent de notre propre factionnalisme et de nos divisions, combiné à de la technologie universellement disponible qui pourrait causer une destruction étenduée ou des décès indicible si elles tombaient dans les mauvaises mains. Certains viennent de notre manque de vision à long terme, incapables de voir les dangers dans lesquels nous nous sommes plongés entraînant notre nevironnement à cause d'actions imprudentes. D'autres proviennent de nos prorpres créations qui se sont retournées contre nous, comme les TITANs l'ont prouvé. D'autres risques peuvent venir d'intelligence étrangère aux motivations que nous ne pouvons pas encore deviner, et dont nous pourrions ne jamais avoir conscience. D'autres enfin pourraient nous menacer par pur hasard et la plus stupide, mais néanmoins meurtrière, causalité d'un univers dans lequel nous ne sommes rien d'autre que d'insignifiantes poussières. 

Firewall existe pour identifier, analyser et contrer ces risques. Nous sommes tous volontaires. Nous mettons tous nos vies en danger afin d'assurer la survie de la tranhsumanité. 

Firewall a existé, sous des noms et des formes différentes, bien avant la Chute. De nombreuses agences avec des plans similaires se sont regroupé à l'aube de ces évènements cataclysmique pour faire un point sur notre situation et nous préparer au pire. Maintenant, nous opérons sous une seule bannière. 

Nous sommes un réseau privé pour deux raisons. Premièrement, notre existence et nos capacités opératoires sont protégées par notre secret. Moins notre opposition sait de choses sur nous, plus nous pouvons les contrer de manière efficace. De manière similaire, certaines autorités pourraient être hostiles à une organisation telle que la notre opérant dans les territoires qu'elles proclament comme les leurs. Bien que certains d'entre eux doivent être au courant de notre existence, nous passons outre de nombreux obstacles juridiques et légaux qui pourraient sinon entraver nos actions et nos objectifs. Deuxièmement, il arrive que notre mission révèle des informations qui ne sont pas seulement dangereuses dans les mauvaises mains, mais qui pourraient en plus déclencher une panique généralisée si elles étaient rendues publique. Dans certains cas, l'existence même d'une telle connaissance peut-être problématique. En conservant ces secrets et en opérant sur lprincipe que vous savez ce que vous avez besoin de savoir, nous controns automatiquement certains risques. 

Firewall est un réseau décentralisé, de pair à pair. Nous avons une hiérarchie minimale et nous ne répondons à personne d'autre qu'à nous-même. Notre structure nodale nous permets de partager des ressources et des talents sans sacrifier la sécurité et la vie privée de nos agents de terrain. Vous avez été recrutés à cause de vos connaissances, possession ou compétences, et/ou parceque vous êtes entrés en contact avec certaines données d'accès restreint. Vous avez prouvé votre volonté à défendre nos objectifs. Nos vies et nos existences - et le futur de la transhumanité - peuvent reposer entre vos mains. 

Voici donc le futur - que nous puissions tous survivre pour le voir. 

[Fin du Message] 

[Ce message s'est auto-supprimé] 
=======
\textbf{Ce que vous avez réellement besoin de savoir} 

[Message Entrant Reçus. Source: Inconnue] 

[Analyse Quantique: Pas d'interception Détectée] 

[Déchiffrement Complet] 

Assieds toi, et prends toi un putain de verre. 

Oublies toute cette intro merdique générée par des IA que tu viens de lire. Voici la vraie affaire. 

Tu meurs sans doute d'impatience pour savoir ce dans quoi tu t'es fourré. On t'as peut-être déjà donné la ligne du parti; que nous sommes tout ce qui sépare la transhumanité de l'extinction. Ou peut-être que quelqu'un t'as murmuré que nous somme sune opération clandestine qui se mèle de sacré merdiers dans lesquels nous ne sommes pas censés intervenir, et que parfois nous faisons tuer des gens. Tu dois être curieux. Peut-être que tu as une envie de faire justice toi-même, et que tu cherches à faire couler du sang pour une bonne cause. Est-ce que ça aurai une importance pour toi que cette cause ne soit qu'une illusion? Tu es peu-être un adepte des complots et tu meurs d'envie de savoir quels secrets Firewall serre sur sa poitrine collective. Et si ces secrets éparpillés était consciencisuement assemblés en mensonges que nous nous racontons à nous-même pour préserver notre santé mentale? 

Tout ce que tu as entendu, de bon ou de mauvais, à propos de Firewall pourrait bien être vrai. Nous ne sommes pas des anges. Nous avons perdu la clarté de nos idées lorsque les TITANs ont forcé l'upload de leur premier esprit humain. En ce moment, tu dois te demander dans quel bordel tu t'es engagé. Je l'ai fait. 
=======
\textbf{Ce que vous avez réellement besoin de savoir} 

[Message Entrant Reçus. Source: Inconnue] 

[Analyse Quantique: Pas d'interception Détectée] 

[Déchiffrement Complet] 

Assieds toi, et prends toi un putain de verre. 

Oublies toute cette intro merdique générée par des IA que tu viens de lire. Voici la vraie affaire. 

Tu meurs sans doute d'impatience pour savoir ce dans quoi tu t'es fourré. On t'as peut-être déjà donné la ligne du parti; que nous sommes tout ce qui sépare la transhumanité de l'extinction. Ou peut-être que quelqu'un t'as murmuré que nous somme sune opération clandestine qui se mèle de sacré merdiers dans lesquels nous ne sommes pas censés intervenir, et que parfois nous faisons tuer des gens. Tu dois être curieux. Peut-être que tu as une envie de faire justice toi-même, et que tu cherches à faire couler du sang pour une bonne cause. Est-ce que ça aurai une importance pour toi que cette cause ne soit qu'une illusion? Tu es peu-être un adepte des complots et tu meurs d'envie de savoir quels secrets Firewall serre sur sa poitrine collective. Et si ces secrets éparpillés était consciencisuement assemblés en mensonges que nous nous racontons à nous-même pour préserver notre santé mentale? 

Tout ce que tu as entendu, de bon ou de mauvais, à propos de Firewall pourrait bien être vrai. Nous ne sommes pas des anges. Nous avons perdu la clarté de nos idées lorsque les TITANs ont forcé l'upload de leur premier esprit humain. En ce moment, tu dois te demander dans quel bordel tu t'es engagé. Je l'ai fait. 

Firewall est tout un tas de choses. La plupart d'entre elles sont bonnes, mais pas mal d'entre elles sont tellement monstrueuse que tu préfèreras te tirer une balle dans la pile et retourner à un précédent backup, just pour pouvoir l'oublier. Si tu avais des visions romantiques à propos de devenir un héros, oublies-les. Maintenant. Tu ne te sentiras pas héroïque lorsque tu balanceras dans le vide un gamin parcequ'il est infecté par un nanovirus. Tu ne te sentiras pas courageux lorsque tu courras à travers un truc étrange et que tu te chieras dessus. Et tu ne te sentiras même plus humain lorsque tu passeras un coup de fil qui coutera la vie à des douzaines, des centaines, ou même des milliers de gens, même si tu en sauves des millions d'autres. 

Donc. Pourquoi quelqu'un serait suffisament cinglé pour faire partie de ça? Parceque le boulot doit être fait. Notre survie en dépend. Pour certains, c'est de l'altruisme, la défense de la transhumanité. Mais en fait, il s'agît surtout de sauver ta putain de tête. Bien sûr, tu pourrais t'abstenir de prendre des responsabilité et laisser une quelconque autorité auto-proclamée s'en occuper. Mais si les anarchistes ont bien compris quelque chose, c'est que l'on ne peut pas faire confiance aux personnes qui ont le pouvoir. Ils sont, plus souvent qu'à leur tour, une partie du problème. Donc, Firewall fait les choses de manière collective. Nous sommes undergound, mais nous sommes une organisation open source. Nous partageons informations et ressources pour atteindre un but commun. Nous sommes organisés en un réseau de cellulle ad-hoc, comme une foule intelligente. Nous ne laissons personne acquérir trop de pouvoir ou de contrôle. Toutes les personnes impliquées dans une opération ont leur mot à dire. Nous nous surveillons nous-même. Nous venons de tout type d'origine et de factions, mais nous faisons face à un ennemi commun - et nous nous battons pour gagner. Il n'y a pas d'alternative. 

Tu as peut-être entendu parler du paradoxe de Fermi? La question était: pourquoi, avec une galaxie si grande, il y avait tellement peu de signe d'une autre vie? Même si nous avons rencontrés les Facteurs et trouvé des preuves d'autres étrangers, notre voisinnage galactique devrait grouiller d'intelligence - mais ce n'est pas le cas. 

Je vais te dire pourquoi. Ce putain d'univers n'est pas juste. Si la transhumanité avait été rayée de la carte, la galaxie ne l'aurai même pas remarqué. Regarde la Terre par exemple. Cette planète existe encore, acceuille toujours la vie, même si nous sommes partis depuis longtemps. La réalité est une putain indifférente. Oublies toute cette merde utopiste à propos de la vie éternelle. On sera chanceux si on survit une année de plus. Nous avons développés des technologies qui mettent les armes de destructions massives entre les mains de n'importe qui, mais nous sommes toujours une espèce adolescente incapable de passer outre les petites conneries tribales. Si tu veux réellement aller de l'avant et explorer l'univers en tant que postmortel, tu vas devoir y travailler trés dur. La survie n'est pas un droit, c'est un privilège. 

Quand tu t'engages à Firewall, tu te mets à dispo. A chaque fois qu'une merde sort des bois pour te tomber dessus ou que tu pourrais être particulièrement utile pour gérer le bordel, tu reçois un appel. On s'attend à ce que tu abandonnes tout ce que tu es en train de faire et de mettre tout le reste en attente comme si ta vie en dépendait - et c'est probablement le cas. Quand tu seras sur le terrain, pour une opé - on appelles ça "aller au docteur" - ta cellulle aura tout pouvoir pour agir comme elle le juge bon ... garde juste en tête que tu devras répondre à nos question plus tard. Tu as aussi le réseau de Firewall qui te couvre - même si les ressources sont souvent limitées, ne t'attends donc pas à ce qu'on sauve ton cul à chaque fois. D'autres sentinelles peuvent être appellées pour tirer quelques ficelles, mais à chaque fois qu'on le fait, cela menace de révéler un agent, créant un bourbier que nous devons nettoyer, et compliquant les choses de toutes manière. L'autonomie est la clef. 

Une dernière chose: ne jamais, jamais oublier que nous avons des ennemis. je ne parles pas simplement de la tête de nœud qui veut utiliser une tête nuke sur un habitat pour faire une revendication politique ou de ces néo-luddites qui pensent que les fléaux de la guerre biologique nous apprendront une leçon, je parle des agences qui connaissent l'existence de Firewall et qui le considèrent comme une menace. Si ils t'étiquettent comme sentinelle, tes jours sont comptés. Probablement ceux de tes backups aussi. Donc, surveilles tes putains d'arrières. 

Voilà le vrai bazar, aussi honnètement que je peux te le donner. Bienvenue dans notre club-house secrète, camarade. Rappelles-toi: la mort est juste la routine du boulot. 

[Fin du Message] 

[Ce message s'est auto-supprimé] \end{quotation} 
>>>>>>> origin/french

Firewall est tout un tas de choses. La plupart d'entre elles sont bonnes, mais pas mal d'entre elles sont tellement monstrueuse que tu préfèreras te tirer une balle dans la pile et retourner à un précédent backup, just pour pouvoir l'oublier. Si tu avais des visions romantiques à propos de devenir un héros, oublies-les. Maintenant. Tu ne te sentiras pas héroïque lorsque tu balanceras dans le vide un gamin parcequ'il est infecté par un nanovirus. Tu ne te sentiras pas courageux lorsque tu courras à travers un truc étrange et que tu te chieras dessus. Et tu ne te sentiras même plus humain lorsque tu passeras un coup de fil qui coutera la vie à des douzaines, des centaines, ou même des milliers de gens, même si tu en sauves des millions d'autres. 

Donc. Pourquoi quelqu'un serait suffisament cinglé pour faire partie de ça? Parceque le boulot doit être fait. Notre survie en dépend. Pour certains, c'est de l'altruisme, la défense de la transhumanité. Mais en fait, il s'agît surtout de sauver ta putain de tête. Bien sûr, tu pourrais t'abstenir de prendre des responsabilité et laisser une quelconque autorité auto-proclamée s'en occuper. Mais si les anarchistes ont bien compris quelque chose, c'est que l'on ne peut pas faire confiance aux personnes qui ont le pouvoir. Ils sont, plus souvent qu'à leur tour, une partie du problème. Donc, Firewall fait les choses de manière collective. Nous sommes undergound, mais nous sommes une organisation open source. Nous partageons informations et ressources pour atteindre un but commun. Nous sommes organisés en un réseau de cellulle ad-hoc, comme une foule intelligente. Nous ne laissons personne acquérir trop de pouvoir ou de contrôle. Toutes les personnes impliquées dans une opération ont leur mot à dire. Nous nous surveillons nous-même. Nous venons de tout type d'origine et de factions, mais nous faisons face à un ennemi commun - et nous nous battons pour gagner. Il n'y a pas d'alternative. 

Tu as peut-être entendu parler du paradoxe de Fermi? La question était: pourquoi, avec une galaxie si grande, il y avait tellement peu de signe d'une autre vie? Même si nous avons rencontrés les Facteurs et trouvé des preuves d'autres étrangers, notre voisinnage galactique devrait grouiller d'intelligence - mais ce n'est pas le cas. 

Je vais te dire pourquoi. Ce putain d'univers n'est pas juste. Si la transhumanité avait été rayée de la carte, la galaxie ne l'aurai même pas remarqué. Regarde la Terre par exemple. Cette planète existe encore, acceuille toujours la vie, même si nous sommes partis depuis longtemps. La réalité est une putain indifférente. Oublies toute cette merde utopiste à propos de la vie éternelle. On sera chanceux si on survit une année de plus. Nous avons développés des technologies qui mettent les armes de destructions massives entre les mains de n'importe qui, mais nous sommes toujours une espèce adolescente incapable de passer outre les petites conneries tribales. Si tu veux réellement aller de l'avant et explorer l'univers en tant que postmortel, tu vas devoir y travailler trés dur. La survie n'est pas un droit, c'est un privilège. 

Quand tu t'engages à Firewall, tu te mets à dispo. A chaque fois qu'une merde sort des bois pour te tomber dessus ou que tu pourrais être particulièrement utile pour gérer le bordel, tu reçois un appel. On s'attend à ce que tu abandonnes tout ce que tu es en train de faire et de mettre tout le reste en attente comme si ta vie en dépendait - et c'est probablement le cas. Quand tu seras sur le terrain, pour une opé - on appelles ça "aller au docteur" - ta cellulle aura tout pouvoir pour agir comme elle le juge bon ... garde juste en tête que tu devras répondre à nos question plus tard. Tu as aussi le réseau de Firewall qui te couvre - même si les ressources sont souvent limitées, ne t'attends donc pas à ce qu'on sauve ton cul à chaque fois. D'autres sentinelles peuvent être appellées pour tirer quelques ficelles, mais à chaque fois qu'on le fait, cela menace de révéler un agent, créant un bourbier que nous devons nettoyer, et compliquant les choses de toutes manière. L'autonomie est la clef. 

Une dernière chose: ne jamais, jamais oublier que nous avons des ennemis. je ne parles pas simplement de la tête de nœud qui veut utiliser une tête nuke sur un habitat pour faire une revendication politique ou de ces néo-luddites qui pensent que les fléaux de la guerre biologique nous apprendront une leçon, je parle des agences qui connaissent l'existence de Firewall et qui le considèrent comme une menace. Si ils t'étiquettent comme sentinelle, tes jours sont comptés. Probablement ceux de tes backups aussi. Donc, surveilles tes putains d'arrières. 

Voilà le vrai bazar, aussi honnètement que je peux te le donner. Bienvenue dans notre club-house secrète, camarade. Rappelles-toi: la mort est juste la routine du boulot. 

[Fin du Message] 

[Ce message s'est auto-supprimé] \end{quotation} 
>>>>>>> ab07e7255a391f48de47c3ed67a243ec64317163

\end{quotation} 

\begin{quotation} 

\textbf{Ce que vous avez réellement besoin de savoir} 

[Message Entrant Reçus. Source: Inconnue] 

[Analyse Quantique: Pas d'interception Détectée] 

[Déchiffrement Complet] 

Assieds toi, et prends toi un putain de verre. 

Oublies toute cette intro merdique générée par des IA que tu viens de lire. Voici la vraie affaire. 

Tu meurs sans doute d'impatience pour savoir ce dans quoi tu t'es fourré. On t'as peut-être déjà donné la ligne du parti; que nous sommes tout ce qui sépare la transhumanité de l'extinction. Ou peut-être que quelqu'un t'as murmuré que nous somme sune opération clandestine qui se mèle de sacré merdiers dans lesquels nous ne sommes pas censés intervenir, et que parfois nous faisons tuer des gens. Tu dois être curieux. Peut-être que tu as une envie de faire justice toi-même, et que tu cherches à faire couler du sang pour une bonne cause. Est-ce que ça aurai une importance pour toi que cette cause ne soit qu'une illusion? Tu es peu-être un adepte des complots et tu meurs d'envie de savoir quels secrets Firewall serre sur sa poitrine collective. Et si ces secrets éparpillés était consciencisuement assemblés en mensonges que nous nous racontons à nous-même pour préserver notre santé mentale? 

Tout ce que tu as entendu, de bon ou de mauvais, à propos de Firewall pourrait bien être vrai. Nous ne sommes pas des anges. Nous avons perdu la clarté de nos idées lorsque les TITANs ont forcé l'upload de leur premier esprit humain. En ce moment, tu dois te demander dans quel bordel tu t'es engagé. Je l'ai fait. 

Firewall est tout un tas de choses. La plupart d'entre elles sont bonnes, mais pas mal d'entre elles sont tellement monstrueuse que tu préfèreras te tirer une balle dans la pile et retourner à un précédent backup, just pour pouvoir l'oublier. Si tu avais des visions romantiques à propos de devenir un héros, oublies-les. Maintenant. Tu ne te sentiras pas héroïque lorsque tu balanceras dans le vide un gamin parcequ'il est infecté par un nanovirus. Tu ne te sentiras pas courageux lorsque tu courras à travers un truc étrange et que tu te chieras dessus. Et tu ne te sentiras même plus humain lorsque tu passeras un coup de fil qui coutera la vie à des douzaines, des centaines, ou même des milliers de gens, même si tu en sauves des millions d'autres. 

Donc. Pourquoi quelqu'un serait suffisament cinglé pour faire partie de ça? Parceque le boulot doit être fait. Notre survie en dépend. Pour certains, c'est de l'altruisme, la défense de la transhumanité. Mais en fait, il s'agît surtout de sauver ta putain de tête. Bien sûr, tu pourrais t'abstenir de prendre des responsabilité et laisser une quelconque autorité auto-proclamée s'en occuper. Mais si les anarchistes ont bien compris quelque chose, c'est que l'on ne peut pas faire confiance aux personnes qui ont le pouvoir. Ils sont, plus souvent qu'à leur tour, une partie du problème. Donc, Firewall fait les choses de manière collective. Nous sommes undergound, mais nous sommes une organisation open source. Nous partageons informations et ressources pour atteindre un but commun. Nous sommes organisés en un réseau de cellulle ad-hoc, comme une foule intelligente. Nous ne laissons personne acquérir trop de pouvoir ou de contrôle. Toutes les personnes impliquées dans une opération ont leur mot à dire. Nous nous surveillons nous-même. Nous venons de tout type d'origine et de factions, mais nous faisons face à un ennemi commun - et nous nous battons pour gagner. Il n'y a pas d'alternative. 

Tu as peut-être entendu parler du paradoxe de Fermi? La question était: pourquoi, avec une galaxie si grande, il y avait tellement peu de signe d'une autre vie? Même si nous avons rencontrés les Facteurs et trouvé des preuves d'autres étrangers, notre voisinnage galactique devrait grouiller d'intelligence - mais ce n'est pas le cas. 

Je vais te dire pourquoi. Ce putain d'univers n'est pas juste. Si la transhumanité avait été rayée de la carte, la galaxie ne l'aurai même pas remarqué. Regarde la Terre par exemple. Cette planète existe encore, acceuille toujours la vie, même si nous sommes partis depuis longtemps. La réalité est une putain indifférente. Oublies toute cette merde utopiste à propos de la vie éternelle. On sera chanceux si on survit une année de plus. Nous avons développés des technologies qui mettent les armes de destructions massives entre les mains de n'importe qui, mais nous sommes toujours une espèce adolescente incapable de passer outre les petites conneries tribales. Si tu veux réellement aller de l'avant et explorer l'univers en tant que postmortel, tu vas devoir y travailler trés dur. La survie n'est pas un droit, c'est un privilège. 

Quand tu t'engages à Firewall, tu te mets à dispo. A chaque fois qu'une merde sort des bois pour te tomber dessus ou que tu pourrais être particulièrement utile pour gérer le bordel, tu reçois un appel. On s'attend à ce que tu abandonnes tout ce que tu es en train de faire et de mettre tout le reste en attente comme si ta vie en dépendait - et c'est probablement le cas. Quand tu seras sur le terrain, pour une opé - on appelles ça "aller au docteur" - ta cellulle aura tout pouvoir pour agir comme elle le juge bon ... garde juste en tête que tu devras répondre à nos question plus tard. Tu as aussi le réseau de Firewall qui te couvre - même si les ressources sont souvent limitées, ne t'attends donc pas à ce qu'on sauve ton cul à chaque fois. D'autres sentinelles peuvent être appellées pour tirer quelques ficelles, mais à chaque fois qu'on le fait, cela menace de révéler un agent, créant un bourbier que nous devons nettoyer, et compliquant les choses de toutes manière. L'autonomie est la clef. 

Une dernière chose: ne jamais, jamais oublier que nous avons des ennemis. je ne parles pas simplement de la tête de nœud qui veut utiliser une tête nuke sur un habitat pour faire une revendication politique ou de ces néo-luddites qui pensent que les fléaux de la guerre biologique nous apprendront une leçon, je parle des agences qui connaissent l'existence de Firewall et qui le considèrent comme une menace. Si ils t'étiquettent comme sentinelle, tes jours sont comptés. Probablement ceux de tes backups aussi. Donc, surveilles tes putains d'arrières. 

Voilà le vrai bazar, aussi honnètement que je peux te le donner. Bienvenue dans notre club-house secrète, camarade. Rappelles-toi: la mort est juste la routine du boulot. 

[Fin du Message] 

[Ce message s'est auto-supprimé] \end{quotation} 



=======
=======
>>>>>>> ruskov
\chapter{Enter the singularity} \label{chap:enter-the-singularity} 

We humans have a special way of pulling ourselves up and kicking ourselves down at the same time. We'd achieved more progress than ever before, at the cost of wrecking our planet and destabilizing our own governments. But things were starting to look up. 

With exponentially accelerating technologies, we reached out into the solar system, terraforming worlds and seeding new life. We re-forged our bodies and minds, casting off sickness and death. We achieved immortality through the digitization of our minds, resleeving from one biological or synthetic body to the next at will. We uplifted animals and AIs to be our equals. We acquired the means to build anything we desired from the molecular level up, so that no one need want again. 

Yet our race toward extinction was not slowed, and in fact received a machine-assist over the precipice. Billions died as our technologies rapidly bloomed into something beyond control ... further transforming humanity into something else, scattering us throughout the solar system, and reigniting vicious conflicts. Nuclear strikes, biowarfare plagues, nanoswarms, mass uploads ... a thousand horrors nearly wiped humanity from existence. 

We still survive, divided into a patchwork of restrictive inner system hypercorp-backed oligarchies and libertarian outer system collectivist habitats, tribal networks, and new experimental societal models. We have spread to the outer reaches of the solar system and even gained footholds in the galaxy beyond. But we are no longer solely ``human'' ... we have evolved into something simultaneously more and different— something transhuman. 



\section{Starting out} \label{sec:starting-out} 

Eclipse Phase is a post-apocalyptic roleplaying game of transhuman conspiracy and horror. Humans are enhanced and improved, but humanity is battered and bitterly divided. Technology allows the re-shaping of bodies and minds and liberates us from material needs, but also creates opportunities for oppression and puts the capability for mass destruction in the hands of everyone. Many threats lurk in the devastated habitats of the Fall, dangers both familiar and alien. 



\subsection{What is a roleplaying game?} \label{sec:what-roleplaying} 

Have you ever read a book or seen a movie or a television show where a character does something really stupid, like heading into a basement at night when the character knows the serial killer is around? The whole time, you're thinking: ``I wouldn't walk down those creepy stairs to the dark basement, especially without a flashlight. I'd do X, Y, or Z instead!'' Since you're in the passenger's seat for the plot you're reading or watching, however, you simply have to sit back and let it unfold. 

What if you could take hold of the driver's seat? What if you could take the plot in the direction you'd choose? That is the essence of a roleplaying game. 

A roleplaying game (or RPG, for short) is part improvisational theater, part storytelling, and part game. A single person (the gamemaster) runs the game for a group of players that pretend to be characters in a fictitious world. The world could be a mystery game set in the 1920s that takes you adventuring around the globe, a fantasy realm inhabited by dragons and trolls and sword-wielding barbarians, or a science fiction setting with aliens and spaceship and world-crushing weaponry. The players pick a setting that they find cool and want to play in. The players then craft their own characters, providing a detailed history and personality to bring each to life. These characters have a set of statistics (numerical values) that represent skills, attributes, and other abilities. The gamemaster then explains the situation in which the characters find themselves. The players, through their characters, interact with the storyline and each others' characters, acting out the plot. As the players roleplay through some scenarios, the gamemaster will probably ask a given player to roll some dice and the resulting numbers will determine the success or failure of a character's attempted action. The gamemaster uses the rules of the game to interpret the dice rolls and the outcome of the character's actions. 

As a group exercise, the players control the storyline (the adventure), which evolves much like any movie or book but within the flexible plot created by the gamemaster. This gamemaster plot provides a framework and ideas for potential courses of action and outcomes, but it is simply an outline of what might happen—it is not concrete until the players become involved. If you don't want to walk down those stairs, you don't. If you think you can talk yourself out of a situation in place of pulling a gun, then try and make it happen. The script of any roleplaying session is written by the players, and the story, based upon the character's actions and their responses to the events of the plot, will constantly change and evolve. 

The best part is that there is no ``right'' or ``wrong'' way to play an RPG. Some games may involve more combat and dice rolling-related situations, where other games may involve more storytelling and improvised dialogue to resolve a situation. Each group of players decides for themselves the type and style of game they enjoy playing! 



\subsection{What is transhumanism?} \label{sec:what-transhumanism} 

Transhumanism is a term used synonymously to mean ``human enhancement.'' It is an international cultural and intellectual movement that endorses the use of science and technology to enhance the human condition, both mentally and physically. In support of this, transhumanism also embraces using emerging technologies to eliminate the undesirable elements of the human condition such as aging, disabilities, diseases, and involuntary death. Many transhumanists believe these technologies will be arriving in our near future at an exponentially accelerated pace and work to promote universal access to and democratic control of such technologies. In the long scheme of things, transhumanism can also be considered the transitional period between the current human condition and an entity so far advanced in capabilities (both physical and mental faculties) as to merit the label ``posthuman.'' 

As a theme, transhumanism embraces heady questions. What defines human? What does it mean to defeat death? If minds are software, where do you draw the line with programming them? If machines and animals can also be raised to sentience, what are our responsibilities to them? If you can copy yourself, where does ``you'' end and someone new begin? What are the potentials of these technologies in terms of both oppressive control and liberation? How will these technologies change our society, our cultures, and our lives? 



\subsection{Post-apocalyptic, conspiracy and horror themes} \label{sec:post-apoc-consp} 

Several themes pervade Eclipse Phase, some of which the reader may not be intimately familiar with. The following helps define these themes so that as play ers read further into this rulebook, they gain a solid understanding of how Eclipse Phase builds on such themes to create its unique setting. 

Post-apocalyptic is a term used to describe fiction set after a cataclysmic event has ended human civili zation as we know it (usually accompanied by loss of human life on an almost unthinkable scale). The exact mechanism of the disaster is usually unimportant nuclear war, plague, asteroid strike, and so on. The importance of the theme is the human condition. If the world we know is torn away from us and humans suffer horrors beyond imagining in this transforma tion to a post-apocalyptic setting, how does human ity cope? Do we survive and thrive and overcome? Or do we lose our own humanity in the process, o ultimately fall to extinction? Those are the questions that drive this genre. 

To conspire means ``to join in a secret agreement to do an unlawful or wrongful act or to use such means to accomplish a lawful end.'' As such, a con spiracy theory attributes the ultimate cause of an event or a chain of events (whether political, societa or historical) to a secret group of individuals with immense power (including political, wealth and so on) who hide their activities from public view while manipulating events to achieve their goals, regard less of consequences. Many conspiracy theories contend that a host of the greatest events of history were initiated and ultimately controlled by such secret organizations. Of equal importance is the silent struggle between clandestine groups, waging a secret war behind the scenes to determine who influences the future. 

Horror takes many forms, but in Eclipse Phase it is more psychological than gore. It is the uncertainty of survival, the suspense of finding malevolent things among the stars, the fear of the unknown, the dread of facing Things That Should Not Be, the revulsion when encountering alien things, and the sickening realization of the wrong and ghastly things that transhumans are capable of doing to themselves and each other. Horror also arises both from the comprehension that there are scary things beyond our understanding nhabiting our universe and that transhumanity may be its own worst enemy. Despite all of the technological tools and advances available to future transhumans, they still face terrors like losing control of their own dentities, their perceptions, and their mental faculties—not to mention their future as a species. 

Eclipse Phase takes all of these themes and weaves them together in a transhuman setting. The postapocalyptic angle covers the understanding of all that transhumanity has lost, the fight against extinction, and how much of that is a struggle against our own nature. The conspiracy side delves into the nature of the secret organizations that play key roles n determining transhumanity's future and how the actions of determined individuals can change the ives of many. The horror perspective explores the results of humanity's self-inflicted transformations and how some of these changes effectively make us non-human. Tying it all together is an awareness of the massive indifference and the terrible alien-ness that pervades the universe and how transhumanity is insignificant against such a backdrop. 

Offsetting these themes, however, Eclipse Phase also asserts that there is still hope, that there is still something worth fighting for, and that transhumanity can pave its own path toward the future. 



\subsection{But how do you actually play?} \label{sec:but-how-do} 

To play a game of Eclipse Phase, you need the following: 

\end{itemize} \item A group of players and a place to meet (real life or online!) \item One player to act as the gamemaster \item The contents of this book \item Something for everyone to take notes with (note pads, laptops, whatever!) \item Two 10-sided dice per player (or a digital equivalent) \item Imagination \end{itemize} 

\subsubsection{A group of players and a place to meet} \label{sec:group-players} 

While roleplaying games are flexible enough to allow any number of people, most gaming groups number around four to eight players. That number of people brings a good mix of personalities to the table and ensures great cooperative play. 

Once a group of players have determined to play Eclipse Phase, they'll need to designate someone as the gamemaster (see below). Then they'll need to determine a time and place to meet. 

Most roleplaying groups meet once a week at a regularly scheduled time and place: 7:00 PM, Thursday night, Rob's house, for example. However, each group determines where, how they'll play, and how often. One group may decide they can only get together once a month, while another group is so excited to dive into the story potential of Eclipse Phase that they want to meet twice a week (they decide to rotate between their houses, though, so as not to overload a particular player). If a group is lucky enough to have a favorite local gaming store that supports instore play, the group might meet there. Other gaming groups meet in libraries, common rooms at their school, bookstores that have generously-sized ``reading rooms,'' quiet restaurants, and so on. Whatever fits for your gaming group, make it work! 

When getting together for a game, most RPGs use the phrase ``gaming session.'' The length of each gaming session is completely dependent upon the consensus of the playing group, as well as the limitations of the locale where they're playing. The particular story that unfolds in a given session can also impact a session's length. If playing in a game store, the group may only have a four-hour slot and the gamemaster and group may have determined—through several sessions of play—that this is a perfect time frame to enjoy the story they're participating in each week. Another group, however, may want an even shorter length of time. Yet another group may decide that while they'll usually do four-hour sessions, once a month they'll set aside an entire Saturday for a great all-day gaming session. Players will need to dive in and start playing and be flexible to decide what will provide the ultimate enjoyment for their gaming group. 

While the camaraderie of a shared experience of playing face-to-face with a group of friends remains the strength of roleplaying games, groups need not confine themselves to a single mode of play. There are myriad options that can be used. Email, instant messages, message boards, video chats, phone/voip calls, text messages, wikis, (micro-)blogs: any and all of these can be utilized to play the game without having warm bodies in seats directly across the table from one another. 

Finally, when playing groups meet for the first time, they should generate their characters (as opposed to generating characters by themselves). While a gaming group can decide to generate characters individually, often it is far easier once the players are together. This allows those more experienced in roleplaying games to help those new to RPGs. Even more important, it enables the entire group to tailor the characters so there is not too much overlap in capabilities and style. After all, with the wealth of character opportunities available, you don't want to show up at the table with an almost identical character to the player next to you. 

\subsubsection{The gamemaster} \label{sec:gamemaster} 

Once a group has been organized, someone needs to step up and take the reins of the gamemaster. Some groups have a single gamemaster that runs all their gaming sessions month after month. Other groups rotate a gamemaster, with a single gamemaster running a given portion of the unfolding story for several sessions before handing the work off to another player. Once again, the participants should be flexible. Some groups may have the perfect person who loves the work involved and is more than willing to run session after session, while other groups may decide that they all want to take turns both as the gamemaster and as players. 

The gamemaster controls the story. They keep track of what is supposed to happen when, describes events as they occur so that the players (as characters) can react to them, keep track of other characters in the game (referred to as non-player characters, or NPCs), and resolve attempts to take action using the game system. The game system comes into play when characters seek to use their skills or otherwise do something that requires a test to see whether or not they succeed. Specific rules are presented for situations that involve rolling dice to determine the outcome (see Game Mechanics, p. 112). 

The gamemaster describes the world as the characters see it, functioning as their eyes, ears, and other senses. Gamemastering is not easy, but the thrill of creating an adventure that engages the other players' imaginations, testing their gaming skills and their characters' skills in the game world, makes it worthwhile. Posthuman Studios and Catalyst Game Labs will follow the publication of Eclipse Phase with supporting supplements and adventures to help this process along, but experienced gamemasters can always adapt the game universe to suit their own styles. In fact, since Eclipse Phase is published under a Creative Commons License (see p. 5), players are encouraged to tailor the universe to their style of play and also to share that with other players. You never know when a specific choice you've made in the running of a campaign is exactly what another gamemaster and his group is looking for. 

\subsubsection{The contents of this book} \label{sec:contents-this-book} 

Whether you have purchased the print or electronic version, this book is specifically organized to present the information you need to know to start telling your stories in the Eclipse Phase universe. Below you'll find a summary of each chapter of the book. 

\paragraph{A Time of Eclipse:} A comprehensive history and setting fully describes the Eclipse Phase universe and how humanity transitioned from here to there. See p. 30. 

\paragraph{Game Mechanics:} The player's desired actions become reality within the universe through quick and easy-to-use game mechanics. See p. 112. 

\paragraph{Character Creation and Advancement:} Creating a unique character can be one of the most enjoyable experiences of roleplaying. Even more rewarding is watching that character evolve and grow across numerous gaming sessions, far beyond anything your imagination first envisioned. See p. 128. 

\paragraph{Skills:} Beyond a character's innate abilities, their skills are what set them apart. This is what your character knows and what they know how to do. See p. 170. 

\paragraph{Action and Combat:} What is a dramatic story without action and violence? When words fail, weapons will blaze. See p. 186. 

\paragraph{Mind Hacks:} The unusual possibilities offered by psi abilities and mental reprogramming. See p. 216. 

\paragraph{The Mesh:} The all-pervasive nature of the mesh ensures that it is a key element to any story telling. See p. 234. 

\paragraph{Accelerated Future:} The wonders of advanced technologies and how they work. See p. 266. 

\paragraph{Gear:} Personal enhancements, weapons, robots, and everything else in between. See p. 294. 

\paragraph{Game Information:} The quintessential set of insider secrets for gamemasters. See p. 350. 



\subsubsection{Taking notes} \label{sec:taking-notes} 

Whether a gamemaster or player, you'll need a way to track information. Players will be generating characters and making changes to those characters from session to session. Meanwhile, the gamemaster will have a host of information to track: notes on how the story is unfolding due to player character interaction that you'll need to fold into next week's session; changes to NPCs; changes to player characters that the players are not yet aware off (such as a character has been mind hacked but doesn't yet know it); and so on. 

Additionally, some groups enjoy a synopsis of each session that can be compiled and read at a later time in order to enjoy and share their exploits, just as you might fileshare clips from your favorite video game to show off your skill in taking the bad guy down (traditionally this has been called ``bluebooking''). This can be particularly useful if a player was unable to attend a given session, providing a quick re-cap that they can read before attending the next gaming session and thus avoiding a bog-down up-front as that player tries to catch up on current events in the game. The session scribe can be a shared responsibility or assigned, all based upon what a given playing group finds works best for them. Likewise, some gaming groups audiorecord their entire game session, both for later reference and for ``actual play'' podcasts. 

The old standard of a pencil and paper still works wonders. A host of additional technologies, however, provide many new options for players. From a text file on a laptop to a shared wiki, the ability to track large amounts of information in a quick and useful fashion—while simultaneously making appropriate information available to each player from session to session—significantly decreases how much time everyone needs to spend tracking information. That time can now be redirected into the enjoyment of participating in a great story. 



\subsubsection{Dice} \label{sec:dice} 

As described in the Game Mechanics section (p. 112), two ten-sided dice are required to play Eclipse Phase. While most players enjoy the feel of tossing dice onto a table, there are many other mechanisms for rolling two ten-sided dice to achieve a 00 to 99 result. Players who make heavy use of any online technologies for game play—such as using online chatting or video blogging—should find it easy to track down and implement a quick dice-rolling program. 



\subsubsection{Imagination} \label{sec:imagination} 

All too often, it's easy for someone looking at an RPG to be intimidated. So many concepts to grasp, so many ideas that seem overwhelming. Just as described under What is a Roleplaying Game?, however, how often have you read a book or watched that movie and decided that you would have done it better? That's your imagination at work. Just dive in and you'll be amazed at how quickly you can immerse yourself in the Eclipse Phase universe. Soon you'll be spinning stories with the best of them. 

Also, don't forget to tap your resources. Your gaming group is your best resource. What's going on, ideas for how to handle a situation, or how to take on a bad guy: these are just some of the things that can and should be discussed by the gaming group in between sessions, and each is an opportunity to strengthen your imagination. 

Another resource is simply watching TV or reading a good book. Pay attention to how the story is put together, how the characters are built, and how the plot unfolds. Push your imagination and soon you'll be figuring out subplots and who the bad guy is long before it's revealed. Knowing how a story is put together enables you to put together your own stories during each gaming session. 

Finally, eclipsephase.com is the offi cial site for Eclipse Phase. If you have questions about the game or want to see how another group of players handles a given situation, post on the forums. The online community can be just as helpful and enjoyable as a local gaming group. 



\subsection{What do players do?} \label{sec:what-do-players} 

The players can take on a variety of roles in Eclipse Phase. Due to advances in digital mind emulation technology, uploading, and downloading into new morphs (physical bodies, biological or synthetic), it is possible to literally be a new person from session to session. With bodies taking on the role of gear, players can customize their forms for the task at hand. 



\subsubsection{The default campaign} \label{sec:default-campaign} 

In the default story (also known as ``campaign setting''), every player character is a ``sentinel,'' an agent-on-call (or potential recruit) for a shadowy network known as ``Firewall.'' Firewall is dedicated to counteracting ``existential risks''—threats to the existence of transhumanity. These risks can and do include biowar plagues, nanotech swarm outbreaks, nuclear proliferation, terrorists with WMDs, netbreaking computer attacks, rogue AIs, alien encounters, and so on. Firewall isn't content to simply counteract these threats as they arise, of course, so characters may also be sent on information- gathering missions or to put in place pre-emptive or failsafe measures. Characters may be tasked to investigate seemingly innocuous people and places (who turn out not to be), make deals with shady criminal networks (who turn out not to be trustworthy), or travel through a Pandora's Gate wormhole to analyze the relics of some alien ruin (and see if the threat that killed them is still real). Sentinels are recruited from every faction of transhumanity; those who aren't ideologically loyal to the cause are hired as mercenaries. These campaigns tend to mix a bit of mystery and investigation with fierce bouts of action and combat, also stirring in a nice dose of awe and horror. 



\subsubsection{Alternate campaigns} \label{sec:alternate-campaigns} 

When they're not saving the solar system, sentinels are free to pursue their own endeavors. The gamemaster and players can use this rulebook to generate any type of story they wish to tell. However, the following examples provide a brief look at the most obvious opportunities for adventure in Eclipse Phase. 

After each campaign variant below, a list of ``archetypes'' for Eclipse Phase are provided in parenthesis. Archetypes are the names applied to the most common character types featured in those scenarios. For example, in a traditional detective story, the archetypes would be the Detective, the Damsel In Distress, the Hard-bitten Cop, and so on. In a cowboy movie, the archetypes would be the Gunfighter, the Bartender, the Marshal, the Indian Brave, and so on. Players will note that some archetypes fit into multiple story settings. The character creation system (p. 128) allows players to create any of the suggested archetypes. Just as roleplaying games are designed for players to build their own stories, however, these archetypes are just suggestions and players can mix and match how they will. 

\paragraph{Salvage and Rescue/Retrieval Ops:} The Fall left two worlds and numerous habitats in ruins—but these devastated cities and stations contain untold riches for those who are brave and foolhardy enough. Potential hauls include: weapon systems; physical resources; lost databanks; left-behind uploads of friends, family, or important people; new technologies developed and lost in the brief singularity takeoff; valued heirlooms of immortal oligarchs; and much more. Outside of these once-inhabited realms, space itself is a big place and lots of people and things get lost out there. Some need to be saved and some are beyond saving. This option lets players explore the unknown or seek out specific targets on contract. (Archeologist/Scavenger/Pirate/Free Trader/ Smuggler/Black Marketeer) 

\paragraph{Exploration:} There are plenty of opportunities to be had as an explorer, colonist, or long-range scout—perhaps even as one of the few lucky or suicidal individuals who explore through an untested Pandora's Gate. Even the Kuiper Belt, on the fringe of our solar system, is still sparsely explored; there may be riches and mysteries still to be found. Many dangers also lurk in odd corners of the system, from isolationist posthuman factions to secretive criminal cartels, as well as pirates, aliens, and others wishing to remain out of sight. (Explorer/Archeologist/ Scavenger/Singularity Seeker/Techie/Medic) 

\paragraph{Trade:} While the majority of inner system trade is controlled by sleek hypercorporations, many of the smaller or more independent stations rely on small traders. In the post-scarcity outer system, trade takes on a different form, with information, favors, and creativity serving as currency among those who no longer want for anything due to the availability of cornucopia machines. (Free Trader/Smuggler/Black Marketeer/Pirate) 

\paragraph{Crime:} The patchwork of city-state habitats and widely varying laws throughout the system create ample opportunity for those who would make a living from this situation. Black market commodities and activities include infomorph-slave trading, pleasure pod sex industries, data brokerage and theft, extracting/smuggling advanced technologies and scientists, political/economic espionage, assassination, drug and XP dealing, soul-trading, and much more. Whether as an independent or part of an organized criminal element, there are always opportunities for those with a thirst for adventure or profit and questionable morals. (Criminal/Smuggler/ Pirate/Fixer/Black Marketeer/Genehacker/Hacker/ Covert Ops) 

\paragraph{Mercenaries:} The constant maneuvering of ideologically-driven factions, the squabbling over contested resources, and the rush to colonize new exoplanets beyond the Pandora Gates all spark new conflicts on a regular basis. Some of these simmer and seeth as low-intensity conflicts for years, occasionally flaring into raids and clashes. Others break out into all-out warfare. Women and men willing to bear arms for credits are always in demand for good wages. Players can engage in commando and military campaigns in habitats, between the stars, or in hostile planetary environments. (Merc/Security Specialist/Fixer/Bounty Hunter/Ex-Cop/Medic) 

\paragraph{Socio-Political Intrigue:} The corporations and political factions that span the solar system do not always play nice with each other, but neither is it wise for them to openly confront each other except under extreme circumstances. Many battles are fought with diplomacy and political maneuvering, using words and ideas more potent than weapons. Even within factions, social cliques can compete ruthlessly, or heated class confl icts can come to a boil, tearing a society apart from within. In this campaign, the players can start as pawns of some entity who rise through the ranks as they become more enmeshed in the intrigues of their sponsor, play a group of ambassadors and spies stationed in the opposition's capital, or can play a group of activists and radicals fighting for social change. (Politico/ Socialite/Covert Ops/Hacker/Security Specialist/ Journalist/Memeticist) 



\subsection{Where does it take place?} \label{sec:where-does-it} 

While Eclipse Phase is set in the not-too-distant future, the changes that have taken place due to the advancements of technology have transformed the Earth and its inhabitants almost beyond recognition. As players dive into the universe, they'll generally encounter one of the following settings. 



\subsubsection{Humanity's habitats} \label{sec:humanitys-habitats} 

The Earth has been left an ecologically-devastated ruin, but humanity has taken to the stars. When Earth was abandoned, so too were the last of the great nation-states; transhumanity now lacks a single unifying governing body and is instead subject to the laws and regulations of whomever controls a given habitat. 

The majority of transhumanity is confined to orbital habitats or satellite stations scattered throughout the Sol system. Some of these were constructed from scratch in the orbit or Lagrange points of planetary bodies, others have been hewn out of solid satellites and large asteroids. These stations have myriad purposes from trade to warfare, espionage to research. 

Mars continues to be one of transhumanity's largest settlements, though it too, suffered heavily during the Fall. Numerous cities and settlements remain, however, though the planet is only partially terraformed. Venus, Luna, and Titan are also home to significant populations. Additionally, there are a small number of colonies that have been established on exoplanets (on the other side of the Pandora Gates) with environments that are not too hostile towards humanity. 

Some transhumans prefer to live on large colony ships or linked swarms of smaller spacecraft, moving nomadically. Some of these rovers intentionally exile themselves to the far limits of the solar system, far from everyone else, while others actively trade from habitat to habitat, station to station, serving as mobile black markets. 



\subsubsection{The great unknown} \label{sec:great-unknown} 

The areas of the galaxy that have felt the touch of humanity are few and far between. Lying betwixt these occasional outposts of questionable civilization are mysteries both dangerous and wonderful. Ever since the discovery of the Pandora Gates, there has been no shortage of adventurers brave or foolhardy enough to strike out on their own into the unknown regions of space in hopes of finding more alien artifacts, or even establishing contact with one of the other sentient races in the universe. 



\subsubsection{The mesh} \label{sec:mesh} 

While not a ``setting'' in the traditional sense, as the sections describe above, the computer networks known as the ``mesh'' are all-pervasive. This ubiquitous computing environment is made possible thanks to advanced computer technologies and nanofabrication that allow unlimited data storage and near-instantaneous transmission capacities. With micro-scale, cheap-to-produce wireless transceivers so abundant, literally everything is wirelessly connected and online. Via implants or small personal computers, characters have access to archives of information that dwarf the entire 21st-century internet and sensor systems that pervade every public place. People's entire lives are recorded and lifelogged, shared with others on one of numerous social networks that link everyone together in a web of contacts, favors, and reputation systems. 



\subsection{Ego vs. Morph} \label{sec:ego-vs.-morph} 

The distinction between ego (your mind and personality, including memories, knowledge, and skills) and morph (your physical body and its capabilities) is one of the defi ning characteristics of Eclipse Phase. A good understanding of the concept right up front will allow players a glimpse at all the story possibilities out of the gate. 

Your body is disposable. If it gets old, sick, or too heavily damaged, you can digitize your consciousness and download it into a new one. The process isn't cheap or easy, but it does guarantee you effective immortality—as long as you remember to back yourself up and don't go insane. The term morph is used to describe any type of form your mind inhabits, whether a vat-grown clone sleeve, a synthetic robotic shell, a part-bio/part-synthetic ``pod,'' or even the purely electronic software state of an infomorph. 

A character's morph may die, but the character's ego may live on, assuming appropriate backup measures have been taken. Morphs are expendable, but your character's ego represents the ongoing, continuous life path of your character's mind and personality. This continuity may be interrupted by an unexpected death (depending on how recently the backup was made), but it represents the totality of the character's mental state and experiences. 

Some aspects of your character—particularly skills, along with some stats and traits—belong to your character's ego and so stay with them throughout the character's development. Some stats and traits, however, are determined by morph, as noted, and so will change if your character leaves one body and takes on another. Morphs may also affect other skills and stats, as detailed in the morph description. 



\subsection{Where to go from here?} \label{sec:where-go-from} 

Now that you know what this game is about, we suggest that you next read the Time of Eclipse chapter (p. 30), to get a feel for the game's default setting (which you are, of course, free to change to suit your whims). Then read the Game Mechanics chapter (p. 112) to get a grasp of the rules. After that, you can move on to Character Creation and Advancement (p. 128) and create your first character! 



\subsection{Terminology} \label{sec:terminology} 

Eclipse Phase uses a host of jargon to simply convey the numerous concepts covered within the pages of this book. While not all-inclusive, this list of terminology will allow players to quickly acclimate themselves for their journey into Eclipse Phase. If you read something and are confused, don't worry. These concepts are fully explained in later sections of this book. 

Note that several of the words on this list are standard scientific terms, often used in astronomy. As Eclipse Phase attempts to remain as close to ``hard science'' as possible—while allowing players to interact with the great stories waiting to unfold—such terms are used liberally. 

\end{itemize} \item Aerostat: A habitat designed to float like a balloon in a planet's upper atmosphere. \item AF: After the Fall (used for reference dating). \item AGI: Artificial General Intelligence. An AI that has cogni tive faculties comparable to that of a human or higher. Also known as ``strong AI'' (differentiating from more specialized ``weak AI''). See also ``seed AI.'' \item AI: Artificial Intelligence. Generally used to refer to weak AIs; i.e., AIs that do not encompass (or in some cases, are completely outside of) the full range of human cognitive abilities. AIs differ from AGIs in that they are usually specialized and/or intentionally crippled/limited. \item Anarchist: Someone who believes government is unnecessary, that power corrupts, and that people should control their own lives through self-organized individual and collective action. \item Arachnoid: A spider-like robotic synthmorph. \item Argonauts: A faction of techno-progressive scientists that promote responsible and ethical use of technology. \item AR: Augmented Reality. Information from the mesh (universal data network) that is overlaid on your real-world senses. AR data is usually entoptic (visual), but can also be audio, tactile, olfactory, kinesthetic (body awareness), emotional, or other types of input. \item Async: A person with psi abilities. \item AU: Astronomical unit. The distance between the Earth and the Sun, equal to 8.3 light minutes, or about 150 million kilometers. \item Autonomists: The alliance of anarchists, Barsoomians, Extropians, scum, and Titanians. \item Barsoomian: A rural Martian, typically resentful of hypercorp control. \item Basilisk Hack: An image or other sensory input that affects the brain's visual cortex and pattern recognition abilities in such a way as to cause a glitch and possibly exploit it and rewrite neural code. \item Beehive: A microgravity habitat made from a tunneledout asteroid or moon. \item BF: Before the Fall (used for reference dating). \item Bioconservative: An anti-technology movement that argues for strict regulation of nanofabrication, AI, uploading, forking, cognitive enhancements, and other disruptive technologies. \item Biomorph: A biological body, whether a flat, splicer, genetically engineered transhuman, or pod. \item Body Bank: A service for leasing, selling, acquiring, or storing a morph. Aka dollhouse, morgue. \item Bots: Robots. AI-piloted synthetic shells. \item Bracewell Probe: A type of autonomous monitoring deep- space probe meant to make contact with alien civilizations. \item Brinkers: Exiles who live on the fringes of the system, as well as other isolated and well-hidden nooks and crannies. Also called isolates, fringers, drifters. \item Case: A cheap, common, mass-produced synthetic shell. \item Chimeric: Transgenic, containing genetic traits from other species. \item Circumjovian: Orbiting Jupiter. \item Circumlunar: Orbiting the Moon. \item Circumsolar: Orbiting the Sun. \item Cislunar: Between the Earth and the Moon. \item Clade: A species or group of organisms with common features. Used to refer to transhuman subspecies and morph types. \item Cole Bubble: A habitat made from a hollowed-out asteroid or moon, spun for gravity. \item Cornucopia Machine: A general-purpose nanofabricator. \item Cortical Stack: An implanted memory cell used for ego backup. Located where the spine meets the skull; can be cut out. \item Cyberbrain: An artificial brain, housing an ego. Used in both synthmorphs and pods. \item Darkcast: Illegal and black market farcasting and egocasting services. \item Domain Rules: The rules that govern the reality of a virtual reality simulspace. \item Drone: A robot controlled through teleoperation (rather than directly via onboard AI). \item Ecto: Personal mesh devices that are flexible, stretchable, self-cleaning, translucent, and solar-powered. From ecto-link (external link). \item Ego: The part of you that switches from body to body. Also known as ghost, soul, essence, spirit, persona. \item Egocasting: Term for sending egos via farcasting. \item Entoptics: Augmented-reality images that you ``see'' in your head. (``Entoptic'' means ``within the eye.'') \item ETI: Extraterrestial intelligence. The term Firewall uses to refer to the god-like post-singularity alien intelligence theorized to be responsible for the Exsurgent virus. \item Exalts: Genetically-enhanced humans (between genefixed and transhumans). Aka genefreaks, the ascended, the elevated. \item Exoplanet: A planet in another solar system. \item Exsurgent: Someone infected by the Exsurgent virus. \item Exsurgent Virus: The multi-vector virus created by an unknown ETI and seeded throughout the galaxy in Bracewell probes. The Exsurgent virus is self-morphing and can infect both computer systems and biological creatures. \item Extrasolar: Outside the solar system. \item Factors: The alien ambassadorial race that deals with transhumanity. Also called Brokers. \item The Fall: The apocalypse; the singularity and wars that nearly brought about the downfall of transhumanity. \item Farcasting: Intrasolar communication utilizing classical communication technologies (radio, laser, etc.) and quantum teleportation. \item Farhauler: Long distance space shipper. \item Firewall: The secret cross-faction conspiracy that works to protect transhumanity from ``existential threats'' (risks to transhumanity's continued existence). \item Flatlander: Someone born or used to living on a planet or moon with gravity. \item Flats: Baseline humans (not genetically modified). Also called norms. \item Flexbot: A shape-changing synthmorph also capable of joining together with other flexbots in a modular fashion to create larger shapes. \item Forking: Copying an ego. Not all forks are full copies. AKA backups. \item FTL: Faster-Than-Light. \item Fury: A transhuman combat morph. \item Gatecrashers: Explorers who take their chances using a Pandora gate to go somewhere previously unexplored. \item Genehacker: Someone who manipulates genetic code to create genetic modifications or even new life. \item Ghost: A transhuman combat morph optimized for stealth and infiltration. \item Ghost-riding: The act of carrying an infomorph in a special implant module inside your head. \item Greeks: Trojan asteroids or moons that share the same orbit as a larger planet or moon, but are 60 degrees ahead in the orbit at the L4 Lagrange point. The term Greeks normally refers to the asteroids orbiting around Jupiter's L4 point. See also ``Trojans.'' \item Habtech: A habitat technician. \item Heliopause: The point where pressure from the solar wind balances with the interstellar medium (about 100 AU out). \item Hibernoid: A transhuman modified for hibernation, for extensive travel in space. \item Iceteroid: An asteroid made from mostly ice rather than rock or metals. \item Iktomi: The name given to the mysterious alien race whose relics have been found beyond the Pandora Gates. \item Indentures: Indentured servants who have contracted their labor to a hypercorp or other authority, usually in exchange for a morph. \item Infolife: Artificial general intelligences and seed AIs. \item Infomorph: A digitized ego; a virtual body. Also known as datamorphs, uploads, backups. \item Infugee: ``Infomorph refugee,'' or someone who left everything behind on Earth during the Fall—even their own body. \item Isolates: Those who live in isolated communities far outside the system (in the Kuiper Belt and Oort Cloud); aka outsters, fringers. \item Jamming: The act of ``becoming'' a teleoperated drone thanks to XP technology. Also sometimes applied to accesing the real-time XP feed from lifeloggers and others. \item Kuiper Belt: A region of space extending from Neptune's orbit out to about 55 AU, lightly populated with asteroids, comets, and dwarf planets. \item Lagrange Point: One of five areas in respect to a small planetary body orbiting a larger one in which the gravitational forces of those two bodies are neutralized. Lagrange points are considered stable and ideal locations for habitats. \item Lifelog: A recording of one's entire life experience, made possible due to near unlimited computer memory. \item Lost Generation: In an effort to repopulate post-Fall, a generation of children were reared using forced-growth methods. The results were disastrous: many died or went insane, and the rest were stigmatized. \item Main Belt: The main asteroid belt, a torus ring orbiting between Mars and Jupiter. \item Meme: A viral idea. \item Mentons: Transhumans optimized for mental and cognitive ability. \item Mercurials: The non-human sentient elements of the transhuman ``family,'' including AGIs and uplifted animals. \item Mesh: The omnipresent wireless mesh data network. Also used as a verb (to mesh) and adjective (meshed or unmeshed). \item Mesh ID: The unique signature attached to one's mesh activity. \item Microgravity: Zero-g or near weightless environments. \item Mist: The clouds of AR data that sometimes fog up your perception/displays. \item Morph: A physical body. Aka suit, jacket, sleeve, shell, form. \item Muse: Personal AI helper programs. \item Nanobot: A nano-scale machine. \item Nano-ecology: Pro-tech ecological movement. \item Nanoswarm: A mass of tiny nanobots unleashed into an environment. \item Neo-Avians: Uplifted ravens and gray parrots. \item Neogenesis: The creation of new life forms via genetic manipulation and biotechnology. \item Neo-Hominids: Uplifted chimpanzees, gorillas, and orangutans. \item Neotenics: Transhumans modified to retain a child-like form. \item Novacrab: A pod created from genetically-engineered spider crab stock. \item Olympian: A transhuman biomorph modified for athleticism and endurance. \item O'Neill Cylinder: A soda-can shaped habitat, spun for gravity. \item Oort Cloud: The spherical ``cloud'' of comets that surrounds the solar system out to about one light-year from the sun. \item PAN: Personal area network. The network created when you slave all of your minor personal electronics to your ecto or mesh inserts. \item Pandora Gates: The wormhole gateways left behind by the TITANs. \item Pods: Mixed biological-synthetic morphs. Pod clones are force-grown and feature computer brains. Also known as bio-bots, skinjobs, replicants. From ``pod people.'' \item Posthuman: A human or transhuman individual or species that has been genetically or cognitively modified so extensively as to no longer be human (a step beyond transhuman). Aka parahuman. \item Prometheans: A group of transhuman-friendly seed AIs that were created by the Lifeboat Project (precursors to the argonauts) years before the TITANs became self- aware and that (mostly) avoided Exsurgent infection. The Prometheans secretly back Firewall and work to defeat existential threats. \item Proxies: Members of the Firewall internal structure. \item Psi: Parapsychological powers acquired due to infection by the Watts-MacLeod strain of Exsurgent virus. \item Reaper: A warbot synthmorph. \item Reclaimers: A transhuman faction that seeks to lift the interdiction and reclaim Earth. \item Redneck: A rural Martian. See Barsoomian. Aka Reds. \item Reinstantiated: Refugees from Earth who escaped only as bodiless infomorphs, but who have since been resleeved. \item Resleeving: Changing bodies, or being downloaded into a new one. Also called remorphing, reincarnation, shifting, rebirthing. \item Rusters: Biomorphs optimized for life on Mars. \item Scorching: Hostile programs that can damage or affect cyberbrains. \item Scum: The nomadic faction of space punks/gypsies that travel from station to station in heavily-modified barges or swarms of ships. Notorious for being a roving black market. \item Seed AI: An AGI that is capable of recursive self-improvement, allowing it to reach god-like levels of intelligence. \item Sentinels: Agents of Firewall. \item Shell: A synthetic physical morph. Aka synthmorph. \item Simulmorph: The avatar you use in VR simulspace programs. \item Simulspace: Full-immersion virtual reality environments. \item Singularity: A point of rapid, exponential, and recursive technological progress, beyond which the future becomes impossible to predict. Often used to refer to the ascension of seed AI to god-like levels of intelligence. \item Singularity Seeker: People who pursue relics and evidence of the TITANs or other possible avenues to super-intelligence, either to learn more about it or to become part of a super-intelligence themselves. \item Skin: A biological physical morph. Aka meat, flesh. \item Skinning: Changing your perceived environment via augmented reality programming. \item Sleight: A psi power. \item Slitheroid: A snake-like robotic synthmorph. \item Smart Animals: Partially-uplifted animal species (including dogs, cats, rats, and pigs). Some other large smart animals (whales, elephants) are nearly extinct. \item Spime: Meshed, self-aware, location-aware devices. \item Splicers: Humans that are genetically modified to eliminate genetic diseases and some other traits. Also known as genefixed, cleangenes, tweaks. \item Swarmanoid: A synthetic morph composed from a swarm of tiny insect-sized robots. \item Sylphs: Transhuman biomorphs with exotic good looks. \item Synthmorph: Synthetic morphs. Robotic shells possessed by transhuman egos. \item Synths: A specific type of synthmorph. Synths are standard androids/gynoids; robots that are designed to look humanoid, though they are usually noticeably not human. \item Teleoperation: Remote control. \item Titanian: Someone from Titan, a moon of Saturn. \item TITANs: The human-created, recursively-improving, military seed AIs that underwent a hard-takeoff singularity and prompted the Fall. Original military designation was TITAN: Total Information Tactical Awareness Network. \item Torus: A donut-shaped habitat, spun for gravity. \item Transgenic: Containing genetic traits from other species. \item Transhuman: An extensively modified human. \item Trojans: Asteroids or moons that share the same orbit as a larger planet or moon, but follow about 60 degrees ahead or behind at the L4 and L5 Lagrange points. The term Trojans normally refers to the asteroids orbiting at Jupiter's Lagrange points, but Mars, Saturn, Neptune, and other bodies also have Trojans. See also ``Greeks.'' \item Uplifting: Genetically transforming an animal species to sapience. \item Vacworker: Space laborer. \item Vapor: A failed mind emulation or crippled fork/infomorph (from vaporware). \item VPNs: Virtual private networks. Networks that operate within the mesh, usually encrypted for privacy/security. \item VR: Virtual Reality. Imposing an artificially-constructed hyper-real reality over one's physical senses. \item X-Caster: Someone who transmits/sells XP recordings of their experiences. \item Xenomorph: Alien life form. \item Xer: As in ``X-er''—someone who is addicted or obsessed with XP. Sometime used to refer to people making XP as well. \item XP: Experience Playback. Experiencing someone else's sensory input (in real-time or recorded). Also called experia, sim, simsense, playback. \item X-Risk: Existential risk. Something that threatens the very existence of transhumanity. \item Zeroes: People without wireless mesh access. Common with some indentures. \end{itemize} 

\begin{quotation} 

\textbf{Welcome to Firewall} 

[Incoming Message Received. Source: Unknown] 

[Quantum Analysis: No Interception Detected] 

[Decryption Complete] 

Greetings, 

Your references and background have been triple-checked and confirmed, and you are now vetted as a sentinel operative. Welcome to Firewall, friend. 

For those new to our private network, Firewall is an organization dedicated to protecting transhumanity from threats—both internal and external—to our continued existence as a species. The Fall may have reminded us that our ability to survive and prosper is not guaranteed, but our kind has a remarkably short attention span. Despite our achievement of functional near-immortality, we continue to face numerous dangers that may contribute to our extinction. Some of these risks come from our own factionalism and divisiveness, combined with universally available technology that could cause widespread destruction and untold deaths in the wrong hands. Some stem from our short-sightedness, failing to see the dangers in which we place ourselves and our environments through careless actions. Some arise from our own creations turned against us, as the TITANs proved. Other risks may come from alien intelligences whose motivations we cannot yet fathom, and of whom we may not even be aware. Still others may threaten us by sheer chance and the mindless but deadly cause-and-effect of a universe in which we are but an insignificant speck. 

Firewall exists to identify, analyze, and counter these risks. We are all volunteers. We are all placing our own lives at risk in order to ensure the survival of transhumanity. 

Firewall has existed, under other names and guises, since before the Fall. Numerous agencies with a similar agenda banded together in the wake of those cataclysmic events to assess our situation and prepare for the worst. Now we operate under a single umbrella. 

We are a private network for two reasons. First, our existence and operational abilities are protected by our secrecy. The less our opposition knows about us, the more effectively we can counter them. Similarly, certain authorities might be hostile to an organization such as ours operating in their claimed territory. Though some may be aware of our existence, we bypass numerous legal and jurisdictional hurdles that might otherwise hamper our actions and goals. Second, our mission sometimes brings to light information that is not only dangerous in the wrong hands, but might even trigger widespread panic if made public. In some cases, the very existence of such knowledge could be problematic. By retaining secrecy and operating on a need-to-know basis, we automatically counter certain risks. 

Firewall is a decentralized, peer- to-peer network. We have minimal hierarchy and we answer to no one but ourselves. Our node structure enables us to share resources and talents without sacrificing the privacy and security of our operatives. You have been recruited because of your knowledge, assets or skills, and/or because you have come into contact with certain restricted data. You have proven your willingness to support our goals. Our lives and existence—and the future of transhumanity—may rest in your hands. 

So here's to the future—may we all live to see it. 

[End Message] 

[This Message Has Self-Erased] 

\end{quotation} 

\begin{quotation} 

\textbf{What you really need to know} 

[Incoming Message Received. Source: Unknown] 

[Quantum Analysis: No Interception Detected] 

[Decryption Complete] 

Sit down, and grab yourself a fucking drink. 

Forget all of that AI-generated intro crap you just read. Here's the real deal. 

You're probably dying to know what you've been dragged into. Maybe you've been told the party line already: that we're all that stands between transhumanity and extinction. Or maybe someone whispered to you that we're a rogue operation that meddles in heavy shit that we have no authority to get involved in, and that we sometimes get people killed as a result. You must be curious. Maybe you've got a vigilante streak, and you're looking to spill blood for a good cause. Would it matter to you if the cause was a deluded one? Maybe you're a conspiracy wingnut and you're dying to know what secrets Firewall is clutching to its collective chest. What if those secrets shattered the carefully constructed lies that we all tell to ourselves to keep our sanity intact? 

Everything you've heard, good or bad, about Firewall very well may be true. We're not angels. We lost the sheen on our ideals when the TITANs forcibly uploaded their first human mind. Right now, you should be asking yourself what the fuck you just signed up for. I did. 

Truth is, Firewall is lots of things. Most of it is good, but a lot of it so fucking horrible you'll be thinking about planting a bullet in your stack and resorting to an earlier backup, just so you can forget it all. If you have any romantic visions about being a hero, though, drop them now. You won't feel like a hero when you airlock some kid because he's carrying an infectious nanovirus. You won't feel brave when you run across some alien thing and crap your pants. And you won't even feel human anymore when you make a call that will cost dozens, hundreds, or even thousands of people their lives, even if you are saving millions more. 

So why would anyone be crazy enough to be part of this thing? Because it needs to be done. Our survival depends on it. To some people, it's altruism, defending transhumanity. But really, it's about saving your own fucking neck too. Sure, you could abstain from taking responsibility and let some self- described authority take care of it. But if the anarchists have anything right, it's that people in power can't be trusted. As often as not, they're part of the problem. So Firewall does things the collective way. We're underground, but we're an open source operation. We share information and resources towards a common goal. We organize in networked ad-hoc cells, smart-mob style. We don't let anyone accrue too much power or control. Everyone involved in an op has an equal say. We police ourselves. We come from all sorts of backgrounds and factions, but we face a common enemy—and we fight to win. There is no alternative. 

Maybe you've heard of the Fermi Paradox? That question asked why, with a galaxy so huge, there were so few signs of other life? Even though we've met the Factors and seen evidence of other aliens, our galactic neighborhood should be crawling with intelligence—but it's not. 

I'll tell you why. The universe is not fucking fair. If transhumanity were wiped out, the galaxy wouldn't even notice. Just look at the Earth. That planet still exists, still supports life, even though we're far gone. Reality is an uncaring asshole. Forget all that utopian crap about living forever. We'll be lucky to survive another year. We've developed technologies that put weapons of mass destruction in the hands of everyone, but we're still an adolescent species that has trouble overcoming petty tribal bullshit. If you're really looking forward to exploring the universe as a postmortal, you're going to have to work hard at it. Survival isn't a right, it's a privilege. 

When you sign up with Firewall, you put yourself on call. Anytime some shit goes down in your neck of the woods or that you might be particularly helpful in dealing with, you'll get a call. You'll be expected to drop whatever you're doing and put everything else on hold as if your life depended on it—it probably does. When you're in the field, on an op— ''going to the doctor,'' as we call it—your cell is empowered to act as it sees fit ... just keep in mind that you'll be answering to the rest of us later. You'll also have the Firewall network to back you up—though resources are often limited, so don't expect us to always save your ass. Other sentinels can be called on to pull strings, but every time we do so, it threatens to unveil an agent, create a trail that we need to clean up, and otherwise complicates matters. Self-reliance is key. 

One last thing: don't ever, ever forget that we have enemies. I'm not just talking about the nutjob who wants to nuke a habitat to make a political statement or the neo-luddites who think biowar plagues will teach us all a lesson, I'm talking about the agencies that know Firewall exists and consider it a threat. If they tag you as a sentinel, your days are numbered. Maybe your backups too. So watch yer friggin' back. 

So that's the real deal, as honest as I can give it. Welcome to our secret clubhouse, comrade. Remember: death is just another day on the job. 

[End Message] 

[This Message Has Self-Erased] \end{quotation} 



<<<<<<< HEAD
>>>>>>> ruskov
=======
>>>>>>> ruskov
